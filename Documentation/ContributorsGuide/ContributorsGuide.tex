\documentclass[a4paper,11pt,twoside,pdftex]{article}

% Caption package for formating figure and label captions
\usepackage[font=bf,format=hang,labelsep=colon,justification=justified,singlelinecheck=false,compatibility=false]{caption}


% Misc packages
\usepackage{ifthen} % for conditional expressions
\usepackage{float} % for floating tables and graphics
\usepackage{setspace} % for defining spacing between lines
\usepackage{cmap} % allow searching in pdf documents
\usepackage{tabulary} % better tables
\usepackage{comment} % block comments
\usepackage[T1]{fontenc} % for searching for terms with underscores
\usepackage{listings} % for writing Code
\usepackage{csquotes} % for writing Code
\usepackage[usenames, dvipsnames]{color} % Highlighting text for editing
\usepackage{xcolor}
% hyperref for HTML links within pdf 
\usepackage[bookmarks=true,bookmarksopen=false,bookmarksnumbered=true,
            pdfpagelayout=TwoPageRight,
            pdftitle={Casal2 Contributors Guide}, 
            pdfauthor={Casal2 Development Team} %authors
            pdfsubject={Casal2 Contributors Guide},
            pdftex]{hyperref}
\hypersetup{
  breaklinks=true,      % allow line breaks in URLs
  colorlinks=true,      % use colour to define links
  linkcolor=black,      % colour of internal links
  citecolor=black,      % colour of links to bibliography
  filecolor=black,      % colour of file links
  urlcolor=darkgray     % colour of external links
}
\pdfadjustspacing=1

\definecolor{codegreen}{rgb}{0,0.6,0}
\definecolor{codegray}{rgb}{0.5,0.5,0.5}
\definecolor{codepurple}{rgb}{0.58,0,0.82}
\definecolor{backcolour}{rgb}{0.95,0.95,0.92}

\lstdefinestyle{Rstyle}{
	backgroundcolor=\color{backcolour},   
	commentstyle=\color{codegreen},
	keywordstyle=\color{magenta},
	numberstyle=\tiny\color{codegray},
	stringstyle=\color{codepurple},
	basicstyle=\ttfamily\footnotesize,
	breakatwhitespace=false,         
	breaklines=true,                 
	captionpos=b,                    
	keepspaces=true,                 
	numbers=left,                    
	numbersep=5pt,                  
	showspaces=false,                
	showstringspaces=false,
	showtabs=false,                  
	tabsize=2
}
\lstset{style=Rstyle}
% Use AMS Maths 
\usepackage[fleqn]{amsmath}
\newcommand\AddVspace{\\[0 pt]} % artificial method of adding vertical space within equations

% Package for including and pretty printing of external config files and program code
\usepackage{listings} 
\lstset{ %
basicstyle=\ttfamily\footnotesize,
breaklines=true,
columns=fullflexible,
showspaces=false,               % show spaces adding particular underscores
showstringspaces=false,         % underline spaces within strings
showtabs=false,                 % show tabs within strings adding particular underscores
tabsize=2,			            % sets default tabsize to 2 spaces
breakatwhitespace=false,	    % sets if automatic breaks should only happen at whitespace
escapeinside={\%*}{*)}          % if you want to add a comment within your code
}

% highlighting package for editing
\usepackage{color,soul}

% Allow colour for HTML links
\usepackage{color}
\definecolor{darkgray}{gray}{0.20}
\definecolor{lightgray}{gray}{0.95}

% Geometery for A4 layout like MS-Word defaults
\usepackage[left=2.54cm,top=2.54cm,bottom=3.17cm,right=3.17cm]{geometry}

% Natlib to better cite references (round brackets, commas between refs, and sorted)
\usepackage[round,comma,sort]{natbib}

% Making the index
\usepackage{makeidx}
\makeindex

% Graphics (no postscript files.. just use jpeg, png, etc)
\usepackage[dvips]{graphicx}
% Changes fonts to Times, Helvetica, Courier
\usepackage{pslatex} 

% Section header fonts
\usepackage{sectsty}
\allsectionsfont{\sffamily\large} % Normal sized arial style section headings
% Add a dot after section headings
\makeatletter
 \def\@seccntformat#1{\csname the#1\endcsname.\quad}
 \renewcommand\paragraph{\@startsection{paragraph}{4}{\z@}%
             {-2.5ex\@plus -1ex \@minus -.25ex}%
             {1.25ex \@plus .25ex}%
             {\normalfont\normalsize\bfseries}}
\makeatother

\setcounter{secnumdepth}{4} % how many sectioning levels to assign numbers to
\setcounter{tocdepth}{4}    % how many sectioning levels to show in ToC

% Page style
\usepackage{fancyhdr}
\pagestyle{fancy}
\fancyhead{}
\fancyfoot{}
\headheight 15pt
\renewcommand{\headrulewidth}{0pt} % rule line under header
\renewcommand{\footrulewidth}{0pt}
\setlength{\parindent}{0pt} % No indentation at start of paragraph
\setlength{\baselineskip}{1ex plus 0.2ex minus 0.1ex}
\setlength{\parskip}{1.1ex} % Gap between paragraphs
\raggedbottom % prefer space at the bottom of page
						
% Make equation numbers be section number then equation number
\makeatletter
\@addtoreset{equation}{section}
\@addtoreset{figure}{section}
\@addtoreset{table}{section}
\def\thefigure{\thesection.\@arabic\c@figure}
\def\thetable{\thesection.\@arabic\c@table}
\def\theequation{\thesection.\@arabic\c@equation}
\makeatother

%% stop figures from going onto a page by themselves 
\renewcommand{\topfraction}{0.85}
\renewcommand{\textfraction}{0.1}
\renewcommand{\floatpagefraction}{0.75}

% Minimise hyphen use
\hyphenpenalty=5000
\tolerance=1000

% Compact titles
%\usepackage[small,compact]{titlesec} 

% New commands to define macros and other aids to text and layout
\newcommand{\config}{input configuration file}
\newcommand{\command}[1] {\texttt{@#1}}
\newcommand{\subcommand}[1] {\texttt{#1}}
\newcommand{\commandsub}[2] {\command{#1}\subcommand{.#2}}
\newcommand{\commandlabsub}[2] {\command{#1\texttt{[label]}}\subcommand{.#2}}
\newcommand{\argument}[1] {\texttt{#1}}
\newcommand{\commandsubarg}[3] {\command{#1}\subcommand{.#2}\argument{=#3}}
\newcommand{\commandlabsubarg}[3] {\command{#1\texttt{[label]}}\subcommand{.#2}\argument{=#3}}

\newcommand{\commentline}{\#}
\newcommand{\commentstart}{\{}
\newcommand{\commentend}{\}}

\newcommand{\textlow}[1]{\raisebox{-.4ex}{\scriptsize #1}}

% command shortcuts:
\newcommand{\Rzero}{\emph{R}$_0$}
\newcommand{\Bzero}{\emph{B}$_0$}
\newcommand{\R}{\textbf{R}}
\newcommand{\SSBoff}{$SSB_{\text{offset}}$}

%quick index:
\newcommand{\I}[1]{#1\index{#1}}

% New commands to template syntax definitions: copied from SPM, setup needs defining
% Define a command without a label
\newcommand{\defCom}[2]{\texttt{\textbf{@#1}\index{Command ! #1}} \hspace{0.5cm} {#2}}
% Define a command with a label
\newcommand{\defComLab}[2]{\texttt{\textbf{@#1}\ \emph{label}\index{Command ! #1}} \hspace{0.5cm} {#2}}
% Define a subcommand
\newcommand{\defSub}[2]{\texttt{#1} \hspace{0.5cm} #2 \\*}% Define a command with an argument
\newcommand{\defComArg}[3]{\texttt{\textbf{@#1}\ \emph{#2}\index{Command ! #1}} \hspace{0.5cm} {#3}}
% Define a Command\index{Command} argument
\newcommand{\defArg}[2]{\emph{\texttt{#1}} \hspace{0.5cm} #2 \\*}

% Generic definition for subcommand syntax: copied from SPM, setup needs defining
\newcommand{\defText}[2]{\hangindent=0.3cm \small{#1\ #2}\normalsize \\*}
% Define subcommand syntax for Type / Default / Condition / Value / Note / Example / Lower Bound / Upper Bound
\newcommand{\defType}[1]{\defText{Type:}{#1}}
\newcommand{\defDefault}[1]{\defText{Default:}{#1}}
\newcommand{\defCondition}[1]{\defText{Condition:}{#1}}
\newcommand{\defValue}[1]{\defText{Value:}{#1}}
\newcommand{\defNote}[1]{\defText{Note:}{#1}}
\newcommand{\defExample}[1]{\defText{Example:}{#1}}
\newcommand{\defLowerBound}[1]{\defText{Lower Bound:}{#1}}
\newcommand{\defUpperBound}[1]{\defText{Upper Bound:}{#1}}
\newcommand{\defAllowedValues}[1]{\defText{Allowed Values:}{#1}}

% Input CASAL2 version definitions
% WARNING: THIS FILE IS AUTOMATICALLY GENERATED BY doBuild version. DO NOT EDIT THIS FILE
\newcommand{\Version}{23.09 (2023-09-27)}
\newcommand{\SourceRepos}{https://github.com/alistairdunn1/CASAL2:Development}
\newcommand{\VersionNumber}{23.09}
\newcommand{\SourceControlRevision}{14bf9327022b81d480d17a5cdf880161d5b4f83b}
\newcommand{\SourceControlDateDoc}{2023-09-27}
\newcommand{\SourceControlYearDoc}{2023}
\newcommand{\SourceControlMonthDoc}{September}
\newcommand{\SourceControlTimeDoc}{21:48:31}
\newcommand{\SourceControlVersion}{2023-09-27 21:48:31 UTC (rev. 14bf93270)}

\newcommand{\DocYear}{\SourceControlYearDoc}
\newcommand{\DocMonth}{\SourceControlMonthDoc}
\newcommand{\DocDate}{\SourceControlMonthDoc\ \SourceControlYearDoc}
\newcommand{\DocVer}{\SourceControlDateDoc}

%New commands to automate document dates, manual titles, document reference, etc.
\newcommand{\VER}{v\SourceControlDateDoc} % CASAL2 program version
\newcommand{\CNAME}{Casal2}
\newcommand{\cname}{\texttt{casal2}} % casal2 binary name
\newcommand{\authors}{Casal2 development team}
\newcommand{\email}{scott@zaita.com}
\newcommand{\github}{\url{https://github.com/Casal2/CASAL2}}
\newcommand{\authorlink}{\href{mailto:"Scott Rasmussen "<scott@zaita.com>?subject=Casal2:}{authors}} 
%hyper ref for email
\newcommand{\Organisation}{National Institute of Water \& Atmospheric Research Ltd.}
\newcommand{\ManualRef}{\authors\ (\DocYear). \CNAME\ Contributors Guide, \VER. \ref{TotPages} p. (Using source code from \SourceRepos)} % full document reference

% Define \clearemptydoublepage so-as to have truly blank pages between sections
\let\origdoublepage\cleardoublepage
\newcommand{\clearemptydoublepage}{%
  \clearpage
  {\pagestyle{empty}\origdoublepage}%
}
%% Commands for the license section taken from http://www.gnu.org/licenses/lgpl-3.0.tex
\renewcommand{\labelenumii}{\alph{enumii})}
\renewcommand{\labelenumiii}{\arabic{enumiii})}

% Load package to count the number of pages in document
% For getting number of pages in document (NOT the last page number printed), use \ref{TotPages}
% Load this last to ensure its macros are not overwritten
\usepackage{totpages} 

\makeatother

%Begin the document
\begin{document} 
\hbadness=10000 % to deal with underfull hbox warnings
\sloppy % use sloppy paras

% Title page
\pdfbookmark[1]{Casal2 Contributors Guide}{title}
\pagenumbering{alph} % alpha not used, but used to remove warnings when page 1 is re-defined below

\begin{titlepage}
  \thispagestyle{empty} % no header/footer/page number on this page
	\begin{center}

		\vspace*{2.5cm}
		\Huge \CNAME\ Contributors Guide \\

		\vspace{2.0cm}
		\huge \authors \\ %Document authors

		\vspace{2cm}
		\begin{figure}[htp]
			\begin{center}
			\includegraphics[height=7cm]{Figures/CASAL2.png}
			\end{center}
		\end{figure}

		~\vfill
%		\Large NIWA Technical Report 139 \\%Document Date
%		\Large ISSN 1174-2631 \\%Document Date
%		\Large \DocYear \\%Document Date

		\Large
		\vspace{1.0cm}
		\CNAME\ Contributors Guide for use with \CNAME~(\VER) \\ \SourceRepos
	
	\end{center}
\end{titlepage}

% Citation page
%\cleardoublepage{}
\fancyfoot[C]{\thepage}
\pagenumbering{roman}

~\vfill

\begin{center}
Citation: \ManualRef    
\end{center}

% Table of contents
\clearemptydoublepage{}
\pdfbookmark[1]{Contents}{contents}

\begin{spacing}{0.8} % Reduce space between lines in contents list
\tableofcontents
\end{spacing}

% Table of figures
%\clearemptydoublepage{}
%\pdfbookmark[1]{List of figures}{figures}
%\begin{spacing}{0.8} % Reduce space between lines in contents list
%\renewcommand\listfigurename{List of figures}
%\listoffigures
%\end{spacing}

% Table of tables
%\clearemptydoublepage{}
%\pdfbookmark[1]{List of tables}{tables}
%\begin{spacing}{0.8} % Reduce space between lines in contents list
%\renewcommand\listtablename{List of tables}
%\listoftables
%\end{spacing}

% Document body
\clearemptydoublepage{}
\renewcommand{\headrulewidth}{0.2pt}
\fancyhead[RO]{\slshape \nouppercase \rightmark} % Section headings at top of page (header, odd pages)
\fancyhead[LE]{\slshape \nouppercase \leftmark}  % Section headings at top of page (header, even pages)
\pagenumbering{arabic} % Page numbers a arabic numerals

%\include{equations}

\section{\I{Introduction}\label{sec:Introduction}}

\subsection{\I{About \CNAME}}

\CNAME\ is an open-source integrated statistical catch-at-age or catch-at-length assessment tool for modelling the population dynamics of marine populations. \CNAME\ is designed for quantitative assessments of marine populations, including fish, invertebrates, marine mammals and seabirds.

\CNAME\ implements generalised age (\CNAME\ for age-based models) or length structured (\CNAME\ for length-based models) population models that allows for a great deal of choice in specifying the population dynamics, parameters and which parameters should be estimated, and the model outputs. \CNAME\ is designed for flexibility. It allows implementation of age or length structured models from single species or stocks, to multiple species or stocks, using user-defined categories such as area, sex, and maturity stage. The categories are generic, are not predefined, and are easily specified. \CNAME\ models can be used for a single population with a single anthropogenic event (i.e., a single fish stock with a single fishery), or for multiple species and populations, areas, and/or anthropogenic or exploitation methods, and can include predator-prey interactions.

In \CNAME\ the processes and observations that occur over each year are defined by the user. Processes include recruitment, natural mortality, and anthropogenic mortality. Observations used to fit the models can be from many different sources, including removals-at-size or -age (e.g., a fishery), research survey or other biomass indices, and mark-recapture data. Model parameters can be estimated using penalised maximum likelihood or Bayesian methods.

As well as the point estimates of the parameters, \CNAME\ can calculate the likelihood or posterior distribution profiles for estimated parameters, and can generate Bayesian posterior distributions using Markov chain Monte Carlo methods. \CNAME\ can project the population status into the future using either deterministic or stochastic population dynamics. \CNAME\ can also simulate observations from a given model for both existing and potential observations.

The \CNAME\ user manual has been split into two separate manuals for the age-based functionality and for the length-based functionality. These two manuals contain many common components but differ in processes and observations. \ifAgeBased
This manual is for age-based models. For length-based models, see the \CNAME\ user manual for length-based models. 
\else
This manual is for length-based models. For age-based models, see the \CNAME\ user manual for age-based models. 
\fi % end if


\subsection{\I{Citing \CNAME}}

The reference for this document is: \ManualRef

The peer-reviewed journal article reference for \CNAME\ is \cite{doonan_casal2}.

\CNAME\ has also been simulation tested using simulated data from CASAL to validate results \citep{dunn_integrated_2022}.

\subsection{\I{\CNAME\ Contributors}}

The \CNAME\ project is maintained by the \CNAME\ Development Team. \CNAME\ was initiated by Alistair Dunn. The software architect and lead author of the software code was Scott Rasmussen. Contributors to the development of \CNAME\ are Scott Rasmussen, Alistair Dunn, Ian Doonan, Craig Marsh, Teresa A'mar, Kath Large, Sophie Mormede, Samik Datta, Matt Dunn, Jingjing Zhang, Marco Kienzle, and Arnaud Gr\"{u}ss.

The development of \CNAME\ was funded by \href{http://www.niwa.co.nz}{National Institute of Water \& Atmospheric Research Ltd. (NIWA)}, with additional funding from the New Zealand \href{http://www.mpi.govt.nz}{Ministry for Primary Industries}. More recent developments of this version were funded by Ocean Environmental Ltd.

\subsection{\I{Software license}}

This program and the accompanying materials are made available under the terms of the \href{http://www.opensource.org/licenses/GPL-2.0}{GNU General Public License version 2} which accompanies this software (see Section \ref{sec:License}).

Copyright \copyright 2016-\SourceControlYearDoc, \href{https://www.niwa.co.nz}{\Organisation}. All rights reserved.

\subsection{\I{Where to get \CNAME }}

This version of \CNAME\ is hosted on GitHub, and can be found at \url{https://github.com/alistairdunn1/CASAL2}\index{GitHub}.

There are installation packages available for Linux and Microsoft Windows. Release versions of the package includes the \CNAME\ binary, the \CNAME\ \R\ library, the \CNAME\ User Manual and associated documentation, example models, and other information. The installation packages for older versions of \CNAME\ can be downloaded at \url{https://github.com/NIWAFisheriesModelling/CASAL2/releases}. For more recent versions of the compiled packages, please contact the \CNAME\ Development Team. 

See \url{https://www.niwa.co.nz/} or \url{https://github.com/alistairdunn1/CASAL2} for more information about \CNAME.

\subsection{\I{System requirements}}

\CNAME\ is available for most x86 compatible machines running 64-bit \I{Linux} and \I{Microsoft Windows} operating systems. \CNAME\ has not been compiled nor tested on MacOS.

Several of \CNAME's tasks are computer intensive and a fast processor is recommended. Depending on the model implemented, some of the \CNAME\ tasks can take a considerable amount of processing time (minutes to hours), and in extreme cases may even take several days to complete an MCMC estimate.

Output files have the potential to be large, and the output from developing a model, sensitivity analyses, and running multiple MCMC chains can take up significant amounts of disk space\index{Disk space}. Depending on the number and type of user output requests, the output could range from a few hundred kilobytes to several hundred megabytes. When estimating model fits, several hundred megabytes of RAM may be required, depending on the spatial size of the model, number of categories, and complexity of processes and observations. For larger models, several gigabytes of RAM and disk space may occasionally be required.

\subsection{\I{Necessary files}}

For both 64-bit Linux and Microsoft Windows, we recommend using the zip files available on \github\ for Microsoft Windows and Linux; older Linux systems can use Casal2.tar.gz. Running \CNAME\ on a system requires the main binary ("casal2.exe" on Windows, or "casal2" on Linux) and the associated dynamically linked libraries (DLL) for Windows or shared objects (.so) for Linux, and cannot be (easily) run by copying the binary to a working directory. \CNAME\ is not available for 32-bit operating systems or MacOS.

\CNAME\ does not post-process model output. \CNAME\ writes all output to text files --- either to standard output or directed to files. A package that allows tabulation and graphing of model outputs is recommended. Software such as \href{http://www.r-project.org}{\R}\ \citep{R} is recommended. The \CNAME\ \R\ package is provided for extracting \CNAME\ output from reports and output files into \R\ (see Section \ref{sec:PostProcessing}). A separate \R\ package, r4Casal2, is also available on GitHub and provides some examples of diagnostic and plot functionality.

\subsection{Getting help\index{Getting help}\index{User assistance}\index{Repoerting issues or errors}}

\CNAME\ is distributed as unsupported software. Please notify the \CNAME\ Development Team of any issues with or errors in \CNAME. Please contact the \emaillink. See Section \ref{sec:ReportingErrors} for the guidelines for reporting issues. Note that \CNAME\ is a complex program, with many different options and possibilities, and we may not be able to provide any useful help if you submit an error report that does not follow the guidelines.

\subsection{Technical details\index{Technical specifications}}\label{sec:TechnicalDetails}

The source code\index{\CNAME\ source code} for \CNAME\ is available in the GitHub repository at \github. \CNAME\ is compiled on GitHub using GitHub Actions on host operating systems Microsoft Windows Server 2019 and Ubuntu 20.04.

\CNAME\ was compiled on Linux using \texttt{gcc} (\url{https://gcc.gnu.org}), the C/C++ compiler developed by the GNU Project. The 64-bit Linux \index{Linux} version was compiled using \texttt{gcc} version 10.3.0). 

The Microsoft Windows  (\url{https://www.microsoft.com})\index{Microsoft Windows} version was compiled using TDM-gcc (\url{https://jmeubank.github.io/tdm-gcc/}) using \texttt{gcc} 10.3.0 (\url{http://gcc.gnu.org})\index{gcc}. The Microsoft Windows (\url{https://www.microsoft.com}) installer was previously built using the Inno Setup (\url{https://jrsoftware.org/isdl.php}). \textbf{Note:} for some previous gcc versions, there have been issues related to threading. This was indicated by failed unit tests which rely on threading such MCMCHamiltonian etc.

\CNAME\ includes several minimisers; different minimisers may perform better for some models than others. These include both numerical differences minimisers and automatic differentiation minimisers. Numerical differences minimisers will usually work for most problems, albeit more slowly than auto-differentiation based minimisers. The numerical differences minimisers are:

\begin{enumerate}
\item Numerical differences: A minimiser that is closely based on the main algorithm of \cite{779}, that uses finite difference gradients\index{Finite differences minimiser}.
\item deltadiff: A multithreaded version of the numerical differences, but with $tan$ rescaling instead of $arcsin$ and available for use with the Hamiltonian Monte Carlo MCMC algorithm.
\item DESolver: The differential evolution solver\index{Differential evolution minimiser} \citep{1442}, based on code by \href{mailto:<godwin@pushcorp.com>}{Lester E. Godwin} of \href{http://www.pushcorp.com}{PushCorp, Inc.}
\end{enumerate}

There are two auto-differentiation minimisers, both based on ADOL-C. These are
\begin{enumerate}
\item Betadiff: A minimiser using an older version of ADOL-C (v1.8.4) that was used as the automatic differentiation minimiser in CASAL \citep{1388}, with the same minimising algorithm as used for the finite differences minimiser above, but based on the gradient from the auto-differentiation chain.
\item ADOL-C: A minimiser using version of ADOL-C (v2.5.1) \citep{walther1996adolc};, with a similar minimising algorithm as used for the finite differences minimiser above, but based on the gradient from the auto-differentiation chain.
\end{enumerate}

The random number generator\index{Random number generator} used in \CNAME\ uses an implementation of the Mersenne twister random number generator \citep{796}. This functionality, the command line functionality, matrix operations, and a number of other functions use the \href{http://www.boost.org/}{Boost} C++ library (Version 1.71.0)\index{Boost C++ library}.

Note that the output from \CNAME\ may differ slightly on the different operating systems and operating system versions due to different precision arithmetic or other platform-dependent (including CPU hardware) implementation details. In particular, the implementation of the standard C++ library \texttt{math.h} differs slightly on different platforms, and hence the results from different platforms may be different.

\CNAME\ uses unit tests and post-compile model validations to verify the source code. Unit tests of the underlying \CNAME\ code are run at build time using the \href{https://github.com/google/googletest}{Google Test and Mock} unit testing and mocking framework. The unit test framework aims to cover a significant proportion of the key functionality within the \CNAME\ code base. The unit test code\index{Unit tests} for \CNAME\ is available as a part of the underlying source code. Post-compile models are run using Python scripts at the end of every build and with each commit to the GitHub repository. See Appendix \ref{sec:buildrules} for more information. 




\section{Creating a local repository\label{sec:local_repo}}

This section covers the following:

\begin{enumerate}
	\item Registering a GitHub username
	\item Download git software
	\item Fork the master repository
\end{enumerate}

\subsection{Git username and profile}

The first step is to create a username and profile on GitHub (if you do not already have one). Creating a username and profile on GitHub is free and easy to do. 

Go to \url{https://www.github.com} to register a username and set up a profile if you do not already have one. See the help at GitHub for more information. Once you have set up a GitHub account, you need to download the git software (see next section) so you can push and pull changes between your local repository and the main online repository.

\subsection{Git software}

You will need to acquire a git client in order to clone the repository to your local machine.

\CNAME\ also requires a command line version of git in order to compile. The \CNAME\ build environment requires \texttt{git} in order to evaluate the version of the code used at compile time to include into the executable, manuals, etc. when being built. 

You will need to download the git client from \href{https://git-scm.com/downloads}{git software} and you will need to make sure that it has been included into your system path. This can be checked by opening a terminal (powershell or command prompt and typing in \texttt{git -v}. Git is a command line program, you can also download a GUI interface which helps using git a lot more user-friendly. My personal preference is the \href{https://desktop.github.com/}{github desktop application}. 

\subsection{Cloning a repository}

The publicly available \CNAME\ code is in the master repository. Only the \CNAME\ Development Team have permission to add, delete, or change code directly to the master repository. 

To clone the master repository, navigate to \url{https://github.com/Casal2/CASAL2} and use the green code button, which is shown below to clone. There are multiple protocols (https, ssh etc.) for cloning highlighted in the red box. We recommend using the GitHub desktop method which is circled in Figure~\ref{fig:clone}.

\begin{figure}[H]
	\centering
	\includegraphics[scale=0.6]{Figures/clone_repo.png}
	\caption{Cloning a repository}\label{fig:clone}
\end{figure}


\subsection{Adding an SSH authentication key to your account on GitHub.com}

One of the reasons contributors may have difficulty pushing and pulling changes to \CNAME\ is because they don't have a SSH authentication key linked to their GitHub account. The error message may look like that shown in Figure~\ref{fig:permissiondenied}. Please see the information from \href{https://docs.github.com/en/authentication/connecting-to-github-with-ssh/adding-a-new-ssh-key-to-your-github-account}{this link} for detailed instructions on how to create a public SSH key on your computer and then link it to your GitHub account.

\begin{figure}[H]
	\centering
	\includegraphics[scale=0.6]{Figures/permissiondenied.png}
	\caption{Example of the error message when SSH authentication has not been enabled}\label{fig:permissiondenied}
\end{figure}
 

\include{MaintainRepo}

%\section{Setting up the \CNAME\ BuildSystem\label{sec:build_environment}}

This section describes how to set up the environment on your local machine that will allow you to build and compile \CNAME. The build environment can be on either Microsoft Windows or Linux systems. At present the \CNAME\ build system supports Microsoft Windows 7+ and Linux (with GCC/G++ 4.9.0+). Apple OSX or other platforms are not currently supported. 

Appendix A of the \CNAME\ User Manuals contain the most up-to-date information on how to build \CNAME\, and this would be the best place to look if you are attempting to build \CNAME\ locally.

\subsection{Overview}

The build system is made up of a collection of python scripts that do various tasks. These are located in \path{CASAL2/BuildSystem/buildtools/classes/} directory. Each python script has it's own set of functionality and undertakes a set of actions.

The top level of the build system can be found at \path{CASAL2/BuildSystem/}. In this directory you can run \texttt{doBuild.bat help} from a command terminal in Microsoft Windows systems or \texttt{./doBuild.sh help} from a terminal in Linux systems.

The script will take one or two parameters depending on what style of build you'd like to achieve. These commands allow the building of various stand-alone binaries, shared libraries, and the documentation. Note that you will need additional software installed on your system in order to build \CNAME. These requirements are described later.

A summary of all of the doBuild arguments can be found using the command \texttt{doBuild help} in the BuildSystem directory.

The current arguments to doBuild are:

\begin{itemize}
  \item \texttt{debug}:  Build standalone debug executable
  \item \texttt{release}: Build standalone release executable
  \item \texttt{test}: Build standalone unit tests executable
  \item \texttt{documentation}: Build the user manual
  \item \texttt{thirdparty}: Build all required third party libraries
  \item \texttt{thirdpartylean}: Build minimal third party libraries
  \item \texttt{clean}: Remove any previous debug/release build information
  \item \texttt{cleanall}: Remove all previous build information
  \item \texttt{archive}: Build a zipped archive of the application. The application is built using shared libraries so a single \CNAME executable is created.
  \item \texttt{check}: Do a check of the build system
  \item \texttt{rlibrary}: Build the \CNAME\ \R\ Library
  \item \texttt{modelrunner}: Run the test suite of models
  \item \texttt{installer}: Build an installer package
  \item \texttt{deb}: Create Linux .deb installer
  \item \texttt{library}: Build shared library for use by front end application
  \item \texttt{frontend}: Build single \CNAME executable with all minimisers and unit tests
\end{itemize}

Valid Build Parameters: (thirdparty only)
\begin{itemize}
  \item \texttt{<libary name>}: Target third party library to build or rebuild
\end{itemize}

Valid Build parameters: (debug/release only) e.g. \texttt{doBuild.bat release betadiff}
\begin{itemize}
  \item \texttt{betadiff}: Use BetaDiff auto-differentiation (from CASAL)
  \item \texttt{cppad}: Use CppAD auto-differentiation
  \item \texttt{adolc}: Use ADOLC auto-differentiation in compiled executable
\end{itemize}

Valid Build parameters: (library only e.g. \texttt{doBuild.bat library betadiff})
\begin{itemize}
  \item \texttt{adolc}: Build ADOLC auto-differentiation library
  \item \texttt{betadiff}: Build BetaDiff auto-differentiation library (from CASAL)
  \item \texttt{cppad}: Build CppAD auto-differentiation library
  \item \texttt{test}: Build Unit Tests library
  \item \texttt{release}: Build release library
\end{itemize}

The outputs from the build system commands will be placed in sub-folders of \path{CASAL2/BuildSystem/bin/<operating system>/<build_type>}

For example:

\path{CASAL2/BuildSystem/windows_gcc/debug}

\path{CASAL2/BuildSystem/windows_gcc/library_release}

\path{CASAL2/BuildSystem/windows_gcc/thirdparty/}

\path{CASAL2/BuildSystem/linux/library_release}

\subsection{Building on Microsoft Windows}

\subsubsection{Prerequisite software}

The building of \CNAME\ requires additional build tools and software, including git version control, GCC compiler, LaTeX compiler, and an Windows package builder. \CNAME\ requires specific implementations and versions of these in order to build.

\textbf{C++ and Fortran Compiler}

Source: tdm-gcc (MingW64) from \url{http://www.tdm-gcc.tdragon.net/}.

\CNAME\ is designed to compile under GCC on Microsoft Windows and Linux  platforms. While it may be possible to build the package using different compilers, the \CNAME\ Development Team does provide any assistance or recommendations. We recommend using 64-bit TDM-GCC version 5.1.0. Ensure you have the \enquote{fortran} and \enquote{openmp} options installed as a part of the \enquote{gcc} dropdown tickboxes otherwise \CNAME\ will not compile.. \textbf{Note}: A common error that can be made is having a different GCC compiler in your path when attempting to compile. For example, rtools includes a version of the GCC compiler. We recommend removing these from your path prior to compiling \CNAME.

\textbf{GIT Version Control}

Source: Command line GIT from \url{https://www.git-scm.com/downloads}.

\CNAME\ automatically adds a version number based on the GIT version of the latest commit to its repository. The command line version of GIT is used  to generate a version number for the compiled binaries, R libraries, and the manuals.

\textbf{MiKTeX Latex Processor}

Source: Portable version from \url{http://www.miktex.org/portable}.

The main user documentation for \CNAME\ is a PDF manual generated from LaTeX. The LaTeX syntax sections of the documentation are generated, in part, directly from the code. In order to regenerate the user documentation, you will need the MiKTeX LaTeX compiler.

\textbf{7-Zip}

Source: 7-Zip from \url{http://www.7-zip.org/download.html}.

The BuildSystem calls 7zip.exe to unzip files in the build system; it is advised to have this in the path.

\textbf{Inno Setup Installer Builder (optional)}

Source: Inno Setup 5 from \url{http://www.jrsoftware.org/isdl.php}

If you wish to build a Microsoft Windows compatible Installer for \CNAME\ then you will need the Inno Setup 5 application installed on the machine. The installation path must be \path{C:\Program Files (x86)\Inno Setup 5\} in order for the build scripts to fins and use it.

\subsubsection{Pre-build requirements}

Prior to building \CNAME\ you will need to ensure you have both G++ and GIT in your path. You can check both of these by typing:

\texttt{g++ --version}

\texttt{git --version}

This also allows you to check that there are no alternative versions of a GCC compiler that may confuse the \CNAME\ build.

It’s worth checking to ensure GFortran has been installed with the G++ compiler by typing:

\texttt{gfortran --version}

If you wish to build the documentation bibtex will also need to be in the path:

\texttt{bibtex -version}

\subsubsection{Building \CNAME}

The build process is relatively straightforward. You can run \texttt{doBuild check} to see if your build environment is ready.

\begin{enumerate}
  \item Get a copy (clone) of the forked code on your local machine, mentioned in Section~\ref{sec:local_repo}:
  \item Navigate to the BuildSystem folder in \path{CASAL2/BuildSystem}
  \item You need to build the third party libraries with:
  \begin{itemize}
    \item \texttt{doBuild thirdparty}
  \end{itemize}
  \item You need to build the binary you want to use:
  \begin{itemize}
    \item \texttt{doBuild release}
  \end{itemize}
  \item You can build the documentation if you want:
  \begin{itemize}
    \item \texttt{doBuild documentation}
  \end{itemize}
\end{enumerate}

\subsection{Building on Linux}

This guide has been written against a fresh install of Ubuntu 15.10. With Ubuntu we use apt-get to install new packages. You’ll need to be familiar with the package manager for your distribution to correctly install the required prerequisite software.

\subsubsection{Prerequisite Software}

\textbf{Compiler G++}

Ubuntu 15.10 comes with G++ 15.10, gfortran is not installed though so we can install it with: \texttt{sudo apt-get install gfortran}.

\textbf{GIT Version Control}

Git isn't installed by default but we can install it with \texttt{sudo apt-get install git}

\CNAME\ automatically adds a version number based on the GIT version of the latest commit to its repository. The command line version of GIT is used  to generate a version number for the compiled binaries, R libraries, and the manuals.

\textbf{CMake}

CMake is required to build multiple third-party libraries and the main code base. You can do this with \texttt{sudo apt-get install cmake}

\textbf{Python2 Modules}

There are a couple of Python2 modules that are required to build \CNAME. These can be installed with \texttt{sudo apt-get install python-dateutil}

You may also need to install \textbf{datetime}, re and \textbf{distutils}. \textbf{Texlive} Latex Processor. No supported latex processors are installed with Ubuntu by default. You can install a suitable latex process with:

\texttt{sudo apt-get install texlive-binaries}
\texttt{sudo apt-get install texlive-latex-base}
\texttt{sudo apt-get install texlive-latex-recommended}
\texttt{sudo apt-get install texlive-latex-extra}

Alternatively you can install the complete package:
\texttt{sudo apt-get install texlive-full}

\subsubsection{Building \CNAME}

The build process is relatively straightforward. You can run \texttt{./doBuild.sh check} to see if your build environment is ready.

\begin{enumerate}
	\item Get a copy (clone) of the forked code on your local machine, mentioned in Section~\ref{sec:local_repo}:
	\item Navigate to the BuildSystem folder in \path{CASAL2/BuildSystem}
	\item You need to build the third party libraries with:
	\begin{itemize}
	    \item \texttt{./doBuild.sh thirdparty} for Linux or \texttt{doBuild.bat thirdparty} for windows
	\end{itemize}
	\item Build all libraries:
	\begin{itemize}
		\item \texttt{./doBuild.sh archive} for Linux or \texttt{doBuild.bat archive} for windows
	\end{itemize}
	\item You can build the documentation if you want:
	\begin{itemize}
		\item \texttt{./doBuild.sh documentation}
	\end{itemize}
\end{enumerate}

\subsection{Troubleshooting}

\subsubsection{Third-party libraries}

It's possible that there will be build errors or issues building the third-party libraries. If you encounter an error, then it’s worth checking the log files. Each third-party build system stores a log of everything it’s doing. The files will be named

\begin{itemize}
	\item casal2\_unzip.log
	\item casal2\_configure.log
	\item casal2\_make.log
	\item casal2\_build.log
	\item \dots etc,.
\end{itemize}

Some of the third-party libraries require very specialised environments for compiling under GCC on Windows. These libraries are packaged with MSYS (MinGW Linux style shell system). The log files for these will be found in \path{ThirdParty/<library name>/msys/1.0/<library name>/}

e.g.,: \path{ThirdParty/adolc/msys/1.0/adolc/ADOL-C-2.5.2/casal2_make.log}\\
e.g: \path{ThirdParty/boost/boost_1_58_0/casal2_build.log}

\subsubsection{Main code base}

If the unmodified code base does not compile, the most likely cause is an issue with the third-party libraries not being built. Ensure they have been built correctly. As they are outside the control of the Development Team, problems can arise that may require the developers of the third party libraries to resolve first. Contact the \CNAME\ development lead developer at \texttt{scott@zaita.com} for help.


\section{\I{Compiling \CNAME}}\label{sec:build_environment}

This section describes how to set up the environment on your local machine that will allow you to build and compile \CNAME. The build environment can be on either Microsoft Windows or Linux systems. At present the \CNAME\ build system supports Microsoft Windows 10+ and Linux (with GCC/G++ 4.9.0+). Apple OSX or other platforms are not currently supported.

\subsection{Overview}

The build system is made up of a collection of Python scripts that do the various tasks. These are located in \path{CASAL2/BuildSystem/buildtools/classes/}. Each Python script has its own set of functionality and undertakes one set of actions.

The top level of the build system can be found at \path{CASAL2/BuildSystem/}. In this directory you can run \texttt{doBuild.bat help} from a command terminal in Microsoft Windows systems or \texttt{./doBuild.sh help} from a terminal in Linux systems.

The script will take one or two parameters depending on what style of build you want to undertake. These commands allow the building of various stand-alone binaries, shared libraries, and the documentation. Note that you will need additional software installed on your system in order to build \CNAME.  The software requirements are described below.

A summary of all of the doBuild arguments can be found using the command \texttt{doBuild help} in the BuildSystem directory.

The arguments to doBuild are:

Usage: doBuild $<$build\_target$>$ $<$argument$>$
\begin{description}
  \item{\texttt{help}} Print out the doBuild help (this output)
  \item{\texttt{check}} Do a check of the build system
  \item{\texttt{clean}} Remove debug/release build files
  \item{\texttt{clean\_all}} Remove all build files and ALL prebuilt binaries
  \item{\texttt{version}} Build the current version files for C++, R, and LaTeX
\end{description}

Build required libraries (DLLs/shared objects for Casal2)
\begin{description}
  \item{\texttt{thirdparty}} Build the third party libraries
  \begin{description}
    \item{$<$option$>$} Optionally specify the target third party library to build, either adolc or betadiff (default is none)
  \end{description}
\end{description}

Build development and test versions (for development builds only)
\begin{description}
\item{\texttt{release}} Build stand-alone release executable
  \begin{description}
	\item{$<$option$>$} Optionally specify the target third party library to build, either adolc or betadiff (default is none)
  \end{description}
  \item{\texttt{debug}} Build stand-alone debug executable
  \begin{description}
	\item{$<$option$>$} Optionally specify the target third party library to build, either adolc or betadiff (default is none)
  \end{description}
  \item{\texttt{test}} Build stand-alone unit tests executable
  \item{\texttt{unittests}} Run the unit tests (requires that `test' has been built)
  \item{\texttt{modelrunner}} Run the test case models
\end{description}

Build the Casal2 end-user application
\begin{description}
  \item{\texttt{library}} Build shared library for use by front end application
  \begin{description}
    \item{$<$argument$>$} Required argument to specify the target library to build: release, adolc, betadiff, or test
  \end{description}
  \item{\texttt{frontend}} Build \CNAME\ front end application
\end{description}

Create the archive, R Library, documentation, and the installers
\begin{description}
  \item{\texttt{documentation}} Build the \CNAME\ user manuals
  \item{\texttt{rlibrary}} Create the R library
  \item{\texttt{archive}} Create a zipped archive of the \CNAME\ application.
  \begin{description}
     \item{$<$true$>$} if specified build skips everything but frontend
  \end{description}
  \item{\texttt{installer}} Create the Microsoft Windows installer package
  \item{\texttt{deb}} Create Linux Debian installer
\end{description}

The outputs from the build system commands will be placed in sub-folders of \path{CASAL2/BuildSystem/bin/<operating system>/<build_type>}

For example:

\path{CASAL2/BuildSystem/windows/debug}

\path{CASAL2/BuildSystem/windows/library_release}

\path{CASAL2/BuildSystem/windows/thirdparty/}

\path{CASAL2/BuildSystem/linux/library_release}

 The files \texttt{Casal2\_build.bat} for Windows and \texttt{Casal2\_build.sh} for Linux in the root folder contain all the calls in the correct order of \texttt{doBuild} required to successfully build \CNAME, the documentation, the Windows installer (Windows) or the Debian installer (Linux), the R-Libraries, and run all the test cases and unit tests.

\subsection{Building on Microsoft Windows}

\subsubsection{Prerequisite software}

The building of \CNAME\ requires additional build tools and software, including Python, git version control, GCC compiler, LaTeX compiler, and a Windows package builder. \CNAME\ can require specific implementations of these packages and versions in order to build without modifying the build scripts.

\paragraph*{C++ and Fortran compiler}

Source: tdm-gcc (MinGW-w64) from \url{https://jmeubank.github.io/tdm-gcc/}.

\CNAME\ is designed to compile under GCC on Microsoft Windows and Linux. While it may be possible to build the package using different compilers, the \CNAME\ Development Team does provide any assistance or recommendations. We recommend using 64-bit TDM-GCC with a version of at least 10.3.0. Ensure you have the "fortran" and "openmp" options installed as a part of the "gcc" install option drop-down tick boxes as these are required. For example, from  \url{https://jmeubank.github.io/tdm-gcc/articles/2021-05/10.3.0-release}, select the 64+32-bit MinGW-w64 edition, then select the Custom install and tick all boxes. 

Note that a common error that can be made is having a different GCC compiler in your path when attempting to compile. For example, \texttt{rtools} includes a version of the GCC compiler. We recommend removing these from your path prior to compiling.

\paragraph*{GIT version control}

Source: Command line GIT from \url{https://www.git-scm.com/downloads}.

\CNAME\ automatically adds a version details to its files and any output based on the GIT version of the latest commit to its repository. This includes the name of source repository that was used. The command line version of GIT is used  to generate the version details.

\paragraph*{MiKTeX LaTeX Processor}

Source: Portable version from \url{http://www.miktex.org/portable}.

The main user documentation for \CNAME\ is a PDF document generated from LaTeX. The LaTeX syntax sections of the documentation are generated, in part, directly from the code. In order to generate the user documentation, you will need the MiKTeX LaTeX compiler.

A number of additional LaTeX styles are used --- these will usually be identified doing the doBuild process and can be installed as required. 

\paragraph*{7-Zip}

Source: 7-Zip from \url{http://www.7-zip.org/download.html}.

The build system calls \texttt{7zip.exe} to unzip files in the build system; it is advised to have this in the path.

\paragraph*{Python}

Source: Python3 from \url{https://www.python.org/downloads/windows/}

Python is used to run the build scripts and set the required environment variables required to build \CNAME. 

\paragraph*{Python modules}

There are a number of Python3 modules that are required to build \CNAME. These can be installed with \texttt{python -m pip install \emph{module-name}}. For example, You may need to install \texttt{datetime}, \texttt{re}, and \texttt{distutils} Python modules. 

\paragraph*{Inno setup installer build (optional)}

Source: Inno Setup 5 from \url{http://www.jrsoftware.org/isdl.php}

If you wish to build a Microsoft Windows compatible Installer for \CNAME\ (recommended) then you will need the  `Inno Setup 5' application installed on the machine. The installation path must be \path{C:\Program Files (x86)\Inno Setup 5\} in order for the build scripts to find and use it.

\subsubsection{Pre-build requirements}

Prior to building \CNAME\ you will need to ensure you have both G++ and GIT in your path. You can check both of these by typing the following commands and checking that they return the correct version number:

\texttt{g++ -{}-version}

\texttt{git -{}-version}

This also allows you to check that there are no alternative versions of a GCC compiler that may confuse the \CNAME\ build. It’s also worth checking to ensure GFortran has been installed with the G++ compiler by typing:

\texttt{gfortran -{}-version}

If you wish to build the documentation, \texttt{bibtex} will also need to be in the path, e.g., to check, try:

\texttt{bibtex -{}-version}

\subsubsection{Building \CNAME}

The build process is relatively straightforward. Before you start the build process, you can run \texttt{doBuild check} from the command prompt to check if your build environment is complete. Make sure that you are within \path{CASAL2/BuildSystem/} to run \texttt{doBuild}. 

\texttt{doBuild check} will summarise Windows environment PATH as a part of its output, and this can be used to check that the paths for g++ and gfortran and the g++ point to where the correct version of GCC is installed. 

The build process is as follows: 
\begin{enumerate}
  \item Download a clone of the code on your local machine
  \item Navigate to the BuildSystem folder in \path{CASAL2/BuildSystem}
  \item You need to build the third party libraries with the following commands from the command prompt:
  \begin{itemize}
    \item \texttt{doBuild thirdparty}
  \end{itemize}
  \item You need to build the binary you want to use:
  \begin{itemize}
    \item \texttt{doBuild release}
  \end{itemize}
  \item You can build the documentation if you want:
  \begin{itemize}
    \item \texttt{doBuild documentation}
  \end{itemize}
\end{enumerate}

\subsection{Building on Linux}

This guide has been written against a fresh install of Ubuntu 20.04. With Ubuntu we use apt-get to install new packages. You’ll need to be familiar with the package manager for your distribution to correctly install the required prerequisite software. For this you will require administrator level access.

\subsubsection{Prerequisite software}

\paragraph*{G++ compiler}

If gfortran is not installed, install this with: \texttt{sudo apt-get install gfortran}.

\paragraph*{GIT version control}

Git may not be installed by default and it can be installed with \texttt{sudo apt-get install git}

\CNAME\ automatically adds a version details to its files and any output based on the GIT version of the latest commit to its repository. This includes the name of source repository that was used. The command line version of GIT is used  to generate the version details.

\paragraph*{CMake}

CMake is required to build multiple third-party libraries and the main code base. You can do this with \texttt{sudo apt-get install cmake}

\paragraph*{Python}

Python3 is used to run the build scripts and set the required environment variables required to build \CNAME. This is usually installed by default on Linux systems, but if not, it can be installed using: \texttt{sudo apt-get install python3}

\paragraph*{Python modules}

There are a number of Python3 modules that are required to build \CNAME. These can be installed with \texttt{sudo apt-get install \texttt{module-name}}. For example, You may need to install \texttt{datetime}, \texttt{re}, and \texttt{distutils} Python modules. 

\paragraph*{LaTeX}

LaTeX on Linux is required, and the Texlive LaTeX Processor is recommended. This can be installed with:

\texttt{sudo apt-get install texlive-binaries}
\texttt{sudo apt-get install texlive-latex-base}
\texttt{sudo apt-get install texlive-latex-recommended}
\texttt{sudo apt-get install texlive-latex-extra}

Alternatively you can install the complete package with 
\texttt{sudo apt-get install texlive-full}

A number of additional LaTeX styles are used --- these will usually be identified doing the doBuild process and can be installed as required. 

\subsubsection{Building \CNAME}

The build process is relatively straightforward. You can run \texttt{./doBuild.sh check} to see if your build environment is ready.

\begin{enumerate}
	\item Download a clone of the code on your local machine
	\item Navigate to the BuildSystem folder in \path{CASAL2/BuildSystem}
	\item You need to build the third party libraries with:
	\begin{itemize}
	    \item \texttt{./doBuild.sh thirdparty}
	\end{itemize}
	\item You need to build the binary you want to use:
	\begin{itemize}
		\item \texttt{./doBuild.sh release}
	\end{itemize}
	\item You can build the documentation:
	\begin{itemize}
		\item \texttt{./doBuild.sh documentation}
	\end{itemize}
\end{enumerate}

\subsection{Troubleshooting}

\subsubsection{Third-party C++ libraries}

It's possible that there will be build errors or issues building the C++ third-party libraries. If you encounter an error, then check the log files to locate the source of the problem. Each third-party build system stores a log of everything that was done. The files will be named

\begin{itemize}
	\item casal2\_unzip.log
	\item casal2\_configure.log
	\item casal2\_make.log
	\item casal2\_build.log
	\item \dots etc,.
\end{itemize}

Some of the third-party libraries require very specialised environments for compiling under GCC on Windows. These libraries are packaged with MSYS (MinGW Linux style shell system). The log files for these will be found in \path{ThirdParty/<library name>/msys/1.0/<library name>/}

e.g., \path{ThirdParty/adolc/msys/1.0/adolc/ADOL-C-2.5.2/casal2_make.log}\\
e.g., \path{ThirdParty/boost/boost_1_58_0/casal2_build.log}

A common issue when running doBuild thirdparty are Python error messages about missing modules, e.g., ModuleNotFoundError: No module named 'dateutil'. This type of error message indicates that a Python module (library) is missing and will need to be installed. For instance, to install the 'dateutil' module, type the following into a command prompt or terminal window: pip3 install python-dateutil.  

\subsubsection{Main code base}

If the unmodified code base does not compile, the most likely cause is an issue with the third-party libraries not being built correctly. As updates and revisions are outside the control of the Development Team, problems can arise that may require the developers of the third party libraries to resolve first. However, versions of these libraries are included in the \CNAME\ source code and these should work. For any specific issues contact a local expert with regard to your specific system environment, or else the \CNAME\ Development Team for help.

\clearpage{}

\section{\I{\CNAME\ build guidelines and validation}\label{sec:buildrules}}

\subsection{\CNAME\ coding practice and style}\label{subsec:codepractice}

\CNAME\ is written in C++ and uses the Google C++ style guide (see \url{https://google.github.io/styleguide/cppguide.html}). 

In general when editing or writing code for \CNAME\:

\begin{enumerate}
  \item Using consistent indentations inside functions and loops, and descriptive and human readable variable or function names.
  \item Use of the characters `\_' on the end of class variables defined in the .h files. 
  \item Annotate and comment the code, especially where it would help another contributor understand the program logic and rationale.
  \item Add descriptive log messages to aid in debugging and checking of the program logic flow.
  \item Implement unit tests, internal models, and external models to test and validate the new or changed functionality.
  \item Document the functionality in the \CNAME\ User Manual(s).
\end{enumerate}

\CNAME\ allows printing of logging messages art runtime using the --loglevel command line argument. The levels of logging in \CNAME\ are:

\begin{itemize}
	\item LOG\_MEDIUM()  usually reserved for iterative functionality (e.g. estimates during estimation phase)
	\item LOG\_FINE() the level of reporting between an actual report and a fine scale detail that end users are not interested in (Developers)
	\item LOG\_FINEST() Minor fine scale details within a function or routine.
	\item LOG\_TRACE() put at the beginning of every function
\end{itemize}

e.g., to run \CNAME\ with logging, use 

\texttt{casal2 -r --loglevel finest > my\_run.log 2> my\_run.err}

This will output all the logged information to \texttt{my\_run.err}.

\subsection{Units tests and model validation}

The \CNAME\ development places an emphasis on maintaining software integrity and reproducibility between revisions. \CNAME\ uses model validations and built in unit tests to validate and verify the code each time \CNAME\ is compiled and built. 

There are three different validation approaches in \CNAME. These are:

\begin{enumerate}
	\item Mocking specific classes.
	\item Implementing internal models (implemented in C++ source code with extension \texttt{.Test.cpp}) that have variable test cases for specific classes. 
	\item Implementing externally run models (found in the TestModel folder) that are validated to generate expected output.
\end{enumerate}

To implement mocking of classes and internal models, \CNAME\ uses the Google testing framework and the Google mocking framework.

To implement testing of full models, input configuration files are run using the compiled \CNAME\ binaries, and the output compared with expected output using @assert commands.

\subsubsection{Mocking specific classes}

Classes are unit tested using unit tests that are a part of the source code. These are designed to check the components of the code to validate that functions provide expected output. These unit tests are run each time \CNAME\ is compiled.

When adding unit tests, they should to be developed and tested outside of \CNAME\. This gives confidence that the test does not contain any calculation errors. 

Mocking specific classes is used to validate specific functionality and is encouraged because it is the easiest to isolate simple errors that may be introduced with code changes. 

As examples, see (i) the file \texttt{VonBertalanffy.Test.cpp} mocks the von Bertalanffy age-length class and tests the mean length calculation, and (ii) the \texttt{Partition} class has the file \texttt{Partition.Test.cpp} that validates user inputs and model expectations.

\subsubsection{Internal mocking of simple models}

Mocking of simple models is done using a number of internal models. Most of the functionality for implementing these are in the source folder \texttt{/CASAL2/source/TestResources}. 

These implements simple models and run test cases with differing class implementations by running an internal empty model and testing the output of classes from a model run. 

As examples, see (i) the \texttt{LogNormal.Test.cpp} in the \texttt{Projects} class that test the lognormal distribution when used for projections, and (ii) the \texttt{TagByLength} process in \texttt{TagByLength.Test.cpp} that tests functionality of the tagging process.

\subsubsection{External testing using test models}

External tests are run following compilation using the Python modelrunner.py scripts (i.e., using the \texttt{DoBuild modelrunner} script in the BuildSystem folder). These models are used to test model runs, minimisation routines, and MCMC output.

The test model input configuration files are located in the \texttt{TestModel} folder and the command calls to run these are in the modelrunner.py script.  

Contributors are encouraged to add additional models to the list of test models as these be used to validate the combined functionality of a range of interrelated commands and subcommands in \CNAME. 

\subsection{Verification}
After \CNAME\ executes Validate and Build it runs sanity checks in the verify state. These are business rules that can be checked across the entire system. This can be useful to suggest dependencies or configurations. For an example see the directory \subcommand{Processes\textbackslash Verification\textbackslash} in the source code. 

\subsection{Reporting (optional)}

Currently \CNAME\ has reports that are \R\ compatible, i.e., all output reports produced by \CNAME\ can be read into \R\ using the standard  \textbf{CASAL2} \R\ package. If you create a new report or modify an old one, you most follow the standard so that the report is \R\ compatible.

All reports must start with,
\texttt{*label (type)}
and end with,
\texttt{*end}

Depending on what type of information you wish to report, will depend on the syntax you need to use. For example

\paragraph*{$\{$d$\}$ (Dataframe)}
Report a dataframe
{\small{\begin{verbatim}
			*estimates (estimate_value)
			values {d}
			process[Recruitment_BOP].r0 process[Recruitment_ENLD].r0 
			2e+006 8e+006
			*end
\end{verbatim}}}

\paragraph*{$\{$m$\}$ (Matrix)}
Report a matrix
{\small{\begin{verbatim}
			*covar (covariance_matrix)
			Covariance_Matrix {m}
			2.29729e+010 -742.276 -70160.5
			-110126 -424507 -81300 
			-36283.4 955920 -52736.2 
			*end
\end{verbatim}}}

\paragraph*{$\{$L$\}$ (List)}
Report a List
{\small{\begin{verbatim}
			*weight_one (partition_mean_weight)
			year: 1900
			ENLD.EN.notag {L}
			mean_weights {L}
			0.0476604 0.111575 0.199705
			end {L}
			age_lengths {L}
			12.0314 16.2808 20.0135
			end {L}
			end {L}
			*end
\end{verbatim}}}

\subsection{Update the \CNAME\ User Manual(s)}

Contributors will need to add or modify sections of the user manual(s) to document their changes. This includes the section that describes the methods and the section where the specific syntax is defined. 

\subsection{Builds to pass before merging changes}

Once you have made changes, you must run the following before your changes can be included in the master code. 

\begin{itemize}
	\item Build the unittest version. See Section~\ref{sec:build_environment} for how to build unittest depending on your system.
	\item Run the standard and new unit tests to check that they all pass. To do this first compile the test executable using the script \texttt{DoBuild test}. Then move to the directory with the location of the executable (\texttt{BuildSystem/bin/OS/test}) and run it (open a command terminal and run \texttt{casal2}) to check all the unit-tests pass.
	\item Test that the debug and release of \CNAME\ compiles and runs with \texttt{DoBuild debug}
	\item Run the second phase of unit tests (requires that the debug version is built). This runs the tests that comprise of complete model runs using \texttt{DoBuild modelrunner}
	\item Build the archive using \texttt{DoBuild archive} which builds all required libraries. There are small nuances between Double classes, especially when reporting the class that mean seemingly simple changes can sometimes cause a break in the full build.
\end{itemize}




\include{Example}

\include{AddingReports}

\include{PullRequest}

\include{ModeratingPullRequest}

\section{\I{Acknowledgements}\label{sec:acknowledgements}}

We thank the developers of CASAL \citep{1388} for their ideas that led to the development of \CNAME. The \CNAME\ logo was designed by Ian Doonan and Erika Mackay \href{http://www.niwa.co.nz}{(NIWA)}.

Much of the structure of \CNAME, the methods and equations, and documentation draw heavily on similar components of the fisheries population modelling application CASAL \citep{1388} and  the spatial population model SPM \citep{SPM}. We thank the authors of CASAL and SPM for their permission to use their work as the basis for parts of \CNAME\ and allow the use of the definitions, concepts, and documentation.

The development of \CNAME\ was funded by the New Zealand \href{http://www.mpi.govt.nz}{Ministry for Primary Industries} and the \href{http://www.niwa.co.nz}{National Institute of Water \& Atmospheric Research Ltd. (NIWA)}. More recent developments of this version were funded by Ocean Environmental Ltd.


\clearpage{}
% Referencing
\bibliographystyle{plainnat}
\renewcommand{\bibsection}{%
	\section{References}}
\setcitestyle{round,aysep={},yysep={;}%
}
\bibliography{../UserManual/CASAL2}

\end{document}
%End
