\section{\I{Partition \& Categories}\label{sec:PartitionCategories}}

Dividing the population into categories is fundamental to modelling the dynamics of a fish stock. The grouping of the population into categories and either ages or lengths is called the partition. 

In \CNAME\, the concept of user defined categories allows for flexibility in grouping of categories or parts of the modelled population. Note that \CNAME\ does not know about sex or area and their properties; these are explicitly defined by the user by specifying processes that act on the categories. The cost is that users need to follow good practice to achieve clarity and readability of the input files, i.e., poor specifications can result in input files that are more difficult to understand.

CASAL had a fixed set of hard-wired categories (e.g., factors like sex, maturity, area, or stock) and each category type had a predefined set of allowed processes (or transitions in CASAL-speak), e.g., immature fish moving into the mature category \citep{1388}. This made sense when CASAL was coded, but now it is seen as a limitation, e.g., changing sex was not allowed and only male and female sexes were allowed, not an unknown sex that sometimes occurs in data. 

\subsection{Specifying the partition using categories}

A key element of the \CNAME\ model is the partition which holds the current state of the population. The partition can be conceptualised as a matrix, where each row represents a category and the columns are the \ifAgeBased age \else length \fi \ classes (Figure~\ref{Fig:part}). Each row represents all individuals in that category as a \ifAgeBased numbers-at-age \else numbers-at-length \fi vector.  There must be at least one category defined for each model.

\begin{figure}[H]
	\centering
	\ifAgeBased
	\includegraphics[scale=0.4]{Figures/partition2-age.png}
	\caption{A visual representation of the partition for a simple age-based model.}
	\else
	\includegraphics[scale=0.4]{Figures/partition2-length.png}
	\caption{A visual representation of the partition for a simple length-based model.}
	\fi
    \label{Fig:part}
\end{figure}

The categories can include combinations of levels from one or more factors such as sex, maturity state, area, stock, or even species. \CNAME\ has no predefined categories; \emph{all} categories are defined by the user. Note that the partition only has the current state of the model; past states are not kept (\textit{See} the section on derived quantities about saving summary values from the partition, p. \pageref{sec:DerivedQuantity}).

To illustrate categories, consider a model of a fish population with two fisheries, one on spawning fish at the spawning grounds and another on the non-spawning population in the rest of the stock area. The mature fish will migrate to the spawning area, where the spawning fishery occurs. At the end of spawning, these fish, along with the recruits from the previous year, migrate back to the non-spawning area. The fish population can be represented  by factors sex (levels \textit{male} and \textit{female}), maturity (levels \textit{immature} and \textit{mature}), and area (levels \textit{spawning} and \textit{non-spawning}). So the partition has 8 rows of \ifAgeBased numbers-at-age\else numbers-at-length\fi, for 2 sexes $\times$ 2 maturity levels $\times$ 2 areas.

These categories are specified in a categories block which starts with a \textit{@categories} line followed on the next line by a \textit{format} subcommand that specifies the factors to use and their order. Factor names are user defined and have no intrinsic meaning in \CNAME.

The command block for this example is:

{\small{\begin{lstlisting}
	@categories
	format area.sex.mature
	names spawn.male.immature spawn.male.mature spawn.female.immature spawn.female.mature nonspawn.male.immature nonspawn.male.mature nonspawn.female.immature #all on one line nonspawn.female.mature
\end{lstlisting}}}  %{verbatimIJD}}}

Note the "." syntax to separate the factor names.

Next comes the \textit{names} subcommand which specifies the combinations of levels that makes up each category. In a sense, the \textit{format} subcommand is not needed since the \textit{names} subcommand can define all categories. However, \textit{format} allows a more  digestible and shorter syntax to define categories here and in other blocks such as matching observation to categories that provided the data (including combinations of categories, e.g., \ifAgeBased age \else length \fi compositions that combine both sexes).

The \textit{names} subcommand can also be specified with:

{\small{\begin{verbatim}names spawn,nonspawn.male,female.immature,mature
\end{verbatim}}}

which defines the categories above in a more efficient manner, (again, note the ``.'' to separate the factors and ``,'' to separate the levels within each factor (see the next section for more details). A visualisation of the partition is in Figure \ref{Fig:part}.

When using this short-cut syntax in \textit{names}, the order of level combinations is for the levels of the first factor to change the slowest, then the next factor will change faster, and so on with the last factor to changing levels the fastest. The order is important because linking categories to their characteristics, e.g., growth curve or selectivity, is done in other subcommands where these must be specified in the same order.

To exclude unused categories from the partition, the long form must be used in the \textit{names} subcommand, e.g., to exclude  \textit{spawn.female.immature} and \textit{spawn.male.immature} since they are never in the spawning area.

To make recruitment to enter the partition in the non-spawning area, use

{\small{\begin{lstlisting}
	@categories
	format area.sex.mature
	names spawn.male.mature spawn.female.mature nonspawn.male.immature nonspawn.male.mature nonspawn.female.immature nonspawn.female.mature
\end{lstlisting}}}

\subsection{Shorthand syntax for categories}\label{sec:ShorthandSyntax}

Some specifications have long lists of categories or years or initial values for parameters and the like, e.g., for YCS from 1900 to 2019, 120 years and 120 initial vales of YCS must be specified; this is hard to do by hand and it can be error prone as well as difficult to match values for each year. Here, the range short cut (\textit{:)} can be used so the  the year specification is \textit{1900:2019}, and the multiplier short cut (\textit{*)} to give the initial values specification as \textit{1*120}.

There is also shorthand notation for categories since each category can be quite complicated\label{sec:Categories}. First use the \texttt{format} subcommand in the \textit{@categories} block to define the factors that make up the sections of the category names. A ``.'' (period) character delineates each factor and this structure allows a shorthand syntax to compose category names.

The \texttt{names} subcommand is used to list the category names. Sections within the shorthand syntax for \textit{names} are required to match the order of factors in the \textit{format} subcommand so \CNAME\ can organise and search on them. In these sections, levels for each factor use the "list specifier" and range characters, e.g.,

{\small{\begin{lstlisting}
@categories
format sex.stage.tag   # 2 sexes, 2 stages, tag years 2001 to 2005 = 20 categories

names male.immature               # Invalid: No tag information
names female                      # Invalid: no stage of tag information
names female.immature.notag.1    # Invalid: Additional format segment not defined

names male,female.immature,mature.notag,2001:2005 # Valid shortcut

# Without the shorthand syntax these categories would be written:

names male.immature.notag male.immature.2001 male.immature.2002 male.immature.2003 male.immature.2004 male.immature.2005 male.mature.notag male.mature.2001 male.mature.2002 male.mature.2003 male.mature.2004 male.mature.2005 female.immature.notag female.immature.2001 female.immature.2002 female.immature.2003 female.immature.2004 female.immature.2005 female.mature.notag female.mature.2001 female.mature.2002 female.mature.2003 female.mature.2004 female.mature.2005
\end{lstlisting}}}

The shorthand syntax available are: \TODO{edit list}

\begin{itemize}
	\item \textit{*} Specify all categories
	\item \textit{+} Categories join, e.g., \textit{categories *+} joins all categories together into one unit; \textit{categories male+female}
specifies that the observation covers both sexes combined.
    \item \textit{:} Specify a range of integers $[$int1$]$:$[$int2$]$, e.g., \textit{2000:2005} expands to \textit{2000 2001 2002 2003 2004 2005}
    \item Lists using "," $[$item1$]$,$[$item2$]$,$[$item3$]$, e.g., \textit{male,female,unsexed} are the levels for the factor \textit{sex}.
    \item Repeats a number or label: $[$number $\mid$ label$]$ * $[$integer$]$, e.g., \textit{1 * 5} $\rightarrow$ \textit{1 1 1 1 1}
    \item \textit{format=$[$X$]$=$[$x$]$=$[$int$]$}   \TODO{this may need additional explanation as to how it works}
    \textit{$[$factor$]$=$[$level$]$=$[$year range$]$}, e.g., \textit{tag=2001=1999:2003} the categories with level 2001 in the tag factor are accessible from year 1999 to 2003 inclusive.
    \item \textit{[]} replace label to a command block with the block defined inline, e.g., \textit{catchability [q = 1e-5]} rather than \textit{catchability CHATq} where \textit{CHATq} labels a command block somewhere in the input files
\end{itemize}


Example of specifying categories using the short cuts:

This syntax is the long way:
{\small{\begin{verbatim}
		@categories
		format sex.stage
		names male.immature male.mature female.immature female.mature
\end{verbatim}}}

A shorter way to specify the exact same partition structure  using \textit{lists}:

{\small{\begin{verbatim}
		@categories
		format sex.stage
		names male,female.immature,mature
\end{verbatim}}}

\CNAME\ requires categories in processes and observations so that the correct model dynamics can be applied to the correct categories of the partition.

\ifAgeBased
This block illustrates using categories required for the ageing process:

{\small{\begin{verbatim}
		# 1. The long-hand way
		@ageing my_ageing
		categories male.immature male.mature female.immature female.mature

		# 2. The first shorthand way
		@ageing my_ageing
		categories male,female.immature,mature

		# 3. Wild Card (all categories)
		@ageing my_ageing
		categories *

		# 4. The second shorthand way
		@ageing my_ageing
		categories sex=male sex=female
\end{verbatim}}}
\fi
To combine/aggregate categories together, use the "\texttt{+}" special character. For example, this feature can be used to specify that the total biomass of the population is made up of both males and females.

For example,

{\small{\begin{verbatim}
		@observation CPUE
		type biomass   # observation using an index of biomass
		categories male+female
        ...   # other subcommands to link index to the fishery etc
\end{verbatim}}}

This combination/aggregation functionality can be used to compare an observation to the total combined population:

{\small{\begin{verbatim}
		@observation CPUE
		type biomass
		categories *+
 ...   # other subcommands to link index to the fishery etc
		\end{verbatim}}}

If the levels \subcommand{male} and \subcommand{female} are the only categories in a population (i.e., factor \textit{sex}), then this is the same syntax as the command block above it.

Shorthand syntax can be useful when applying processes to a specific group of categories from the partition.

For example, to apply a spawning migration to the mature categories in the partition given the partition definition

{\small{\begin{verbatim}
		@categories
		format area.maturity.tag
		names north,south.immature,mature.notag,2001:2005
\end{verbatim}}}

To migrate a portion of the mature population from the southern area to the northern area:

{\small{\begin{verbatim}
		@process spawn_migration
		type transition_category   # process to move fish from one category to another
		from format=south.mature.* # move all south mature fish, both notag and tagged fish
		to format=north.mature.*   # into the relevant north categories
\end{verbatim}}}

An easy way to determine if you have specified the syntax correctly is to look at a report. \CNAME\ will expand most shorthand category labels in reports, and this can be used to check the order that \CNAME\ has assumed, and that these have been specified in the correct order for other related parameters . 

\subsection{\I{Referencing vector and map parameters}\label{sec:params}}

To build relationships between command blocks, \CNAME\ uses  a referencing system so that blocks and parameters within blocks can be accessed. In its simplest form, command blocks are referenced by their label. To access specified parameters within a command block, the syntax used is:

{\small{\begin{verbatim}
<syntactic element>     #<>  enclosing a description of the element

# most used version
<block type>[<label of block>].<parameter name>
# e.g., identify a fishery
<block type>[<label of block>].method_<parameter name>

## Examples
# recruitment multiplier (ycs) parameter in the process block called recruitment
process[recruitment].recruitment_multiplier

# natural mortality in the process called Fishing
process[Fishing].m

# pot fishery in the process called Fishing
# it is usual to define all fisheries in one
# mortality process block so we need a way to
# identify each one
process[Fishing].method_pot
\end{verbatim}}}

Parameters can be scalars (one value), vectors (several values), or maps. A map consists of two vectors: one containing a key value (for searching or uniquely indexing), and another vector that contains values associated with the index, e.g., for specifying recruitment multiplier values for each year, the years are the key (or index). To reference one or more components of a vector or map use the \textit{\{\}} syntax. This may be needed when specifying which element(s) in a vector or map are to be estimated.

An example of a map parameter is \texttt{recruitment\_multipliers} in a recruitment process

{\small{\begin{verbatim}
@process   WestRecruitment
# Beverton-Holt function
type       recruitment_beverton_holt
# initial values of the recruitment_multipliers (YCS) (a vector with 9 values)
recruitment_multipliers 1 1 1 1 1 1 1 1
standardise_years 1975:1983

# An alternative method to specify a sequence of values
# use an asterisks to represent a vector of repeating integers
recruitment_multipliers 1*8
\end{verbatim}}}

To specify that only the last four years of the recruitment multipliers (YCS) parameter \texttt{process[WestRecruitment].recruitment\_multipliers} are to be estimated:

{\small{\begin{verbatim}
@estimate RecMult    # RecMult is a label to identify this block
# estimate 4 values only: 1980, 1981, 1982, & 1983
parameter process[WestRecruitment].recruitment_multipliers{1980:1983}
\end{verbatim}}}

To estimate a common value for a block of years in a map parameter use the \textit{same} subcommand. We illustrate the idea within the process \command{time\_varying}\texttt{[label].type=constant}, where we want to fix \textit{q} over a specified block of years, 1992 to 1995.

First specify the relationships in a \command{time\_varying} block:

{\small{\begin{verbatim}
@time_varying q_step1
# specify a set value for a year
type          constant
# parameter ref for q in block Fishq
parameter     catchability[Fishq].q
# or 1992:1995 = key into value
years 	1992	1993	1994	1995
value 	0.2		0.2		0.2		0.2
# or 0.2*4, initial values of q
\end{verbatim}}}

Next, to estimate only one \textit{q} value for the time block, pick one element of the map (say 1992), and then force all other years to have the same value:

{\small{\begin{verbatim}
@estimate q_block_1992
# estimate this one
parameter time_varying[q_step1].value{1992}
# set these to the value for 1992
same      time_varying[q_step1].value{1993:1995}
# uniform prior on q
type      uniform
lower_bound 0.1
upper_bound  10
\end{verbatim}}}

Keys are restricted in \CNAME\ to years and categories. An example using categories as a key in a map:

{\small{\begin{verbatim}
@category
factor sex
names male female

@process recruit
categories male female
# natural mortality values indexed by categories
m  0.17 0.17
...

@estimate M
# prior = uniform
type uniform
# estimate male M, "male" is a level for factor sex
parameter recruitment.[m]{male}
# set female M to the same value as male's
same recruitment.[m]{female}
\end{verbatim}}}

For vector parameters (i.e., no key values), the index is an integer starting with 1 for the first value, i.e., similar to R syntax. An example is the selectivity \textit{all.values.bounded} which can be defined by:

{\small{\begin{verbatim}
@selectivity MatSel
type         all_values_bounded
# lower bound at age (if age-based) or length class (if length-based) of 2
L            2
# upper bound at age (if age-based) or length class (if length-based) of 4
H            4
# 3 values, one for each 2, 3, and 4
v            0.1 0.2 0.7

@estimate  mature
# prior = uniform
type       uniform
# estimate the 2nd value only, i.e., at 3
parameter selectivity[MatSel].v{2}
# lower parameter range
lower_bound 0.1
# upper parameter range
upper_bound 1.0
\end{verbatim}}}

The integer \textit{{2}} cannot be used to specify the \textit{q} parameter for 1993 in the above example labelled \textit{q\_block\_1992}. This will pass the syntax test, but it will fail at the validate stage in \CNAME.


\paragraph*{\I{In-line declaration, avoiding extra command blocks}\label{sec:declare}}

In-line declarations can help shorten models by defining \command{} blocks within the subcommand line instead of having a label that points to a command block define somewhere else in the input files.

For example, catchability for a CPUE index can be defined in-line:

{\small{\begin{verbatim}
@observation chatCPUE
type biomass                   # biomass index
catchability [q=6.52606e-005]  # define catchability here
categories male+female         # index cover both sexes together


@estimate chatCPUE_q
parameter catchability[chatTANbiomass.one].q # how to reference q
type uniform_log     # prior
lower_bound 1e-2
upper_bound 1
\end{verbatim}}}

In the above code catchability is defined and estimated without explicitly creating a \command{catchability} block.

\TODO{Add Examples}

