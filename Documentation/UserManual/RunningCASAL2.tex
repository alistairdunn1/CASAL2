\section{Running \CNAME\label{sec:RunningCasal2}\index{Running \CNAME}}

\CNAME\ is run from a console window (i.e., the command line) on \I{Microsoft Windows} or from a terminal window on \I{Linux}. \CNAME\ uses information from input data files -{}- the \emph{\config\index{Input configuration file}}\ being the input files that are supplied to \CNAME.

The \config\ is required and defines the model structure, processes, observations, parameters (both the fixed parameters and the parameters to be estimated)\index{Estimable parameters}, and the requested reports (outputs).

By convention, the name of the \config\ ends with the suffix \texttt{.csl2}. However, any suffix is acceptable. The default name for the initial \config\ is \emph{config.csl2} and if used it does not have to be specified as one of the command line arguments to to \CNAME. Note that the \config\ can include other files so the specification can be split into parts, e.g., files for specifying information on the 'population', 'estimation', 'observation', and 'reports'.

Command line arguments\index{Command line arguments} are used to specify the actions or \emph{tasks}\index{Tasks} of \CNAME, e.g., to run a model with a set of parameter values, to estimate parameter values (either point estimates or MCMC), to project quantities, or to simulate observations.For example, \texttt{-r} is the \emph{run} mode, \texttt{-e} is the \emph{estimation} mode, and \texttt{-m} is the \emph{MCMC} mode. The \emph{command line arguments} are described in Section \ref{sec:CommandLineArguments}.

\subsection{\I{Using \CNAME}}

To use \CNAME, open a console window (i.e. the command prompt) window on Microsoft Windows or a terminal window on Linux. Navigate to the directory where the model \config s are located. Then enter \CNAME\ with arguments for a specific mode to start the \CNAME\ mode running; see Section \ref{sec:CommandLineArguments} for the list of possible arguments. \CNAME\ will print output to the screen.

For both 64-bit Linux and Microsoft Windows, we recommend using the zip files available on \github\ for Microsoft Windows and Linux; there is also file 'Casal2.tar.gz' for older versions of Linux. Running \CNAME\ on a system requires the main binary (casal2.exe on Windows, or casal2 on Linux) and the associated dynamically linked libraries (DLL) for Windows or shared objects (.so) for Linux to be installed into appropriate directories, and cannot be (easily) run by copying the binary to a working directory. \CNAME\ is not available for 32-bit operating systems or MacOS.

\subsection{\I{Redirecting standard output}\label{sec:RedirectingStandardOut}}

\CNAME\ uses the \texttt{standard output}\index{standard output} stream to display runtime information. The \I{standard error} stream is used by \CNAME\ to output the program exit status and runtime errors. We suggest redirecting both the standard output and standard error into a file or separate files\index{Redirecting standard out}\index{Redirecting standard error}.

With the bash shell (on Linux systems), you can do this using the command structure

\begin{verbatim} (casal2 [arguments] > run.out) >& run.err &\end{verbatim}

It may be useful to redirect the standard input, especially if you're using \CNAME\ inside a batch job, i.e.

\begin{verbatim} (casal2 [arguments] > run.out < /dev/null) >& run.err &\end{verbatim}

On Microsoft Windows systems, you can redirect to standard output using

\begin{verbatim} casal2 [arguments] > run.out\end{verbatim}

And, on some Microsoft Windows systems (e.g., Windows 10), you can redirect to both standard output and standard error, using the syntax

\begin{verbatim} casal2 [arguments] > run.out 2> run.err\end{verbatim}

\CNAME\ outputs header information to standard output. The header\index{Output header information} consists of the program name and version, the arguments passed to \CNAME\ from the command line, the date and time that the program was called (derived from the system time), the user name, and the machine name (including the operating system and the process identification number). This information can be used to track output across runs, dates, and versions of \CNAME .

\vspace*{4mm}

\subsection{\I{Command line arguments}\label{sec:CommandLineArguments}}

\CNAME\ is called using:

\texttt{\cname\ [-c \emph{config\_file\_1} \emph{config\_file\_2} \emph{config\_file\_N}] [\emph{task}] [\emph{options}]}

where

\begin{description}
  \item [\texttt{-c \emph{config\_file\_1 config\_file\_N}}] Define the \config for \CNAME\ (if this argument is omitted, the default \config\ is \texttt{config.csl2}).
  \CNAME\ supports one or many configuration files. The first file specified must start with @model definition, subsequent files will be treated the same as an include from the end of the previous file.
\end{description}

and where \emph{task} must be one of the following (\textbf{[]} indicates a secondary label to call the task, e.g. \textbf{\texttt{-h}} will execute the same task as \textbf{\texttt{-{}-help}}),

\begin{description}
\item [\texttt{-h [-{}-help]}] Display help (this page)
\item [\texttt{-l [-{}-licence]}] Display the reference for the software license (GPL v2)
\item [\texttt{-v [-{}-version]}] Display the \CNAME\ version number and version details (including the location of the GitHub repository from which the source code was compiled)

\item [\texttt{-r [-{}-run]}] \emph{Run} the model once using the parameter values in the \config, or optionally with the free parameter values from the file specified with argument \texttt{-i} or \texttt{-I}.
\item [\texttt{-e [-{}-estimate]}] Do a point \emph{estimate} using the values in the \config\ as the starting point for the parameters to be estimated, or optionally with the free parameter values from the file specified with argument \texttt{-i} or \texttt{-I}.
\item [\texttt{-E [-{}-Estimate]} \emph{filename}] Do a point \emph{estimate} and generate an MPD file (ie.e., a file containing the free parameters and the covariance matrix). As with \texttt{-e}, this uses the values in the \config\ as the starting point for the parameters to be estimated, or optionally the free parameter values from the file specified with argument \texttt{-i} or \texttt{-I}.
\item [\texttt{-{p}[-profiling]}] Do an objective function \emph{profile} using the parameter values in the \config\ as the starting point, or optionally with the free parameter values from the file specified with argument \texttt{-i} or \texttt{-I}
\item [\texttt{-m [-{}-mcmc]}] Do an \emph{MCMC}. An estimate run is fist carried out to estimate the covariance matrix for the MCMC proposal distribution, using the values in the \config\ as the starting point for the parameters to be estimated. Optionally the free parameter values from the file specified with argument \texttt{-i} or \texttt{-I} can be used as the starting point.
\item [\texttt{-M [-{}-mcmc-from-estimate] \emph{filename}}] Do an \emph{MCMC} run using the covariance and free parameters in the MPD file.
\item [\texttt{-R [-{}-resume] \emph{filename}}] Resume a previously stopped \emph{MCMC} run using the covariance and free parameters in the MPD file. Additional arguments must be supplied to specify the sample and objective files from the previous MCMC with \texttt{-{}-objective-file} and \texttt{-{}-sample-file}.
\item [\texttt{-f [-{}-projection]}] \emph{n}. Project the model \emph{forward} in time using the parameter values in the \config\ as the starting point for the estimation, or optionally with the parameter values from the file specified with the argument \texttt{-i} or \texttt{-I}. Projections are repeated for each parameter set (i.e., each line of data in the free parameter file) \emph{n} times (the default is 1). Typically, the MCMC sample output will be used with \texttt{-i}.
\item [\texttt{-s [-{}-simulation]} \emph{n}] \emph{Simulate} \emph{n} sets of observation data using values in the \config\ as the parameter values, or optionally with the parameter values from the file specified with the argument \texttt{-i} or \texttt{-I}.

\end{description}

The following optional arguments\index{Optional command line arguments} [\emph{options}] may be specified

\begin{description}
\item [\texttt{-i [-{}-input] \emph{filename}}] \emph{Input} one or more sets of free (estimated) parameter values from \texttt{\emph{filename}} (see Section \ref{sec:Report} for details about the format of \texttt{\emph{filename}}).
\item [\texttt{-I [-{}-input-force] \emph{filename}}] \emph{Input} one or more sets of parameter values from \texttt{\emph{filename}}. This contains both the free parameters and also force the \emph{overwrite} addressable (non-estimated) values in the \config\ (see Section \ref{sec:Report} for details about the format of \texttt{\emph{filename}})
\item [\texttt{-o [-{}-output] \emph{filename}}] \emph{Output} a report of the free (estimated) parameter values in a format suitable for \texttt{-i \emph{filename}} (see Section \ref{sec:Report} for details about the format of \texttt{\emph{filename}})
\item [\texttt{-O [-{}-Output] \emph{filename}}] \emph{Output} and append a report of the free (estimated) parameter values in a format suitable for \texttt{-i \emph{filename}} (see Section \ref{sec:Report} for details about the format of \texttt{\emph{filename}})
\item [\texttt{-g [-{}-seed] \emph{seed}}] Initialise the random number \emph{generator} with \texttt{\emph{seed}}, a positive (long) integer value (note, if \texttt{-g} is not specified, then \CNAME\ will  generate a random number seed based on the computer clock time)
\item [\texttt{-L [-{}-loglevel] \emph{arg}}] Set the level for information or logging messages from \CNAME. Valid options are (from more verbose to less verbose) trace, finest, fine, medium, info, important, and warning. The default is 'info' (see Section \ref{sec:TroubleShooting-logging} for more information).
\item [\texttt{-t [-{}-tabular]}] Print \command{report} in tabular format (see Section \ref{sec:Report} for more information).
\item [\texttt{-{}-single-step}] Run with \texttt{-r} to pause the model and ask the user to specify parameters and their values to use for the next iteration (see Section \ref{sec:SingleStepping})
\item [\texttt{-q [-{}-query]\emph{object type}}] \emph{Query} an object type to print an extract of the object description and parameter definitions.  An object can be defined as \texttt{\emph{block.type}}, e.g., \texttt{casal2 -{}-query process.recruitment\_constant} will query the constant recruitment block, printing the inputs for this process (should be consistent with syntax section).
\item [\texttt{-V [-{}-verifylevel] \emph{arg}}] If \CNAME\ exits with a verify message (the default), then it will halt. If \emph{arg = warning} \CNAME\ will complete the model run and print the verify statement.
\end{description}
%
Combinations of these command line arguments can also be implemented. Examples of some useful ones are below.
\begin{description}
	\item \texttt{casal -r -i par.file > multi\_run.out} will conduct multiple model runs, one for each row of parameters in \texttt{par.file}. This can be useful for investigating the effect of individual parameters in the model or summarising profiled outputs.
	
	\item \texttt{casal -e -i par.file > multi\_estimate.out} will conduct multiple estimation routines. One for each row of parameters in \texttt{par.file}. This can be useful for assessing convergence to a global minimum. All base models should be run from multiple starting parameter values to assess model convergence/sensitivity to starting values.
	
	\item \texttt{casal -s 10 -i par.file > multi\_simulation.out} This command instructs \CNAME\ to simulate 10 sets of simulated data sets for each each row of parameters in \texttt{par.file}. The \texttt{-s} component adds observation error in simulated data sets through the likelihood distribution assumptions and the \texttt{-i} adds parameter uncertainty into the simulated data sets if each row differs.
	
	\item \texttt{casal -r --loglevel trace > run.log 2> run.err} This command runs the \CNAME\ model with parameters based on the configuration files and will print logging information into the file \texttt{run.err} (useful when debugging models). See Section~\ref{sec:TroubleShooting-logging} for details on logging.
\end{description}


\subsection{Constructing the \CNAME\ \config s \label{ConstructingConfig}}\index{Input configuration file syntax}

\begin{itemize}
	\item the description of the population structure, dynamics, and parameters. See Section \ref{sec:Population},
	\item the estimation methods and estimated variables. See Section \ref{sec:Estimation},
	\item the observations and their associated properties and likelihoods. See Section \ref{sec:Observation}, and
	\item the reports that \CNAME\ will output. See Section \ref{sec:Report}.
\end{itemize}

Note that \config\ files can \emph{include} other \config s to assist in file management, using the command \texttt{!include "\emph{filename}"}. See Section \ref{syntax:General} for more details.

\subsubsection{Commands}\index{Commands}

\CNAME\ has a range of commands that define the model structure, processes, parameters, observations, and how tasks are carried out. There are three types of commands

\begin{itemize}
	\item Commands that have an argument only and do not have subcommands (for example, \texttt{!include}\ \argument{\emph{filename}})
	\item Commands that have a label and subcommands (for example \command{process} must have a label and has subcommands)
	\item Commands that do not have either a label or argument, but have subcommands (for example \command{model} or \command{categories})
\end{itemize}

Apart from \texttt{!include}, commands start with an \texttt{@} in the first column (i.e., may not have a space or tab character before them on the line). After each command, the subcommands are listed and must occur before the next command. Otherwise, the commands and subcommands are free form with each command or subcommand on a separate line (see Section \ref{sec:CommandBlockFormat}).

Commands that have a label must have a unique label, i.e., the label cannot be used for more than one command. \CNAME\ checks and will report an error if two commands of the same type have the same label. The labels can contain alpha numeric characters, period (`.'), underscore (`\_') and dash (`-'), but cannot start with a double underscore ('\_\_'). Labels that start with a double underscore are reserved, and used for internal reports that \CNAME\ can automatically generate in some circumstances. Otherwise labels must not contain white space (tabs or spaces) or any characters that are not letters, numbers, dashes, periods, or underscores. For example,

{\small{\begin{verbatim}
@process NaturalMortality
\end{verbatim}}}
or
{\small{\begin{verbatim}
!include MyModelSpecification.csl2
\end{verbatim}}}

\subsubsection{Subcommands}\index{Commands ! Subcommands}

\CNAME\ subcommands define options and parameter values related to a particular command. Subcommands always take an argument which is one of a specific \emph{type}. The \emph{types} for each subcommand are defined in Section \ref{sec:syntax}, and are summarised below.

Like commands (\command{command}), subcommands and their arguments are not order specific, except that that all subcommands of a given command must appear before the next \command{command} block. \CNAME\ may report an error if they are not supplied in this way. However, in some circumstances a different order may result in a valid, but unintended, set of actions, leading to unexpected results.

The argument type for a subcommand can be\index{Subcommand argument type}:

\begin{description}
    \item[switch] true/false
    \item[integer] an integer number
	\item[integer vector] a vector of integer numbers
	\item[integer range] a range of integer numbers separated by a colon, e.g. 1994:1996 is expanded to an integer vector of values (1994 1995 1996)
	\item[constant] a real number (i.e., a double)
	\item[constant vector] a vector of real numbers (i.e., a vector of doubles)
	\item[estimable] a real number that can be estimated (i.e., a double)
	\item[estimable vector] a vector of real numbers that can be estimated (i.e., a vector of doubles)
	\item[addressable] a real number that can be referenced but not estimated (i.e., an addressable double)
	\item[addressable vector] a vector of real numbers that can be referenced but not estimated (i.e., a vector of addressable doubles)
	\item[string] a categorical (string) value
	\item[string vector] a vector of categorical values
\end{description}

Switches are characteristics which are either true or false. Enter \emph{true} as \argument{true} or \argument{t}, and \emph{false} as \argument{false} or \argument{f}.

Integers must be entered as whole numbers without decimal points (i.e., if \subcommand{year}\ is an integer then it is specified as \texttt{2008}, not \texttt{2008.0})

Arguments of type integer vector, constant vector, estimable vector, addressable vector, or categorical vector must contain one or more entries on a row, separated by white space (tabs or spaces). Arguments of type integer range must contain a colon (\texttt{:}) and no white space (tabs or spaces).

Parameters are defined in the population section and most (but not all) numeric parameters can be estimated. See Section \ref{sec:syntax} for the list of available parameters and if they are can be estimated. Note that parameters will only be estimated if requested using an \command{estimate} command, and are otherwise treated as a constant.

Parameters can also be addressable, i.e., they can be referred to within another command or command block by using their addressable name. See Section \ref{sec:syntax} to determine of a subcommand is addressable.

\subsubsection{The command block format}\index{Command block format}\label{sec:CommandBlockFormat}

The command block is a basic unit within the \config. Each command begins with the symbol \command{} and then the command name, usually followed by a user defined label or a valid argument. The end of each command block is denoted by the start of the next command block or end of the file. For example, the layout of a \config\ will be

\begin{description}
	\item \command{command} \subcommand{label}
	\item \subcommand{first\_subcommand} \subcommand{argument}
	\item \subcommand{second\_subcommand} \subcommand{argument}
	\item ... etc.
	\item \command{another\_command} \subcommand{label}
	\item \subcommand{another\_subcommand} \subcommand{argument}
	\item \subcommand{another\_subcommand} \subcommand{argument}
	\item ... etc.
\end{description}

Note that subcommands can be in any order within each command block. And command blocks can be in any order within the input files, except \command{model} --- this must be the first command block encountered by \CNAME.

Blank lines are ignored, as is extra white space (tabs and spaces) between arguments. However, to start command block the \command{} character must be the first character on the line and must not be preceded by any white space. Each input file must end with a carriage return.

Commands, subcommands, and arguments in the \config s are not case sensitive. However, labels and variable values are case sensitive. Note that on Linux (unlike Microsoft Windows) specification of any file names or file paths will be case sensitive.

\subsubsection{\I{Commenting out lines}}\index{Comments}

Text on a line that starts with the symbol \commentline\ is considered to be a comment and is ignored. To comment out a group of commands or subcommands, use \commentline\ at the beginning of each line to be ignored.

Alternatively, to comment out an entire block or section, use \commentstart\ at the beginning of a line to start the comment block, then end the block with \commentend. All lines (including line breaks) between \commentstart\ and \commentend\ inclusive are ignored.

\small{\begin{verbatim}
  # This line is a comment and will be ignored
  @process NaturalMortality
  m 0.2
  /* 
  This block of text
  is a comment and
  will be ignored
  */
\end{verbatim}}

\subsubsection{How to reference parameters\label{sec:parameter-names}\index{Determining parameter names}\index{Parameter names}}

All parameters have a unique name, allowing it to be referenced in other command blocks. When \CNAME\ processes the \config\ it translates each command block (see section~\ref{sec:CommandBlockFormat}) and each subcommand block into an object, each with a unique parameter name. For commands, this parameter name is simply the command label. For subcommands, the parameter name format is one of the following:

\begin{description}
	\item \texttt{command[label].subcommand} if the command has a label, or
	\item \texttt{command.subcommand} if the command has no label, or
	\item \texttt{command[label].subcommand\{i\}} if the command has a label and the subcommand arguments are a vector, and we are accessing the  \emph{i}th element of that vector.
	\item \texttt{command[label].subcommand\{i:j\}} if the command has a label, and the subcommand arguments are a vector, and we are accessing the elements from $i$ to $j$ (inclusive) of that vector.
\end{description}

For example, the parameter name of a process of instantaneous mortality (i.e., natural mortality) is the subcommand \subcommand{m} of a \command{process} of type \subcommand{mortality\_constant\_rate}, i.e., the command block may be

\small{\begin{verbatim}
				@process NaturalMortality
				type mortality_constant_rate
				categories male female
				m 0.2 0.2
\end{verbatim}}
%
\texttt{process[NaturalMortality].m} is the unique reference for the vector of male then female natural mortality values (\textbf{note:} order will follow categories order). To reference just the 'female' value then the form is
\texttt{process[NaturalMortality].m\{female\}}.

\subsection{\I{Reading a command block}\label{sec:Readingcommandblock}}\index{Reading\ command\ block}\index{Reading\ command\ block section}

Here, we illustrate reading a command block using two important commands, \texttt{@process} and \texttt{@estimate}.

The command \texttt{@process} specifies a process that can be used in the model. There are a fixed set of predefined processes (subroutines in C++ code). The way to specify which process is used is with the \texttt{type} subcommand. Processes can take one or more parameters and some will need other data to be supplied as well. Some parameters are mandatory and others can take a default value if they are not specified.

For this example we have categories male and female, and two fisheries, line and pot. The command block starts with a\textit{@process}:

{\small{\begin{verbatim}
@process Fishing
type mortality_instantaneous
\end{verbatim}}}

This sets up a process block using the \textit{mortality\_instantaneous} process which simultaneously depletes the population by natural mortality and from two types of fishing. Its label is \textit{Fishing}.

Next we specify the values for natural mortality (\textit{m}), an argument for this process, to 0.17 and specify that fisheries acts on all categories. Note there are two values for natural mortality, one for each category. The parameters \textit{m} can be estimated, if required. The command block fragment:

\ifAgeBased 
{\small{\begin{verbatim}
			m 0.17 0.17   # natural mortality  for each category
			relative_m_by_age One One # natural mortality multiplier
			categories *  # fishing acts on all categories ("*" shorthand for male female)
\end{verbatim}}}
\else 
{\small{\begin{verbatim}
			m 0.17 0.17   # natural mortality  for each category
			relative_m_by_length One One # natural mortality multiplier
			categories *  # fishing acts on all categories ("*" shorthand for male female)
\end{verbatim}}}
\fi

Catches are supplied via a \textit{table} format using three columns: one for year and one for each of the two fisheries, which take the labels \textit{line} and \textit{pot}. Column names are on the first line of the table and these columns can be in any order,

{\small{\begin{verbatim}
#catches
table catch         # define catches by fishery in table format
year line pot       #names columns so can identity catch for each fishery
2000 1000 2000      # catches by year
2001 500  1000
2002 1000 5000
end_table           # end of table marker
\end{verbatim}}}

Other information required is supplied in the methods table which has a fixed number of columns (again these can be in any order), one for each piece of information needed to specify a fishery. The method column defines the fishery name which is used in the catch table and also in other observations like \ifAgeBased age \else length \fi \ composition from that fishery. The categories that the fishery operates on (all in this case, but it could be just males for one and females for the other) are in the category column, the fishing selectivity to be used is given as a selectivity block name which is define somewhere else in the files, $U_max$ is the maximum exploitation rate that is allowed in any year, then the time step the fishing operates in, and lastly the block name of a penalty function that is used to penalise estimable parameter values that result in the supplied catch not being caught. Again, the penalty block is define elsewhere in the files. After the row with the column names, there is one row for each fishery:

{\small{\begin{verbatim}
table method        # supply arguments and name selectivity etc
method     category  selectivity    u_max    time_step      penalty
pot        *         potFSel        0.7      1              CatchMustBeTaken1
line       *         lineFSel       0.7      1              CatchMustBeTaken1
end_table
\end{verbatim}}}

To estimate natural mortality, you need to supply an \textit{@estimate} block with a reference name back to \textit{m} in the \textit{Fishing} block. For \textit{@estimate}, \textit{type} specifies the prior to be used in the estimation, which in this case is a normal distribution:

{\small{\begin{verbatim}
@estimate estimate.m
type normal # prior type
parameter process[fishing].m  
# this is a comment
/*
Fishing is unique amongst the @process command blocks
so this defines the unique reference to the parameter m
*/
mu  0.2 0.2   #argument to prior = mean
sd  0.02 0.02 #another argument to the prior = standard deviation
\end{verbatim}}}

Note that there are two \textit{m} values, one for each category, so there are two priors specified. The \textit{\@estimate} label \textit{estimate.m} is often redundant, but it may be needed in some circumstances.

To estimate a common \textit{m} over both sexes, we estimate one \textit{m}, say the female category, and use the \textit{same} subcommand to apply the same value to the male category \textit{m},

{\small{\begin{verbatim}
@estimate estimate.m
type normal
parameter process[fishing].m{male}
# {} is used to index one or more elements in a vector
same process[fishing].m{female} # set female value = male estimated value
# The mean of the prior
mu  0.2
# The standard deviation of the prior
sd  0.02
\end{verbatim}}}


\subsection{\I{Single-stepping \CNAME}\label{sec:SingleStepping}}\index{Single\_stepping}\index{single\_stepping section}


\TODO{Move this section to somewhere else?}

Single-stepping means \CNAME\ can 'pause' after each year in the annual cycle during a model run, write reports, then wait and process user input of updated estimable parameters for the next year (see the command line argument \texttt{ -{}-single-step}). Note this is still an experimental feature.

%This enables \CNAME\ to implement models for management simulations or scenarios that require feedback and can be used, for example, in operational management procedures (OMPs). The single-stepping process can be automated using \R, so that \CNAME\ may be used with \R\ to update input harvest values (e.g., catches from a fishery in a fisheries model) to evaluate a particular harvest control rule.

\subsection{\I{Logging and verifying Models}}\label{sec:LogandVerify}\index{Logging and Verifying Models}

\CNAME\ has a  number of standard information, warning, and error message outputs. Additional logging and debugging information is available using the \subcommand{--loglevel} or \subcommand{-L} command line option. See Section~\ref{sec:TroubleShooting-logging} for details on logging. 

\CNAME\ also applies sanity checks on model configurations. These can be bypassed using the command line option \subcommand{--verifylevel} or \subcommand{-V}. These sanity checks are based on expected model structures, i.e., in age-based models \CNAME\ verifies models have an ageing process. Currently only a few sanity checks are implemented.

\subsection{\I{Validating models between different versions of Casal2}}\label{sec:Assert}\index{Asserts}\index{Unit tests}

\CNAME\ can validate or check addressables parameters for testing and validation using the \command{assert} command. Asserts check the value of a specific addressable (for example, an observation, parameter, or the objective function) against a user defined predefined value. See Section \ref{syntax:Assert} for more information.

Asserts are one of a number of the internal and system tests used by \CNAME\ to ensure consistency across versions and revisions. \CNAME\ also uses unit tests and post-compile model validations to verify the source code. See Appendix \ref{sec:buildrules} for more information. 


\subsection{\CNAME\ exit status values\index{Exit status value}}

When \CNAME\ is run, it will either complete its task successfully or output an error. \CNAME\ will return a single exit status value 'completed' to the standard output. Error messages will be printed to the console. When input file configuration errors are found, \CNAME\ will print error messages, along with the associated filename(s) and line number(s) where the errors were identified, for example,

{\small{\begin{verbatim}
	[ERROR] At line 15 in Reports.csl2: Parameter '{' is not supported
\end{verbatim}}}
