% Standard A4 2-sided layout
\documentclass[a4paper,11pt,twoside,pdftex]{article}

%
% Define the switch here using "newif" and start its name with "if"
% Here, NAME_OF_SWITCH == "AgeBased"
\newif\ifAgeBased

% By default, a switch is "false". Use \NAME_OF_SWITCHtrue to set to true
% Uncomment the line below to set the switch to "true"
\AgeBasedtrue


% Caption package for formatting figure and label captions
\usepackage[font=bf,format=hang,labelsep=colon,justification=justified,singlelinecheck=false,compatibility=false]{caption}

% Misc packages
\usepackage{ifthen}   % for conditional expressions
\usepackage{float}    % for floating tables and graphics
\usepackage{setspace} % for defining spacing between lines
\usepackage{cmap}     % allow searching in pdf documents
\usepackage{tabulary} % better tables
\usepackage{comment}  % block comments
\usepackage[toc,page]{appendix}
\usepackage{csquotes} % better tables

% hyperref for HTML links within pdf
% EXAMPLE: \href{https://www.niwa.co.nz}{NIWA}
\usepackage[bookmarks=true,bookmarksopen=false,bookmarksnumbered=true,
            pdfpagelayout=TwoPageRight,
            pdftitle={Casal2 user manual for age-based models},
            pdfauthor={Casal2 Development Team} %authors
            pdfsubject={Casal2 user manual for age-based models, NIWA Technical Report},
            pdftex]{hyperref}
\hypersetup{
  breaklinks=true,      % allow line breaks in URLs
  colorlinks=true,      % use colour to define links
  linkcolor=black,      % colour of internal links
  citecolor=black,      % colour of links to bibliography
  filecolor=black,      % colour of file links
  urlcolor=darkgray     % colour of external links
}
\pdfadjustspacing=1

% Use AMS Maths
\usepackage[fleqn]{amsmath}
% artificial method of adding vertical space within equations
\newcommand\AddVspace{\\[0 pt]}

% Package for including and pretty printing of external config files and program code
\usepackage{listings}
\lstset{
  basicstyle=\ttfamily\footnotesize,
  breaklines=true,
  columns=fullflexible,
  showspaces=false,               % show spaces adding particular underscores
  showstringspaces=false,         % underline spaces within strings
  showtabs=false,                 % show tabs within strings adding particular underscores
  tabsize=2,			                % sets default tabsize to 2 spaces
  breakatwhitespace=false,	      % sets if automatic breaks should only happen at whitespace
  escapeinside={\%*}{*)}          % if you want to add a comment within your code
}

% Allow colour for HTML links
\usepackage{color}
\definecolor{darkgray}{gray}{0.20}

% Geometry for A4 layout like MS-Word defaults
\usepackage[left=2.54cm,top=2.54cm,bottom=3.17cm,right=3.17cm]{geometry}

% Natlib to better cite references (round brackets, commas between references, and sorted)
\usepackage[round,comma,sort]{natbib}

% Making the index
\usepackage{makeidx}
\makeindex

% Graphics (no postscript files.. just use jpeg, png, etc)
\usepackage[dvips]{graphicx}
% Changes fonts to Times, Helvetica, Courier
\usepackage{pslatex}

% Section header fonts
\usepackage{sectsty}
% Normal sized arial style section headings
\allsectionsfont{\sffamily\large}
% Add a dot after section headings
\makeatletter
\makeatother

% Page style
\usepackage{fancyhdr}
\pagestyle{fancy}
\fancyhead{}
\fancyfoot{}
\headheight 15pt
\renewcommand{\headrulewidth}{0pt} % rule line under header
\renewcommand{\footrulewidth}{0pt}
\setlength{\parindent}{0pt} % No indentation at start of paragraph
\setlength{\baselineskip}{1ex plus 0.2ex minus 0.1ex}
\setlength{\parskip}{1.05ex} % Gap between paragraphs
\raggedbottom % prefer space at the bottom of page

% Make equation numbers be section number then equation number
\makeatletter
\@addtoreset{equation}{section}
\@addtoreset{figure}{section}
\@addtoreset{table}{section}
\def\thefigure{\thesection.\@arabic\c@figure}
\def\thetable{\thesection.\@arabic\c@table}
\def\theequation{\thesection.\@arabic\c@equation}
\makeatother

% stop figures from going onto a page by themselves
\renewcommand{\topfraction}{0.85}
\renewcommand{\textfraction}{0.1}
\renewcommand{\floatpagefraction}{0.75}

% make list gaps small
\usepackage{tweaklist} % (a local style) for tweaking spacing between list elements
\renewcommand{\enumhook}{\setlength{\topsep}{0.5ex}%
  \setlength{\itemsep}{0.1ex}}
\renewcommand{\itemhook}{\setlength{\topsep}{0.5ex}%
  \setlength{\itemsep}{0.1ex}}
\renewcommand{\descripthook}{\setlength{\topsep}{0.5ex}%
  \setlength{\itemsep}{0.1ex}}

% Minimise hyphen use
\hyphenpenalty=5000
\tolerance=1000

% Compact titles
\usepackage[small,compact]{titlesec}

% New commands to define macros and other aids to text and layout
\newcommand{\config}{input configuration file}
\newcommand{\command}[1] {\texttt{@#1}}
\newcommand{\commandlab}[1] {\command{#1\texttt{[label]}}}
\newcommand{\subcommand}[1] {\texttt{#1}}
\newcommand{\commandsub}[2] {\command{#1}\subcommand{.#2}}
\newcommand{\commandlabsub}[2] {\command{#1\texttt{[label]}}\subcommand{.#2}}
\newcommand{\argument}[1] {\texttt{#1}}
\newcommand{\commandsubarg}[3] {\command{#1}\subcommand{.#2}\argument{=#3}}
\newcommand{\commandlabsubarg}[3] {\command{#1\texttt{[label]}}\subcommand{.#2}\argument{=#3}}

\newcommand{\commentline}{\#}
\newcommand{\commentstart}{/*}
\newcommand{\commentend}{*/}

\newcommand{\textlow}[1]{\raisebox{-.4ex}{\scriptsize #1}}

% Command shortcuts:
\newcommand{\Rzero}{\emph{R}$_0$}
\newcommand{\Bzero}{\emph{B}$_0$}
\newcommand{\R}{\textbf{R}}
\newcommand{\SSBoff}{$SSB_{\text{offset}}$}

% Quick index:
\newcommand{\I}[1]{#1\index{#1}}

% New commands to template syntax definitions: setup needs defining
% Define a command without a label
\newcommand{\defCom}[2]{\texttt{\textbf{@#1}\index{Command ! #1}} \hspace{0.5cm} {#2}}
% Define a command with a label
\newcommand{\defComLab}[2]{\texttt{\textbf{@#1}\ \emph{label}\index{Command ! #1}} \hspace{0.5cm} {#2}}
% Define a subcommand
\newcommand{\defSub}[2]{\texttt{#1} \hspace{0.5cm} #2 \\*}% Define a command with an argument
\newcommand{\defComArg}[3]{\texttt{\textbf{@#1}\ \emph{#2}\index{Command ! #1}} \hspace{0.5cm} {#3}}
% Define a Command\index{Command} argument
\newcommand{\defArg}[2]{\emph{\texttt{#1}} \hspace{0.5cm} #2 \\*}

% Generic definition for subcommand syntax: setup needs defining
\newcommand{\defText}[2]{\hangindent=0.3cm \small{#1\ #2}\normalsize \noindent \\*}
% Define subcommand syntax for Type / Default / Condition / Value / Note / Example / Lower Bound / Upper Bound
\newcommand{\defType}[1]{\defText{Type:}{#1}}
\newcommand{\defDefault}[1]{\defText{Default:}{#1}}
\newcommand{\defValue}[1]{\defText{Value:}{#1}}
\newcommand{\defLowerBound}[1]{\defText{Lower bound:}{#1}}
\newcommand{\defUpperBound}[1]{\defText{Upper bound:}{#1}}
\newcommand{\defNote}[1]{\defText{Note:}{#1}}
\newcommand{\defallowed}[1]{\defText{Allowed values:}{#1}}
\newcommand{\defRef}[1]{See Section \ref{#1} for more information.}

% Input Casal2 version definitions
% WARNING: THIS FILE IS AUTOMATICALLY GENERATED BY doBuild version. DO NOT EDIT THIS FILE
\newcommand{\Version}{23.09 (2023-09-27)}
\newcommand{\SourceRepos}{https://github.com/alistairdunn1/CASAL2:Development}
\newcommand{\VersionNumber}{23.09}
\newcommand{\SourceControlRevision}{14bf9327022b81d480d17a5cdf880161d5b4f83b}
\newcommand{\SourceControlDateDoc}{2023-09-27}
\newcommand{\SourceControlYearDoc}{2023}
\newcommand{\SourceControlMonthDoc}{September}
\newcommand{\SourceControlTimeDoc}{21:48:31}
\newcommand{\SourceControlVersion}{2023-09-27 21:48:31 UTC (rev. 14bf93270)}

\newcommand{\DocYear}{\SourceControlYearDoc}
\newcommand{\DocMonth}{\SourceControlMonthDoc}
\newcommand{\DocDate}{\SourceControlMonthDoc\ \SourceControlYearDoc}
\newcommand{\DocVer}{\SourceControlDateDoc}

% New commands to automate document dates, manual titles, document reference, etc.
\newcommand{\VER}{v\Version} % Casal2 program version
\newcommand{\CNAME}{Casal2}
\newcommand{\cname}{\texttt{casal2}} % Casal2 binary name
\newcommand{\authors}{\CNAME\ Development Team}
% hyper ref for Development Team email
\newcommand{\emaillink}{\href{mailto:"Casal2 Development Team"<casal2@niwa.co.nz>?subject=Casal2:}{\texttt{Casal2 Development Team}}}
\newcommand{\github}{\url{https://github.com/alistairdunn1/CASAL2}}
\newcommand{\authorlink}{\href{mailto:"Casal2 Development Team"<casal2@niwa.co.nz>?subject=Casal2:}{authors}}
% hyper ref for email
\newcommand{\Organisation}{National Institute of Water \& Atmospheric Research Ltd.} %NIWA
\newcommand{\ManualRef}{\authors\ (\DocYear). \CNAME\ user manual for age-based models, \VER. \Organisation\ \emph{NIWA Technical Report 139}. \ref{TotPages} p. (Using source code from \SourceRepos)} % full document reference

% Special commands for TODOs and Status checking
%% Use these for developers
% \newcommand{\TODO}[1]{TODO: {#1}}
% \newcommand{\STATUS}[1]{Code test status: {#1}}
%% Use these for external PDF manual versions
\newcommand{\TODO}[1]{}
\newcommand{\STATUS}[1]{}


% Define \clearemptydoublepage so-as to have truly blank pages between sections
\let\origdoublepage\cleardoublepage
\newcommand{\clearemptydoublepage}{%
  \clearpage
  {\pagestyle{empty}\origdoublepage}%
}

% Commands for the license section taken from https://www.gnu.org/licenses/lgpl-3.0.tex
\renewcommand{\labelenumii}{\alph{enumii}}
\renewcommand{\labelenumiii}{\arabic{enumiii}}

% Load package to count the number of pages in document
% For getting number of pages in document (NOT the last page number printed), use \ref{TotPages}
% Load this last to ensure its macros are not overwritten
\usepackage{totpages}
\makeatother

% Begin the document
\begin{document}
\hbadness=10000 % to deal with under-full hbox warnings
\sloppy % use sloppy paras

% Title page
\pdfbookmark[1]{Casal2 user manual for age-based models}{title}
\pagenumbering{alph} % alpha not used, but used to remove warnings when page 1 is re-defined below

\begin{titlepage}
  \thispagestyle{empty} % no header/footer/page number on this page
	\begin{center}

		\vspace*{2.5cm}
		\Huge \CNAME\ User Manual for Age-Based Models\\

		\vspace{2.0cm}
		\huge \authors % Document authors

		\vspace{2cm}
		\begin{figure}[htp]
			\begin{center}
			 \includegraphics[height=7cm]{Figures/CASAL2.png}
			\end{center}
		\end{figure}

		~\vfill
		\Large NIWA Technical Report 139 \\% Document Date
		\Large ISSN 1174-2631 \\% Document Date
		\Large \DocYear \\% Document Date

		\vspace{1.0cm}
		User manual for age-based models with \CNAME\ \VER. \\ \SourceRepos.

	\end{center}
\end{titlepage}

% Citation page
\cleardoublepage{}
\fancyfoot[C]{\thepage}
\pagenumbering{roman}

~\vfill

\begin{center}
Citation: \ManualRef
\end{center}

% Table of contents
\clearemptydoublepage{}
\pdfbookmark[1]{Contents}{contents}
\begin{spacing}{0.8} % Reduce space between lines in contents list
\tableofcontents
\end{spacing}

% Table of figures (if required)
%\clearemptydoublepage{}
%\pdfbookmark[1]{List of figures}{figures}
%\begin{spacing}{0.8} % Reduce space between lines in contents list
%\renewcommand\listfigurename{List of figures}
%\listoffigures
%\end{spacing}

% Table of tables (if required)
%\clearemptydoublepage{}
%\pdfbookmark[1]{List of tables}{tables}
%\begin{spacing}{0.8} % Reduce space between lines in contents list
%\renewcommand\listtablename{List of tables}
%\listoftables
%\end{spacing}

% Document body
\clearemptydoublepage{}
\renewcommand{\headrulewidth}{0.2pt}
\fancyhead[RO]{\slshape \nouppercase \rightmark} % Section headings at top of page (header, odd pages)
\fancyhead[LE]{\slshape \nouppercase \leftmark}  % Section headings at top of page (header, even pages)
\pagenumbering{arabic} % Page numbers a arabic numerals

\section{\I{Introduction}\label{sec:Introduction}}

\subsection{\I{About \CNAME}}

\CNAME\ is an open-source integrated statistical catch-at-age or catch-at-length assessment tool for modelling the population dynamics of marine populations. \CNAME\ is designed for quantitative assessments of marine populations, including fish, invertebrates, marine mammals and seabirds.

\CNAME\ implements generalised age (\CNAME\ for age-based models) or length structured (\CNAME\ for length-based models) population models that allows for a great deal of choice in specifying the population dynamics, parameters and which parameters should be estimated, and the model outputs. \CNAME\ is designed for flexibility. It allows implementation of age or length structured models from single species or stocks, to multiple species or stocks, using user-defined categories such as area, sex, and maturity stage. The categories are generic, are not predefined, and are easily specified. \CNAME\ models can be used for a single population with a single anthropogenic event (i.e., a single fish stock with a single fishery), or for multiple species and populations, areas, and/or anthropogenic or exploitation methods, and can include predator-prey interactions.

In \CNAME\ the processes and observations that occur over each year are defined by the user. Processes include recruitment, natural mortality, and anthropogenic mortality. Observations used to fit the models can be from many different sources, including removals-at-size or -age (e.g., a fishery), research survey or other biomass indices, and mark-recapture data. Model parameters can be estimated using penalised maximum likelihood or Bayesian methods.

As well as the point estimates of the parameters, \CNAME\ can calculate the likelihood or posterior distribution profiles for estimated parameters, and can generate Bayesian posterior distributions using Markov chain Monte Carlo methods. \CNAME\ can project the population status into the future using either deterministic or stochastic population dynamics. \CNAME\ can also simulate observations from a given model for both existing and potential observations.

The \CNAME\ user manual has been split into two separate manuals for the age-based functionality and for the length-based functionality. These two manuals contain many common components but differ in processes and observations. \ifAgeBased
This manual is for age-based models. For length-based models, see the \CNAME\ user manual for length-based models. 
\else
This manual is for length-based models. For age-based models, see the \CNAME\ user manual for age-based models. 
\fi % end if


\subsection{\I{Citing \CNAME}}

The reference for this document is: \ManualRef

The peer-reviewed journal article reference for \CNAME\ is \cite{doonan_casal2}.

\CNAME\ has also been simulation tested using simulated data from CASAL to validate results \citep{dunn_integrated_2022}.

\subsection{\I{\CNAME\ Contributors}}

The \CNAME\ project is maintained by the \CNAME\ Development Team. \CNAME\ was initiated by Alistair Dunn. The software architect and lead author of the software code was Scott Rasmussen. Contributors to the development of \CNAME\ are Scott Rasmussen, Alistair Dunn, Ian Doonan, Craig Marsh, Teresa A'mar, Kath Large, Sophie Mormede, Samik Datta, Matt Dunn, Jingjing Zhang, Marco Kienzle, and Arnaud Gr\"{u}ss.

The development of \CNAME\ was funded by \href{http://www.niwa.co.nz}{National Institute of Water \& Atmospheric Research Ltd. (NIWA)}, with additional funding from the New Zealand \href{http://www.mpi.govt.nz}{Ministry for Primary Industries}. More recent developments of this version were funded by Ocean Environmental Ltd.

\subsection{\I{Software license}}

This program and the accompanying materials are made available under the terms of the \href{http://www.opensource.org/licenses/GPL-2.0}{GNU General Public License version 2} which accompanies this software (see Section \ref{sec:License}).

Copyright \copyright 2016-\SourceControlYearDoc, \href{https://www.niwa.co.nz}{\Organisation}. All rights reserved.

\subsection{\I{Where to get \CNAME }}

This version of \CNAME\ is hosted on GitHub, and can be found at \url{https://github.com/alistairdunn1/CASAL2}\index{GitHub}.

There are installation packages available for Linux and Microsoft Windows. Release versions of the package includes the \CNAME\ binary, the \CNAME\ \R\ library, the \CNAME\ User Manual and associated documentation, example models, and other information. The installation packages for older versions of \CNAME\ can be downloaded at \url{https://github.com/NIWAFisheriesModelling/CASAL2/releases}. For more recent versions of the compiled packages, please contact the \CNAME\ Development Team. 

See \url{https://www.niwa.co.nz/} or \url{https://github.com/alistairdunn1/CASAL2} for more information about \CNAME.

\subsection{\I{System requirements}}

\CNAME\ is available for most x86 compatible machines running 64-bit \I{Linux} and \I{Microsoft Windows} operating systems. \CNAME\ has not been compiled nor tested on MacOS.

Several of \CNAME's tasks are computer intensive and a fast processor is recommended. Depending on the model implemented, some of the \CNAME\ tasks can take a considerable amount of processing time (minutes to hours), and in extreme cases may even take several days to complete an MCMC estimate.

Output files have the potential to be large, and the output from developing a model, sensitivity analyses, and running multiple MCMC chains can take up significant amounts of disk space\index{Disk space}. Depending on the number and type of user output requests, the output could range from a few hundred kilobytes to several hundred megabytes. When estimating model fits, several hundred megabytes of RAM may be required, depending on the spatial size of the model, number of categories, and complexity of processes and observations. For larger models, several gigabytes of RAM and disk space may occasionally be required.

\subsection{\I{Necessary files}}

For both 64-bit Linux and Microsoft Windows, we recommend using the zip files available on \github\ for Microsoft Windows and Linux; older Linux systems can use Casal2.tar.gz. Running \CNAME\ on a system requires the main binary ("casal2.exe" on Windows, or "casal2" on Linux) and the associated dynamically linked libraries (DLL) for Windows or shared objects (.so) for Linux, and cannot be (easily) run by copying the binary to a working directory. \CNAME\ is not available for 32-bit operating systems or MacOS.

\CNAME\ does not post-process model output. \CNAME\ writes all output to text files --- either to standard output or directed to files. A package that allows tabulation and graphing of model outputs is recommended. Software such as \href{http://www.r-project.org}{\R}\ \citep{R} is recommended. The \CNAME\ \R\ package is provided for extracting \CNAME\ output from reports and output files into \R\ (see Section \ref{sec:PostProcessing}). A separate \R\ package, r4Casal2, is also available on GitHub and provides some examples of diagnostic and plot functionality.

\subsection{Getting help\index{Getting help}\index{User assistance}\index{Repoerting issues or errors}}

\CNAME\ is distributed as unsupported software. Please notify the \CNAME\ Development Team of any issues with or errors in \CNAME. Please contact the \emaillink. See Section \ref{sec:ReportingErrors} for the guidelines for reporting issues. Note that \CNAME\ is a complex program, with many different options and possibilities, and we may not be able to provide any useful help if you submit an error report that does not follow the guidelines.

\subsection{Technical details\index{Technical specifications}}\label{sec:TechnicalDetails}

The source code\index{\CNAME\ source code} for \CNAME\ is available in the GitHub repository at \github. \CNAME\ is compiled on GitHub using GitHub Actions on host operating systems Microsoft Windows Server 2019 and Ubuntu 20.04.

\CNAME\ was compiled on Linux using \texttt{gcc} (\url{https://gcc.gnu.org}), the C/C++ compiler developed by the GNU Project. The 64-bit Linux \index{Linux} version was compiled using \texttt{gcc} version 10.3.0). 

The Microsoft Windows  (\url{https://www.microsoft.com})\index{Microsoft Windows} version was compiled using TDM-gcc (\url{https://jmeubank.github.io/tdm-gcc/}) using \texttt{gcc} 10.3.0 (\url{http://gcc.gnu.org})\index{gcc}. The Microsoft Windows (\url{https://www.microsoft.com}) installer was previously built using the Inno Setup (\url{https://jrsoftware.org/isdl.php}). \textbf{Note:} for some previous gcc versions, there have been issues related to threading. This was indicated by failed unit tests which rely on threading such MCMCHamiltonian etc.

\CNAME\ includes several minimisers; different minimisers may perform better for some models than others. These include both numerical differences minimisers and automatic differentiation minimisers. Numerical differences minimisers will usually work for most problems, albeit more slowly than auto-differentiation based minimisers. The numerical differences minimisers are:

\begin{enumerate}
\item Numerical differences: A minimiser that is closely based on the main algorithm of \cite{779}, that uses finite difference gradients\index{Finite differences minimiser}.
\item deltadiff: A multithreaded version of the numerical differences, but with $tan$ rescaling instead of $arcsin$ and available for use with the Hamiltonian Monte Carlo MCMC algorithm.
\item DESolver: The differential evolution solver\index{Differential evolution minimiser} \citep{1442}, based on code by \href{mailto:<godwin@pushcorp.com>}{Lester E. Godwin} of \href{http://www.pushcorp.com}{PushCorp, Inc.}
\end{enumerate}

There are two auto-differentiation minimisers, both based on ADOL-C. These are
\begin{enumerate}
\item Betadiff: A minimiser using an older version of ADOL-C (v1.8.4) that was used as the automatic differentiation minimiser in CASAL \citep{1388}, with the same minimising algorithm as used for the finite differences minimiser above, but based on the gradient from the auto-differentiation chain.
\item ADOL-C: A minimiser using version of ADOL-C (v2.5.1) \citep{walther1996adolc};, with a similar minimising algorithm as used for the finite differences minimiser above, but based on the gradient from the auto-differentiation chain.
\end{enumerate}

The random number generator\index{Random number generator} used in \CNAME\ uses an implementation of the Mersenne twister random number generator \citep{796}. This functionality, the command line functionality, matrix operations, and a number of other functions use the \href{http://www.boost.org/}{Boost} C++ library (Version 1.71.0)\index{Boost C++ library}.

Note that the output from \CNAME\ may differ slightly on the different operating systems and operating system versions due to different precision arithmetic or other platform-dependent (including CPU hardware) implementation details. In particular, the implementation of the standard C++ library \texttt{math.h} differs slightly on different platforms, and hence the results from different platforms may be different.

\CNAME\ uses unit tests and post-compile model validations to verify the source code. Unit tests of the underlying \CNAME\ code are run at build time using the \href{https://github.com/google/googletest}{Google Test and Mock} unit testing and mocking framework. The unit test framework aims to cover a significant proportion of the key functionality within the \CNAME\ code base. The unit test code\index{Unit tests} for \CNAME\ is available as a part of the underlying source code. Post-compile models are run using Python scripts at the end of every build and with each commit to the GitHub repository. See Appendix \ref{sec:buildrules} for more information. 




\clearemptydoublepage{}
\section{Model overview\label{sec:Overview}}\index{Model overview}

\CNAME\ is a generalised age- or length-structured population dynamics modelling framework for undertaking age- or length-structured integrated assessments \citep{Maunder_2013}. \CNAME\ allows for multiple sources of information to be combined into a single analysis using a statistical framework so that error sources are fully propagated into the uncertainty in the outcomes. The model follows cohorts as numbers-at-age (or numbers-at-length for length-structure models) through time, recording the changes that occur from the population dynamic processes.\index{Model ! About \CNAME}\index{About \CNAME}

\CNAME\ is run from the console window on Microsoft Windows or from a terminal window on Linux. \CNAME\ has two sources of information: the \emph{\config} which  define the model structure, provides observations, defines active parameters, and specifies outputs (reports); and the run-time mode (e.g., estimate active parameters, run projections, etc..) that is given by command line options and arguments (see Section \ref{sec:RunningCasal2} for specific details). Typically a 'run' will simply run the model (without estimation) and generate the expected values and other models outputs from the parameter values assumed, and an estimation will minimise the model objective function to derive the 'best fit' parameters.

A \CNAME\ model is defined by its initial conditions and the processes that occur within and to the population over the years of the model run. The annual cycle defines the time steps that occur within each year and the order of processes within each time step. At each point in time, the model updates the \emph{state}\index{Model ! state}\index{State} of the model, where the state consists of two parts, the \emph{partition}\index{Model ! partition}\index{Partition}, and any \emph{derived quantities}\index{Model ! derived quantities}\index{Derived quantities} requested.

The \emph{partition} is a representation of the population at each time step, and can be considered a matrix of the numbers of individuals within each category (i.e., row) and at each age (i.e., column). The partition will change after each process\index{Model ! processes}\index{Processes}, and hence after each \emph{time-step}\index{Model ! time-steps}\index{time-steps} of every year. The rows of the partition (categories) and columns \ifAgeBased(ages)\else(lengths)\fi \ define the population structure. For example, categories can define males and females, area, and/or maturity stage. Note that any user-defined category or category label is possible, for example species, stock, tagged status. The number of categories, what they represent, and how they interact is completely defined by the user. The model records the numbers of individuals within each category and \ifAgeBased age\else length\fi \  (e.g., for model with a two sexes, the numbers of males and females at \ifAgeBased age\else length\fi ). In general, cohorts are added via a recruitment event, are aged annually, and are removed from the population via various forms of mortality (e.g., fishing or natural mortality).

A \emph{derived quantity} is a value that is calculated for the partition at some point in time. An example of a derived quantity is spawning stock biomass ($SSB$) -- the sum of the biomass of individuals that are mature (or spawning) at some specific point in the annual cycle. Unlike the partition, which is updated in each time step, a derived quantity calculates a single value for each year of the model run. Hence, derived quantities are a vector of values over the time period represented by the model. Each derived quantity can be reported or used as an input into a process. The most commonly used derived quantity is spawning stock biomass ($SSB$) in the stock-recruitment relationship which determines recruitment of the population into the model. Another example might be a density dependent mortality process for a species that is based on the biomass of another species in the model.

Observations are the data that was observed for some aspect of the population. Each observation block includes the observed values, the sampling distribution, relationship with the partition, and the time within the model that these occur (time step and year). For example, indices of abundance or biomass from a research survey, or \ifAgeBased age\else length\fi \ compositions from a commercial catch in a fishery. The partition is queried to generate the expected values for the observations, and then the sampling distribution and sample sizes are used to calculate their likelihood. In broad terms, the model parameters are estimated to provide the best fit of the expected values to the observations, by minimising an objective function. Best fit is judged by the lowest objective function value, with the objective function equal to the sum of the negative log-likelihood values, the priors, and any model constraints (penalties). The evaluation of observations and the calculation of the expected values for each observation type is described in Section \ref{sec:Observation}.

The method that \CNAME\ uses to find a minimum, the parameters to estimate and their priors is given in the estimation section, Section \ref{sec:Estimation}. This includes the choice of minimiser, MCMC algorithms and associated parameters, as well as any transformations and penalties used to constrain the model.

Outputs (reports) are defined in the report section. \CNAME\ has a large number of reports to summarise or generate a variety of output for a given model. See Section \ref{sec:Report} for more information.

The model, its population structure, observations, methods of estimation, and output reports are all defined in the \config\. The run mode of \CNAME\ is determined by the command line arguments given when 'running' \CNAME.

The \config\ are text files with the \CNAME\ commands and subcommands. The \config\ completely describe a model implemented in \CNAME. See Sections \ref{syntax:Population}, \ref{syntax:Estimation}, \ref{syntax:Observations}, and \ref{syntax:Reports} for details of \CNAME's command and subcommand syntax.  The default name for the initial file is \texttt{config.csl2}, however any file name can be used if given as an argument to the command line when calling \CNAME. Generally, it can be useful to split the \config\ into a number of smaller files using the \texttt{@include} command or on the command line with the \texttt{-c} argument. We recommend using separate files for the different section for the configuration commands and subcommands to assist readability.


\clearemptydoublepage{}
\section{\I{Running \CNAME}\label{sec:RunningCasal2}\index{Running \CNAME}}

\CNAME\ is run from a console window (i.e., the command line) on \I{Microsoft Windows} or from a terminal window on \I{Linux}. \CNAME\ uses information from input data files -{}- the \emph{\config\index{Input configuration file}}\ being the input file that is supplied to \CNAME.

The \config\ is required and defines the model structure, processes, observations, parameters (both the fixed parameters and the parameters to be estimated)\index{Estimable parameters}, and the requested reports (outputs).

By convention, the name of the \config\ ends with the suffix \texttt{.csl2}. However, any suffix is acceptable. The default name for the \config\ is \emph{config.csl2} and if used it does not have to be specified as one of the command line arguments to to \CNAME. Note that the \config\ can include other files so the specification can be split into parts, e.g., files for specifying information on the 'population', 'estimation', 'observation', and 'reports'.

Command line arguments\index{Command line arguments} are used to specify the actions or \emph{tasks}\index{Tasks} of \CNAME, e.g., to run a model with a set of parameter values, to estimate parameter values (either point estimates or MCMC), to project quantities, or to simulate observations.For example, \texttt{-r} is the \emph{run} mode, \texttt{-e} is the \emph{estimation} mode, and \texttt{-m} is the \emph{MCMC} mode. The \emph{command line arguments} are described in Section \ref{sec:CommandLineArguments}.

\subsection{\I{Using \CNAME}}

To use \CNAME, open a console window (i.e. the command prompt) window on Microsoft Windows or a terminal window on Linux. Navigate to the directory where the model \config s are located. Then enter \CNAME\ with arguments for a specific mode to start the \CNAME\ mode running; see Section \ref{sec:CommandLineArguments} for the list of possible arguments. \CNAME\ will print output to the screen.

For both 64-bit Linux and Microsoft Windows, we recommend using the zip files available on \github\ for Microsoft Windows and Linux; there is also file 'Casal2.tar.gz' for older versions of Linux. Running \CNAME\ on a system requires the main binary (casal2.exe on Windows, or casal2 on Linux) and the associated dynamically linked libraries (DLL) for Windows or shared objects (.so) for Linux to be installed into appropriate directories, and cannot be (easily) run by copying the binary to a working directory. \CNAME\ is not available for 32-bit operating systems or MacOS.

\subsection{\I{Redirecting standard output}\label{sec:RedirectingStandardOut}}

\CNAME\ uses the \texttt{standard output}\index{standard output} stream to display runtime information. The \I{standard error} stream is used by \CNAME\ to output the program exit status and runtime errors. We suggest redirecting both the standard output and standard error into a file or separate files\index{Redirecting standard out}\index{Redirecting standard error}.

With the bash shell (on Linux systems), you can do this using the command structure

\begin{verbatim} (casal2 [arguments] > run.out) >& run.err &\end{verbatim}

It may be useful to redirect the standard input, especially if you're using \CNAME\ inside a batch job, i.e.

\begin{verbatim} (casal2 [arguments] > run.out < /dev/null) >& run.err &\end{verbatim}

On Microsoft Windows systems, you can redirect to standard output using

\begin{verbatim} casal2 [arguments] > run.out\end{verbatim}

And, on some Microsoft Windows systems (e.g., Windows 10), you can redirect to both standard output and standard error, using the syntax

\begin{verbatim} casal2 [arguments] > run.out 2> run.err\end{verbatim}

\CNAME\ outputs header information to standard output. The header\index{Output header information} consists of the program name and version, the arguments passed to \CNAME\ from the command line, the date and time that the program was called (derived from the system time), the user name, and the machine name (including the operating system and the process identification number). This information can be used to track output across runs, dates, and versions of \CNAME .

\vspace*{4mm}

\subsection{\I{Command line arguments}\label{sec:CommandLineArguments}}

\CNAME\ is called using:

\texttt{\cname\ [-c \emph{config\_file}] [\emph{task}] [\emph{options}]}

where

\begin{description}
  \item [\texttt{-c \emph{config\_file}}] Define the \config for \CNAME\ (if this argument is omitted, the default \config\ is \texttt{config.csl2})
\end{description}

and where \emph{task} must be one of the following (\textbf{[]} indicates a secondary label to call the task, e.g. \textbf{\texttt{-h}} will execute the same task as \textbf{\texttt{-{}-help}}),

\begin{description}
\item [\texttt{-h [-{}-help]}] Display help (this page)
\item [\texttt{-l [-{}-licence]}] Display the reference for the software license (GPL v2)
\item [\texttt{-v [-{}-version]}] Display the \CNAME\ version number and version details (including the location of the GitHub repository from which the source code was compiled)

\item [\texttt{-r [-{}-run]}] \emph{Run} the model once using the parameter values in the \config, or optionally with the free parameter values from the file specified with argument \texttt{-i} or \texttt{-I}.
\item [\texttt{-e [-{}-estimate]}] Do a point \emph{estimate} using the values in the \config\ as the starting point for the parameters to be estimated, or optionally with the free parameter values from the file specified with argument \texttt{-i} or \texttt{-I}.
\item [\texttt{-E [-{}-Estimate]} \emph{filename}] Do a point \emph{estimate} and generate an MPD file (i.e., a file containing the free parameters and the covariance matrix). As with \texttt{-e}, this uses the values in the \config\ as the starting point for the parameters to be estimated, or optionally the free parameter values from the file specified with argument \texttt{-i} or \texttt{-I}.
\item [\texttt{-{p}[-profiling]}] Do an objective function \emph{profile} using the parameter values in the \config\ as the starting point, or optionally with the free parameter values from the file specified with argument \texttt{-i} or \texttt{-I}
\item [\texttt{-m [-{}-mcmc]}] Do an \emph{MCMC}. An estimate run is fist carried out to estimate the covariance matrix for the MCMC proposal distribution, using the values in the \config\ as the starting point for the parameters to be estimated. Optionally the free parameter values from the file specified with argument \texttt{-i} or \texttt{-I} can be used as the starting point.
\item [\texttt{-M [-{}-mcmc-from-estimate] \emph{filename}}] Do an \emph{MCMC} run using the covariance and free parameters in the MPD file.
\item [\texttt{-R [-{}-resume] \emph{filename}}] Resume a previously stopped \emph{MCMC} run using the covariance and free parameters in the MPD file. Additional arguments must be supplied to specify the sample and objective files from the previous MCMC with \texttt{-{}-objective-file} and \texttt{-{}-sample-file}.
\item [\texttt{-f [-{}-projection]}] \emph{n}. Project the model \emph{forward} in time using the parameter values from a file specified with the argument \texttt{-i} or \texttt{-I}. Projections are repeated for each parameter set (i.e., for each line of data in the free parameter file) \emph{n} times. Typically, the MCMC sample output will be used with \texttt{-i}.
\item [\texttt{-s [-{}-simulation]} \emph{n}] \emph{Simulate} \emph{n} sets of observation data using values in the \config\ as the parameter values, or optionally with the parameter values from the file specified with the argument \texttt{-i} or \texttt{-I}.

\end{description}

The following optional arguments\index{Optional command line arguments} [\emph{options}] may be specified

\begin{description}
\item [\texttt{-i [-{}-input] \emph{filename}}] \emph{Input} one or more sets of free (estimated) parameter values from \texttt{\emph{filename}} (see Section \ref{sec:Report} for details about the format of \texttt{\emph{filename}}).
\item [\texttt{-I [-{}-input-force] \emph{filename}}] \emph{Input} one or more sets of parameter values from \texttt{\emph{filename}}. This contains both the free parameters and also force the \emph{overwrite} addressable (non-estimated) values in the \config\ (see Section \ref{sec:Report} for details about the format of \texttt{\emph{filename}})
\item [\texttt{-o [-{}-output] \emph{filename}}] \emph{Output} a report of the free (estimated) parameter values in a format suitable for \texttt{-i \emph{filename}} (see Section \ref{sec:Report} for details about the format of \texttt{\emph{filename}})
\item [\texttt{-O [-{}-Output] \emph{filename}}] \emph{Output} and append a report of the free (estimated) parameter values in a format suitable for \texttt{-i \emph{filename}} (see Section \ref{sec:Report} for details about the format of \texttt{\emph{filename}})
\item [\texttt{-g [-{}-seed] \emph{seed}}] Initialise the random number \emph{generator} with \texttt{\emph{seed}}, a positive (long) integer value (note, if \texttt{-g} is not specified, then \CNAME\ will  generate a random number seed based on the computer clock time)
\item [\texttt{-L [-{}-loglevel] \emph{arg}}] Set the level for information or logging messages from \CNAME. Valid options are (from more verbose to less verbose) trace, finest, fine, medium, info, important, and warning. The default is 'info' (see Section \ref{sec:TroubleShooting-logging} for more information).
\item [\texttt{-t [-{}-tabular]}] Print \command{report} in tabular format (see Section \ref{sec:Report} for more information).
\item [\texttt{-{}-single-step}] Run with \texttt{-r} to pause the model and ask the user to specify parameters and their values to use for the next iteration (see Section \ref{sec:SingleStepping})
\item [\texttt{-q [-{}-query]\emph{object type}}] \emph{Query} an object type to print an extract of the object description and parameter definitions.  An object can be defined as \texttt{\emph{block.type}}, e.g., \texttt{casal2 -{}-query process.recruitment\_constant} will query the constant recruitment block, printing the inputs for this process (should be consistent with syntax section).
\item [\texttt{-V [-{}-verifylevel] \emph{arg}}] If \CNAME\ exits with a verify message (the default), then it will halt. If \emph{arg = warning} \CNAME\ will complete the model run and print the verify statement.
\end{description}
%
Combinations of these command line arguments can also be implemented. Examples of some useful ones are below.
\begin{description}
	\item \texttt{casal -r -i par.file > multi\_run.out} will conduct multiple model runs, one for each row of parameters in \texttt{par.file}. This can be useful for investigating the effect of individual parameters in the model or summarising profiled outputs.
	
	\item \texttt{casal -e -i par.file > multi\_estimate.out} will conduct multiple estimation routines. One for each row of parameters in \texttt{par.file}. This can be useful for assessing convergence to a global minimum. All base models should be run from multiple starting parameter values to assess model convergence/sensitivity to starting values.
	
	\item \texttt{casal -s 10 -i par.file > multi\_simulation.out} This command instructs \CNAME\ to simulate 10 sets of simulated data sets for each each row of parameters in \texttt{par.file}. The \texttt{-s} component adds observation error in simulated data sets through the likelihood distribution assumptions and the \texttt{-i} adds parameter uncertainty into the simulated data sets if each row differs.
	
	\item \texttt{casal -r --loglevel trace > run.log 2> run.err} This command runs the \CNAME\ model with parameters based on the configuration files and will print logging information into the file \texttt{run.err} (useful when debugging models). See Section~\ref{sec:TroubleShooting-logging} for details on logging.
\end{description}


\subsection{Constructing the \CNAME\ \config s \label{ConstructingConfig}}\index{Input configuration file syntax}

\begin{itemize}
	\item the description of the population structure, dynamics, and parameters. See Section \ref{sec:Population},
	\item the estimation methods and estimated variables. See Section \ref{sec:Estimation},
	\item the observations and their associated properties and likelihoods. See Section \ref{sec:Observation}, and
	\item the reports that \CNAME\ will output. See Section \ref{sec:Report}.
\end{itemize}

Note that \config\ files can \emph{include} other \config s to assist in file management, using the command \texttt{!include "\emph{filename}"}. See Section \ref{syntax:General} for more details.

\subsubsection{Commands}\index{Commands}

\CNAME\ has a range of commands that define the model structure, processes, parameters, observations, and how tasks are carried out. There are three types of commands

\begin{itemize}
	\item Commands that have an argument only and do not have subcommands (for example, \texttt{!include}\ \argument{\emph{filename}})
	\item Commands that have a label and subcommands (for example \command{process} must have a label and has subcommands)
	\item Commands that do not have either a label or argument, but have subcommands (for example \command{model} or \command{categories})
\end{itemize}

Apart from \texttt{!include}, commands start with an \texttt{@} in the first column (i.e., may not have a space or tab character before them on the line). After each command, the subcommands are listed and must occur before the next command. Otherwise, the commands and subcommands are free form with each command or subcommand on a separate line (see Section \ref{sec:CommandBlockFormat}).

Commands that have a label must have a unique label, i.e., the label cannot be used for more than one command. \CNAME\ checks and will report an error if two commands of the same type have the same label. The labels can contain alpha numeric characters, period (`.'), underscore (`\_') and dash (`-'), but cannot start with a double underscore ('\_\_'). Labels that start with a double underscore are reserved, and used for internal reports that \CNAME\ can automatically generate in some circumstances. Otherwise labels must not contain white space (tabs or spaces) or any characters that are not letters, numbers, dashes, periods, or underscores. For example,

{\small{\begin{verbatim}
@process NaturalMortality
\end{verbatim}}}
or
{\small{\begin{verbatim}
!include MyModelSpecification.csl2
\end{verbatim}}}

\subsubsection{Subcommands}\index{Commands ! Subcommands}

\CNAME\ subcommands define options and parameter values related to a particular command. Subcommands always take an argument which is one of a specific \emph{type}. The \emph{types} for each subcommand are defined in Section \ref{sec:syntax}, and are summarised below.

Like commands (\command{command}), subcommands and their arguments are not order specific, except that that all subcommands of a given command must appear before the next \command{command} block. \CNAME\ may report an error if they are not supplied in this way. However, in some circumstances a different order may result in a valid, but unintended, set of actions, leading to unexpected results.

The argument type for a subcommand can be\index{Subcommand argument type}:

\begin{description}
    \item[switch] true/false
    \item[integer] an integer number
	\item[integer vector] a vector of integer numbers
	\item[integer range] a range of integer numbers separated by a colon, e.g. 1994:1996 is expanded to an integer vector of values (1994 1995 1996)
	\item[constant] a real number (i.e., a double)
	\item[constant vector] a vector of real numbers (i.e., a vector of doubles)
	\item[estimable] a real number that can be estimated (i.e., a double)
	\item[estimable vector] a vector of real numbers that can be estimated (i.e., a vector of doubles)
	\item[addressable] a real number that can be referenced but not estimated (i.e., an addressable double)
	\item[addressable vector] a vector of real numbers that can be referenced but not estimated (i.e., a vector of addressable doubles)
	\item[string] a categorical (string) value
	\item[string vector] a vector of categorical values
\end{description}

Switches are characteristics which are either true or false. Enter \emph{true} as \argument{true} or \argument{t}, and \emph{false} as \argument{false} or \argument{f}.

Integers must be entered as whole numbers without decimal points (i.e., if \subcommand{year}\ is an integer then it is specified as \texttt{2008}, not \texttt{2008.0})

Arguments of type integer vector, constant vector, estimable vector, addressable vector, or categorical vector must contain one or more entries on a row, separated by white space (tabs or spaces). Arguments of type integer range must contain a colon (\texttt{:}) and no white space (tabs or spaces).

Parameters are defined in the population section and most (but not all) numeric parameters can be estimated. See Section \ref{sec:syntax} for the list of available parameters and if they are can be estimated. Note that parameters will only be estimated if requested using an \command{estimate} command, and are otherwise treated as a constant.

Parameters can also be addressable, i.e., they can be referred to within another command or command block by using their addressable name. See Section \ref{sec:syntax} to determine of a subcommand is addressable.

\subsubsection{The command block format}\index{Command block format}\label{sec:CommandBlockFormat}

The command block is a basic unit within the \config. Each command begins with the symbol \command{} and then the command name, usually followed by a user defined label or a valid argument. The end of each command block is denoted by the start of the next command block or end of the file. For example, the layout of a \config\ will be

\begin{description}
	\item \command{command} \subcommand{label}
	\item \subcommand{first\_subcommand} \subcommand{argument}
	\item \subcommand{second\_subcommand} \subcommand{argument}
	\item ... etc.
	\item \command{another\_command} \subcommand{label}
	\item \subcommand{another\_subcommand} \subcommand{argument}
	\item \subcommand{another\_subcommand} \subcommand{argument}
	\item ... etc.
\end{description}

Note that subcommands can be in any order within each command block. And command blocks can be in any order within the input files, except \command{model} --- this must be the first command block encountered by \CNAME.

Blank lines are ignored, as is extra white space (tabs and spaces) between arguments. However, to start command block the \command{} character must be the first character on the line and must not be preceded by any white space. Each input file must end with a carriage return.

Commands, subcommands, and arguments in the \config s are not case sensitive. However, labels and variable values are case sensitive. Note that on Linux (unlike Microsoft Windows) specification of any file names or file paths will be case sensitive.

\subsubsection{\I{Commenting out lines}}\index{Comments}

Text on a line that starts with the symbol \commentline\ is considered to be a comment and is ignored. To comment out a group of commands or subcommands, use \commentline\ at the beginning of each line to be ignored.

Alternatively, to comment out an entire block or section, use \commentstart\ at the beginning of a line to start the comment block, then end the block with \commentend. All lines (including line breaks) between \commentstart\ and \commentend\ inclusive are ignored.

\small{\begin{verbatim}
  # This line is a comment and will be ignored
  @process NaturalMortality
  m 0.2
  /* 
  This block of text
  is a comment and
  will be ignored
  */
\end{verbatim}}

\subsubsection{How to reference parameters\label{sec:parameter-names}\index{Determining parameter names}\index{Parameter names}}

All parameters have a unique name, allowing it to be referenced in other command blocks. When \CNAME\ processes the \config\ it translates each command block (see section~\ref{sec:CommandBlockFormat}) and each subcommand block into an object, each with a unique parameter name. For commands, this parameter name is simply the command label. For subcommands, the parameter name format is one of the following:

\begin{description}
	\item \texttt{command[label].subcommand} if the command has a label, or
	\item \texttt{command.subcommand} if the command has no label, or
	\item \texttt{command[label].subcommand\{i\}} if the command has a label and the subcommand arguments are a vector, and we are accessing the  \emph{i}th element of that vector.
	\item \texttt{command[label].subcommand\{i:j\}} if the command has a label, and the subcommand arguments are a vector, and we are accessing the elements from $i$ to $j$ (inclusive) of that vector.
\end{description}

For example, the parameter name of a process of instantaneous mortality (i.e., natural mortality) is the subcommand \subcommand{m} of a \command{process} of type \subcommand{mortality\_constant\_rate}, i.e., the command block may be

\small{\begin{verbatim}
				@process NaturalMortality
				type mortality_constant_rate
				categories male female
				m 0.2 0.2
\end{verbatim}}
%
\texttt{process[NaturalMortality].m} is the unique reference for the vector of male then female natural mortality values (\textbf{note:} order will follow categories order). To reference just the 'female' value then the form is
\texttt{process[NaturalMortality].m\{female\}}.

\subsection{\I{Reading a command block}\label{sec:Readingcommandblock}}\index{Reading\ command\ block}\index{Reading\ command\ block section}

Here, we illustrate reading a command block using two important commands, \texttt{@process} and \texttt{@estimate}.

The command \texttt{@process} specifies a process that can be used in the model. There are a fixed set of predefined processes (subroutines in C++ code). The way to specify which process is used is with the \texttt{type} subcommand. Processes can take one or more parameters and some will need other data to be supplied as well. Some parameters are mandatory and others can take a default value if they are not specified.

For this example we have categories male and female, and two fisheries, line and pot. The command block starts with a\textit{@process}:

{\small{\begin{verbatim}
@process Fishing
type mortality_instantaneous
\end{verbatim}}}

This sets up a process block using the \textit{mortality\_instantaneous} process which simultaneously depletes the population by natural mortality and from two types of fishing. Its label is \textit{Fishing}.

Next we specify the values for natural mortality (\textit{m}), an argument for this process, to 0.17 and specify that fisheries acts on all categories. Note there are two values for natural mortality, one for each category. The parameters \textit{m} can be estimated, if required. The command block fragment:

\ifAgeBased 
{\small{\begin{verbatim}
			m 0.17 0.17   # natural mortality  for each category
			relative_m_by_age One One # natural mortality multiplier
			categories *  # fishing acts on all categories ("*" shorthand for male female)
\end{verbatim}}}
\else 
{\small{\begin{verbatim}
			m 0.17 0.17   # natural mortality  for each category
			relative_m_by_length One One # natural mortality multiplier
			categories *  # fishing acts on all categories ("*" shorthand for male female)
\end{verbatim}}}
\fi

Catches are supplied via a \textit{table} format using three columns: one for year and one for each of the two fisheries, which take the labels \textit{line} and \textit{pot}. Column names are on the first line of the table and these columns can be in any order,

{\small{\begin{verbatim}
#catches
table catch         # define catches by fishery in table format
year line pot       #names columns so can identity catch for each fishery
2000 1000 2000      # catches by year
2001 500  1000
2002 1000 5000
end_table           # end of table marker
\end{verbatim}}}

Other information required is supplied in the methods table which has a fixed number of columns (again these can be in any order), one for each piece of information needed to specify a fishery. The method column defines the fishery name which is used in the catch table and also in other observations like \ifAgeBased age \else length \fi \ composition from that fishery. The categories that the fishery operates on (all in this case, but it could be just males for one and females for the other) are in the category column, the fishing selectivity to be used is given as a selectivity block name which is define somewhere else in the files, $U_max$ is the maximum exploitation rate that is allowed in any year, then the time step the fishing operates in, and lastly the block name of a penalty function that is used to penalise estimable parameter values that result in the supplied catch not being caught. Again, the penalty block is define elsewhere in the files. After the row with the column names, there is one row for each fishery:

{\small{\begin{verbatim}
table method        # supply arguments and name selectivity etc
method     category  selectivity    u_max    time_step      penalty
pot        *         potFSel        0.7      1              CatchMustBeTaken1
line       *         lineFSel       0.7      1              CatchMustBeTaken1
end_table
\end{verbatim}}}

To estimate natural mortality, you need to supply an \textit{@estimate} block with a reference name back to \textit{m} in the \textit{Fishing} block. For \textit{@estimate}, \textit{type} specifies the prior to be used in the estimation, which in this case is a normal distribution:

{\small{\begin{verbatim}
@estimate estimate.m
type normal # prior type
parameter process[fishing].m  
# this is a comment
/*
Fishing is unique amongst the @process command blocks
so this defines the unique reference to the parameter m
*/
mu  0.2 0.2   #argument to prior = mean
sd  0.02 0.02 #another argument to the prior = standard deviation
\end{verbatim}}}

Note that there are two \textit{m} values, one for each category, so there are two priors specified. The \textit{\@estimate} label \textit{estimate.m} is often redundant, but it may be needed in some circumstances.

To estimate a common \textit{m} over both sexes, we estimate one \textit{m}, say the female category, and use the \textit{same} subcommand to apply the same value to the male category \textit{m},

{\small{\begin{verbatim}
@estimate estimate.m
type normal
parameter process[fishing].m{male}
# {} is used to index one or more elements in a vector
same process[fishing].m{female} # set female value = male estimated value
# The mean of the prior
mu  0.2
# The standard deviation of the prior
sd  0.02
\end{verbatim}}}


\subsection{\I{Single-stepping \CNAME}\label{sec:SingleStepping}}\index{Single\_stepping}\index{single\_stepping section}


\TODO{Move this section to somewhere else?}

Single-stepping means \CNAME\ can 'pause' after each year in the annual cycle during a model run, write reports, then wait and process user input of updated estimable parameters for the next year (see the command line argument \texttt{ -{}-single-step}). Note this is still an experimental feature.

%This enables \CNAME\ to implement models for management simulations or scenarios that require feedback and can be used, for example, in operational management procedures (OMPs). The single-stepping process can be automated using \R, so that \CNAME\ may be used with \R\ to update input harvest values (e.g., catches from a fishery in a fisheries model) to evaluate a particular harvest control rule.

\subsection{\I{Logging and verifying Models}}\label{sec:LogandVerify}\index{Logging and Verifying Models}

\CNAME\ has a  number of standard information, warning, and error message outputs. Additional logging and debugging information is available using the \subcommand{--loglevel} or \subcommand{-L} command line option. See Section~\ref{sec:TroubleShooting-logging} for details on logging. 

\CNAME\ also applies sanity checks on model configurations. These can be bypassed using the command line option \subcommand{--verifylevel} or \subcommand{-V}. These sanity checks are based on expected model structures, i.e., in age-based models \CNAME\ verifies models have an ageing process. Currently only a few sanity checks are implemented.

\subsection{\I{Validating models between different versions of Casal2}}\label{sec:Assert}\index{Asserts}\index{Unit tests}

\CNAME\ can validate or check addressables parameters for testing and validation using the \command{assert} command. Asserts check the value of a specific addressable (for example, an observation, parameter, or the objective function) against a user defined predefined value. See Section \ref{syntax:Assert} for more information.

Asserts are one of a number of the internal and system tests used by \CNAME\ to ensure consistency across versions and revisions. \CNAME\ also uses unit tests and post-compile model validations to verify the source code. See Appendix \ref{sec:buildrules} for more information. 


\subsection{\CNAME\ exit status values\index{Exit status value}}

When \CNAME\ is run, it will either complete its task successfully or output an error. \CNAME\ will return a single exit status value 'completed' to the standard output. Error messages will be printed to the console. When input file configuration errors are found, \CNAME\ will print error messages, along with the associated filename(s) and line number(s) where the errors were identified, for example,

{\small{\begin{verbatim}
	[ERROR] At line 15 in Reports.csl2: Parameter '{' is not supported
\end{verbatim}}}


\clearemptydoublepage{}
\section{\I{Partition \& Categories}\label{sec:PartitionCategories}}

Dividing the population into categories is fundamental to modelling the dynamics of a fish stock. The grouping of the population into categories and either ages or lengths is called the partition. 

In \CNAME\, the concept of user defined categories allows for flexibility in grouping of categories or parts of the modelled population. Note that \CNAME\ does not know about sex or area and their properties; these are explicitly defined by the user by specifying processes that act on the categories. The cost is that users need to follow good practice to achieve clarity and readability of the input files, i.e., poor specifications can result in input files that are more difficult to understand.

CASAL had a fixed set of hard-wired categories (e.g., factors like sex, maturity, area, or stock) and each category type had a predefined set of allowed processes (or transitions in CASAL-speak), e.g., immature fish moving into the mature category \citep{1388}. This made sense when CASAL was coded, but now it is seen as a limitation, e.g., changing sex was not allowed and only male and female sexes were allowed, not an unknown sex that sometimes occurs in data. 

\subsection{Specifying the partition using categories}

A key element of the \CNAME\ model is the partition which holds the current state of the population. The partition can be conceptualised as a matrix, where each row represents a category and the columns are the \ifAgeBased age \else length \fi \ classes (Figure~\ref{Fig:part}). Each row represents all individuals in that category as a \ifAgeBased numbers-at-age \else numbers-at-length \fi vector.  There must be at least one category defined for each model.

\begin{figure}[H]
	\centering
	\ifAgeBased
	\includegraphics[scale=0.4]{Figures/partition2-age.png}
	\caption{A visual representation of the partition for a simple age-based model.}
	\else
	\includegraphics[scale=0.4]{Figures/partition2-length.png}
	\caption{A visual representation of the partition for a simple length-based model.}
	\fi
    \label{Fig:part}
\end{figure}

The categories can include combinations of levels from one or more factors such as sex, maturity state, area, stock, or even species. \CNAME\ has no predefined categories; \emph{all} categories are defined by the user. Note that the partition only has the current state of the model; past states are not kept (\textit{See} the section on derived quantities about saving summary values from the partition, p. \pageref{sec:DerivedQuantity}).

To illustrate categories, consider a model of a fish population with two fisheries, one on spawning fish at the spawning grounds and another on the non-spawning population in the rest of the stock area. The mature fish will migrate to the spawning area, where the spawning fishery occurs. At the end of spawning, these fish, along with the recruits from the previous year, migrate back to the non-spawning area. The fish population can be represented  by factors sex (levels \textit{male} and \textit{female}), maturity (levels \textit{immature} and \textit{mature}), and area (levels \textit{spawning} and \textit{non-spawning}). So the partition has 8 rows of \ifAgeBased numbers-at-age\else numbers-at-length\fi, for 2 sexes $\times$ 2 maturity levels $\times$ 2 areas.

These categories are specified in a categories block which starts with a \textit{@categories} line followed on the next line by a \textit{format} subcommand that specifies the factors to use and their order. Factor names are user defined and have no intrinsic meaning in \CNAME.

The command block for this example is:

{\small{\begin{lstlisting}
	@categories
	format area.sex.mature
	names spawn.male.immature spawn.male.mature spawn.female.immature spawn.female.mature nonspawn.male.immature nonspawn.male.mature nonspawn.female.immature #all on one line nonspawn.female.mature
\end{lstlisting}}}  %{verbatimIJD}}}

Note the "." syntax to separate the factor names.

Next comes the \textit{names} subcommand which specifies the combinations of levels that makes up each category. In a sense, the \textit{format} subcommand is not needed since the \textit{names} subcommand can define all categories. However, \textit{format} allows a more  digestible and shorter syntax to define categories here and in other blocks such as matching observation to categories that provided the data (including combinations of categories, e.g., \ifAgeBased age \else length \fi compositions that combine both sexes).

The \textit{names} subcommand can also be specified with:

{\small{\begin{verbatim}names spawn,nonspawn.male,female.immature,mature
\end{verbatim}}}

which defines the categories above in a more efficient manner, (again, note the ``.'' to separate the factors and ``,'' to separate the levels within each factor (see the next section for more details). A visualisation of the partition is in Figure \ref{Fig:part}.

When using this short-cut syntax in \textit{names}, the order of level combinations is for the levels of the first factor to change the slowest, then the next factor will change faster, and so on with the last factor to changing levels the fastest. The order is important because linking categories to their characteristics, e.g., growth curve or selectivity, is done in other subcommands where these must be specified in the same order.

To exclude unused categories from the partition, the long form must be used in the \textit{names} subcommand, e.g., to exclude  \textit{spawn.female.immature} and \textit{spawn.male.immature} since they are never in the spawning area.

To make recruitment to enter the partition in the non-spawning area, use

{\small{\begin{lstlisting}
	@categories
	format area.sex.mature
	names spawn.male.mature spawn.female.mature nonspawn.male.immature nonspawn.male.mature nonspawn.female.immature nonspawn.female.mature
\end{lstlisting}}}

\subsection{Shorthand syntax for categories}\label{sec:ShorthandSyntax}

Some specifications have long lists of categories or years or initial values for parameters and the like, e.g., for YCS from 1900 to 2019, 120 years and 120 initial vales of YCS must be specified; this is hard to do by hand and it can be error prone as well as difficult to match values for each year. Here, the range short cut (\textit{:)} can be used so the  the year specification is \textit{1900:2019}, and the multiplier short cut (\textit{*)} to give the initial values specification as \textit{1*120}.

There is also shorthand notation for categories since each category can be quite complicated\label{sec:Categories}. First use the \texttt{format} subcommand in the \textit{@categories} block to define the factors that make up the sections of the category names. A ``.'' (period) character delineates each factor and this structure allows a shorthand syntax to compose category names.

The \texttt{names} subcommand is used to list the category names. Sections within the shorthand syntax for \textit{names} are required to match the order of factors in the \textit{format} subcommand so \CNAME\ can organise and search on them. In these sections, levels for each factor use the "list specifier" and range characters, e.g.,

{\small{\begin{lstlisting}
@categories
format sex.stage.tag   # 2 sexes, 2 stages, tag years 2001 to 2005 = 20 categories

names male.immature               # Invalid: No tag information
names female                      # Invalid: no stage of tag information
names female.immature.notag.1    # Invalid: Additional format segment not defined

names male,female.immature,mature.notag,2001:2005 # Valid shortcut

# Without the shorthand syntax these categories would be written:

names male.immature.notag male.immature.2001 male.immature.2002 male.immature.2003 male.immature.2004 male.immature.2005 male.mature.notag male.mature.2001 male.mature.2002 male.mature.2003 male.mature.2004 male.mature.2005 female.immature.notag female.immature.2001 female.immature.2002 female.immature.2003 female.immature.2004 female.immature.2005 female.mature.notag female.mature.2001 female.mature.2002 female.mature.2003 female.mature.2004 female.mature.2005
\end{lstlisting}}}

The shorthand syntax available are: \TODO{edit list}

\begin{itemize}
	\item \textit{*} Specify all categories
	\item \textit{+} Categories join, e.g., \textit{categories *+} joins all categories together into one unit; \textit{categories male+female}
specifies that the observation covers both sexes combined.
    \item \textit{:} Specify a range of integers $[$int1$]$:$[$int2$]$, e.g., \textit{2000:2005} expands to \textit{2000 2001 2002 2003 2004 2005}
    \item Lists using "," $[$item1$]$,$[$item2$]$,$[$item3$]$, e.g., \textit{male,female,unsexed} are the levels for the factor \textit{sex}.
    \item Repeats a number or label: $[$number $\mid$ label$]$ * $[$integer$]$, e.g., \textit{1 * 5} $\rightarrow$ \textit{1 1 1 1 1}
    \item \textit{format=$[$X$]$=$[$x$]$=$[$int$]$}   \TODO{this may need additional explanation as to how it works}
    \textit{$[$factor$]$=$[$level$]$=$[$year range$]$}, e.g., \textit{tag=2001=1999:2003} the categories with level 2001 in the tag factor are accessible from year 1999 to 2003 inclusive.
    \item \textit{[]} replace label to a command block with the block defined inline, e.g., \textit{catchability [q = 1e-5]} rather than \textit{catchability CHATq} where \textit{CHATq} labels a command block somewhere in the input files
\end{itemize}


Example of specifying categories using the short cuts:

This syntax is the long way:
{\small{\begin{verbatim}
		@categories
		format sex.stage
		names male.immature male.mature female.immature female.mature
\end{verbatim}}}

A shorter way to specify the exact same partition structure  using \textit{lists}:

{\small{\begin{verbatim}
		@categories
		format sex.stage
		names male,female.immature,mature
\end{verbatim}}}

\CNAME\ requires categories in processes and observations so that the correct model dynamics can be applied to the correct categories of the partition.

\ifAgeBased
This block illustrates using categories required for the ageing process:

{\small{\begin{verbatim}
		# 1. The long-hand way
		@ageing my_ageing
		categories male.immature male.mature female.immature female.mature

		# 2. The first shorthand way
		@ageing my_ageing
		categories male,female.immature,mature

		# 3. Wild Card (all categories)
		@ageing my_ageing
		categories *

		# 4. The second shorthand way
		@ageing my_ageing
		categories sex=male sex=female
\end{verbatim}}}
\fi
To combine/aggregate categories together, use the "\texttt{+}" special character. For example, this feature can be used to specify that the total biomass of the population is made up of both males and females.

For example,

{\small{\begin{verbatim}
		@observation CPUE
		type biomass   # observation using an index of biomass
		categories male+female
        ...   # other subcommands to link index to the fishery etc
\end{verbatim}}}

This combination/aggregation functionality can be used to compare an observation to the total combined population:

{\small{\begin{verbatim}
		@observation CPUE
		type biomass
		categories *+
 ...   # other subcommands to link index to the fishery etc
		\end{verbatim}}}

If the levels \subcommand{male} and \subcommand{female} are the only categories in a population (i.e., factor \textit{sex}), then this is the same syntax as the command block above it.

Shorthand syntax can be useful when applying processes to a specific group of categories from the partition.

For example, to apply a spawning migration to the mature categories in the partition given the partition definition

{\small{\begin{verbatim}
		@categories
		format area.maturity.tag
		names north,south.immature,mature.notag,2001:2005
\end{verbatim}}}

To migrate a portion of the mature population from the southern area to the northern area:

{\small{\begin{verbatim}
		@process spawn_migration
		type transition_category   # process to move fish from one category to another
		from format=south.mature.* # move all south mature fish, both notag and tagged fish
		to format=north.mature.*   # into the relevant north categories
\end{verbatim}}}

An easy way to determine if you have specified the syntax correctly is to look at a report. \CNAME\ will expand most shorthand category labels in reports, and this can be used to check the order that \CNAME\ has assumed, and that these have been specified in the correct order for other related parameters . 

\subsection{\I{Referencing vector and map parameters}\label{sec:params}}

To build relationships between command blocks, \CNAME\ uses  a referencing system so that blocks and parameters within blocks can be accessed. In its simplest form, command blocks are referenced by their label. To access specified parameters within a command block, the syntax used is:

{\small{\begin{verbatim}
<syntactic element>     #<>  enclosing a description of the element

# most used version
<block type>[<label of block>].<parameter name>
# e.g., identify a fishery
<block type>[<label of block>].method_<parameter name>

## Examples
# recruitment multiplier (ycs) parameter in the process block called recruitment
process[recruitment].recruitment_multiplier

# natural mortality in the process called Fishing
process[Fishing].m

# pot fishery in the process called Fishing
# it is usual to define all fisheries in one
# mortality process block so we need a way to
# identify each one
process[Fishing].method_pot
\end{verbatim}}}

Parameters can be scalars (one value), vectors (several values), or maps. A map consists of two vectors: one containing a key value (for searching or uniquely indexing), and another vector that contains values associated with the index, e.g., for specifying recruitment multiplier values for each year, the years are the key (or index). To reference one or more components of a vector or map use the \textit{\{\}} syntax. This may be needed when specifying which element(s) in a vector or map are to be estimated.

An example of a map parameter is \texttt{recruitment\_multipliers} in a recruitment process

{\small{\begin{verbatim}
@process   WestRecruitment
# Beverton-Holt function
type       recruitment_beverton_holt
# initial values of the recruitment_multipliers (YCS) (a vector with 9 values)
recruitment_multipliers 1 1 1 1 1 1 1 1
standardise_years 1975:1983

# An alternative method to specify a sequence of values
# use an asterisks to represent a vector of repeating integers
recruitment_multipliers 1*8
\end{verbatim}}}

To specify that only the last four years of the recruitment multipliers (YCS) parameter \texttt{process[WestRecruitment].recruitment\_multipliers} are to be estimated:

{\small{\begin{verbatim}
@estimate RecMult    # RecMult is a label to identify this block
# estimate 4 values only: 1980, 1981, 1982, & 1983
parameter process[WestRecruitment].recruitment_multipliers{1980:1983}
\end{verbatim}}}

To estimate a common value for a block of years in a map parameter use the \textit{same} subcommand. We illustrate the idea within the process \command{time\_varying}\texttt{[label].type=constant}, where we want to fix \textit{q} over a specified block of years, 1992 to 1995.

First specify the relationships in a \command{time\_varying} block:

{\small{\begin{verbatim}
@time_varying q_step1
# specify a set value for a year
type          constant
# parameter ref for q in block Fishq
parameter     catchability[Fishq].q
# or 1992:1995 = key into value
years 	1992	1993	1994	1995
value 	0.2		0.2		0.2		0.2
# or 0.2*4, initial values of q
\end{verbatim}}}

Next, to estimate only one \textit{q} value for the time block, pick one element of the map (say 1992), and then force all other years to have the same value:

{\small{\begin{verbatim}
@estimate q_block_1992
# estimate this one
parameter time_varying[q_step1].value{1992}
# set these to the value for 1992
same      time_varying[q_step1].value{1993:1995}
# uniform prior on q
type      uniform
lower_bound 0.1
upper_bound  10
\end{verbatim}}}

Keys are restricted in \CNAME\ to years and categories. An example using categories as a key in a map:

{\small{\begin{verbatim}
@category
factor sex
names male female

@process recruit
categories male female
# natural mortality values indexed by categories
m  0.17 0.17
...

@estimate M
# prior = uniform
type uniform
# estimate male M, "male" is a level for factor sex
parameter recruitment.[m]{male}
# set female M to the same value as male's
same recruitment.[m]{female}
\end{verbatim}}}

For vector parameters (i.e., no key values), the index is an integer starting with 1 for the first value, i.e., similar to R syntax. An example is the selectivity \textit{all.values.bounded} which can be defined by:

{\small{\begin{verbatim}
@selectivity MatSel
type         all_values_bounded
# lower bound at age (if age-based) or length class (if length-based) of 2
L            2
# upper bound at age (if age-based) or length class (if length-based) of 4
H            4
# 3 values, one for each 2, 3, and 4
v            0.1 0.2 0.7

@estimate  mature
# prior = uniform
type       uniform
# estimate the 2nd value only, i.e., at 3
parameter selectivity[MatSel].v{2}
# lower parameter range
lower_bound 0.1
# upper parameter range
upper_bound 1.0
\end{verbatim}}}

The integer \textit{{2}} cannot be used to specify the \textit{q} parameter for 1993 in the above example labelled \textit{q\_block\_1992}. This will pass the syntax test, but it will fail at the validate stage in \CNAME.


\paragraph*{\I{In-line declaration, avoiding extra command blocks}\label{sec:declare}}

In-line declarations can help shorten models by defining \command{} blocks within the subcommand line instead of having a label that points to a command block define somewhere else in the input files.

For example, catchability for a CPUE index can be defined in-line:

{\small{\begin{verbatim}
@observation chatCPUE
type biomass                   # biomass index
catchability [q=6.52606e-005]  # define catchability here
categories male+female         # index cover both sexes together


@estimate chatCPUE_q
parameter catchability[chatTANbiomass.one].q # how to reference q
type uniform_log     # prior
lower_bound 1e-2
upper_bound 1
\end{verbatim}}}

In the above code catchability is defined and estimated without explicitly creating a \command{catchability} block.

\TODO{Add Examples}



\clearemptydoublepage{}
\section{\I{The population section: model structure and the population dynamics}}\label{sec:Population}

The command and subcommand syntax for the estimation section is given in Section \ref{syntax:Population}.

\subsection{Introduction}

This section\index{Population section} shows how to specify a model for the population dynamics. It describes the model time and \ifAgeBased age \else length \fi scope, the population processes used (e.g., recruitment, ageing or growth transition, migration, and mortality), the selectivity ogives, and how to set values for their associated parameters, or starting values if they are going to be estimated.

The basic structure of the population is defined in terms of its partitions and the succession of processes that act on them throughout a year. \CNAME\ assumes an annual cycle, i.e., rates like natural mortality are assumed to be for a year. To place certain processes or observations (e.g., a research survey) into the right part of the year, the year can be divided into one or more time steps, and each time step needs  at least one process. Each time step can represent a specific period of the calendar year, or it can be an abstract sequence of events. Certain processes like natural mortality and growth can have a proportion of the effects of the process assigned to different time steps to crudely mimic seasonal effects, or fisheries that occur in short periods of the year, as well as place a survey within the year relative to the proportion of annual natural mortality that has occurred (see Section \ref{sec:DerivedQuantity}).

The \emph{state} is the current status of the population at any given time and it can change one or more times during the year. The state object must contain sufficient information to determine how the population changes over time, given a model and a complete set of parameters. The partition is key to the state, but it has no "memory". Thus, other information must also be kept, such as the mature biomass from a previous year or time step to calculate the recruit numbers into the first \ifAgeBased age \else length \fi class via the spawner-recruitment relationship. Quantities like mature biomass are defined as \emph{derived variables} and are calculated for each year of the model. However, the \emph{derived variables} record only summary information from the partition at a specified time step and year.

Processes can change the partition and, for example, include recruitment, natural mortality, fishing mortality, ageing (in age-based models) or a growth transition (in length-based models), migration, and maturation. These processes are repeated for each year of the model.

The specification and ordering of processes in multiple time steps can be used to represent complex dynamics, with the intermingling of multiple species and stocks, migration patterns occurring over multiple areas, and/or multiple sources of anthropogenic impacts using a range of methods which cover different areas and times.

However, the complexity of a stock structure definition is constrained by the available data. It is challenging to use a complex structure to model a population when there are no observations to support that structure.  For information on how to define categories and use the shorthand syntax see Section \ref{sec:ShorthandSyntax}.

Topics covered are:

\begin{itemize}
	\item The model scope, such as the \ifAgeBased ages \else length \fi covered, the years over which the model runs, and the end year for projections (Section~\ref{sec:Model});
    \item Linking processes \ifAgeBased, such as length-at-age,\fi to each category;
    \item The number of time steps and the processes that are applied in each time step\index{Annual cycle} (Section~\ref{sec:TimeStep});
    \item The specification of and the parameters for the population processes: processes that add or remove individuals from a partition, or shift individuals between \ifAgeBased ages \else length classes \fi and categories in a partition;
    \item The initialisation process: the state of the partition at the start of the first year\index{Initialisation}\index{Model ! initialisation};
   \item Defining selectivity ogives and linking them to observations;
   \item The parameters: their definitions, initial values, prior distributions, and other characteristics; and
   \item Derived quantities, e.g., mature biomass, to include in density-dependent processes such as the spawner-recruit relationship
\end{itemize}

\subsection{\I{Model scope and structure}}\label{sec:Model}

The model needs scoping for \ifAgeBased ages \else length classes \fi  and year covered. This is done in the \command{model} command block.

Each \CNAME\ model requires:

\begin{itemize}
\ifAgeBased
\item The minimum and maximum population ages
\item Whether the maximum age is a plus group
\else 
\item The length classes used
\item Whether the maximum length class is a plus group
\item The mean length of the last plus group if its a plus group
\fi
\item The start and final year
\item The names of all of the categories
\end{itemize}

\ifAgeBased The ages used starts at the minimum age through to the maximum age in steps of one. \else The length classes used using the length bins defined. The last group can be a plus group and, if so, the mean length of this group must be defined. \fi The model is run from the start year through to the final year. It can also be run past the final year to project the state of the population through the final projection year.

An example of how to specify a potential model with two categories is outlined below;  the \command{model} and \command{categories} blocks are:

\ifAgeBased 
{\small{\begin{verbatim}
		@model
		start_year 1981
		final_year 2000
		projection_final_year 2010
		base_weight_units     tonnes
		min_age    1
		max_age   20
		age_plus_group        true
		initialisation_phases Equilibrium_phase
		time_steps            step1 step2 step3

		@categories
		format      sex
		names       male female
		age_lengths male_growth female_growth  # labels for growth blocks
\end{verbatim}}}

This model runs for 20 years, starting in 1981, and will do a projection over 10 years for a population with ages from  1 through 20, with age 20 being a plus-group. Each year is divided into three time-steps. The categories are male and female (i.e., there is one category factor, labelled \textit{sex}) and each category has an age-length relationship.

Whist \CNAME\ generally uses generic formulation, it does have some specific population concepts, in this case, growth which can be different for each category. Additionally, there is a length-weight characteristic which is specified in the age-length blocks, which in this example are command blocks starting with \command{age\_size male\_growth} and \command{age\_size female\_growth} that are placed elsewhere in the input files (not shown).

\else
{\small{\begin{verbatim}
			@model
			start_year 1981
			final_year 2000
			projection_final_year 2010
			base_weight_units     tonnes
			length_bins	    1:68
			length_plus True
			length_plus_group 69
			initialisation_phases Equilibrium_phase
			time_steps            step1 step2 step3
			
			@categories
			format      sex
			names       male female
			growth_increment male_growth female_growth
\end{verbatim}}}

This model runs for 20 years, starting in 1981, and will do a projection over 10 years for a population with length classes 1--68, with the last class being a plus-group with mean length 69 cms. Each year is divided into three time-steps. The categories are male and female (i.e., there is one category factor, labelled \textit{sex}) and each category has an associated growth transition matrix .

Whist \CNAME\ generally uses generic formulation, it does have some specific population concepts, in this case, the growth transition matrix can be different for each category. These are specified in the growth increment blocks that define how individuals size increments with each growth episode, which in this example are command blocks starting with \command{growth\_increment male\_growth} and \command{growth\_increment female\_growth} that are placed elsewhere in the input files (not shown).
\fi

\CNAME\ allows categories of the partition to exist for a subset of years of a model. This feature enables more efficient computations when models contain categories that do not persist over all model years. A model may define one-off processes that transition individuals from one category into another in a subset of the model initialisation phases or years (e.g., tagging events). Excluding categories for certain years can be more efficient as \CNAME\ will not initialise these categories or apply processes to categories in years or time steps in which they do not exist.

The structure of the partition is defined in a configuration block with the \command{categories} block (Section \ref{sec:Model}).

Derived quantities are an important component of the state object. An example of a derived quantity is spawning stock biomass (SSB; the biomass of [female] spawning fish calculated at the mid point of the spawning season). \CNAME\ calculates derived quantities using the command \command{derived\_quantity}, required for some processes. In fisheries stock assessment models, a recruitment process which includes a stock-recruitment relationship requires the definition of a derived quantity that specifies the mid-season spawning stock biomass. See Section \ref{sec:DerivedQuantity} for more details.

\subsubsection{\I{The implicit annual cycle}}\label{sec:TimeStep}

There is an implicit annual cycle that orders the sequence of processes within the year, but there is no command block as such. The implementation is by ordering processes within the time-steps. This sequence is repeated for every year. Time steps are used to break the year into separate components and allow observations to be associated with specific time periods and processes. Any number of processes can occur within each time step, in any order, although there are restrictions for mortality-based processes (see Section~\ref{sec:Process-Mortality}); processes can occur multiple times within each time step. Time steps are not implemented during the initialisation phases (effectively there is only one initialisation time step), and the annual cycle in the initialisation phases can be different from the annual cycle specified for the model years (\ref{sec:Initialisation}).

Figure \ref{Fig:annual} shows an example of the annual cycle for an age-based models using three time-steps (for length-based models, the example is the same, except that the growth process would be replaced by a growth increment process).

\begin{figure}
	\centering
	\includegraphics[scale=0.5]{Figures/annual_cycle.jpg}
	\caption{A example sequence for an annual cycle for an age-based model.}\label{Fig:annual}
\end{figure}

This would be specified using \command{time\_step} block:

{\small{\begin{verbatim}
@model
time_steps step1 step2 step3
\end{verbatim}}}

This gives the order and labels for each time step, i.e., 3. Processes are sequenced using order within the \command{time\_step} block:

\ifAgeBased
{\small{\begin{verbatim}
@time_step step1
processes Recruitment Fishing

@time_step step2
processes Spawn_migration Fishing

@time_step step3
processes Home_migration Ageing
\end{verbatim}}}
\else
{\small{\begin{verbatim}
			@time_step step1
			processes Recruitment Fishing
			
			@time_step step2
			processes Spawn_migration Fishing
			
			@time_step step3
			processes Home_migration Growth
\end{verbatim}}}
\fi

The \emph{Recruitment}, \emph{Fishing}, \emph{Spawn\_migration}, \emph{Home\_migration} and \ifAgeBased \emph{Ageing} \else \emph{Growth} \fi are all labels of command blocks that defines a process (see Section \ref{sec:Process} for the list of available processes). The order that the  processes are executed is in the same order as specified. The process \emph{Fishing} could be the process type \texttt{Instantaneous\_Mortality} (Section \ref{sec:Process-MortalityInstantaneous}) which takes natural mortality as a parameter as well as specifying the catches in the time-steps, so it is possible to have all catch taken in time-step \emph{step1} with some natural mortality, and no fishing in time-step \emph{step2} where the rest of the natural mortality occurs.

Although the process \emph{Spawn} represents a biological process, spawning, in the \CNAME\ model it is the time that the spawning stock biomass ($SSB$) is calculated since this is needed to calculate recruitment if there is a spawner-recruitment relationship. A related concept is maturity which can be in the partition, so there needs to be a process to transfer immature fish into the mature category, but it is only indirectly related to spawning. Hence, in modelling, spawning is not a process that affects the partition directly, but it the time to calculate the $SSB$ which must be defined as a derived quantity (from the partition). Hence, \emph{Spawn} is located in Figure \ref{Fig:annual}.

To calculate the $SSB$ a \command{derived\_quantity} command block is needed in which the "timing" of the $SSB$ calculation in terms of which time-step and the proportion of natural mortality within it is specified (\ref{sec:DerivedQuantity}).

\subsubsection{\I{The initialisation phases}}\label{sec:Initialisation}

Initialisation is the process of determining the model starting state at the start of the first year (\texttt{Start\_year}. The initial state can be equilibrium/steady state or some other initial state for the model (e.g., exploited), prior to the start year of the model.

There are multiple options for partition initialisation in \CNAME, including

\begin{itemize}
	\item Iterative: run the model for a specified number of years to get the converged state.
	\item Derived: Use the analytical solution (i.e., faster than iterative) for the initial state, but it does not work with some processes (e.g., density-dependent migration)
	\ifAgeBased
	\item Cinitial: Estimate the initial partition's numbers-at-age
	\item state\_category\_by\_age: specify the partition's numbers-at-age
	\else
	%\item Cinitial: Estimate the initial partition's numbers-at-length
    %\item state\_category\_by\_length: specify the partition's numbers-at-length
    \fi
\end{itemize}

Initialisation definitions start with specifying the initialisation label in the \command{model} command block followed by a \command{initialisation\_phase} command block specifying the type and other settings:

{\small{\begin{verbatim}
@model
...     # other subcommands
initialisation_phase int_label

@initialisation_phase int_label
type iterative  #choose one from the list above
...             # specify option values

\end{verbatim}}}

If needed, the processes used and their order in the initialisation are those specified in the annual cycle, but these can by changed by either excluding some processes or including others by using the  \texttt{exclude\_processes} or  \texttt{insert\_processes} subcommands in the \textit{initialisation\_phase} command blocks,

{\small{\begin{verbatim}

@initialisation_phase int_label
type iterative
exclude_processes Fishing
insert_processes step1(recruitment)=initialFishing
            # format: <step>(<insert before process label>)=<new block label>
...         # specify option values

\end{verbatim}}}

where \textit{Fishing} is the normal fishing process which defines natural mortality so when excluded, initialisation can use another value that incorporates some unrecorded fishing before the start of the assessment period by setting natural mortality to a higher value in the process \textit{initialFishing}. The place to insert \textit{initialFishing} is in the time-step labelled \textit{step1} before the process \textit{recruitment} which must be in that time-step (process label is enclosed in brackets). To insert at the end of the time-step use \textit{()}, e.g. \textit{step1()=initialFishing}.

For most models the most common type of initialisation phase to define an initial equilibrium age structure is \subcommand{derived}, whereas for length-based models the only type available is \subcommand{iterative}. Additional initialisation phases can be included by sequencing other phases one after another

{\small{\begin{verbatim}
@model
...     # other subcommands
initialisation_phase int_label int_label2


@initialisation_phase int_label
type derived    #choose one from the list above
...             # specify option values

@initialisation_phase int_label2
type iterative    #choose one from the list above
...             # specify option values

\end{verbatim}}}

which may be faster overall since fewer iterations may be required used in the second phase. The order of applying each initialisation is that given in the \command{model} command block.

The multi-phased initialisation allows for flexibility in the number and type of initialisation processes, for initialising a non-equilibrium starting state, or applying simple processes before applying more complex ones.

In each initialisation phase, the processes defined for that phase are applied and used as the starting point for the following phase or, if it is the last phase, the start year of the model.

The \emph{first} initialisation phase is always initialised with each \ifAgeBased age \else length class \fi and category set to zero. Care must be taken when using complex category inter-relationships or density-dependent processes that depend on a previously calculated state, as they may fail when used in the first phase of an initialisation.

Multi-phase iterations\index{Multi-phase iteration} can also be used to determine if an initialisation has converged. A second initialisation phase can be added for 1 year, with the same processes applied as in the first phase. The state at the end of the first and second phase is then output. If these states are identical, then it is likely that the initialisation has converged to an equilibrium state.

For multi-phase initialisation models, it is advised to include the \command{report} of type \subcommand{initialisation\_partition}. This will print the partition at the end of each initialisation phase, which can be useful for assessing the impact of each phase on the partition.

{\small{\begin{verbatim}
@report initial_partitions
type initialisation_partition
\end{verbatim}}}

\paragraph{\I{Iterative Initialisation}}\label{sec:InitialisationPhase-Iterative}

The \subcommand{iterative} initialisation is a general solution for initialising the model, but can be slow to converge, depending on the model. Its value is that it can work on complex structured models that may be difficult or impossible to implement using analytic approximations.

The number of iterations in the iterative initialisation can increase the model output, and the number of iterations should be chosen to be large enough to allow the population state to fully converge. In an age-based model, a period of about two times the maximum age is recommended to ensure convergence. In length-based models a longer time may be required depending on the nature of the growth increment process. \CNAME\ can report a convergence statistic to assist in determining if adequate convergence has been obtained.

In addition, the iterative initialisation phase can optionally be stopped early if the user-defined convergence criteria is met. For a list of supplied years in the initialisation phase, the convergence criteria is met if the proportional absolute summed difference between the state in year $t-1$ and the state in year $t$ ($\widehat{\lambda}$) is less than the user-defined value of $\lambda$, where

\begin{equation}
  \widehat{\lambda} = \frac{\sum\limits_{i,j}  \left|\text{element}(t)_{i,j} - \text{element}(t-1)_{i,j} \right|}{\sum\limits_{i,j} \frac{}{}\text{element}(t)_{i,j}}
\end{equation}

where $\text{element}(t)_{i,j}$ denotes the numbers at time step $t$ in category $j$ and \ifAgeBased age \else length class \fi $i$.

Hence, for the initialisation define:

\begin{itemize}
  \item The number of initialisation phases,
  \item The number of years in each phase, and
  \item The processes to apply in each phase, where the default processes are those applied in the annual cycle.
\end{itemize}

An example with one initialisation phase:

{\small{\begin{verbatim}
@model
...
initialisation_phases Iterative_initialisation

@initialisation_phase Iterative_initialisation
type iterative
years 50                # do 50 annual cycle iterations
lambda 0.0001
convergence_years 20 40 # test for convergence at 20 and 40 iterations
\end{verbatim}}}

A report on the outcome of the iterative convergence evaluation is available (\command{report} of type \subcommand{initialisation}). This will print the years when convergence was tested and the result of the convergence tests.

\ifAgeBased
\paragraph{\I{Derived Initialisation}}\label{sec:InitialisationPhase-Derived}

The \subcommand{derived} initialisation is an analytical solution that calculates the equilibrium age structure and the plus group using a geometric series solution. The benefit of this method is it can be solved in \texttt{max\_age - min\_age + 1} years or time-steps units, so it is computationally faster than the iterative initialisation phase. Under some process combinations (e.g., one-way migrations) this initialisation does not calculate the exact equilibrium partition. When using this initialisation, users can confirm that the plus group has reached an equilibrium state by either comparing with an iterative initialisation, or by adding a second iterative initialisation phase with a limited number of iterations for comparison.

An example with one initialisation phase:

{\small{\begin{verbatim}
		@model
		...
		initialisation_phases Equilibrium_initialisation

		@initialisation_phase Equilibrium_initialisation
		type derived
\end{verbatim}}}
\fi

When a model is initialised with \subcommand{derived} (for age-based models only) or \subcommand{iterative} (for either age-based or length-based models),  and recruitment is defined by \(B_0\), the model initialises the partition with \(R_0 = 1\). Once the initialisation phase is complete, it scales all the categories defined in each recruitment process by
\[
N_{a,c}  = N_{a,c} \times B_0^R / SSB^R  \ .
\]
where, \(R\) denotes each recruitment block and \(N_{a,c}\) are categories defined in that recruitment block. For this case, it is advised to associate all categories to the recruitment so they are accounted for in this scaling process. If maturity is in the partition, it is not intuitive, but they must be defined in the recruitment dynamic with a proportion set = 0 (for more information on specifying this see Section 
\ifAgeBased \ref{sec:Process-Recruitment} \else \ref{sec:Process-Recruitment} \fi). \CNAME\ will flag a warning if a model doesn't have all categories defined in the available recruitment blocks. A case where this can be ignored is in models with tagged categories, these categories don't exist during initialisation and so don't need to be scaled, and thus can be omitted from the recruitment definition. \CNAME\ will still output a warning for this, but can be ignored if users understand its purpose.

\ifAgeBased
\paragraph{\I{Cinitial Initialisation}}\label{sec:InitialisationPhase-Cinitial} \STATUS{Untested}

The \subcommand{cinitial} initialisation can only be applied after \subcommand{derived} or \subcommand{iterative} initialisation phases. This initialisation can be a method for estimating the non-equilibrium state of population if there is exploitation before observations are collected. The estimated \subcommand{cinitial} factors shift the initial population away from an equilibrium state prior to the start year.

After the first initialisation phase we have an equilibrium age-structure denoted by $N_{equil}$.

\subcommand{Ciniital} specifies an age structure denoted by $N_{cinit}$ (in numbers), but this can be combinations of categories, for example, both sexes by two areas.

$Multiplier =  N_{cinit} / N_{equil}^{combined}$

where $N_{equil}^{combined} $ is summed over the same combined categories as \textit{Cinitial}. Then

$N_{init} =  N_{equil} * Multiplier $

$N_{init}$ is the numbers-at-age by category for the start of the model run.

It would be helpful to include an observation of age composition data for the first year of the model in order to estimate the non-equilibrium population state.

An example with two initialisation phases:

{\small{\begin{verbatim}
		@model
		...
		initialisation_phases Iterative Cinitial

		@initialisation_phase Iterative
		type iterative
		years 10
		lambda 0.0001
		convergence_years 10 20

		@initialisation_phase Cinitial
		type cinitial
		categories spawn.male+nonspawn.male spawn.female+nonspawn.female
		table n
		spawn.male+nonspawn.male     5e7 5e7 7e6 6e6 5e6 4e6 3e6 2e6 1e6 1e6 1e1 1e1 1e1 1e1
		spawn.female+nonspawn.female 5e7 5e7 7e6 6e6 5e6 4e6 3e6 2e6 1e6 1e6 1e1 1e1 1e1 1e1
		end_table
		\end{verbatim}}}

The Cinitial factors can also be estimated with the syntax

{\small{\begin{verbatim}
	@estimate cinit_male
	parameter initialisation_phase[Cinitial].spawn.male+nonspawn.male
	same initialisation_phase[Cinitial].spawn.female+nonspawn.female
	lower_bound  2e2  2e2  2e2  2e2  2e2  2e2  2e2  2e2  2e2  2e2  2e0  2e0  2e0  2e0
	upper_bound  2e9  2e9  2e9  2e9  2e9  2e9  2e9  2e9  2e9  2e9  2e9  2e9  2e9  2e9
	type uniform
	\end{verbatim}}}

\paragraph{\I{State\_category\_by\_age}}\label{sec:InitialisationPhase-StateCategoryByAge}\label{sec:Process-LoadPartition}

The \subcommand{state\_category\_by\_age} initialisation uses a user-defined table as the initial partition numbers-at-age for the beginning of the  start year. Models can be initialised by specifying the numbers-at-age for each category.

An example with one initialisation phase:

{\small{\begin{verbatim}
		@model
		...
		initialisation_phases Fixed

		@initialisation_phase Fixed
		type state_category_by_age
		categories male female
		min_age 3
		max_age 10
		table n
		male   1000 900 800 700 600 500 400 700
		female 1000 900 800 700 600 500 400 700
		end_table
		\end{verbatim}}}

When initialising models with this type, undefined behaviour may result if the model applies processes that require derived quantities to be calculated in the initialisation phase. (e.g., $SSB$ so that recruitment can be calculated for the start year). In the latter case, the user would have to use a subsequent initialisation phase \subcommand{iterative} that has natural mortality set to zero (i.e., \subcommand{insert\_processes} subcommand to introduce zero natural mortality and \subcommand{exclude\_processes} to exclude the mortality process that defines natural mortality) for as many year needed to calculate the $SSB$ values.

\subsubsection{\I{Non-equilibrium initialisation phases}}\label{sec:Initialisation-NonEquilibrium}

This section provides tips and advice for configuring \CNAME\ models to have non-equilibrium age or length structures at the model \subcommand{start\_year}. An equilibrium age or length structure in this context, roughly refers to an age-structure that would result if the annual cycle was repeatedly run with no fishing. For many models this will result in an age or length structure that has an exponential decay from natural mortality and constant recruitment.

As mentioned in the above section, the \subcommand{cinitial} initialisation type can be used to after either \subcommand{derived} or \subcommand{iterative} to produce a non-equilibrium age structure. However, recent simulations have shown difficulties in estimating the parameters from this phase \citep{roberts_dunn_estimate_start_M_init}. Another approach also explored by \cite{roberts_dunn_estimate_start_M_init}, was to start the model \subcommand{max\_age} years before the intended \subcommand{start\_year} and estimate additional recruitment parameters during this phase so that by the intended \subcommand{start\_year}, the age structure would be in a non-equilibrium state.
\fi % currently these are only availble in age-based models.

\ifAgeBased
Another approach that is similar to the \subcommand{cinitial} initialisation type is to first run either \subcommand{derived} or \subcommand{iterative} phase and then to apply a second \subcommand{iterative} phase that has an additional initialisation mortality process.

There are a range of mortality processes that can only be applied during initialisation. These are

\begin{itemize}
	\item \subcommand{mortality\_initialisation\_event} see Section~\ref{sec:Process-MortalityInitialisationEvent}
	\item \subcommand{mortality\_initialisation\_event\_biomass} see Section~\ref{sec:Process-MortalityInitialisationEventBiomass}
	\item \subcommand{mortality\_initialisation\_baranov} see  Section~\ref{sec:Process-MortalityInitialisationBaranov}
\end{itemize}
\fi % end if

\subsection{\I{Population processes}}\label{sec:Process}

\ifAgeBased

Population processes are processes that change the model state. These processes produce changes in the partition by adding or removing individuals, or by moving individuals between ages and/or categories.

Current population processes available include:

\begin{itemize}
\item recruitment\index{Recruitment} (Section~\ref{sec:Process-Recruitment}),
\item ageing\index{Ageing} (Section~\ref{sec:Process-Ageing}),
\item growth\index{Growth} (Section~\ref{sec:AgeLength}),
\item maturation\index{Maturation} (Section~\ref{sec:Process-TransitionCategory}),
\item mortality\index{Mortality} events (e.g., natural and fishing) (Section~\ref{sec:Process-Mortality}), 
\item Markovian movement\index{Markovian movement} which is a specialised version of a category transition processes, 
\item category transition processes\index{Category transition}, i.e., processes that move individuals between categories while preserving their overall age structure (Section~\ref{sec:Process-TransitionCategory}), and
\item tagging and tag loss (Section~\ref{sec:Process-TagByAge}).
\end{itemize}

There are two types of processes: (1) processes that occur across multiple time steps in the annual cycle, e.g., \subcommand{mortality\_constant\_rate} and \subcommand{mortality\_instantaneous}; and (2) processes that occur only within the time step in which they are specified.

\subsubsection{\I{Recruitment}}\label{sec:Process-Recruitment}

Recruitment processes add new individuals to the partition. Recruitment depends on virgin biomass or alternatively recruitment in the virgin state and so these parameters are located in this process (as \textit{b0} and \textit{r0}). The other factors needed are Spawning Stock Biomass ($SSB$) if there is a stock-recruitment relationship and recruitment multipliers which can be standardised to have an arithmetic mean to be 1 over some specified year range (\subcommand{standardise\_years}).

In the recruitment processes, a number of individuals are added to a single age class (subcommand \textit{age}) within the partition, with the number determined by the type of stock-recruitment process specified. If recruits are added to more than one category, then the proportion of recruits to be added to each category is specified by the \argument{proportions} subcommand. For example, if recruiting to categories labelled \texttt{male} and \texttt{female}, then the proportions may be set to $0.5$ and $0.5$, so that half of the recruits are added to the male category and the other half to the female category.

Recruitment can differ between a spawning event or the creation of a cohort/year class. One view for fisheries is that recruitment usually refers to individuals \enquote{recruiting} to a fishery. This definition is used because there is usually not a lot of information/observations on younger age classes between the spawning events and being vulnerable to a survey or fishery for data collection. However, in \CNAME\, recruitment is to a specified age class for one or more categories.

The year offset for an age cohort between spawning and recruitment to the partition is by default automatically generated by \CNAME. This offset is a function of \subcommand{age} specified in the recruitment \command{process} (defaults to the model \subcommand{min\_age}) and sequence of processes within the annual cycle. Users can also supply this offset using the \CNAME\ parameter \texttt{ssb\_offset}. This is analogous to the CASAL parameter \texttt{y\_enter}.

\CNAME\ has the following recruitment processes, constant recruitment\index{Recruitment ! Constant}, the \I{Beverton-Holt stock-recruitment relationship}\index{Recruitment ! Beverton-Holt} \citep{1203}, and the \I{Ricker stock-recruitment relationship}\index{Recruitment ! Ricker}. The  number of individuals following recruitment in year $y$ is

\begin{equation}
N_{y,a,j} \leftarrow N_{y,a - 1,j} + p_j(R_y)
\end{equation}

where $N_{y,a,j}$ is the numbers in year $y$ and category $j$ at age $a$, $p_j$ is the proportion added to category $j$, and $R_y$ is the total number of recruits in year $y$.

\paragraph{\I{Constant recruitment}}\label{sec:Process-RecruitmentConstant}

In the constant recruitment process the total number of recruits added in each year $y$ in age $a$ is $R_y$, with $R_y = R_0$ for all years

\begin{equation}
  R_{y,j} = p_j(R_0)
\end{equation}

Constant recruitment is equivalent to a Beverton-Holt recruitment process with steepness ($h$) set to 1.

For example, to specify a constant recruitment process where individuals are added to the male and female immature categories at $age=1$ in equal proportion (\texttt{proportions} = 0.5), and the number to add is $R_0=5 \times 10^5$, the syntax is

{\small{\begin{verbatim}
	@process Recruitment
	type constant_recruitment
	categories male.immature female.immature
	proportions 0.5 0.5
	r0 500000
	age 1
\end{verbatim}}}

\paragraph{\I{Beverton-Holt and Ricker recruitment}}\label{sec:Process-RecruitmentBevertonHolt}\label{sec:Process-RecruitmentRicker}

In the Beverton-Holt and the Ricker recruitment process the total number of recruits added each year is $R_y$. $R_y$ is the product of the average recruitment $R_0$, the annual recruitment multipliers ($YCS$, also called year class strength), and the stock-recruit relationship $SR(SSB_y)$

\begin{equation}\label{eq:BH}
  R_{y,a,j} = p_j(R_0 \times YCS_{y} \times SR(SSB_{spawn\_year}))
\end{equation}

where

\begin{equation}\label{eq:year_class}
spawn\_year = y - \texttt{ssb\_offset}
\end{equation}

and $a$ is age, $p_j$ is the proportion of recruits to enter category $j$, and \texttt{ssb\_offset} is the number of years lag between spawning and recruitment.

Recruitment refers to recruitment into the population and may differ from the spawning event. See below on more information about \texttt{ssb\_offset}. In general this parameter should not be specified by the user.

$SR(SSB_y)$ in the Beverton-Holt stock-recruit relationship is parametrised by the steepness $h$, and based on \cite{mace_doonan_88} parametrisation

\begin{equation}\label{eq:BH_SR}
SR(SSB_y) = \frac{SSB_y}{B_0} / \left( 1-\frac{5h-1}{4h} \left( 1-\frac{SSB_y}{B_0} \right) \right)
\end{equation}

$SR(SSB_y)$ in the Ricker stock-recruit relationship is also parametrised by the steepness $h$, and based on \cite{mace_doonan_88} parametrisation

\begin{equation}\label{eq:RICKER_SR}
	SR(SSB_y) = \frac{SSB_y}{B_0} \left(\left(\frac{1}{5h}\right) ^ {\frac{5}{4} \left(\frac{SSB}{B_0}-1\right)} \right)
\end{equation}

The Beverton-Holt and Ricker recruitment processes require a value for \Bzero\ (or \Rzero) and $SSB_y$ to calculate the number of recruits. A derived quantity (see Section \ref{sec:DerivedQuantity}) must be defined that provides the annual $SSB_y$ for the recruitment process. \Bzero\ is then defined as the value of the $SSB$ calculated during initialisation. If a model has more than one initialisation phase, the user needs to supply the initialisation phase that calculates \Bzero. This is defined by the command \subcommand{b0\_initialisation\_phase}. \CNAME\ will default to the last initialisation phase if users do not specify this command.

During initialisation, the recruitment multipliers ($YCS$) are assumed to be equal to one, and recruitment that happens in the initialisation phases that occur before and during the phase when \Bzero\ is determined are assumed to have steepness $h=1$ (i.e., in those initialisation phases, recruitment is equal to \Rzero).

Recruitment during the initialisation phases after the phase where \Bzero\ was determined are calculated using the Beverton-Holt stock-recruit relationship. \Rzero\ and \Bzero\ have a direct relationship when there are no density-dependent processes in the annual cycle. Models can thus be initialised using either \Bzero\ or \Rzero.

An example of the specification of a Beverton-Holt recruitment process, where individuals are added to the category \enquote{immature} at $age=1$, and the number added is $R_0=5 \times 10^5$; \subcommand{SSB\_derived\_quantity} is the label for an \command{derived\_quantity} that specifies the defines spawning stock biomass, with \Bzero\ the value derived at the end of the initialisation phase labelled \texttt{phase1}; and $YCS$ are standardised to have mean one in the recruited years 1995 to 2004, and recruits enter into the model two years following spawning. A similar specification is used for the Ricker, but with \commandsubarg{process}{type}{recruitment\_ricker}.

{\small{\begin{verbatim}
	@process Recruitment
	type recruitment_beverton_holt
	categories immature
	proportions 1.0
	r0 500000
	steepness 0.75
	age 1
	recruitment_multipliers 0.65 0.87 1.6 1.13 1.02 0.38 2.65 1.35 1 1 1 1 1
	standardise_years 1995:2004
	ssb SSB_derived_quantity
	b0_initialisation_phase phase1
	
	@derived_quantity SSB_derived_quantity
	...
	
	@initialisation_phase phase1
	...
\end{verbatim}}}

In most instances \subcommand{ssb\_offset} should not be specified; \CNAME\ determines \subcommand{ssb\_offset} by the order of ageing, recruitment, spawning, and the recruitment parameter \subcommand{age}. However, users can override this and specify \subcommand{ssb\_offset}, \CNAME\ will return a warning if this differs to the value it expects.

\begin{itemize}
	\item if the annual time step order is recruitment, ageing, spawning, then \subcommand{ssb\_offset} should equal \subcommand{age} + 1, or
	\item if the annual time step order is spawning, ageing, recruitment, then \subcommand{ssb\_offset} should equal \subcommand{age} - 1, or
	\item \subcommand{ssb\_offset} = \subcommand{age}
\end{itemize}

There may be scenarios where the user will input these values, e.g., if there are multiple ageing processes in the annual cycle. \CNAME\ does not have functionality to accommodate this situation, so in this case \subcommand{ssb\_offset} would be manually defined.

There are two variants of the Beverton-Holt stock recruitment function and they differ in how the recruitment multipliers are parametrised. This parametrisation can either be in natural space as recruitment multipliers ($YCS$), or in log space as recruitment deviations. Due to the difference in terminology, these variants are implemented in two separate processes, \subcommand{type recruitment\_beverton\_holt} and \subcommand{type recruitment\_beverton\_holt\_with\_deviations}, respectively.

Currently, the only parametrisation for the Ricker stock recruitment relationship is in natural space with recruitment multipliers ($YCS$).

\paragraph*{YCS ($YCS_y$)}

The $YCS$ parameter is reference by the recruited year. The recruited year is the year when a year class or age-cohort enter the partition. The recruited year differs from the spawning event year defined in Equation~\eqref{eq:year_class}. This is a shift away from CASALs terminology which used \subcommand{ycs\_year} and is equivalent to the spawning event year. Standardisation years are are also now expressed as recruited years. This will differ from \CNAME\ versions before August 2022 and CASAL models. From August 2022 we deprecated the commands \subcommand{ycs\_values}, \subcommand{ycs\_years}, and \subcommand{standardised\_ycs\_years}. These were replaced with \subcommand{recruitment\_multipliers} and \subcommand{standardise\_years}.

This year reference is important when defining \command{estimate}, \command{project}, and \command{time\_varying} blocks for the \subcommand{recruitment\_multipliers} parameter. An example is at the end of the section.

A common practice when estimating $YCS$ is to standardise using the Haist parametrisation, which was described by V. Haist. \CNAME\ will standardise $YCS$ only if subcommand \subcommand{standardise\_years} is defined. The model parameter \texttt{recruitment\_multipliers} is a vector \textbf{Y}, covering the years from \texttt{start\_year} to \texttt{final\_year}. The resulting standardised recruitment multipliers are calculated as $YCS_i=Y_i/\bar{\textbf{Y}}$, where the mean is calculated over the user-specified years \texttt{standardise\_years}.

An alternative to \enquote{standardisation} is to constrain the $YCS$ parameters using the simplex transformation (see Section~\ref{sec:Transformation-Simplex}). This is thought to have estimation benefits over the \enquote{standardisation} as priors can be applied to the \enquote{free} (estimable) parameters (\(Y_i\)).

\[
YCS_i =
\begin{cases}
Y_i / mean_{y \in S}(Y_y) & :y \in S\\
Y_i					 & :y \notin S
\end{cases}
\]

where S is the set of years from \texttt{standardise\_years}. One effect of this parametrisation is that \Rzero\ is then defined as the mean estimated recruitment over the set of years $S$, because the mean $YCS$ multiplier over these years will always be one.

Typically, \texttt{standardise\_years} is defined to span the years over which $YCS$ is reasonably well estimated. For years that are not well estimated, $Y_y$ can be set to 1 for some or all years $y\in S$ (which is equivalent to forcing $R_y$ = \Rzero\ x $SR(SSB_y)$) by setting the lower and upper bounds of these $Y$ values to 1. An exception to this might occur for the most recent $YCS$ values, which the user may estimate but not include in the definition of \Rzero\ (because the estimates may be based on too few data). One or more years may be excluded from the range of years for the averaging process of the Haist parametrisation.

The advantage of the Haist parametrisation is that a large penalty is not necessary to force the mean of the $YCS$ parameter to be 1, although a small penalty should still be used to stop the mean of \textbf{Y} from drifting. These adjustments may improve MCMC performance. Projected $YCS$ values are not affected by this feature. A disadvantage with this parametrisation in a Bayesian analysis is that the prior applies to $Y$, not $YCS$.

In the  example given above, $YCS$ are standardised to have mean one in the period 1995 to 2004, and recruits enter into the model two years following spawning

{\small{\begin{verbatim}
	@process Recruitment
	type recruitment_beverton_holt
	...            #subcommand above
	standardise_years 1995:2004
	recruitment_multipliers 0.65 0.87 1.6 1.13 1.02 0.38 2.65 1.35 1 1 1 1 1
\end{verbatim}}}

\paragraph*{Recruitment deviations, $\epsilon_y$ (\emph{type recruitment\_beverton\_holt\_with\_deviations})} \label{sec:Process-RecruitmentBevertonHoltWithDeviations} 

Recruitment deviations represent the stock-recruitment relationship multipliers in log space, with the link between $YCS_y$ and $\epsilon_y$ as

\begin{equation}\label{eq:recruit_devs}
	YCS_y = exp(\epsilon_y - b_y\sigma^2_R / 2)
\end{equation}

where $\epsilon_y\sim N(0,\sigma^2_R)$, $\sigma^2_R$ is the variance of the stock-recruitment residuals, and $b_y$ is a bias correction defined by \cite{methot2011adjusting}

\begin{equation}\label{eq::bias}
b_y = \left\{\begin{array}{lr}
0, & \text{for }y\leq y_1^b\\
b_{max}(1 - \frac{y - y_1^b}{y_2^b - y_1^b}), & \text{for } y_1^b < y < y_2^b\\
b_{max}, & \text{for } y_2^b\leq y \leq y_3^b\\
b_{max}(1 - \frac{y_3^b - y}{y_4^b - y_3^b}), & \text{for }  y_3^b< y < y_4^b\\
0, & \text{for } y_4^b\leq y
\end{array}\right\}
\end{equation}

The $\epsilon_y$ values are normally distributed in log space and thus lognormal when back-transformed to the resulting stock-recruitment relationship $YCS_y$. Recent work has found that this transformation does not technically lead to the \textit{a priori} assumption that the resulting $YCS_y$ are lognormal. \TODO{See Appendix investigating-two-options-for-ycs-prior-distribution-formulations for more details.}

The ramp function described above for the bias correction has the additional subcommands controlling the ramp

\begin{itemize}
	\item $y_1^b = $ \subcommand{last\_year\_with\_no\_bias}
	\item $y_2^b = $ \subcommand{first\_year\_with\_bias}
	\item $y_3^b = $ \subcommand{last\_year\_with\_bias}
	\item $y_4^b = $ \subcommand{first\_recent\_year\_with\_no\_bias}
	\item $b_{max} = $ \subcommand{b\_max}
\end{itemize}

{\small{\begin{verbatim}
@process Recruitment
type recruitment_beverton_holt_with_deviations
categories immature
proportions 1.0
r0 500000
last_year_with_no_bias 1940
first_year_with_bias 1950
last_year_with_bias 2016
first_recent_year_with_no_bias 2018
b_max 0.85
b0_initialisation_phase phase1
steepness 0.75
age 1
ssb SSB_derived_quantity
deviation_values 0 -0.2 0.4 0 0 0 0 0 0 0 0 0 0
\end{verbatim}}}

\subcommand{deviation\_values} are reference by the recruited year i.e., the  process \subcommand{process[Recruitment].deviation\_values\{1990\}} references the multiplier applied to the recruitment process in model year 1990.

\paragraph*{Recruitment when modelling two stocks (or species)}

To specify a Beverton-Holt recruitment for each stock, the information required is:

\begin{enumerate}
	\item $YCS$, starting from year \subcommand{start\_year} and extending up to year \subcommand{final\_year}
	\item the value of \subcommand{ssb\_offset} and \subcommand{age} (which equate to the \subcommand{y\_enter} in CASAL)
	\item the steepness parameter \subcommand{h}
	\item in a multi category model, the proportion of recruits for each category
	\item a label for the derived quantity
\end{enumerate}

When an \command{initialisation\_phase} (Section~\ref{sec:Initialisation}) type = \subcommand{derived} is specified and the recruitment is defined by \subcommand{b0}, then all categories must be specified in the \command{recruitment} block. Usually in a recruitment processes only the categories that receive recruits need to be defined. For example, a population has a spawning area that is different from the area where recruits enter the population. An area-specific model could then be specified which contains spawning categories and recruiting categories. The recruiting categories would be specified in the subcommand \subcommand{categories}, as these would be the categories receiving recruits.

If \command{initialisation\_phase}, \subcommand{type=derived} is used, then all categories that are a part of that recruitment process need to be specified as well. For example,

{\small{\begin{verbatim}
@process Recruitment_stock1
type recruitment_beverton_holt
categories stock1.immature.M stock1.immature.female stock1.spawn.male
           stock1.spawn.female
proportions 0.5 0.5 0.0 0.0
r0 500000
ssb SSB1
....
\end{verbatim}}}

{\small{\begin{verbatim}
@process Recruitment_stock2
type recruitment_beverton_holt
categories stock2.immature.male stock2.immature.female stock2.spawn.male 
           stock2.spawn.female
proportions 0.5 0.5 0.0 0.0
r0 200000
ssb SSB2
....
\end{verbatim}}}

The \texttt{proportions = 0.0} for \enquote{spawn.male} and \enquote{spawn.female} are needed due to the way the derived initialisation phase works. The derived initialisation finds a solution for when \subcommand{r0} = 1.0 based on an infinite geometric series for the plus group, and scales the initial partition by \subcommand{r0}. Thus, if all categories are not specified, then those that are missed would not be initialised to true values and this could lead to inaccurate model outputs. This set-up extends to multiple-stock fisheries model configurations as well, where all of the categories that make up the stock need to be listed.

\subsubsection{\I{Ageing}\label{sec:Process-Ageing}}

The ageing process "ages" individuals, i.e., this process moves all individuals in the named categories $j$ from one age class $a$ to age class $a + 1$, or accumulates them if the last age class is a plus group.

The ageing process is defined as,
\begin{equation}
  \text{element}(a + 1,j) \leftarrow \text{element}(a,j)
\end{equation}

except in the case of the plus group (if defined),
\begin{equation}
  \text{element}(a_{\text{max}}, j) \leftarrow \text{element}(a_{\text{max}}, j) + \text{element}(a_{\text{max}-1}, j).
\end{equation}

For example, to apply ageing to the categories \texttt{immature} and \texttt{mature}, the syntax is

{\small{\begin{verbatim}
	@process Ageing
	type ageing
	categories immature mature
	\end{verbatim}}}

Note: the ageing process is \emph{NOT} applied by \CNAME\ by default --- it needs to be explicitly specified. As with all other processes, \CNAME\ will not apply a process unless it is defined and specified within the annual cycle. Hence, it is possible to specify a model where a category is not aged. \emph{\CNAME\ will not check or otherwise warn if there is a category defined where ageing is not applied.}

\subsubsection{\I{Mortality}\label{sec:Process-Mortality}}

There are several types of mortality processes available in \CNAME, including tag related processes that can also cause mortality:

\begin{itemize}
	\item constant mortality rate,
    \item constant survival rate,
	\item constant exploitation,
	\item event mortality,
	\item biomass-event mortality,
	\item disease mortality
	\item instantaneous mortality,
	\item instantaneous retained (discards) mortality,
	\item Holling mortality,
	\item initialisation,
	\item a density-dependent relationship based on prey suitability,
	\item tag-release by age,
	\item tag-release by length, and	
	\item tag-loss
\end{itemize}

These processes remove individuals from the partition, either as a rate, as a total number (abundance), as a biomass of individuals or, as a combination of these. \CNAME\ does not (yet) implement the Baranov catch equation. However, instantaneous mortality is considered an approximation to the Baranov catch equation.

To apply both natural and biomass-event mortality, the mortality type \texttt{mortality\_instantaneous} can be specified. Or, you can use \texttt{mortality\_instantaneous\_retained}, where discards are allowed. Mortality blocks are special because they allow for both natural mortality and fishing mortality to be applied at the same time. Note that all mortality processes occur within the mortality block of a time step. See Section~\ref{sec:MortalityBlockObservations} for more information and definitions on mortality blocks.

\textbf{\I{Timing evaluation interval}}

The timing evaluation interval is the timing of the point when observations are fitted or derived quantities are evaluated \ref{sec:MortalityBlockObservations}\label{sec:TimingEvaluationInterval}

Observations (see Section~\ref{sec:Observation})  and derived quantities (see Section~\ref{sec:DerivedQuantity}) need a concept called a \emph{timing evaluation interval} so that the "time" within a year can be specified for their fit or evaluation. This interval is intimately tied into mortality processes.

There can be one or more mortality processes specified within a time-step, but these must be grouped sequentially, i.e., there cannot be a non-mortality process between any two mortality processes within any one time step. The sequence of mortality processes is called a \emph{timing evaluation interval}. If no mortality processes occurs in a time step, then the \emph{timing evaluation interval} is defined to occur at the end of the time step, i.e., it is a virtual, unspecified,  process. Thus  each time step has one \emph{timing evaluation interval}.

\CNAME\ will output an error if more than one \textit{timing evaluation interval} occurs in a single time step.

The "time" for an observation or derived quantity is based on the proportion of mortality that has occurred within the \textit{timing evaluation interval}. The starting and ending partition are saved so that a partition can be estimated by  interpolation between the start and end partitions.

For example, the point of calculation can be set to a point  when 75 \% of the deaths from natural mortality plus catch has occurred The partition at this point is based on interpolating between the start and end of the interval as the partition is known at those points.  Two  methods are available: \texttt{weighted\_sum} and \texttt{weighted\_product}, and are defined as

\begin{itemize}
	\item \texttt{weighted\_sum}: after proportion $p$ through the mortality block, the partition elements are given by $n_{p,j} = (1 - p)n_j + p'_j$

	\item \texttt{weighted\_product}: after proportion $p$ through the mortality block, the partition elements are given by $n_{p,j} = n_j^{1-p} n'^p_j$
\end{itemize}

where $n_{p,j}$ is the derived quantity at proportion $p$ of the mortality block for category $j$, $n_j$ is the quantity at the beginning of the mortality block, and $n'_j$ is the quantity at the end of the mortality block.

In the case of a virtual \textit{timing evaluation interval}, the partition at the end of the time-step is used.

\TODO{REDO FIGURE TO REFLECT TEI rather than mortality blocks; have two mortality processes in step 2}

\begin{figure}[H]
	\centering
	\includegraphics[scale=0.5]{Figures/annual_cycle.jpg}
	\caption{A example sequence for an annual cycle.}\label{Fig:annual2}
\end{figure}


\TODO{GO OVER M and specifying M-by-age HERE FOR ALL M-BASED PROCESSES}
\TODO{MAX U rate specification + Penalities}

\paragraph{Constant mortality rate}\label{sec:Process-MortalityConstantRate} 

To specify a constant annual mortality rate \index{Constant mortality}(e.g. $M=0.2$) for categories "male" and "female"

{\small{\begin{verbatim}
# A process with label NaturalMortality
@process NaturalMortality
type          mortality_constant_rate
categories    male female
# effectively age related mortality
relative_m_by_age One One
m             0.2 0.2
\end{verbatim}}}

The total number of individuals removed from a category

\begin{equation}
D_{j,t} = \sum_a N_{a,j,t} [1 - \exp(-S_{a,j} M_{a,j} p_t)]
\end{equation}

where $D_{j,t}$ is the total number of deaths in category $j$ in time step $t$, $N_{a,j,t}$ is the number of individuals in category $j$ of age $a$ in time step $t$, $S_{a,j}$ is the selectivity value for age $a$ in category $j$, $M_{a,j}$ is the mortality rate for category $j$ for age $a$, and $p_t$ is the proportion of the mortality rate to apply in time step $t$.

The mortality rate process requires the specification of the mortality-by-age curve which is specified using a selectivity. To apply the same mortality rate over all age classes in a category, use a selectivity defined as $S_{a,j}=1.0$ for all ages $a$ in category $j$

{\small{\begin{verbatim}
@selectivity One
type constant
c 1
\end{verbatim}}}

Age-specific mortality rates can also be applied. For example, the hypothesis that mortality is higher for younger and older individuals and lowest when individuals are at their optimal fitness could be defined by using a double exponential selectivity (see Section~\ref{sec:Selectivity})

{\small{\begin{verbatim}
@selectivity age_specific_M
type double_exponential
x0 7.06524
x1 1
x2 17
y0 0.182154
y1 1.43768
y2 1.57169
alpha 1.0

@process      NaturalMortalityByAge
type          mortality_constant_rate
categories    male female
relative_m_by_age age_specific_M age_specific_M
m             1.0 1.0
\end{verbatim}}}

\TODO{INSERT FIG OF M-by-age}

In this definition \subcommand{m} is set to 1.0 and the rate is described through the selectivity. Otherwise, $M_{age} = S_{age} * m$. This concept can be constructed similarly for other mortality methods such as \subcommand{instantaneous\_mortality}.

\paragraph{Constant survival rate}\label{sec:Process-SurvivalConstantRate} 

To specify a constant annual survival rate \index{Constant survival}(e.g. $M=0.2$) for categories "male" and "female"

{\small{\begin{verbatim}
			# A process with label Survival
			@process survival
			type          survival_constant_rate
			categories    male female
			# effectively age related mortality
			relative_m_by_age One One
			s             0.8 0.8
\end{verbatim}}}

The total number of individuals that remain in a category

\begin{equation}
	D_{j,t} = \sum_a N_{a,j,t} [1 - \exp(-S_{a,j} (1-s_{a,j}) p_t)]
\end{equation}

where $D_{j,t}$ is the total number of deaths in category $j$ in time step $t$, $N_{a,j,t}$ is the number of individuals in category $j$ of age $a$ in time step $t$, $S_{a,j}$ is the selectivity value for age $a$ in category $j$, $s_{a,j}$ is the survival rate for category $j$ for age $a$, and $p_t$ is the proportion of the survival rate to apply in time step $t$.

The survival rate process requires the specification of the survival-by-age curve which is specified using a selectivity. To apply the same survival rate over all age classes in a category, use a selectivity defined as $S_{a,j}=1.0$ for all ages $a$ in category $j$

{\small{\begin{verbatim}
			@selectivity One
			type constant
			c 1
\end{verbatim}}}

\paragraph{Constant exploitation rate}\label{sec:Process-MortalityConstantExploitation} 

To specify a constant annual exploitation rate \index{Constant exploitation}(e.g. $U=0.2$) for categories "male" and "female"

{\small{\begin{verbatim}
			# A process with label IncidentalMortality
			@process IncidentalMortality
			type          mortality_constant_exploitation
			categories    male female
			# effectively age related mortality
			relative_u_by_age One One
			u             0.2 0.2
\end{verbatim}}}

The total number of individuals removed from a category

\begin{equation}
	D_{j,t} = \sum_a N_{a,j,t} S_{a,j} U_{a,j} p_t
\end{equation}

where $D_{j,t}$ is the total number of removals from category $j$ in time step $t$, $N_{a,j,t}$ is the number of individuals in category $j$ of age $a$ in time step $t$, $S_{a,j}$ is the selectivity value for age $a$ in category $j$, $U_{a,j}$ is the exploitation rate for category $j$ for age $a$, and $p_t$ is the proportion of the exploitation rate to apply in time step $t$.

The exploitation rate process requires the specification of the mortality-by-age curve which is specified using a selectivity. To apply the same rate over all age classes in a category, use a selectivity defined as $S_{a,j}=1.0$ for all ages $a$ in category $j$

{\small{\begin{verbatim}
			@selectivity One
			type constant
			c 1
\end{verbatim}}}

Age-specific exploitation rates can also be applied.

\paragraph{Disease mortality rate}\label{sec:Process-Age-DiseaseMortalityRate}\label{sec:Process-MortalityDiseaseRate}

Disease mortality is a special, additional, mortality that is implemented to occur after natural and fishing mortality during a time step. This process removes individuals from the partition, is applied to all areas, and can depend on sex/age class.

The partition is updated as follows
\begin{equation}
	n'_{c,j} = n_{c,j}  exp\{-t_y M_{c} S_{c,j} \}
\end{equation}

where \(n_{c,j}\) is the partition for category \(c\) and age class \(j\) before mortality, and \(n'_{c,j}\)  is after the process. \(t_y\) is an annual multiplicative scalar (estimable), \(M_{c}\) is the category specific mortality rate and \(S_{c,j}\) is the selectivity.

{\small{\begin{verbatim}
			@process DiseaseMortality
			type mortality_disease_rate
			disease_mortality_rate 1.0
			selectivities DiseaseSel 
			categories OYS
			year_effect 0.05 0.11 0.39 0.38 0.20 
			years 2000 2001 2002 2003 2004 2005 
\end{verbatim}}}

\paragraph{Event and biomass-event mortality}\label{sec:Process-MortalityEvent}\label{sec:Process-MortalityEventBiomass} \STATUS{Untested?}

\TODO{WHEN NOT DOING M and FISHING AT THE SAME TIME}

The event mortality\index{Event mortality} and biomass-event mortality\index{Biomass-event mortality} processes are applied in a similar manner, except that they remove a specified abundance (number of individuals) or biomass, respectively. These mortality processes can be used to define mortality events where the numbers of removals are known, e.g., fishing, rather than applying mortality as a rate.

In these cases, the abundance or biomass removed is also constrained by a maximum exploitation rate. \CNAME\ removes as many individuals or as much biomass as possible,  while not exceeding the maximum exploitation rate.

Event mortality processes require a penalty to avoid estimating parameter values that will not allow the defined number of individuals to be removed. The model penalises those parameter estimates that result in an too low a number of individuals in the defined categories (after applying selectivities) to allow for removals at the maximum exploitation rate, with a similar penalty for biomass. See Section \ref{sec:Penalty} for more information on how to specify penalties.

The event mortality applied to user-defined categories $i$, with the numbers removed at age $j$ determined by a selectivity-at-age $S_j$:

First, calculate the vulnerable abundance for each category $j$ in $1 \ldots J$ for ages $a = 1 \ldots A$ that are subject to event mortality

\begin{equation}
  V_{a,j} = S_{a,j} N_{a,j}
\end{equation}

and define the total vulnerable abundance $V_{total}$ as

\begin{equation}
  V_{total}  = \sum\limits_j {\sum\limits_a {V_{a,j}}}
\end{equation}

The exploitation rate\index{Maximum exploitation rate} to apply is

\begin{equation}
U = \begin{cases}
  C/V_{total}, & \text{if $C/V_{total} \leq U_{max}$} \\
  U_{max}, & \text{otherwise}\\
  \end{cases}
\end{equation}

The number removed $R_{a,j}$ from each age $a$ in category $j$ is,

\begin{equation}
  R_{a,j} = U V_{a,j}
\end{equation}

For example, to specify an \textbf{abundance-based} fishing mortality process with catches given for a set of specific years over categories "immature" and "mature", with selectivity "FishingSel", and assuming a maximum possible exploitation rate of 0.7, the syntax is

{\small{\begin{verbatim}
	@process     Fishing
	type          event_mortality
	categories    immature mature
	years         2000 2001 2002 2003
	U_max         0.70
	selectivities FishingSel FishingSel
	penalty       event_mortality_penalty
	\end{verbatim}}}

and specified similarly for a \textbf{biomass-based} fishing mortality process

{\small{\begin{verbatim}
		@process      Fishing
		type          mortality_event_biomass
		categories    immature mature
		years         2000 2001 2002 2003
		U_max         0.70
		selectivities FishingSel FishingSel
		penalty      event_mortality_penalty
		\end{verbatim}}}

\paragraph{Instantaneous mortality}\label{sec:Process-MortalityInstantaneous}

The instantaneous mortality process\index{Instantaneous mortality} combines both natural mortality and fishing exploitation into a single process. This allows the simultaneous application of both natural mortality and anthropogenic mortality to occur across multiple time steps. This process accounts for half the natural mortality within a time step before calculating vulnerable biomasses for calculating exploitation rates. The remaining half of the natural mortality is taken after exploitation has been accounted for. The input for this process is removals (either catches in either biomass or abundance, or an exploitation rate of either biomass or abundance). In fisheries models in \CNAME\ this is the most commonly used mortality process.

This process allows for multiple removal events, e.g., a fisheries model with multiple fisheries and/or fleets. A removal method can occur in one time step only, although multiple removals can be defined to cover events during the year.

The equations for instantaneous mortality are based on Pope's discrete catch equation, which assumes catch is known without error. \CNAME\ will try and take the exact catch specified in the input.

\begin{itemize}
	\item An exploitation rate (actually a proportion) is calculated for each fishery, as the catch divided by the selected-and-retained abundance or biomass termed vulnerable biomass. Vulnerable biomass is calculated by accounting for half natural mortality (\(M_{a,c}\)) that occurs at time-step which is defined by the subcommand \subcommand{time\_step\_proportions} and denoted by \(p_t\),
	$$U_{f} = \frac{C_f}{\sum\limits_{c}\sum\limits_a \bar{w}_{a,c} S_{f,a,c} n_{a,c} exp(-0.5 p_t M_{a,c})} \ ,$$
	where \(S_{f,a,c}\) is the fishery selectivity for age \(a\) and category \(c\), \(\bar{w}_{a,c}\) is mean weight and \(n_{a,c}\) numbers at age before applying fishing. The categories \(c\) are user defined for each fishery \(f\), which are defined in the \subcommand{table method} (see below for an example).
	\item The fishing pressure associated with method $f$ is defined as the maximum proportion of fish taken from any element of the partition in the area affected by the method $f$
	$$ U_{f,obs} = max_{a,c}(\sum\limits_k\sum\limits_c S_{k,a,c} U_k) $$
	where the maximum is over all partition elements (age and categories) affected by fishery $f$, and the summation is over all fisheries $k$ which affect these partition elements in the same time step as fishery $f$.

	In cases with a single fishery the fishing pressure will be equal to the exploitation rate (i.e., $U_{f,obs} = U_f$), but can be different if: (a) there is another removal method operating in the same time step as removal method $f$ and affecting some of the same partition elements, and/or (b) the selectivity $S_{f,a}$ does not have a maximum value of 1.

	There is a maximum mortality pressure limit of $U_{f,max}$ for each method of removal $f$. So, no more than proportion $U_{f,max}$ can be taken from any element of the partition affected by removal method $f$ in that time step. Clearly, $0 \leq U_{max} \leq 1$. It is an error if two removal methods, which affect the same partition elements in the same time step, do not have the same $U_{max}$.

	For each $f$, if $U_{f,obs} > U_{f,max}$, then $U_f$ is multiplied by $U_{f,max}/U_{f,obs}$ and the mortality pressures are recalculated. In this case the catch actually taken from the population in the model will differ from the specified catch, $C_f$.

	\item The partition is updated using
		$$ n'_{a,c} = n_{a,c} exp(-p_t M_{a,c})\big[1 - \sum_f S_{f,a,c} U_{f} \big] $$
\end{itemize}

For example, to apply natural mortality of $0.20$ across three time steps on both male and female categories, with two methods of removals (fisheries \texttt{FishingWest} and \texttt{FishingEast}) and their respective catches (kg) known for years 1975:1977 (the catches are given in the \texttt{catches} table and information on selectivities, penalties, and maximum exploitation rates are given in the \texttt{method} table), the syntax is

{\small{\begin{verbatim}
	@process instant_mort
	type mortality_instantaneous
	m 0.20
	time_step_proportions 0.42 0.25 0.33
	relative_m_by_age One
	categories male female
	units kgs

	table catches
	year FishingWest FishingEast
	1975	80000	111000
	1976	152000	336000
	1977	74000	1214000
	end table

	table method
	method      category selectivity u_max time_step penalty      biomass U
	FishingWest stock    westFSel    0.7   step1     CatchPenalty true    true
	FishingEast stock    eastFSel    0.7   step1     CatchPenalty true    true
	end_table
	\end{verbatim}}}

For catches specified as removals as abundance, the column \argument{biomass} in the \argument{methods} table is set to \argument{false}. Exploitation rates can be specified (in which case, the column of catches is not a U) with valid range from $0-1$), by setting the \argument{method} table column \argument{U} to \argument{true}.

The columns \argument{biomass} and \argument{U} are optional, and if omitted, default to \argument{biomass=true} (i.e., the catches are in biomass not abundance) and \argument{U=false} (i.e., the removals are the amount, not an exploitation rate) respectively. 

For referencing catch parameters for use in projecting, time-varying, and estimating, the syntax is
{\small{\begin{verbatim}
		parameter process[mortality_instantaneous].method_"method_label"{2018}
\end{verbatim}}}

where \subcommand{"method\_label"} is the label from the \subcommand{catch} or \subcommand{method} table and continuing the example,

{\small{\begin{verbatim}
		parameter process[instant_mort].method_FishingWest{2018}
\end{verbatim}}}

To calculate weight by empirical weight-at-age matrices as described in Section~\ref{sec:AgeWeight}, the method table would include an additional column to reference weight-at-age objects:

{\small{\begin{verbatim}
		@age_weight jan_weight_at_age
		type data
		table data
		year 1 		2 		3 		4
		1980 3.4	5.6		7.23 	8.123
		end_table

		table method
		method      category selectivity u_max time_step penalty      age_weight
		FishingWest stock    westFSel    0.7   step1     CatchPenalty jan_weight_at_age
		FishingEast stock    eastFSel    0.7   step1     CatchPenalty jan_weight_at_age
		end_table
\end{verbatim}}}


\paragraph{Instantaneous mortality with retained catch and discards}\label{sec:Process-MortalityInstantaneousRetained}

The instantaneous mortality retained process\index{Instantaneous mortality retained} builds on the instantaneous mortality process (\ref{sec:Process-MortalityInstantaneous}) which has simultaneous applications of fishing and natural mortality, but with all catch-at-sea being landed, i.e., no discarding. The process \texttt{mortality\_instantaneous\_retained} allows for retained catch, discards, and also a mortality to be applied to discards, i.e., some are allowed to survive. The method for taking catch from the partition and the constraints used are the same as in \texttt{mortality\_instantaneous}.

This process was implemented to address issues with the pot fishery for blue cod which has a minimum legal size and so some catch is discarded at sea and some of these discards are expected to survive (based on some experimental work). There are length data taken at sea, so the total catch selectivity can be estimated, and length and age data taken from the landed catch (retained), so the retention selectivity can also be estimated.

In this mortality process, discard mortality is specified by defining a selectivity to represent mortality by age or length (e.g., constant or asymptotic descending logistic).  This discard selectivity is not be estimated since there is no observation class associated with it. If discard mortality is not provided, it is assumed that all discards die. Landed catch, and both the retained and total catch selectivities must be specified.

Extending the example shown in instantaneous mortality process (\ref{sec:Process-MortalityInstantaneous}) to use retained weight instead of catch, the commands are:

{\small{\begin{verbatim}
@process FishingRetainedCatch
type mortality_instantaneous_retained
# natural mortality
m 0.20
# the ratio of natural mortality in each of the three time steps
time_step_proportions 0.42 0.25 0.33
relative_m_by_age One
#for natural mortality by age
categories male female
units kgs

table catches
# two fisheries, West and East
year FishingWest FishingEast
# the catches are now landed catch
1975 80000 111000
1976 152000 336000
1977 74000 1214000
end table

table method
# all discards die
method      category selectivity retained_selectivity u_max time_step penalty
FishingWest stock    westFSel    westRetainedSel      0.7   step1     CatchPenalty
FishingEast stock    eastFSel    eastRetainedSel      0.7   step1     CatchPenalty
end_table
\end{verbatim}}}

If discard mortality is less than 1.0, use:

{\small{\begin{verbatim}
table method
# 50% discard mortality
method category selectivity retained_selectivity discard_mortality u_max 
       time_step penalty
FishingWest stock westFSel westRetainedSel DisMort 0.7 step1 CatchPenalty
FishingEast stock eastFSel eastRetainedSel DisMort 0.7 step1 CatchPenalty
end_table

@selectivity DisMort
Type constant
# 50% mortality of discards
c 0.5
\end{verbatim}}}

See the instantaneous mortality process (\ref{sec:Process-MortalityInstantaneous}) for referencing catch parameters and calculating weight using empirical weight-at-age matrices.

The report outputs total catch, actual landed catch, and discards, without and with discard mortality:

{\small{\begin{verbatim}
@report Mortality
type process
process Instantaneous_Mortality_Retained
\end{verbatim}}}

\TODO{redo notation, as it is not consistent with that above, e.g., $S_{a,j}$ and $R_y$}

In the following, fisheries are indexed by $f$, and $a$ indexes both age and category combinations.

The total catch is found by applying a selectivity, $S_{f,a}$, in the same way as in the instantaneous mortality process. Retention, $R_{f,a}$, is defined by specifying a selectivity, which can be a function of length or age. The retained catch is the product of these two values, $R_{f,a} * S_{f,a}$. If sex is in the partition, then there are potentially two retention curves, one for each sex.

In general, there is a retention curve for each category in the partition. This property does not apply to surveys. Discard mortality is also specified as a selectivity, $D_{f,a}$. The fraction of dead fish from fishing activity is $S_{f,a} * [ R_{f,a} + (1.0 - R_{f,a}) * D_{f,a} ]$. If $D_{f,a}$ is 1.0, then all selected fish are dead, and if it is 0.0, then only the retained fish are dead.

The equations for the \texttt{mortality\_instantaneous\_retained} process:

\begin{itemize}
    \item Total catch (catch-on-board), $C_{f}$, is calculated by (retained catch) * VF / VR, where VF is vulnerable retained biomass, $j$ indexes categories and $t$ is the proportion of M in the time step, and VF is the full vulnerable biomass, $VF = \sum_{a,j} \overline{w}_{a} S_{a,j} n_{a,j} \exp(-0.5 t M_{a,j})$.

    \item An exploitation rate (actually a proportion) is calculated for each fishery, as the total catch (retained + discards) divided by the selected biomass (VF above) using selectivity $S_{f,a}$,
        $$ U_f = \frac{C_f}{\sum_a \bar{w}_a S_{f,a} n_a \exp(-0.5 t M_{a})}$$

    \item The mortality pressure associated with method $f$ is defined as the maximum proportion of fish taken from any element of the partition in the area affected by the method $f$,
        $$ U_{f,obs} = max_a (\sum_k S_{k,a} U_k) $$
    where the maximum is over all partition elements affected by fishery $f$, and the summation is over all methods $k$ which affect the $j$th partition element in the same time step as fishery $f$.

    In most cases the mortality pressure will be equal to the exploitation rate (i.e., $U_{f,obs} = U_f$), but can be different if: (a) there is another removal method operating in the same time step as removal method $f$ and affecting some of the same partition elements, and/or (b) the selectivity $S_{f,a}$ does not have a maximum value of 1.

    There is a maximum mortality pressure limit of $U_{f,max}$ for each method of removal $f$. So, no more than proportion $U_{f,max}$ can be taken from any element of the partition affected by removal method $f$ in that time step. Clearly, $0 \leq U_{max} \leq 1$. It is an error if two removal methods, which affect the same partition elements in the same time step, do not have the same $U_max$.

    For each $f$, if $U_{f,obs} > U_{f,max}$, then $U_f$ is multiplied by $U_{f,max}/U_{f,obs}$ and the mortality pressures are recalculated. In this case the catch actually taken from the population in the model will differ from the specified catch, $C_f$.

    \item Discard numbers-at-age (including their share of natural mortality) is $S_{a,j} (1 - R_{a,j}) n_{a,j} \exp(-0.5 t M_{a,j})$, and those that die at the end of the time step (updating the partition) are $D_{a,j} S_{a,j} (1 - R_{a,j}) n_{a,j} \exp(-t M_{a,j})$, where $D_{f,a}$ is the fraction that die on return to the sea.

    \item The partition is updated by removing landed catch, natural mortality, and discard mortality
        $$ n'_{a} = n_{a} \exp(-t M_a) \big[ 1 - \sum_f S_{f,a} U_f (R_{f,a} + D_{f,a} (1 - R_{f,a})) \big] $$
\end{itemize}

\paragraph{\I{Mortality Hybrid}}\label{sec:Process-MortalityHybrid} 

The hybrid fishing mortality process in \CNAME\ uses the methods and algorithms applied in Stock Synthesis \citep{methot2013stock}. The descriptions below are heavily based on the text describing this approach in Appendix of \cite{methot2013stock}.

This process begins by calculating popes discrete approximation (same as \subcommand{mortality\_instantaneous} (Section~\ref{sec:Process-MortalityInstantaneous})), and then converts this to Baranov fishing mortality coefficients. A tuning algorithm is then done to tune these coefficients to match input catch nearly exactly, rather than the full Baranov approach.

Total mortality (\(Z_{t,c,a}\)) is calculated as

\begin{equation*}
	Z_{t,c,a} = M_{c,a,t} + \sum\limits_{f} S_{f,c,a} F_{f,t}
\end{equation*}
%
where, \( M_{c,a,t}\) is the natural mortality rate, \( F_{f,t}\) is fishing mortality and \(S_{f,c,a}\) is the selectivity for category \(c\), age \(a\), time-step \(t\) and fishery \(f\). In the context of this description \(t\) denotes both year and time-step.

The hybrid fishing mortality method allows the \(F\) values to be tuned to match input catch nearly exactly, rather than full model parameters. The process begins by calculating the mid time-step exploitation rate using Pope’s approximation. This exploitation rate is then converted to an approximation of the Baranov continuous \(F\). The \(F\) values for all fisheries operating in that time-step are then tuned over a set number of iterations (\subcommand{f\_iterations}) to match the observed catch for each fishery with its corresponding \(F\). Differentiability is achieved by the use of Pope's approximation to obtain the starting value for each \(F\) and then the use of a fixed number of tuning iterations, typically 4. Tests from Stock Synthesis have shown that modelling \(F\) as hybrid versus \(F\) as a parameter has trivial impact on the estimates of the variances of other model derived quantities. 

The hybrid method calculates the harvest rate using the Pope's approximation then converts to an approximation of the corresponding F as:

\begin{align}
	V_{f,t} &= \sum\limits_c\sum\limits_a N_{a,c,t} \exp\left(-\delta_t M_{c,a,t}\right) \nonumber \\
	\tilde{U}_{t,f} &= \frac{C^{obs}_{f,t}}{V_{f,t} + 0.1 C^{obs}_{f,t}}\\
	j_{t,f} &= \left(1 + \exp \left(30 (\tilde{U}_{t,f} - 0.95) \right)\right)^{-1}\\
	U_{t,f} &= 	j_{t,f} \tilde{U}_{t,f} + 0.95 (1 - j_{t,f} )\\
	\tilde{F}_{f,t} &= \frac{-\log\left(1 - U_{t,f}\right)}{\delta_t}
\end{align}
%,
where, \(C^{obs}_{f,t}\) is the observed catch, \(\delta_t\) is the duration of the period of observation within the time-step. In most situations where the entire catch has been observed in a time-step. This should be one. \CNAME\ will log a warning if not equal to one. \(V_{f,t}\) is partway vulnerable biomass and \(\tilde{F}_{f,t}\) is the initial \(F\).

The formulation above is designed so that high exploitation rates (above 0.95) are converted into an F that corresponds to a harvest rate of close to 0.95, thus providing a more robust starting point for subsequent iterative adjustment of this F. The logistic joiner, \(j\), is used at other places in Stock Synthesis to link across discontinuities. The catch at time \(t\), for category \(c\) \TODO{what is supposed to finish this sentence?}

The tuning algorithm begins by setting \(F_{f,t} = \tilde{F}_{f,t}\) and repeating the following algorithm \subcommand{f\_iteration} times.

\[
\widehat{C}_{t,c,a}  = \sum\limits_f {F}_{f,t}\left(S_{f,c,a} N_{t,c,a}\right) \lambda^*_{t,c,a}
\]

where, \(\lambda^*_{t,c,a}\) denotes the survivorship and is calculated as:

\begin{equation}\label{eq:survival}
\lambda^*_{t,c,a} = \frac{1 - \exp\left(-\delta_t Z_{t,c,a}  \right) }{Z_{t,c,a}}
\end{equation}

Total fishing mortality is then adjusted over several fixed number of iterations (typically four, but more in high F and multiple fishery situations). The first step is to calculate the ratio of the total observed catch over all fleets to the predicted total catch according to the current F estimates. This ratio provides an overall adjustment factor to bring the total mortality closer to what it will be after adjusting the individual \(F\) values.

\[
\widehat{C}_{t}  =  \sum\limits_f \sum\limits_c\sum\limits_a {F}_{f,t}\left(S_{f,c,a} N_{t,c,a}\right) \lambda^*_{t,c,a}
\]

This is different from Equation~A.1.25 in the Appendix of \citep{methot2013stock}. They include \(Z_{t,c,a}\) in the denominator of \({F}_{f,t}\), which is a type error because \(Z_{t,c,a}\) is already included in \(\lambda^*_{t,c,a}\), see Equation~\ref{eq:survival}.

\[
Z^{adj}_t = \frac{\sum\limits_f C^{obs}_{f,t}}{\widehat{C}_{t}}
\]

The total mortality if this adjuster was applied to all the Fs is then calculated:

\[
Z^*_{t,c,a} = M_{t,c,a} + Z^{adj}_t \left(Z_{t,c,a} -M_{t,c,a} \right)
\]

\[
\lambda^*_{t,c,a} = \frac{1 - \exp\left(-\delta_t Z^*_{t,c,a}  \right) }{Z^*_{t,c,a}}
\]

The adjusted mortality rate is used to calculate the total removals for each fishery, and then the new \(F\) estimate is calculated by the ratio of observed catch to total removals, with a constraint to prevent unreasonably high \(F\) calculations (\subcommand{max\_f}):

\begin{align*}
	\tilde{V}_{f,t} &= \sum\limits_c\sum\limits_a \left(N_{t,c,a} \bar{w}_{t,c,a}S_{f,c,a} \right)\lambda^*_{t,c,a} \\
	F^*_{t,f} &= \frac{C^{obs}_{f,t}}{\tilde{V}_{f,t} + 0.0001}\\
	j^*_{t,f} &= \left(1 + \exp \left(30 (F^*_{t,f} - 0.95 F_{max}) \right)\right)^{-1}\\
\end{align*}

where, \(F_{max}\) is a user defined maximum fishing mortality \subcommand{f\_max}. The \(F\) at the end of each tuning iteration follows: 

\[
F_{t,f} = j^*_{t,f} F^*_{t,f} + \left(1 - j^*_{t,f}\right)F_{max}
\]

After the tuning algorithm removals at age, and other derived quantities are recorded. The final total mortality is updated
\[
	Z_{t,c,a} = M_{c,a,t} + \sum\limits_{f} S_{f,c,a} F_{f,t}
\]

and the partition modified,

\[
N^*_{t,c,a} = N_{t,c,a} \exp\left(-Z_{t,c,a}\right)
\]

During initialisation phases and in years or time-steps with no catch, \(Z_{t,c,a} = M_{c,a,t}\).

This process generates numbers at age removed for each year, fishery, category and age (\(C_{t,f,c,a}\)) which can be accessed by \subcommand{process\_removals} observations. Numbers at age are calculated as

\[
\widehat{C}_{t,f,c,a} = \frac{F_{t,f,a}}{Z_{t,c,a}} N_{t,c,a} \exp\left(-Z_{t,c,a}\right)
\]

Total catch is the summed over categories and age for a time-step and fishery as

\[
\widehat{C}_{t,f} = \sum\limits_c\sum\limits_a \widehat{C}_{t,f,c,a} \bar{w}_{t,a,c}
\]

where \(\bar{w}_{t,a,c}\) is the mean weight, abundance can be derived by flagging \subcommand{biomass false}, where mean weight is omitted from above calculation. If users assign a penalty in the \subcommand{method} table, then the penalty is flagged each time-step and for each fishery, which calculates the difference between \(\widehat{C}_{t,f}\) and \(C^{obs}_{f,t}\), to ensure the catch observed is taken. This makes the assumption that catch is known without error, which may differ from other assessment packages.

The following is an excerpt of the input files. However, see Section~\ref{syntax:Process-MortalityHybrid} for details and descriptions on the subcommands.

{\small{\begin{verbatim}
@process total_mortality
type mortality_hybrid
m 0.15
time_step_proportions  1 
relative_m_by_age One*2
categories *
max_f 2.95		## max F
f_iterations 5  ## number of tuning iterations
table catches
year Fishery
1972	200
1973	1000
1974	1000
end_table

table method
method  	category 	selectivity 	annual_duration 	time_step  penalty	
Fishery   	male   		fish_sel 		1					step1 		none
Fishery  	female   	fish_sel 		1 					step1 		none	
end_table
\end{verbatim}}}

\paragraph{\I{Holling mortality rate}}\label{sec:Process-MortalityHollingRate} \STATUS{Untested}

The density-dependent Holling mortality process\index{Holling mortality} applies the Holling Type II or Type III functions \citep{Holling1959}, and is generalised by the Michaelis-Menten equation \citep{MentenMichaelis1913}.

This mortality process removes a number or biomass from a set of categories according to the total (selected) abundance (or biomass) and some "predator" abundance (or biomass), and is constrained by a maximum exploitation rate.

The mortality applied to user-defined categories $k$, with the numbers removed at age $l$, determined by a selectivity-at-age $S_l$ is applied as follows:

First, calculate the total predator abundance (or biomass) over all predator categories $k$ in $1 \ldots K$ and ages $l = 1 \ldots L$ that are applying the mortality

\begin{equation}
	P_{k,l} = S^{predator}_l N^{predator}_{k,l}
\end{equation}

And define the total predator abundance (or biomass) $P_{total}$ as

\begin{equation}
	P_{total}  = \sum\limits_K {\sum\limits_L {P_{k,l}}}
\end{equation}

Then calculate the total vulnerable abundance (or biomass) over all prey categories $k$ in $1 \ldots K$ and ages $l = 1 \ldots L$ that are subject to the mortality

\begin{equation}
	V_{k,l} = S^{prey}_l N^{prey}_{k,l}
\end{equation}

Then define the total vulnerable abundance (or biomass) $V_{total}$ as

\begin{equation}
	V_{total}  = \sum\limits_K {\sum\limits_L {V_{k,l}}}
\end{equation}

The number to remove is then determined by

\begin{equation}
	R_{total} = P_{total} \frac{a  V_{total}^{x-1}}{b + V_{total}^{x-1}}
\end{equation}

where $x=2$ for the Holling type II function, $x=3$ for the Holling type III function, or a different value of $x \geq 1$ for the generalised Michaelis-Menten function; $a > 0$ and $b > 0$ are the Holling function parameters.

The exploitation rate\index{Maximum exploitation rate} to apply is

\begin{equation}
	U = \begin{cases}
		R_{total}/V_{total}, & \text{if $R_{total}/V_{total} \leq U_{max}$} \\
		U_{max}, & \text{otherwise}\\
	\end{cases}
\end{equation}

And the number removed $R$ from each age $l$ in category $k$ is

\begin{equation}
	R_{k,l} = U V_{k,l}
\end{equation}

The density-dependent Holling mortality process is applied either as a function of biomass or abundance, depending on the value of the \texttt{is\_abundance} switch.

For example, a biomass Holling type II mortality process on prey \texttt{prey} by predator \texttt{predator} has the syntax

{\small{\begin{verbatim}
		@process HollingMortality
		type Holling_mortality_rate
		is_abundance F
		a 0.08
		b 10000
		x 2
		categories prey
		selectivities One
		predator_categories predator
		predator_selectivities One
		u_max 0.8
\end{verbatim}}}

\paragraph{Initialisation-event mortality}\label{sec:Process-MortalityInitialisationEvent}\label{sec:Process-MortalityInitialisationEventBiomass}\STATUS{Untested}

Initialisation event mortality\index{Initialisation event mortality} is a process that can occur only in the initialisation phase. It applies abundance or biomass mortality events specifically in initialisation phases. This option can be useful if the population is not in equilibrium before model start.

This process applies a single catch value for all iterations within the initialisation phase, and mortality will not be applied outside of the initialisation phase. This process should not be embedded in the annual cycle.

This process should be used in conjunction with the \texttt{insert\_processes} command in the \command{initialisation\_phase} block.

Example syntax where the \texttt{initialisation\_mortality\_event} has been specified in the initialisation phase \texttt{Predation\_state} but not in the annual cycle:

{\small{\begin{verbatim}
initialisation_phases Equilibrium_state Predation_state
time_steps Oct_Nov Dec_Mar

@initialisation_phase Equilibrium_state
type derived

@initialisation_phase Predation_state
type iterative
insert_processes Oct_Nov()=predation_Initialisation

@process predation_Initialisation
type initialisation_mortality_event
categories male.HOKI female.HOKI
catch 90000
selectivities Hakesl Hakesl

time_step Oct_Nov
processes Mg1 Instantaneous_Mortality

@time_step Dec_Mar
processes Recruitment Instantaneous_Mortality
\end{verbatim}}}

\paragraph{Initialisation-mortality-baranov}\label{sec:Process-MortalityInitialisationBaranov}

Initialisation Baranov mortality \index{Initialisation mortality baranov} is a process that can occur only in the initialisation phase. It applies an instantaneous fishing mortality rate to categories during the initialisation phases. This option can be useful to explore non-equilibrium population structures.

This process applies a single \(F\) for all iterations within the initialisation phase it is assigned to, and mortality will not be applied outside of the initialisation phase. This process should not be embedded in the annual cycle i.e., not be defined in an \command{time\_step}.

This process should be used in conjunction with the \texttt{insert\_processes} command in the \command{initialisation\_phase} block.

Each category defined in this process will have have the following mortality applied each iteration of the initialisation phase

\[
N^*_{a,c}  = N_{a,c} \exp{-F S_{a,c}}
\]

where, \(N^*_{a,c}\) are the number for category \(c\) of age \(a\), \(F\) is an estimable quantity (\subcommand{fishing\_mortality}) and \(S_{a,c}\) is the corresponding selectivity.

Example syntax where the \texttt{initialisation\_mortality\_baranov} has been specified in the initialisation phase but not in the annual cycle:

{\small{\begin{verbatim}
		initialisation_phases Equilibrium_state init_F
		time_steps Oct_Nov Dec_Mar
		
		@initialisation_phase Equilibrium_state
		type derived
		
		@initialisation_phase init_F
		type iterative
		insert_processes Oct_Nov()=init_F 
		## because we didn't specify a proccess label in ()
		## this will occur at the end ot this time-step
		
		@process init_F
		type 		mortality_initialisation_baranov
		categories	*
		selectivities Sel_init_F
		fishing_mortality 0.5
		
		@selectivity Sel_init_F
		type 	all_values_bounded 
		l 2
		h 10
		v 0.4 * 9
		\end{verbatim}}}

\paragraph{Prey-suitability mortality}\label{sec:Process-MortalityPreySuitability} \STATUS{Untested}

The density-dependent prey-suitability mortality process\index{Density-dependent prey-suitability} applies predation mortality from a predator group to its prey groups simultaneously. It removes an abundance (or biomass) from each prey group according to the total (selected) abundance (or biomass) of each prey group, the total (selected) abundance (or biomass) of the other prey groups, some "predator" abundance (or biomass), and the preference (electivity) of the predator for each prey group, constrained by a maximum exploitation rate. The predator-prey suitability functions were based on the multispecies Virtual Population Analysis (MSVPA) functions \citep{JuradoMolina2005}.

The mortality applied to the user-defined prey group $g$ of category $k$, with the numbers removed at age $l$ determined by a selectivity-at-age $S_l$ is applied as follows:

First, calculate the total predator abundance (or biomass) over all predator categories $k$ in $1 \ldots K$ and ages $l = 1 \ldots L$ that are applying the mortality

\begin{equation}
P_{k,l} = S^{predator}_l N^{predator}_{k,l}
\end{equation}

And define the total predator abundance (or biomass) $P_{total}$ as

\begin{equation}
P_{total}  = \sum\limits_K {\sum\limits_L {P_{k,l}}}
\end{equation}

Then, given the total vulnerable abundance (or biomass) of prey group $g$ over all categories $k$ in $1 \ldots K$ and ages $l = 1 \ldots L$ that are subject to the mortality

\begin{equation}
V_{g,k,l} = S^{prey}_l N^{prey}_{k,l}
\end{equation}

And define the total vulnerable abundance (or biomass) of each prey group $V^g_{total}$ as

\begin{equation}
V^g_{total}  = \sum\limits_K {\sum\limits_L {V_{g,k,l}}}
\end{equation}

And the total availability $A^g_{total}$ for each prey group as

\begin{equation}
A^g_{total} = \frac{V^g_{total}}{\sum\limits_G {V^g_{total}}}
\end{equation}

The vulnerable abundance (or biomass) and availability every prey group $g$ in $1 \ldots G$ is calculated simultaneously. Then the abundance (or biomass) to remove from each prey group $g$ is a function of its electivity $E_g$, the availability of all other prey groups $i$ in $1 \ldots G$, the electivity of the predator for each prey group $E_i$, and the total consumption rate of the predator $CR$ and its abundance (or biomass) $P_{total}$

\begin{equation}
R^g_{total}=P_{total} CR \frac{A^g_{total} E_g}{\sum\limits_G {A^i_{total} E_i}}
\end{equation}

The exploitation rate\index{Maximum exploitation rate} to apply to each prey group $g$ is then

\begin{equation}
U_g = \begin{cases}
R^g_{total}/V^g_{total}, & \text{if $R^g_{total}/V^g_{total} \leq U_{max}$} \\
U_{max}, & \text{otherwise}\\
\end{cases}
\end{equation}

And the number removed $R^g$ in each prey group $g$ from each age $l$ in category $k$ is

\begin{equation}
R_{g,k,l} = U_g V_{g,k,l}
\end{equation}

Prey suitability choice occurs only between the prey groups specified by the process. The total predator consumption rate represents the consumption of the predator on those prey groups alone. The electivities must sum to 1. Further, the consumption rate can be modified by a layer to be cell specific.

The density-dependent prey-suitability process is applied as either a biomass or an abundance depending on the value of the \subcommand{is\_abundance} switch.

Individual categories can be aggregated into prey groups using the "+" symbol. To indicate that two (or more) categories are to be aggregated, separate them with a "+" symbol.

For example, to specify two prey groups of two species made up of the males and females in each prey group

{\small{\begin{verbatim}
		prey_categories maleSpeciesA + femaleSpeciesA maleSpeciesB + femaleSpeciesB
\end{verbatim}}}

This syntax indicates that there are two prey groups, \texttt{maleSpeciesA + femaleSpeciesA} and \texttt{maleSpeciesB + femaleSpeciesB}, with each group having its own electivity.

For example, a biomass prey-suitability mortality process with an overall consumption rate of $0.8$ of prey \texttt{species A} and \texttt{species B} (modelled as males and females) by the predator \texttt{predatorSpecies} with electivities between \texttt{species A} and \texttt{species B} of $0.18$ and $0.82$ has syntax

{\small{\begin{verbatim}
		@process PreySuitabilityMortality
		type prey-suitability_predation
		is_abundance F
		consumption_rate 0.8
		categories maleSpeciesA + femaleSpeciesA maleSpeciesB + femaleSpeciesB
		electivities 0.18 0.82
		selectivities One One One One
		predator_categories predatorSpecies
		predator_selectivities One
		u_max 0.8
\end{verbatim}}}

\subsubsection{\I{Markovian Movement}}\label{sec:Process-MarkovianMovement}
This process will move fish from multiple categories to multiple categories in a Markovian movement process. This cannot be done in the transition by category process because that process does not allow categories to be both in the \subcommand{from} and \subcommand{to} categories, which is required to apply Markovian movement.

This process is set up be users specified a one to one relationship between \subcommand{from} and \subcommand{to} category inputs. The process will iterate over each user defined combination of \subcommand{from} and \subcommand{to} category inputs calculating the numbers at age that will move from category \(i\) into category \(j\) denoted by \(\tilde{N}_{a,i,j}\) following,

\begin{align}
	\tilde{N}_{a,i,j} \mathrel{+}=& N_{a,i} \times P_i \times S_{a,i}, & \forall \ i \\
	N_{a,i} \mathrel{-}=& \tilde{N}_{a,i,j}, & \forall \ i \\
	N_{a,j} \mathrel{+}=& \tilde{N}_{a,i,j}, & \forall \ j
\end{align}
where $N_{a,j}$ is the number of individuals that have moved to the \subcommand{to} category $j$ from the \subcommand{from} category $i$ in age $a$, $N_{a,i}$ is the number of individuals in category $i$, $P_i$ is the proportion parameter for category $i$, and $S_{a,i}$ is the selectivity at age $a$ for category $i$. See Section~\ref{syntax:Process-MarkovianMovement} for the syntax description. 


For example, to specify a simple spawning migration of mature males from a western area to an eastern (spawning) area, the syntax is

{\small{\begin{verbatim}
			@process movement
			type markovian_movement
			from R1 * 2 R2 * 2
			to R1  R2 R1  R2
			selectivities One
			proportions 0.7 0.3 0.4 0.6
\end{verbatim}}}

The above example can be thought of as defining a two by two movement matrix as follows
\[
 \kbordermatrix{
	& R1 & R2 & \\
	R1 & 0.7 & 0.3  \\
	R2 & 0.4  & 0.6
}
\]

\CNAME\ will check that each unique \subcommand{from} category will have proportions sum to one across all \subcommand{to} categories for that unique \subcommand{from} category. This means users need to define the proportion of fish that stay in the \subcommand{from} category.

The input \subcommand{proportions} are estimable and they are indexed using the "from-to" category label. For example, if we wanted to estimate or transform the proportion of R1 moving to R2 you would specify the parameter as \subcommand{process[movement].proportions\{R1-R2\}}. If you estimate these movement proportion parameters, it is recommended that you use the simplex transformation (Section~\ref{sec:Transformation}). An example of estimating these movement rates is shown below.

{\small{\begin{verbatim}
			@parameter_transformation movement_simplex_move_from_R1
			type simplex
			parameters process[movement].proportions{R1-R1} process[movement].proportions{R1-R2}
			
			@parameter_transformation movement_simplex_move_from_R2
			type simplex
			parameters process[movement].proportions{R2-R1} 		process[movement].proportions{R2-R2}			
\end{verbatim}}}

This is an inherently spatial process and it can get cumbersome when you have other partition attributes other than space i.e., tag or sex. Currently, if you want to move these other attributes at the same rates among regions you need to specify all of the category combinations. Below is an example with a male and female partition members.

{\small{\begin{verbatim}
# Specify the untagged movement process
@process movement
type markovian_movement
from male.R1 * 2 male.R2 * 2 female.R1 * 2 female.R2 * 2
to male.R1  male.R2 male.R1  male.R2 female.R1 female.R2 female.R1  female.R2
selectivities One
proportions 0.7 0.3 0.4 0.6 0.7 0.3 0.4 0.6

@parameter_transformation simplex_move_from_R1_male
type simplex
parameters process[movement].proportions{male.R1-male.R1}
           process[movement].proportions{male.R1-male.R2}

@parameter_transformation simplex_move_from_R2_male
type simplex
parameters process[movement].proportions{male.R2-male.R1}
           process[movement].proportions{male.R2-male.R2}
			
@parameter_transformation simplex_move_from_R1_female
type simplex
parameters process[movement].proportions{female.R1-female.R1} 
           process[movement].proportions{female.R1-female.R2}

@parameter_transformation simplex_move_from_R2_female
type simplex
parameters process[movement].proportions{female.R2-female.R1} 
           process[movement].proportions{female.R2-female.R2}

@estimate movement_simplex_move_from_R1
type uniform
parameter parameter_transformation[simplex_move_from_R1_male].simplex
same parameter_transformation[simplex_move_from_R1_female].simplex
lower_bound -10
upper_bound 10

@estimate movement_simplex_move_from_R2
type uniform
parameter parameter_transformation[simplex_move_from_R2_male].simplex
same parameter_transformation[simplex_move_from_R2_female].simplex
lower_bound -10
upper_bound 10
\end{verbatim}}}

\subsubsection{\I{Transition By Category}}\label{sec:Process-TransitionCategory}\label{sec:Process-Maturation}

The transition by category process moves individuals between categories. This process is used to specify transitions such as maturation (individuals move from an immature to mature state) and migration (individuals move from one area to another).

There is a one-to-one relationship between the "from" category and the "to" category, i.e., for every source category there is one target category only

\begin{equation}
	N_{a,j} = N_{a,i} \times P_i \times S_{a,i}
\end{equation}

where $N_{a,j}$ is the number of individuals that have moved to category $j$ from category $i$ in age $a$, $N_{a,i}$ is the number of individuals in category $i$, $P_i$ is the proportion parameter for category $i$, and $S_{a,i}$ is the selectivity at age $a$ for category $i$.

To merge categories repeat the "to" category multiple times.

For example, to specify a simple spawning migration of mature males from a western area to an eastern (spawning) area, the syntax is

{\small{\begin{verbatim}
		@process Spawning_migration
		type transition_category
		from West.males
		to East.males
		selectivities MatureSel
		proportions 1
		\end{verbatim}}}

where \texttt{MatureSel} is a selectivity that describes the proportion of age or length classes that are mature and thus move to the eastern area.

If you want to estimate the proportion parameter, the parameter is addressed using the the \texttt{to} category. For example using the above process the estimate for the proportion parameter would follow

{\small{\begin{verbatim}
		@estimate proportion_male_spawning
		parameter process[Spawning_migration].proportions{East.males}
		...
\end{verbatim}}}

The transition by category process can be (optionally) included within a mortality block, but note that this may result in negative or nonsensical derived quantities or observations in either the "from" or the "to" categories. If used within the mortality block, care should be taken to ensure that any derived quantities or observations are correctly specified.

\paragraph{Transition by category by age}\label{sec:Process-TransitionCategoryByAge}

A special process type is the transition by category by age process, which allows a transition to occur for a specific subset of ages in specific years only, where each year can have a different number that are moved between categories.

\subsubsection{\I{Tag Release events}}\label{sec:Process-TagByAge}\label{sec:Process-TagByLength} 

Tagging processes can be age- or length-based processes, whereby numbers of individuals are moved to a tagged category defined in the \command{categories} block. Tag release processes can also account for initial tag-induced mortality\index{Initial tag mortality or tag loss} (i.e., initial tag loss) on individuals.

Age-based tag release events move a known number of individuals tagged for each age to a tagged category, along with applying additional mortality. Individuals are removed from the non-tagged categories and added to tagged categories. Often the ages of tagged individuals are not known, so length-based tag release events are more commonly used.

Length-based tag release processes are more complicated, as the age-length matrix is calculated and the exploitation for each length bin to then move the correct numbers-at-age based on the known lengths of release. \CNAME\ also allows for initial tag loss.

\paragraph*{{Tag Release By Length}}
\subcommand{compatibility\_switch casal2}\\

For each length bin $l$ of the input vector of proportions-at-length ${p}_l$ and total tag releases denoted by \(M\). The vulnerable numbers for age \(a\), length \(l\), and category \(j\) are calculated as
$$V_{a,l,j} = N_{a,j} \phi_{a,l,j} S_{a,j}$$
where, \(\phi_{a,l,j}\) is the age-length matrix (Section~\ref{sec:AgeLength-length_at_age}), and \(S_{a,j}\) is the selectivity. The transition matrix is calculated as,
$$ T_{a,l,j}= \frac{V_{a,l,j}}{\sum_a\sum_j V_{a,l,j}} {p}_l$$
and final tag-release by age and category are
$$ M_{a,j} = \sum_l T_{a,l,j}M  $$
An age based maximum transition rate is calculated to stop tagging more fish than are available in the population, in this case the penalty is flagged, see below
$$ u_{a,j} = \frac{ M_{a,j} }{N_{a,j} }  $$

$$
u_{a,j} =
\begin{cases}
u_{max},& \text{if } u_{a,j} > u_{max} \text{ \textbf{flag a penalty}}\\
u_{a},  & \text{otherwise}
\end{cases}
$$

The resulting tag-releases for each age and category denoted by \(\widetilde{N}_{a,j}\) are thus

$$
\widetilde{N}_{a,j} = N_{a,j} u_{a,j}
$$

Release mortality denoted by \(h_a\) can also be applied as

$$\widetilde{N}_{a,j} = \widetilde{N}_{a,j}\left(1.0 - h_a\right)$$

The syntax for an example of tag release by length process

{\small{\begin{verbatim}
@process 2005Tags_shelf
type tag_by_length
years 2005
from male.untagged+female.untagged
to male.2005  female.2005
selectivities ShelfselMale ShelfselFemale
penalty tagging_penalty
initial_mortality 0.1
table proportions
year 30 40 50 60 70 80 90 100 110 120 130 140 150 160 170 180 190 200 210 220
2005  0 0 0.0580 0.1546 0.3380 0.1981 0.1643 0.0531 0.0242 0.0097 0 0 0 0 0 0 0 0 0 0
end_table
n 207
U_max 0.999
\end{verbatim}}}

This process moves 207 individuals from a combination of male.untagged and female.untagged categories, based on the combination of growth rates and selectivity, into tagged male and tagged female categories.

For \subcommand{compatibility\_switch casal} 
this describes the algorithm applied in CASAL, which should yield identical results when a single untagged category and tagged category are supplied. There is a small difference when input numbers at length are defined for a combination of categories (unsexed) and we allocated into each corresponding category (male and female). This is supplied for backwards compatibility with CASAL. We recommend using \subcommand{compatibility\_switch casal2} for when ${p}_l$ is unsexed but sex is in the partition.

For each length bin $l$ of the input vector of proportions-at-length ${p}_l$ and total tag releases denoted by \(M\). The vulnerable numbers for age \(a\), length \(l\), and category \(j\) are calculated as

$$V_{a,l} = N_{a,j} \phi_{a,l,j} S_{a,j}$$

Note this is where the \subcommand{compatibility\_switch} differ, the \subcommand{casal2} option preserves the category difference, where as \subcommand{casal} amalgamates them. \(\phi_{a,l,j}\) is the age-length matrix (Section~\ref{sec:AgeLength-length_at_age}), and \(S_{a,j}\) is the selectivity. The transition matrix is calculated as

$$ T_{a,l}= \frac{V_{a,l}}{\sum_a V_{a,l}} {p}_l$$

and final tag-release by age are

$$ M_{a} = \sum_l T_{a,l}M  $$

An age-based maximum transition rate is calculated to stop tagging more fish than are available in the population, in this case the penalty is flagged, see below.

$$ u_{a} = \frac{ M_{a} }{ N_{a}}  $$

$$
u_{a} =
\begin{cases}
u_{max},& \text{if } u_{a} > u_{max} \text{ \textbf{flag a penalty}}\\
u_{a},  & \text{otherwise}
\end{cases}
$$

The resulting tag-releases for each age and category denoted by \(\widetilde{N}_{a,j}\) are thus

$$
\widetilde{N}_{a,j} = N_{a,j} u_{a}
$$

Release mortality denoted by \(h_a\) can also be applied as

$$\widetilde{N}_{a,j} = \widetilde{N}_{a,j}\left(1.0 - h_a\right)$$

\subsubsection{\I{Tag loss}}\label{sec:Process-TagLoss} 

Tag loss is the process which accounts for tag failure or loss where tagged individuals lose their tags. It is applied as a removal, where individuals are removed from the partition as an instantaneous mortality rate, that can happen over multiple time steps in the annual cycle. 

Two options are available depending on whether the individuals in the tag release event were tagged with a single tag (\subcommand{tag\_type single}) or were double tagged (\subcommand{tag\_type double}).

For tag loss type \subcommand{single}\index{Tag loss: single}, the annual loss rate is simply an exponential decay, i.e, for elements in the partition indexed by age ($a$) and category ($c$), the number of tags lost ($N^*_{c,a}$) in each time step $t$, assuming a loss rate $r$ for category $c$, selectivity at age $S_a$, and proportion in the time step $p^t$ is; 
\begin{equation}
	N^*_{c,a} = N_{c,a} (1 -e^{-r_c p^t S_a})
\end{equation}

where $N_{c,a}$ is the number of tagged fish before the process is applied, and $r_c$ is the tag loss rate for category $c$.

For tag loss type \subcommand{double}\index{Tag loss: double}, the tag loss rate  for each category ($r_c$) is assumed to be independent and identical for both of the tags. The annual loss (i.e., the proportion that do not retain at least one tag) is a function of the time since tagging occurred, $t$, given the number at time $t-1$. Here, let
\begin{equation}
	A_c = 1-e^{-r_c (t-1)} \text{\ and\ }
	B_c = 1-e^{-r_c t} \text{\ ,} 
\end{equation}
then 
\begin{equation}
	r^t_c = -\ln(1 - (B^2_c - A^2_c) / (1-A^2_c))
\end{equation}
is the negative log proportion that lost at least two tags at time $t$ compared with the number available at time $t-1$. Therefore, 
\begin{equation}
	N^*_{c,a} = N_{c,a} (1-e^{-r^t_c p^t S_a})
\end{equation}

For tag loss type \subcommand{double}, the \subcommand{year} of tagging is assumed to be the initial year for determining the years since tagging occurred ($t$).

The syntax for the tag loss process follows

{\small{\begin{verbatim}
@process Tag_loss
type tag_loss
categories single_tagged_fish
tag_loss_rate 0.02
time_step_proportions 0.25 0.75
selectivities One
tag_loss_type single
year 1985
\end{verbatim}}}

and double tagged

{\small{\begin{verbatim}
	@process Tag_loss
	type tag_loss
	categories double_tagged_fish
	tag_loss_rate 0.02
	time_step_proportions 0.25 0.75
	selectivities One
	tag_loss_type double
	year 1985
\end{verbatim}}}

See Section \ref{syntax:Process-TagLoss} for more information on the syntax for tag loss.

\subsubsection{\I{Tag loss empirical}}\label{sec:Process-TagLossEmpirical} 

Tag loss is the process which accounts for tag failure or loss where tagged individuals lose their tags. It is applied as a removal, where individuals are removed from the partition as an instantaneous mortality rate, that can happen over multiple time steps in the annual cycle. The empirical tag loss process allows the user to specify specific tag loss rates for each year at liberty for tagged individuals.

For empirical tag loss\index{Tag loss: empirical}, the annual loss rate is simply an instantaneous rate applied for each year of liberty for the specified categories, i.e, for elements in the partition indexed by age ($a$) and category ($c$), the number of tags lost ($N^*_{c,a}$) in each time step $t$, assuming the loss rate $r$ for category $c$, selectivity at age $S_a$, and proportion in the time step $p^t$ is; 
\begin{equation}
	N^*_{c,a} = N_{c,a} (1 -e^{-r_c p^t S_a})
\end{equation}

where $N_{c,a}$ is the number of tagged fish before the process is applied, and $r_c$ is the tag loss rate for category $c$.

The syntax for the empirical tag loss process follows

{\small{\begin{verbatim}
			@process Tag_loss_empirical
			type tag_loss_empirical
			categories single_tagged_fish
			tag_loss_rate 0.02 0.03 0.04 0.05
			time_step_proportions 0.25 0.75
			selectivities One
			year 1985
			years_at_liberty 1 2 3 4
\end{verbatim}}}

See Section \ref{syntax:Process-TagLossEmpirical} for more information on the syntax for empirical tag loss.

\else

Population processes are processes that change the model state. These processes produce changes in the partition by adding or removing individuals, or by moving individuals between length bins and/or categories.

Current population processes available include:

\begin{itemize}
\item recruitment\index{Recruitment} (Section~\ref{sec:Process-Recruitment}),
\item growth\index{Growth} (Section~\ref{sec:Growth}),
\item mortality\index{Mortality} events (e.g., natural and fishing) (Section~\ref{sec:Process-Mortality}), 
\item category transition processes\index{Category transition}, i.e., processes that move individuals between categories while preserving their overall length structure (Section~\ref{sec:Process-TransitionCategory}), and
\end{itemize}

There are two types of processes: (1) processes that occur across multiple time steps in the annual cycle, e.g., \subcommand{mortality\_constant\_rate}, \subcommand{mortality\_instantaneous}, \subcommand{growth}; and (2) processes that occur only within the time step in which they are specified.

\subsubsection{\I{Recruitment}}\label{sec:Process-Recruitment}

The recruitment processes adds new individuals to the partition. Recruitment depends on the \(B_{0}\) and \(R_{0}\) parameters which are interpreted as the average spawning stock biomass and recruitment over the period of data when there is no fishing. The other factors needed are spawning stock biomass ($SSB$; see Section~\ref{sec:DerivedQuantity}), stock-recruitment relationship and the CV for the prior on recruitment multipliers (the mean is mandated to be 1 over some specified year range). Thus, a $SSB$ label may have to be included (pointing to a derived quantity).

In the recruitment processes, a number of individuals are added to a range of categories and length bins within the partition, with the overall number determined by the type of stock-recruitment process specified. If recruits are added to more than one category, then the proportion of recruits to be added to each category is specified by the \argument{proportions} subcommand. For example, if recruiting to categories labelled \texttt{male} and \texttt{female}, then the proportions may be set to $0.5$ and $0.5$, so that half of the recruits are added to the male category and the other half to the female category.

Recruitment can differ between a spawning event or the creation of a cohort/year class. One view for fisheries is that recruitment usually refers to individuals \enquote{recruiting} to a fishery. This definition is used because there is generally not a lot of observations for younger fish between the time of spawning and being vulnerable to a survey or fishery for data collection.

The year offset for an age cohort between spawning and recruitment into the partition is required. The \CNAME\ parameter \subcommand{ssb\_offset} defines this offset year and is analogous to the CASAL parameter \subcommand{y\_enter}.

\begin{equation}
N_{y,l,j} \leftarrow N_{y,l,j} + p_j(R_y) * f_j(l,\mu, \sigma)
\end{equation}

where $N_{y,l,j}$ is the numbers in year $y$ and category $j$ at length bin $l$, $p_j$ is the proportion added to category $j$, \(f(l,\mu, \sigma)\) is the density function of recruits among length bins for category \(j\), and $R_y$ is the total number of recruits in year $y$.


\paragraph{\I{Constant recruitment}}\label{sec:Process-RecruitmentConstant}

In the constant recruitment process the total number of recruits added in each year $y$ in age $a$ is $R_y$, with $R_y = R_0$ for all years

\begin{equation}
	R_{y,j} = p_j(R_0)
\end{equation}

Constant recruitment is equivalent to a Beverton-Holt recruitment process with steepness ($h$) set to 1.

For example, to specify a constant recruitment process where individuals are added to the male and female immature categories at $age=1$ in equal proportion (\texttt{proportions} = 0.5), and the number to add is $R_0=5 \times 10^5$, the syntax is

{\small{\begin{verbatim}
			@process Recruitment
			type constant_recruitment
			categories male.immature female.immature
			proportions 0.5 0.5
			r0 500000
			length_bins 1
\end{verbatim}}}

\paragraph{\I{Beverton-Holt recruitment}}\label{sec:Process-RecruitmentBevertonHolt}

In the Beverton-Holt recruitment process the total number of recruits added each year is $R_y$. $R_y$ is the product of the average recruitment $R_0$, the annual recruitment multipliers ($YCS$, also called year class strength), and the stock-recruit relationship $SR(SSB_y)$


\begin{equation}\label{eq:BH}
R_{y} = (R_0 \times YCS_{y} \times SR(SSB_{spawn\_year}))
\end{equation}

where

\begin{equation}\label{eq:year_class}
spawn\_year = y - \texttt{ssb\_offset}
\end{equation}

Recruitment refers to an age cohort recruiting into the partition and may differ from the spawning year. See below on more information about \subcommand{ssb\_offset}.

$SR(SSB_y)$ is the Beverton-Holt stock-recruit relationship parametrised by the steepness $h$, and based on \cite{mace_doonan_88} parametrisation

\begin{equation}\label{eq:BH_SR}
SR(SSB_y) = \frac{SSB_y}{B_0} / \left( 1-\frac{5h-1}{4h} \left( 1-\frac{SSB_y}{B_0} \right) \right)
\end{equation}

The Beverton-Holt recruitment process requires a value for \Bzero\ (or \Rzero) and $SSB_y$ to calculate the number of recruits. A derived quantity (see Section \ref{sec:DerivedQuantity}) must be defined that provides the annual $SSB_y$ for the recruitment process. \Bzero\ is then defined as the value of the $SSB$ calculated during initialisation. If a model has more than one initialisation phase, the user needs to supply the initialisation phase that calculates \Bzero. This is defined by the command \subcommand{b0\_initialisation\_phase}. \CNAME\ will default to the last initialisation phase if users do not specify this command.

During initialisation, the recruitment multipliers ($YCS$) are assumed to be equal to one, and recruitment that happens in the initialisation phases that occur before and during the phase when \Bzero\ is determined are assumed to have steepness $h=1$ (i.e., in those initialisation phases, recruitment is equal to \Rzero).

Recruitment during the initialisation phases after the phase where \Bzero\ was determined are calculated using the Beverton-Holt stock-recruit relationship. \Rzero\ and \Bzero\ have a direct relationship when there are no density-dependent processes in the annual cycle. Models can thus be initialised using either \Bzero\ or \Rzero.

The length apportionment of recruits into the partition are derived using the normal cumulative function over the model length bins. This cumulative function is denoted by \(f_j(l,\mu_c, \sigma_c)\). For each category this will calculate the probability of being in a length bin and will  sum equal to one over all length bins. The mean and variance of this normal cumulative function can be category specific and is defined by the input parameters \subcommand{inital\_mean\_length} and \subcommand{inital\_length\_cv}.

{\small{\begin{verbatim}
		@process Recruitment
		type recruitment_beverton_holt
		categories immature mature
		proportions 1.0 0.0
		r0 500000
		steepness 0.75
		ssb_offset 1
		inital_mean_length 10
		inital_length_cv 0.40
		ssb SSB_derived_quantity
		\end{verbatim}}}

The property \subcommand{ssb\_offset} has to be manually specified.

\paragraph*{YCS ($YCS_y$)}

The $YCS$ parameter is reference by the recruited year. The recruited year is the year when a year class or age-cohort enter the partition. The recruited year differs from the spawning event year defined in Equation~\eqref{eq:year_class}. This is a shift away from CASALs terminology which used \subcommand{ycs\_year} and is equivalent to the spawning event year. Standardisation years are are also now expressed as recruited years. This will differ from \CNAME\ versions before August 2022 and CASAL models. From August 2022 we deprecated the commands \subcommand{ycs\_values}, \subcommand{ycs\_years}, and \subcommand{standardised\_ycs\_years}. These were replaced with \subcommand{recruitment\_multipliers} and \subcommand{standardise\_years}.

This year reference is important when defining \command{estimate}, \command{project}, and \command{time\_varying} blocks for the \subcommand{recruitment\_multipliers} parameter. An example is at the end of the section.

A common practice when estimating $YCS$ is to standardise using the Haist parametrisation, which was described by V. Haist. \CNAME\ will standardise $YCS$ only if subcommand \subcommand{standardise\_years} is defined. The model parameter \texttt{recruitment\_multipliers} is a vector \textbf{Y}, covering the years from \texttt{start\_year} to \texttt{final\_year}. The resulting standardised recruitment multipliers are calculated as $YCS_i=Y_i/\bar{\textbf{Y}}$, where the mean is calculated over the user-specified years \texttt{standardise\_years}.

An alternative to \enquote{standardisation} is to constrain the $YCS$ parameters using the simplex transformation (see Section~\ref{sec:Transformation-Simplex}). This is thought to have estimation benefits over the \enquote{standardisation} as priors can be applied to the \enquote{free} (estimable) parameters (\(Y_i\)).

\[
YCS_i =
\begin{cases}
Y_i / mean_{y \in S}(Y_y) & :y \in S\\
Y_i					 & :y \notin S
\end{cases}
\]

where S is the set of years from \texttt{standardise\_years}. One effect of this parametrisation is that \Rzero\ is then defined as the mean estimated recruitment over the set of years $S$, because the mean $YCS$ multiplier over these years will always be one.

Typically, \texttt{standardise\_years} is defined to span the years over which $YCS$ is reasonably well estimated. For years that are not well estimated, $Y_y$ can be set to 1 for some or all years $y\in S$ (which is equivalent to forcing $R_y$ = \Rzero\ x $SR(SSB_y)$) by setting the lower and upper bounds of these $Y$ values to 1. An exception to this might occur for the most recent $YCS$ values, which the user may estimate but not include in the definition of \Rzero\ (because the estimates may be based on too few data). One or more years may be excluded from the range of years for the averaging process of the Haist parametrisation.

The advantage of the Haist parametrisation is that a large penalty is not necessary to force the mean of the $YCS$ parameter to be 1, although a small penalty should still be used to stop the mean of \textbf{Y} from drifting. These adjustments may improve MCMC performance. Projected $YCS$ values are not affected by this feature. A disadvantage with this parametrisation in a Bayesian analysis is that the prior applies to $Y$, not $YCS$.

In the  example given above, $YCS$ are standardised to have mean one in the period 1995 to 2004, and recruits enter into the model two years following spawning

{\small{\begin{verbatim}
		@process Recruitment
		type recruitment_beverton_holt
		...            #subcommand above
		standardise_years 1995:2004
		recruitment_multipliers 0.65 0.87 1.6 1.13 1.02 0.38 2.65 1.35 1 1 1 1 1
		\end{verbatim}}}

\subsubsection{\I{Mortality}\label{sec:Process-Mortality}}

There are 4 types of mortality processes available in \CNAME length based models, plus the tag release processes that can also cause mortality (See Tagging Section~\ref{sec:Process-Tagging}):

\begin{itemize}
	\item constant rate,
	\item constant exploitation,
	\item instantaneous, and
	\item disease.
\end{itemize}

\paragraph{Constant mortality rate}\label{sec:Process-MortalityConstantRate} 

To specify a constant annual mortality rate \index{Constant mortality}(e.g. $M=0.2$) for categories "male" and "female"
{\small{\begin{verbatim}
		# A process with label NaturalMortality
		@process NaturalMortality
		type          mortality_constant_rate
		categories    male female
		relative_m_by_length One One
		m             0.2 0.2
		\end{verbatim}}}

The total number of individuals removed from a category

\begin{equation}
D_{j,t} = \sum_l N_{l,j,t} [1 - \exp(-S_{l,j} M_{l,j} p_t)]
\end{equation}

where $D_{j,t}$ is the total number of deaths in category $j$ in time step $t$, $N_{l,j,t}$ is the number of individuals in category $j$ of length bin $l$ in time step $t$, $S_{l,j}$ is the selectivity value for length bin $l$ in category $j$, $M_{l,j}$ is the mortality rate for category $j$ for length bin $l$, and $p_t$ is the proportion of the mortality rate to apply in time step $t$.

The mortality rate process requires the specification of the mortality-by-length curve which is specified using a selectivity. To apply the same mortality rate over all length bins in a category, use a selectivity defined as $S_{l,j}=1.0$ for all lengths $l$ in category $j$

{\small{\begin{verbatim}
		@selectivity One
		type constant
		c 1
		\end{verbatim}}}

Length-specific mortality rates can also be applied. For example, the hypothesis that mortality is higher for younger and older individuals and lowest when individuals are at their optimal fitness could be defined by using a double exponential selectivity (see Section~\ref{sec:Selectivity})

{\small{\begin{verbatim}
		@selectivity length_specific_M
		type double_exponential
		x0 12
		x1 1
		x2 37
		y0 0.182154
		y1 1.43768
		y2 1.57169
		alpha 1.0
		
		@process      NaturalMortalityByLength
		type          mortality_constant_rate
		categories    male female
		relative_m_by_length length_specific_M length_specific_M
		m             1.0 1.0
		\end{verbatim}}}


In this definition \subcommand{m} is set to 1.0 and the rate is described through the selectivity. Otherwise, $M_{l} = S_{l} * m$. This concept can be constructed similarly for other mortality methods such as \subcommand{instantaneous\_mortality}.

\paragraph{Constant mortality exploitation}\label{sec:Process-MortalityConstantExploitation} 

To specify a constant annual exploitation rate \index{Constant exploitation}(e.g. $U=0.2$) for categories "male" and "female"
{\small{\begin{verbatim}
			# A process with label IncidentalMortality
			@process IncidentalMortality
			type          mortality_constant_exploitation
			categories    male female
			relative_u_by_length One One
			u             0.2 0.2
\end{verbatim}}}

The total number of individuals removed from a category

\begin{equation}
	D_{j,t} = \sum_l N_{l,j,t} S_{l,j} U_{l,j} p_t
\end{equation}

where $D_{j,t}$ is the total number of removals in category $j$ in time step $t$, $N_{l,j,t}$ is the number of individuals in category $j$ of length bin $l$ in time step $t$, $S_{l,j}$ is the selectivity value for length bin $l$ in category $j$, $U_{l,j}$ is the exploitation rate for category $j$ for length bin $l$, and $p_t$ is the proportion of the exploitation rates to apply in time step $t$.

The exploitation rate process requires the specification of the mortality-by-length curve which is specified using a selectivity. To apply the same mortality rate over all length bins in a category, use a selectivity defined as $S_{l,j}=1.0$ for all lengths $l$ in category $j$

{\small{\begin{verbatim}
			@selectivity One
			type constant
			c 1
\end{verbatim}}}

Length-specific exploitation rates can also be applied. 

\paragraph{Instantaneous mortality}\label{sec:Process-MortalityInstantaneous}

The instantaneous mortality process\index{Instantaneous mortality} combines both natural mortality and fishing exploitation into a single process. This allows the simultaneous application of both natural mortality and anthropogenic mortality to occur across multiple time steps. This process accounts for half the natural mortality within a time step before calculating vulnerable biomasses for calculating exploitation rates. The remaining half of the natural mortality is taken after exploitation has been accounted for. The input for this process is catches and these can either be specified as biomasses or numbers (abundance). In fisheries models in \CNAME\ this is the most commonly used mortality process.

This process allows for multiple removal events, e.g., a fisheries model with multiple fisheries and/or fleets. A removal method can occur in one time step only, although multiple removals can be defined to cover events during the year.

The equations for instantaneous mortality are based on Pope's discrete catch equation, which assumes catch is known without error. \CNAME\ will try and take the exact catch specified in the input.

\begin{itemize}
	\item An exploitation rate (a proportion) is calculated for each fishery, as the catch divided by the selected-and-retained abundance or biomass termed vulnerable biomass. Vulnerable biomass is calculated by accounting for half natural mortality (\(M_{l,c}\)) that occurs at time-step which is defined by the subcommand \subcommand{time\_step\_proportions} and denoted by \(p_t\),
	$$U_{f} = \frac{C_f}{\sum\limits_{c}\sum\limits_l \bar{w}_{l,c} S_{f,l,c} n_{l,c} exp(-0.5 p_t M_{l,c})} \ ,$$
	where \(S_{f,l,c}\) is the fishery selectivity for length bin \(l\) and category \(c\), \(\bar{w}_{l,c}\) is mean weight and \(n_{l,c}\) numbers at length before applying fishing. The categories \(c\) are user defined for each fishery \(f\), which are defined in the \subcommand{table method} (see below for an example).
	\item The fishing pressure associated with method $f$ is defined as the maximum proportion of fish taken from any element of the partition in the area affected by the method $f$
	$$ U_{f,obs} = max_{l,c}(\sum\limits_k\sum\limits_c S_{k,l,c} U_k) $$
	where the maximum is over all partition elements (length and categories) affected by fishery $f$, and the summation is over all fisheries $k$ which affect these partition elements in the same time step as fishery $f$.
	
	In cases with a single fishery the fishing pressure will be equal to the exploitation rate (i.e., $U_{f,obs} = U_f$), but can be different if: (a) there is another removal method operating in the same time step as removal method $f$ and affecting some of the same partition elements, and/or (b) the selectivity $S_{f,l}$ does not have a maximum value of 1.
	
	There is a maximum mortality pressure limit of $U_{f,max}$ for each method of removal $f$. So, no more than proportion $U_{f,max}$ can be taken from any element of the partition affected by removal method $f$ in that time step. Clearly, $0 \leq U_{max} \leq 1$. It is an error if two removal methods, which affect the same partition elements in the same time step, do not have the same $U_{max}$.
	
	For each $f$, if $U_{f,obs} > U_{f,max}$, then $U_f$ is multiplied by $U_{f,max}/U_{f,obs}$ and the mortality pressures are recalculated. In this case the catch actually taken from the population in the model will differ from the specified catch, $C_f$.
	
	\item The partition is updated using
	$$ n'_{l,c} = n_{l,c} exp(-p_t M_{l,c})\big[1 - \sum_f S_{f,l,c} U_{f} \big] $$
\end{itemize}

For example, to apply natural mortality of $0.20$ across three time steps on both male and female categories, with two methods of removals (fisheries \texttt{FishingWest} and \texttt{FishingEast}) and their respective catches (kg) known for years 1975:1977 (the catches are given in the \texttt{catches} table and information on selectivities, penalties, and maximum exploitation rates are given in the \texttt{method} table), the syntax is

{\small{\begin{verbatim}
		@process instant_mort
		type mortality_instantaneous
		m 0.20
		time_step_proportions 0.42 0.25 0.33
		relative_m_by_length One
		categories male female
		biomass true
		units kgs
		
		table catches
		year FishingWest FishingEast
		1975	80000	111000
		1976	152000	336000
		1977	74000	1214000
		end table
		
		table method
		method       category  selectivity u_max   time_step  penalty
		FishingWest   stock     westFSel    0.7     step1     CatchPenalty
		FishingEast   stock     eastFSel    0.7     step1     CatchPenalty
		end_table
		\end{verbatim}}}

and for referencing catch parameters for use in projecting, time-varying, and estimating, the syntax is

{\small{\begin{verbatim}
		parameter process[mortality_instantaneous].method_"method_label"{2018}
		\end{verbatim}}}

where \subcommand{"method\_label"} is the label from the \subcommand{catch} or \subcommand{method} table and continuing the example,

{\small{\begin{verbatim}
		parameter process[instant_mort].method_FishingWest{2018}
		\end{verbatim}}}

\paragraph{Disease mortality rate}\label{sec:Process-Length-DiseaseMortalityRate}

Disease mortality is a special, additional, mortality that is implemented to occur after natural and fishing mortality during a time step. This process removes fish from the partition, is applied to all areas, and can depend on sex/length class.

The partition is updated as follows
\begin{equation}
	n'_{c,j} = n_{c,j}  exp\{-t_y M_{c} S_{c,j} \}
\end{equation}

where \(n_{c,j}\) is the partition for category \(c\) and length class \(j\) before mortality, and \(n'_{c,j}\)  is after the process. \(t_y\) is an annual multiplicative scalar (estimable), \(M_{c}\) is the category specific mortality rate and \(S_{c,j}\) is the selectivity.
 
{\small{\begin{verbatim}
@process DiseaseMortality
type mortality_disease_rate
disease_mortality_rate 1.0
selectivities DiseaseSel 
categories OYS
year_effect 0.05 0.11 0.39 0.38 0.20 
years 2000 2001 2002 2003 2004 2005 
		\end{verbatim}}}

\subsubsection{\I{Transition By Category}}\label{sec:Process-TransitionCategory}

The transition by category process moves individuals between categories. This process is used to specify transitions such as maturation (individuals move from an immature to mature state) and migration (individuals move from one area to another).

There is a one-to-one relationship between the "from" category and the "to" category, i.e., for every source category there is one target category only

\begin{equation}
N_{l,j} = N_{l,i} \times P_i \times S_{l,i}
\end{equation}

where $N_{l,j}$ is the number of individuals that have moved to category $j$ from category $i$ in length bin $l$, $N_{a,i}$ is the number of individuals in category $i$, $P_i$ is the proportion parameter for category $i$, and $S_{l,i}$ is the selectivity at length bin $l$ for category $i$.

To merge categories repeat the "to" category multiple times.

For example, to specify a simple spawning migration of mature males from a western area to an eastern (spawning) area, the syntax is

{\small{\begin{verbatim}
		@process Spawning_migration
		type transition_category
		from West.males
		to East.males
		selectivities MatureSel
		proportions 1
		\end{verbatim}}}

where \texttt{MatureSel} is a selectivity that describes the proportion of length or length classes that are mature and thus move to the eastern area.

If you want to estimate the proportion parameter, the parameter is addressed using the the \texttt{to} category. For example using the above process the estimate for the proportion parameter would follow

{\small{\begin{verbatim}
		@estimate proportion_male_spawning
		parameter process[Spawning_migration].proportions{East.males}
		...
		\end{verbatim}}}

The transition by category process can be (optionally) included within a mortality block, but note that this may result in negative or nonsensical derived quantities or observations in either the "from" or the "to" categories. If used within the mortality block, care should be taken to ensure that any derived quantities or observations are correctly specified.

\subsubsection{\I{Tagging}}\label{sec:Process-Tagging}\label{sec:Process-TagByLength}

Tag release events (also known as mark-recapture events or tag-release events) allow \CNAME\ to incorporate tagging data into the model.

Tagging is a process that moves fish from the general population into specific \enquote{tag} categories. The aim is to get \CNAME\ to track these separately to generate expected recaptures or growth etc.

In addition to creating tag category of the partition, you will need to initialise the values by defining a tag-release event (otherwise they will always be zero). This process moves fish from the \enquote{untagged} category of the partition into a named category of the partition. You will need to define how many fish to move, and the year, time step, area, and stock. Also, you may need to define a penalty (see Section~\ref{sec:Penalty-Process}) to avoid parameter values which do not lead to enough fish being present in the population to allow for the number being tagged (although in cases where only a small proportion of the population is tagged, this is unlikely to be required).

The partition is then updated by moving \(N\) fish from the equivalent \enquote{untagged} category of the partition to the named tag category of the partition, where the numbers at length are defined by a vector of proportions by category. Note that \CNAME\ expects the vector of proportions to sum to 1 over all length bins.

\CNAME\ moves fish using a \enquote{rate} which relates to the penalty. All \enquote{untagged} categories represented by \(\tilde{c}\) are multiplied by a selectivity to calculate total abundance available to be tagged denoted by \(V_{c,j}\)

\begin{equation*}
	V_{c,j} = \sum\limits_{\tilde{c}} n_{\tilde{c}, j} S_{c,j} \ \tilde{c} \in c
\end{equation*}

where \(\tilde{c}\) are categories that are a subset of \(c\) which can be an accumulation of multiple categories. Users define the number of tags released denoted by \(N\) and the proportion by length denoted by \(p_{c,j}\). See below for an example configuration file excerpt.

\begin{equation*}
U_{c,j} = \frac{Np_{c,j}}{V_{c,j}}
\end{equation*}

If this rate is greater than the input subcommand \subcommand{u\_max}

\begin{equation*}
U_{c,j} = 
\begin{cases}
	U_{c,j} \geq u_{max}, & u_{max} \ \text{flag a penalty}\\
	U_{c,j} < u_{max}, & U_{c,j} \\
\end{cases}
\end{equation*}

Tagged fish are moved as follows

\begin{equation*}
	T_{c,j} = U_{c,j} V_{c,j}
\end{equation*}

{\small{\begin{verbatim}
@process tag_1996
type tagging
years 1996
from untagged.male
to 1996_3.male
initial_mortality 0
u_max 0.99
selectivities [type=constant; c=1]
penalty none
N 61
table proportions
year 20	21	22	23	24	25	26	27	28	29	30
1996 0.016	0	0.016	0	0.032	0	0	0.01	0.045	0.048	0.016	
end_table
\end{verbatim}}}

\fi 

\subsection{\I{Derived quantities}\label{sec:DerivedQuantity}}

Some processes require a population value derived from the population state as an argument. These values are \texttt{derived quantities}. Derived quantities are values calculated in a specified time step in every year, and thus have a single value for each year of the model. The time within the time-step is at the end unless otherwise specified (using the \textit{proportion\_mortality} subcommand).

Derived quantities can be calculated as either abundance or biomass. Abundance-derived quantities are the sum over the specified categories (after applying a selectivity)\label{sec:DerivedQuantity-Abundance}. Biomass-derived quantities are calculated similarly\label{sec:DerivedQuantity-Biomass}.

Derived quantities are also calculated during the initialisation phases. Therefore, the time step during each initialisation phase must be specified. If the initialisation time steps are not specified, the derived quantity will be calculated during the initialisation phases. \TODO{review}

A common use of an derived quantities is as input into a stock-recruit relationship  which requires an equilibrium biomass ($B_0$) and annual spawning stock biomass values ($SSB_y$) to calculate recruitment into the first \ifAgeBased age \else length \fi class. $SSB_y$ is an derived quantity based on the mature biomass, usually at spawning time.

Derived quantities can be associated with a \textit{time evaluation interval}; see Section~\ref{sec:Process-Mortality} for more detail on mortality blocks. In this case, the point of calculation can be set to any point within the mortality block, e.g., when 75\% of the deaths from natural mortality plus catch has occurred, which is based on interpolating between the start and end of the block as the partition is known at those points.  Two  methods are available: \texttt{weighted\_sum} and \texttt{weighted\_product}, and are defined as

\begin{itemize}
	\item \texttt{weighted\_sum}: after proportion $p$ through the mortality block, the partition elements are given by $n_{p,j} = (1 - p)n_j + p'_j$

	\item \texttt{weighted\_product}: after proportion $p$ through the mortality block, the partition elements are given by $n_{p,j} = n_j^{1-p} n'^p_j$
\end{itemize}

where $n_{p,j}$ is the derived quantity at proportion $p$ of the mortality block for category $j$, $n_j$ is the quantity at the beginning of the mortality block, and $n'_j$ is the quantity at the end of the mortality block.

For example, to define a biomass-derived quantity spawning stock biomass, $SSB$, calculated at the end of the first time step (labelled \texttt{step\_one}), over all "mature" male and female categories and halfway through the mortality block using the \texttt{weighted\_sum} method, the syntax is

{\small{\begin{verbatim}
@derived_quantity SSB
type          biomass
time_step     step_one
categories    mature.male mature.female
selectivities One
time_step_proportion        0.5
time_step_proportion_method weighted_sum
\end{verbatim}}}

\subsection{\I{Growth}\label{sec:Growth}}

\ifAgeBased
\input{GrowthAge.tex}
\else
\input{GrowthLength.tex}
\fi % end if

\subsection{\I{Length-weight relationship}}\label{sec:MeanWeight}\label{sec:LengthWeight}

There are two length-weight relationships options. The first is the naive "no relationship" relationship, where the weight of an individual is always 1, regardless of length. The second relationship is the "basic" relationship, which is the standard length-weight relationship, $W = aL^b$.

\subsubsection{The `none' relationship}\index{Length-weight relationship!None}\label{sec:LengthWeight-None}

\begin{equation}
  \text{mean weight}=1
\end{equation}

\subsubsection{Basic: the standard length-weight relationship}\index{Length-weight relationship!Basic}\label{sec:LengthWeight-Basic}

The mean weight $\bar{w}$ of an individual of length $l$ is

\begin{equation}
  \bar{w} =a l^b \ .
\end{equation}
\ifAgeBased
For age-based models \(l\) is substituted for $\hat{l}_a$, which is the mean length at age $a$. If a distribution of length-at-age is specified, then the mean weight is calculated over the distribution of lengths.

\begin{equation}\label{eq:mean_weight_with_adjustment}
	\hat{w}_a=(a\hat{l}_a^b)(1+cv^2)^{\frac{b(b-1)}{2}}
\end{equation}

where the $cv$ is the coefficient of variation (CV) of the length-at-age relationship. This adjustment is exact for Lognormal distributions, and an approximation for normal distributions if the CV is not large \citep{1388}. 

When comparing \CNAME\ with CASAL, there is a small difference in the algorithms. If CASAL had to interpolate between CV\_first and CV\_last, it only adjusted the CV values with \subcommand{by\_length} = \texttt{true} when CVs were used in distribution calculations (length-based selectivities, length-based processes, and length-based observations), and not otherwise. \CNAME\ always applies the adjustment.
\else
This is used in length based models where \(l\) is the length bin midpoint. 
\fi

Note: the scale of $a$ can be specified incorrectly. If the catch is in tonnes and the growth curve is in centimetres, then $a$ should convert a length in centimetres to a weight in tonnes. There are reports available that can be used to help check that the units specified are plausible (see Section \ref{sec:Report}).
{\small{\begin{verbatim}
		@length_weight length_weight
		type basic
		units tonnes
		a 0.00000123
		b 3.132
\end{verbatim}}}

\ifAgeBased
\subsection{\I{Age-weight relationship}}\label{sec:AgeWeight} \STATUS{Untested}

Either \texttt{none}\label{sec:AgeWeight-None} or an Empirical weight-at-age matrix. The empirical weight-at-age data can be input\label{sec:AgeWeight-Data}. This option is different from the method above as it uses empirical data for weight-at-age, rather than calculating it from the growth and length-weight relationships (age $\rightarrow$\ length $\rightarrow$\ weight). This bypasses the growth relationship and uses the direct weight-at-age data instead. This needs to be declared in blocks that use this method, i.e., biomass observation blocks, fishery mortality blocks, and biomass derived quantities (e.g., $SSB$). The subcommand to use this is \subcommand{age\_weight\_label} (i.e., with argument  \argument{ageWeight.block.label}). 

In the instantaneous fisheries mortality command, \subcommand{age\_weight\_label} is a column in the \textit{table method} part with the corresponding \texttt{ageWeight.block.label} in the body of the table. 

More than one \command{age\_weight} blocks can be used, and both weight-at-age data and the more common method of using the growth and length-weight relationship (i.e., age $\rightarrow$\ length $\rightarrow$\ weight) can be used in the same model (but not in the same command block).

This option specifies the weight-at-age values for categories at a point in time.

An example

{\small{\begin{verbatim}
		@age_weight age_weight
		type Data
		units tonnes
		table data   #year then ages; 1st row is the column labels
		year 1 2 3 4 5 6 7 8 9 10
		1986	0.134	0.686	1.639	2.719	3.649	4.901	6.329	6.591	7.238	7.491
		1987	0.132	0.724	1.534	2.829	4.092	4.853	5.705	6.143	7.179	8.089
		1988	0.122	0.641	1.533	2.641	3.796	5.054	5.652	6.356	6.95	8.857
		1989	0.137	0.722	1.606	2.416	3.629	5.027	5.561	6.35	6.933	7.217
		1990	0.138	0.773	1.645	2.74	3.711	4.506	5.684	6.929	7.424	7.479
		end_table
\end{verbatim}}}

If weight is defined by the empirical weight-at-age data, then the age-length block in the \command{categories} block can be omitted.

{\small{\begin{verbatim}
@categories
format stock
names Stock
\end{verbatim}}}

\subsection{\I{Weightless model}\label{sec:weightless-model}}

To model abundance (i.e., to model the population in numbers and not convert to biomass), the \command{length\_weight} argument is turned off by specifying the keyword \subcommand{none} in the \command{age\_length} block

{\small{\begin{verbatim}
	@age_length age_size
	type schnute
	...
	length_weight none
	\end{verbatim}}}

In this case any "biomass" generated by \CNAME\ will actually be abundance, and care should be taken with interpretation of the output.
\fi % end if

\subsection{\I{Maturity, in models without maturing in the partition}\label{sec:maturity-notinpartition}}

When maturity is not an attribute (explicit category) in the partition, processes may still depend on maturity. You must then make the assumption that the proportion of mature fish in each element is defined by a selectivity ogive. This approximation is used by derived quantities (Section~\ref{sec:DerivedQuantity}). Selectivity ogives are allowed to vary over time with the time-varying class (Section~\ref{sec:TimeVarying})

\subsection{\I{Selectivities}\label{sec:Selectivity}}

Selectivity is a term used in \CNAME\  to mean an ogive in both age and length based models. They can be used to specify the selection curve for a fishery or observation  (Section \ref{sec:Estimation}) or to modify the effects of processes on the partition, e.g., migration rates by age (Section \ref{sec:Population}). 

\ifAgeBased
% One column version
For age-based models \CNAME\ will either use the age to calculate the selectivity or will use the age-length relationship to integrate over the length distribution for a given age to get length-based selectivity in an age-based model (use the subcommand \texttt{by\_length true}, \texttt{false} is the default). Do not expect too much from length selectivities because in the next time-step or year, the length distribution for each age reverts to being as defined in the \texttt{age\_length} blocks, e.g., normal, rather than being partially truncated because, for example, larger fish in an age class have been preferentially caught.

Length-based selectivities denoted by \(g(.)\) in an age-based models are evaluated by integrating over the length distribution for each age class (see Section~\ref{sec:AgeLength-length_at_age}). For age-class. Given age \(a\) with mean length \(\bar{l}_a\), standard deviation \(\sigma_a\) and length distribution denoted by \(f(l,\bar{l}_a, \sigma_a)\). The selectivity for age class \(a\) denoted by \(s_a\) should be calculated as

\begin{align*}
	s_a = & \int\limits_l g(f(l,\bar{l}_a, \sigma_a)) \ .
\end{align*}
%
An approximation for the above integral is made in \CNAME\ by calculating an average of a set of integration points on \(f(l,\bar{l}_a, \sigma_a)\). The number of integration points is dictated by the subcommand \texttt{intervals}.

\else
% Two column version
In length-based models \CNAME\ will use length midpoints when calculating selectivities. 
\fi % end if

There are a number of different parametric forms, including logistic and double normal curves. Selectivities are defined in command block \command{selectivity <label>}, where the unique label of the selectivity is used by observations and processes to specify which selectivity to apply.

Many selectivities can be forced to apply to ages or lengths from a specified age (or length), i.e., a logistic selectivity 
 can be defined with

\ifAgeBased
{\small{\begin{verbatim}
@selectivity trawlSel    # label for the trawl fishery selectivity
type      logistic       # type of curve
a50       4.4            # age at 50% selection
ato95     1.5            # interval (yr) from a50 to the age at 95% selection
                         #   age at 95% selectivity is 5.9 yr; at 5%, 2.9 yr
beta      2              # minimum age selected, so that individuals with 
                         #   age < beta have selectivity = 0
# at_length true         # if used, then a50, ato95, and beta refer to length
# intervals 10			 # integration points for when at_length = true
\end{verbatim}}}
\else
{\small{\begin{verbatim}
			@selectivity trawlSel    # label for the trawl fishery selectivity
			type      logistic       # type of curve
			a50       4.4            # length class at 50% selection
			ato95     1.5            # interval (yr) from a50 to the length
			                         #   class at 95% selection
			beta      2              # minimum length class selected, so that individuals with 
			                         #   length < beta have selectivity = 0
\end{verbatim}}}
\fi

For some selectivities, the function values for some choices of parameters can result in numeric overflow or underflow errors (i.e., the number calculated from parameter values is either too large or too small to be well represented). \CNAME\ implements range checks on some parameters to test for these errors before calculating function values.

For example, the logistic selectivity is implemented such that if $(a_{50}-x)/a_{to95} > 5$ then the value of the selectivity at $x=0$, i.e., for $a_{50}=5$, $a_{to95}=0.1$, then the value of the selectivity at $x=1$, without range checking would be $7.1 \times 10^{-52}$. With range checking, that value is $0$ (as $(a_{50}-x)/a_{to95}=40 > 5$).

The selectivity options are:

\begin{itemize}
  \item Constant (Section~\ref{sec:Selectivity-Constant})
  \item Knife-edge (Section~\ref{sec:Selectivity-KnifeEdge})
  \item All values (Section~\ref{sec:Selectivity-AllValues})
  \item All values bounded (Section~\ref{sec:Selectivity-AllValuesBounded})
  \item Increasing (Section~\ref{sec:Selectivity-Increasing})
  \item Logistic (Section~\ref{sec:Selectivity-Logistic})
  \item Inverse logistic (descending logistic?) (Section~\ref{sec:Selectivity-InverseLogistic})
  \item Logistic producing (Section~\ref{sec:Selectivity-LogisticProducing})
  \item Double normal (Section~\ref{sec:Selectivity-DoubleNormal})
  \item Double normal plateau (Section~\ref{sec:Selectivity-DoubleNormalPlateau})
  \item Double normal stock synthesis (Section~\ref{sec:Selectivity-DoubleNormalStockSynthesis})
  \item Double exponential (Section~\ref{sec:Selectivity-DoubleExponential})
  \item Compound-left (Section~\ref{sec:Selectivity-CompoundLeft})
  \item Compound-right (Section~\ref{sec:Selectivity-CompoundRight})
  \item Compound-middle (Section~\ref{sec:Selectivity-CompoundMidde})
  \item Compound-all (Section~\ref{sec:Selectivity-CompoundAll})
  \item Multi-selectivity  (Section~\ref{sec:Selectivity-MultiSelectivities})
% \item Cubic spline (Not yet implemented)
\end{itemize}

See Figure \ref{fig:select examples} for example plots of the selectivities (p. \pageref{fig:select examples}).

\subsubsection[Constant]{{constant}}\label{sec:Selectivity-Constant}

The constant selectivity is constant power function ($ax^b+c$) for all age/lengths greater than $\beta$. For $x < \beta$, the selectivity is zero.

\begin{equation}
f(x)= \begin{cases}
	0, & \text{if $x < \beta$} \\
	ax^b + c, & otherwise \\
\end{cases}
\end{equation}

For a simple constant selectivity (i.e., where $f(x) = 1$ for all \ifAgeBased age/lengths \else length bins \fi set $a$ and $b = 0$ and $c = 1$). 

To implement an inverse length selectivity (to use, for example, as a selectivity for natural mortality ($M$) where $M$ is inversely proportional to length --- see \cite{lorenzen_natural_2022}), use \ifAgeBased \subcommand{length\_based}=\argument{true} with \fi $a=1$, $b=-1$, and $c=0$).

The constant selectivity has estimable parameters $a$, $b$, and $c$.

Input fragment: {\small{\begin{verbatim}
type constant
a    0.0
b    0.0
c    1.0
beta 0.0 # the default is 0.0
\end{verbatim}}}

\subsubsection[Knife-edge]{\argument{knife\_edge}}\label{sec:Selectivity-KnifeEdge} 

\begin{equation}
f(x)= \begin{cases}
  0, & \text{if $x < E$} \\
  \alpha, & \text{if $x \ge E$}\\
  \end{cases}
\end{equation}

The knife-edge ogive has the estimable parameter E and a non-estimable scaling parameter $\alpha$, where the default value of $\alpha = 1$.

Input fragment: {\small{\begin{verbatim}
type  knife_edge
e     8
alpha 0.5
\end{verbatim}}}

\subsubsection[All-values]{\argument{all\_values}}\index{Selectivities!All-values}\label{sec:Selectivity-AllValues}

\begin{equation}
f(x)=V_x
\end{equation}

The all-values selectivity has estimable parameters $V_{low}$, $V_{low+1}$ \ldots $V_{high}$. The selectivity value for each age/length class must be set.

\subsubsection[All-values-bounded]{\argument{all\_values\_bounded}}\index{Selectivities!All-values-bounded}\label{sec:Selectivity-AllValuesBounded}

\begin{equation}
f(x)=\begin{cases}
		 0, & \text{if $x < L$} \\
		 V_x, & \text{if $L \le x \le H$} \\
		 V_H, & \text{if $x > H$}
  \end{cases}
\end{equation}

The all-values-bounded selectivity has non-estimable parameters L and H. The estimable parameters are $V_L$, $V_{L+1}$ \ldots $V_H$. Selectivity values for each age/length class from $L \ldots H$ must be set.

Selectivities \subcommand{all\_values} and \subcommand{all\_values\_bounded} can be included in additional priors using the syntax

{\small{\begin{verbatim}
		@selectivity maturity
		type all_values
		v 0.001 0.1 0.2 0.3 0.4 0.3 0.2 0.1

		## encourage classes 3-8 to be smooth.
		@additional_prior smooth_maturity
		type vector_smooth
		parameter selectivity[maturity].values{3:8}
		\end{verbatim}}}

\subsubsection[Increasing ]{\argument{increasing}}\index{Selectivities!Increasing} \STATUS{Untested?}\label{sec:Selectivity-Increasing}

\begin{equation}
f(x)=\begin{cases}
	  0, & \text{if $x < L$} \\
	  f(x-1)+ \pi_x(\alpha-f(x-1)), & \text{if $L \le x \le H$} \\
	  f(\alpha), & \text{if $x \ge H$} \\
  \end{cases}
\end{equation}

The increasing ogive has non-estimable parameters $L$ and $H$. The estimable parameters are $\pi_L$, $\pi_{L+1}$ \ldots $\pi_H$; if these are estimated, they should always be constrained to be between 0 and 1. $\alpha$ is a scaling parameter, with default value of $\alpha = 1$. The increasing ogive is similar to the \textit{all-values-bounded} ogive, and is constrained to be non-decreasing.

Input fragment: {\small{\begin{verbatim}
type  increasing
l     3
h     7
v     0.2 0.3 0.4 0.5 0.6

\end{verbatim}}}
\subsubsection[Logistic]{\argument{logistic}}\index{Selectivities!Logistic}\label{sec:Selectivity-Logistic}

\begin{equation}
  f(x) = \begin{cases}
  	0, & \text{if $x < \beta$} \\
   \alpha / [1+19^{(a_{50}-x)/a_{to95}}], & \text{otherwise} \\
   \end{cases}
\end{equation}

The logistic selectivity has estimable parameters $a_{50}$ and $a_{to95}$. $\alpha$ is a scaling parameter, with default value of $\alpha = 1$. $\beta$ is the minimum age/length to which the selectivity applies. 

The logistic selectivity has values $0.5 \alpha$ at $x=a_{50}$ and $0.95 \alpha$ at $x=a_{50}+a_{to95}$. For $x < \beta$, the selectivity is zero.

\subsubsection[Inverse logistic]{\argument{inverse\_logistic}}\index{Selectivities!Inverse-logistic}\label{sec:Selectivity-InverseLogistic} 

\begin{equation}
  f(x) = \begin{cases}
	0, & \text{if $x < \beta$} \\
    \alpha - \alpha / [1+19^{(a_{50}-x)/a_{to95}}], & \text{otherwise} \\
   \end{cases}
\end{equation}

The inverse logistic selectivity has estimable parameters $a_{50}$ and $a_{to95}$. $\alpha$ is a scaling parameter, with default value of $\alpha = 1$. 

The inverse logistic selectivity has values $0.5 \alpha$ at $x=a_{50}$ and $0.95 \alpha$ at $x=a_{50}-a_{to95}$. For $x < \beta$, the selectivity is zero.


Input fragment: {\small{\begin{verbatim}
type  inverse_logistic
a50   4
ato95 1
alpha 0.5 # the default is 1.0
beta  0.0 # the default is 0.0
\end{verbatim}}}

\subsubsection[Logistic producing]{\argument{logistic\_producing}}\index{Selectivities!Logistic-producing}\label{sec:Selectivity-LogisticProducing}

\begin{equation}
f(x)=\begin{cases}
	  0, & \text{if $x < L$} \\
	  \lambda(L), & \text{if $x=L$} \\
	  \left( \lambda(x)-\lambda(x-1) \right) / \left( 1-\lambda(x-1) \right), & \text{if $L < x < H$} \\
	  1, & \text{if $x \ge H$} \\
  \end{cases}
\end{equation}

The logistic-producing selectivity has non-estimable parameters $L$ and $H$. The estimable parameters are $a_{50}$ and $a_{to95}$. $\alpha$ is a scaling parameter, with default value of $\alpha = 1$.

For category transitions, $f(x)$ represents the proportion moving, not the proportion that have moved. This selectivity was designed for use in an age-based model to model either movement or maturity. In such a model, a logistic-producing  selectivity will, in the absence of other influences, make the proportions moved or mature follow a logistic curve with parameters $a_{50}$ and $a_{to95}$.

Input fragment: {\small{\begin{verbatim}
type  logistic_producing
l      2
h      8
a50    4
ato95  1
# alpha 1.0
\end{verbatim}}}

CASAL's implementation of this selectivity adds the following checks.
\begin{align*}
	&for(\text{i in selectivity\_bins}) \\
	& \ \ if((a_{50} - i)/a_{to95} < -5.0))\\
	&selectivity[i] = 1 \\ \\ 
&for(\text{i in selectivity\_bins}) \\
& \ \ if((a_{50} - i)/a_{to95} > 5.0))\\
&selectivity[i] = 0
\end{align*}

\CNAME\ does not have these checks, so when you plot selectivities they may look different at the edges.

\subsubsection[Double-normal]{\argument{double\_normal}}\index{Selectivities!Double-normal}\label{sec:Selectivity-DoubleNormal}

\begin{equation}
  f(x) = \begin{cases}
     0, & \text{if $x < \beta$} \\
    \alpha 2^{-[(x- \mu)/\sigma_L ]^2}, & \text{if $x \leq \mu$} \\
    \alpha 2^{-[(x- \mu)/\sigma_R ]^2}, & \text{if $x \ge \mu$}\\
  \end{cases}
\end{equation}

The double-normal selectivity has estimable parameters $a_1$, $s_L$, and $s_R$. $\alpha$ is a scaling parameter, with default value of $\alpha = 1$. 

It has values $\alpha$ at $x=a_1$, and $0.5 \alpha$ at $x=a_1-s_L$ and $x=a_1+s_R$. For $x < \beta$, the selectivity is zero.

Input fragment: {\small{\begin{verbatim}
type  double_normal
mu       6   # age/length class at switch over from left to right normal curves
             #  = mean for both normal curves
sigma_1  1   # standard deviation for left normal
sigma_2 10   # standard deviation for right normal
# alpha  1.0
# beta   0.0
\end{verbatim}}}

\subsubsection[Double-normal-plateau]{\argument{double\_normal\_plateau}}\index{Selectivities!Double-normal-plateau}\label{sec:Selectivity-DoubleNormalPlateau}

\begin{equation}
f(x) = \begin{cases}
0, & \text{if $x < \beta$} \\
\alpha 2^{-[(x- a1)/\sigma_L ]^2}, & \text{if $x \leq a1$} \\
\alpha, 							& \text{if $a1 \le x \leq a1 + a2 $}\\
\alpha 2^{-[(x- (a1 + a2))/\sigma_R ]^2}, & \text{if $x \ge a1 + a2 $}\\
\end{cases}
\end{equation}

The \texttt{double\_normal\_plateau} ogive has estimable parameters \(a1\), \(a2\), \(\sigma_l\), \(\sigma_r\), and \(\alpha\). 

When \(\alpha = 1\) and \(a2 = 0\), it is identical to the \texttt{double\_normal}, and otherwise follows a double normal form with values \(\alpha\) at \(a1 \le x \leq a1+a2\), and \(0.5 x \alpha\) at \(x= a1-\sigma_l\) or \(x=a1+a2+\sigma_r\). For $x < \beta$, the selectivity is zero.

Input fragment: {\small{\begin{verbatim}
		type  double_normal_plateau
		a1       6    
		a2 	     2
		sigma_1  1   # standard deviation for left normal
		sigma_2  10  # standard deviation for right normal
		# alpha 1.0
		# beta  0.0
		\end{verbatim}}}

\subsubsection[Double-normal-stocksynthesis]{\argument{double\_normal\_stock\_synthesis}}\index{Selectivities!Double-normal-stocksynthesis}\label{sec:Selectivity-DoubleNormalStockSynthesis}

Double normal with defined initial and final selectivity values which is based on the Stock Synthesis 3 implementation. The \texttt{ascending} and \texttt{descending} are expected in log space (This should be taken out and dealt with by the parameter transformation class in future releases).

It is common to estimate the following parameters \subcommand{peak}, \subcommand{width}, \subcommand{ascending} and \subcommand{descending}. \subcommand{y1} can be explored, but can be difficult to estimate, as generally this represents the age or length categories that are not well observed.

Input fragment: {\small{\begin{verbatim}
		type  double_normal_stock_synthesis
		peak  7.5 # age or length for the plateau, should be between L and H
		y0	 -10  # selectivity at min-age or first length bin see below for units
		y1	  0.5 # selectivity at max-age or last length bin see below for units
		descending 	  # log(age or length) of descending limb (shape of right hand side) 
		ascending 	  # log(age or length) of ascending limb (shape of left hand side) 
		width   3     # width of plateau
		L  		1     # first length bin
		H 		10    # last age bin
		#alpha 1.0
		\end{verbatim}}}

The parameter values \texttt{y0} and \texttt{y1} are transformed by the selectivity class as follows

\[
f(x) = \frac{1}{1+exp(-1.0 x )}
\] 

This is to ensure the values stay between 0 and 1. The down side is that the starting values are a little abstract. The rule of thumb is small numbers, e.g., -10, will result in selectivity values close to zero and large values, e.g., 10, will result in selectivity values close to one.

\subsubsection[Double-exponential]{\argument{double\_exponential}}\index{Selectivities!Double-exponential}\label{sec:Selectivity-DoubleExponential}

\begin{equation}
f(x)=\begin{cases}
	  0, & \text{if $x < \beta$} \\
	  \alpha y_0(y_1 / y_0)^{(x-x_0)/(x_1-x_0)}, & \text{if $x \le x_0$} \\
	  \alpha y_0(y_2 / y_0)^{(x-x_0)/(x_2-x_0)}, & \text{if $x > x_0$} \\
  \end{cases}
\end{equation}

The double-exponential selectivity has non-estimable parameters $x_1$ and $x_2$. The estimable parameters are $x_0$, $y_0$, $y_1$, and $y_2$.  $\alpha$ is a scaling parameter, with default value of $\alpha = 1$. 

This selectivity curve can be "U-shaped". Bounds for $x_0$ must be such that $x_1 < x_0 < x_2$. With $\alpha=1$, the selectivity passes through the points $(x_1, y)$, $(x_0, y_0)$, and $(x_2, y_2)$. If both $y_1$ and $y_2$ are greater than $y_0$ the selectivity is "U-shaped" with minimum at $(x_0, y_0)$.  For $x < \beta$, the selectivity is zero.

Input fragment: {\small{\begin{verbatim}
type  double_exponential
x0    15   # age/length at middle point
y0    0.1  # selection at x0; here a minimum --> U shape
x1    1    # left point
y1    0.5  # selection at x1
x2    30   # right point
y2    0.8  # selection at x2
# alpha 1.0
# beta  0.0
\end{verbatim}}}

%\subsubsection[Spline]{\argument{spline}}\index{Selectivities!Spline}\label{sec:Selectivity-Spline}
%
%The spline selectivity implements a cubic spline that has non-estimable knots, and an estimable value for each knot. The cubic spline is either (i) a natural splines where the second derivatives are set to 0 at the boundaries, i.e., the values at the boundaries are horizontal, (ii) a spline with a fixed first derivative at the boundaries (linear, but not necessarily horizontal) and (iii) spline which turns into a parabola at the boundaries.
%

\subsubsection[Compound-Left]{\argument{compound\_left}}\index{Selectivities!Compound-Left}\label{sec:Selectivity-CompoundLeft}

The compound left selectivity was used in some oyster stock assessments but was not documented in CASAL user manual.

\begin{align*}
y_1 & = \frac{\left(1 - a_{min}\right)}{\left(1 + 19^{(a_{50} - x)/a_{to95}}\right)}  + a_{min}\\
y_2 & = 1.0 - \frac{1}{\left(1 + 19^{(left_{mu} + to\_right_{mu} - x)/\sigma}\right)}\\
f(x)  &= 	y_1 y_2
\end{align*}


\subsubsection[Compound-Right]{\argument{compound\_right}}\index{Selectivities!Compound-Right}\label{sec:Selectivity-CompoundRight}

The compound right selectivity was used in some oyster stock assessments but was not documented in CASAL user manual.

\begin{align*}
y_1 & = \frac{\left(1 - a_{min}\right)}{\left(1 + 19^{(a_{50} - x)/a_{to95}}\right)}  + a_{min}\\
y_2 & = \frac{1}{\left(1 + 19^{(left_{mu} + to\_right_{mu} - x)/\sigma}\right)}\\
f(x)  &= 	y_1 y_2
\end{align*}

\subsubsection[Compound-Middle]{\argument{compound\_middle}}\index{Selectivities!Compound-Middle}\label{sec:Selectivity-CompoundMidde}

The compound middle selectivity was used in some oyster stock assessments but was not documented in CASAL user manual.

\begin{align*}
y_1 & = \frac{\left(1 - a_{min}\right)}{\left(1 + 19^{(a_{50} - x)/a_{to95}}\right)}  + a_{min}\\
y_2 & = \frac{1}{\left(1 + 19^{(left_{mu} + to\_right_{mu} - x)/\sigma}\right)}\\
y_3 & = 1.0 -  \frac{1}{\left(1 + 19^{(left_{mu} + to\_right_{mu} - x)/\sigma}\right)}\\
f(x)  &= 	y_1 y_2 y_3
\end{align*}

\subsubsection[Compound-All]{\argument{compound\_all}}\index{Selectivities!Compound-All}\label{sec:Selectivity-CompoundAll}

The compound all selectivity was used in some oyster stock assessments but was not documented in CASAL user manual.

\begin{align*}
	f(x)  & = \frac{\left(1 - a_{min}\right)}{\left(1 + 19^{(a_{50} - x)/a_{to95}}\right)}  + a_{min}
\end{align*}

\subsubsection[Multi Selectivity ]{\argument{multi\_selectivity}}\index{Selectivities!MultiSelectivity} \label{sec:Selectivity-MultiSelectivities}

This selectivity allows users to configure models that have different selectivity functions applied in different years. For each year of the model, a selectivity is defined using a label of a selectivity defined elsewhere in the input configuration files. Each year can be either unique or else blocks of years can define a different selectivity. For example,

{\small{\begin{verbatim}
			@selectivity fishery_selectivity
			type multi_selectivity
			years 1990:2000 2001:2010
			selectivity_labels early_2000s_sel * 10 post_2000_sel * 9
			default_selectivity post_2000_sel
			
			@selectivity early_2000s_sel
			type logistic
			a50 4
			ato95 2.3
			
			@selectivity post_2000_sel
			type double_normal
			mu 6
			sigma_l 2
			sigma_r 23
\end{verbatim}}}

For missing years, the \argument{default\_selectivity} is applied. Missing years in projections use the \argument{projection\_selectivity} if the subcommand has been supplied, otherwise the \argument{default\_selectivity} will be used for those missing years.

\begin{figure}[H]
	\centering
	\includegraphics[scale = 0.9]{Figures/Selectivities.jpg}
	\caption{Examples of the selectivities}
	\label{fig:select examples}
\end{figure}

\subsection{\I{Time-varying parameters}}\label{sec:TimeVarying} \STATUS{Untested}

Any parameter can be varied annually for blocks of years or in specific years within the model run. For years that are not specified, the parameter will default to the input, or if in an iterative state such as estimation mode, the value being trialled at that iteration. The value used in the configuration file, input parameter file, or trialled value during estimation should be applied during initialisation phases.

Method types for a time-varying parameter are:

\begin{itemize}
\item \subcommand{constant},
\item \subcommand{exogenous},
\item \subcommand{linear},
\item \subcommand{annual\_shift},
\item \subcommand{random\_walk}, and
\item \subcommand{random\_draw}.
\end{itemize}

This option allows for a parameter to be fixed in a year, or be the result of a deterministic or stochastic process. Note that the stochastic time-varying methods (e.g., random\_walk and random\_draw) are intended for simulations or projections --- they should not be used in estimation as they utilise random numbers to generate parameter values. 

To implement a hierarchical model using the time-varying functionality, use MCMC estimation as a way to calculate the integral which is required to obtain unbiased estimates. Here, the prior parameter values need to be estimated using hyper-priors. In an MCMC context, a Gibbs sampler is assumed. That is, every draw is from a conditional distribution and so every draw is a candidate value.

When allowing time-varying parameters (such as in catchability coefficients or selectivities parameters), a model is given freedom to more closely match the observed data. Time-varying parameters can be used to allow the mean or shape parameters of selectivities to change between years, and potentially as a function of a linked variable.

An example of this is in the New Zealand hoki stock assessment where the $\mu$ and $a_{50}$ parameters are allowed to shift depending on when the fishing during the season occurs. Descriptive analysis showed that when fishing was earlier relative to other years, smaller fish were caught and vice versa. This can be shown in the Examples/2stock directory, implemented at line: \texttt{382} in the \texttt{population.csl2} file.

\subsubsection[Constant (year blocks)]{\argument{constant}}\index{Time-varying Parameters!Constant}\label{sec:TimeVarying-Constant}

This option allows a parameter to have an different value during specified years than the rest of the model run. This value can be estimated.

To allow survey catchability to be different in the year block 1975 to 1988 from the rest of the series we write:

{\small{\begin{verbatim}
		@time_varying q_time_var
		type          constant
		parameter     catchability[survey_q].q
		years         1975:1988
		values        0.001 # the same for all years
		\end{verbatim}}}

To estimate catchability for 1975 and 1976, use the following:

{\small{\begin{verbatim}
		@estimate q_time_var
		type uniform   #prior
		parameter time_varying[q_time_var].values{1975:1976}
		lower_bound 1e-6 1e-6
		upper_bound 2    2
		\end{verbatim}}}

To make the catchability be same over the year block we need to estimate it for one year (say 1975) and use the \textit{same} subcommand to make the others take the same value

{\small{\begin{verbatim}
		@estimate q_time_var
		type uniform
		parameter time_varying[q_time_var].values{1975}
		same      time_varying[q_time_var].values{1976:1988}
		lower_bound 1e-6
		upper_bound 2
		\end{verbatim}}}

\textbf{Caution}: do not estimate both the actual parameter and its time-varying counterpart, as the time-varying value will overwrite the actual parameter making the actual value unidentifiable. \TODO{this section required a re-write and clarification of the code and the text}

\subsubsection[Linear]{\argument{linear}}\index{Time-varying Parameters!Linear}\label{sec:TimeVarying-Linear}

Parameters are shifted based on a linear trend, with a given slope and intercept. An example of this is an exploitation selectivity parameters that may increase or decrease between years based a simple linear trend.

\begin{equation}
	\delta_y = a E_y + b
\end{equation}
\begin{equation}
	\theta'_y = \theta_y + \delta_y
\end{equation}

where $\delta_y$ is the shift or deviation in parameter $\theta_y$ in year $y$ to generate the new parameter value in year $y$ ($\theta'_y$). $a$ is an estimable slope parameter and $b$ is the linear trend intercept.

\subsubsection[Exogenous]{\argument{exogenous}}\index{Time-varying Parameters!Exogenous}\label{sec:TimeVarying-Exogenous}

Parameters are shifted based on an exogenous variable. An example of this is an exploitation selectivity parameters that may vary between years based on known changes in exploitation behaviour such as season, start time, and average depth of exploitation.

\begin{equation}
	\delta_y = a(E_y - \bar{E})
\end{equation}
\begin{equation}
	\theta'_y = \theta_y + \delta_y
\end{equation}

where $\delta_y$ is the shift or deviation in parameter $\theta_y$ in year $y$ to generate the new parameter value in year $y$ ($\theta'_y$). $a$ is an estimable shift parameter, $E$ is the exogenous variable, and $E_y$ is the value of this variable in year $y$. For more information readers can see \cite{francis_03}.

\subsubsection[Annual shift]{\argument{annual\_shift}}\index{Time-varying Parameters!Annual shift} \label{sec:TimeVarying-AnnualShift}

A parameter generated in year $y$ ($\theta'_y$) depends on the value specified by the user ($\theta_y$) along with three coefficients $a$, $b$, and $c$

\begin{equation}
	\bar{\theta}_y = \frac{\sum_{y}^Y\theta_y}{Y}
\end{equation}
\begin{equation}
	\theta'_y = a \bar{\theta}_y + b\bar{\theta}_y^{2} + c\bar{\theta}_y^{3}
\end{equation}

\subsubsection[Random Walk]{\argument{random\_walk}}\index{Time-varying Parameters!Random Walk}\label{sec:TimeVarying-RandomWalk}

A random deviate drawn from a standard normal distribution is added to the previous year's value. This option has an estimable parameter $\sigma_p$ for each time-varying parameter $p$. For reproducible modelling when using stochastic functionality, set the random seed (see Section~\ref{sec:CommandLineArguments}).

{\small{\begin{verbatim}
			@time_varying q_time_var
			type          random_walk
			parameter     catchability[survey_q].q
			distribution  normal
			mean          0
			sigma         3
\end{verbatim}}}

If the \texttt{parameter} specified in the \command{time\_varying} block is associated with an \command{estimate} block, then the parameter is constrained to stay within the lower and upper bounds of the \command{estimate} block.

\textbf{WARNING:} if the parameter does not have an associated \command{estimate} block then there is no safeguard against the application of a random deviate resulting in parameter values which cause the model to fail, i.e., generates NA or INF values. To avoid this, specify an \command{estimate} block even though the parameter is not actually being estimated; see the example syntax below.

A constraint whilst using this functionality is that a parameter cannot be less than 0.0. If it is then \CNAME\ sets it equal to 0.01.

{\small{\begin{verbatim}
			@estimate survey_q_est
			type      uniform
			parameter catchability[survey_q].q
			lower_bound 1e-6
			upper_bound 10
\end{verbatim}}}

This configuration will insure the random walk time-varying process will set the any new candidate values within the lower and upper bound of the \command{estimate} block.

\subsubsection[Random Draw]{\argument{random\_draw}}\index{Time-varying Parameters!Random Draw}\label{sec:TimeVarying-RandomDraw}

A random deviate drawn from a standard normal distribution used. This option has an estimable parameter $\sigma_p$ for each time-varying parameter $p$. For reproducible modelling when using stochastic functionality, set the random seed (see Section~\ref{sec:CommandLineArguments}).

{\small{\begin{verbatim}
			@time_varying q_time_var
			type          random_draw
			parameter     catchability[survey_q].q
			distribution  normal
			mean          0
			sigma         3
\end{verbatim}}}

If the \texttt{parameter} specified in the \command{time\_varying} block is associated with an \command{estimate} block, then the parameter is constrained to stay within the lower and upper bounds of the \command{estimate} block.

\textbf{WARNING:} if the parameter does not have an associated \command{estimate} block then there is no safeguard against the application of a random deviate resulting in parameter values which cause the model to fail, i.e., generates NA or INF values. To avoid this, specify an \command{estimate} block even though the parameter is not actually being estimated; see the example syntax below.

A constraint whilst using this functionality is that a parameter cannot be less than 0.0. If it is then \CNAME\ sets it equal to 0.01.

{\small{\begin{verbatim}
			@estimate survey_q_est
			type      uniform
			parameter catchability[survey_q].q
			lower_bound 1e-6
			upper_bound 10
\end{verbatim}}}

This configuration will insure the random draw time-varying process will set the any new candidate values within the lower and upper bound of the \command{estimate} block.

\subsection{\I{Equation parser}\label{sec:eq_parser}} \STATUS{Untested}

\CNAME\ has an equation parser, which is currently implemented in Projections (Section~\ref{sec:Project}), Derived quantities (Section~\ref{sec:DerivedQuantity}), and Reports (Section~\ref{sec:Report}).

Examples of syntax for implementing the equation parser are below. For more information on the parser, see \url{https://github.com/nickgammon/parser/blob/master/parser.cpp}

{\small{\begin{verbatim}
		equation process[Recruitment].r0 * 2 #double the recruitment
\end{verbatim}}}

mathematical functions such as \texttt{sqrt()}, \texttt{log()},  \texttt{exp()},  \texttt{cos()}, \texttt{sin()}, and \texttt{tan()} can be used

{\small{\begin{verbatim}
		equation sqrt(process[Recruitment].r0)
\end{verbatim}}}

exponents can be used with \texttt{pow()}

{\small{\begin{verbatim}
		equation pow(2, 3)
\end{verbatim}}}

the absolute value of an equation using \texttt{abs()}

{\small{\begin{verbatim}
		equation abs(sqrt(process[Recruitment].r0) * 1.33)
\end{verbatim}}}

\texttt{if-else} statements can be used

{\small{\begin{verbatim}
		equation if(process[Recruitment].r0 > 23, 44, 55)
		## if R0 is greater than 23 return 44 else return 55
\end{verbatim}}}

\texttt{if-else} statements can also be linked, more complex syntax

{\small{\begin{verbatim}
# if R0 > 23 return 44
# else if R0 < 23 & r0 > 10 return 55
equation if(process[Recruitment].r0 > 23, 44,
         if(process[Recruitment].r0 > 10, 55, 66))
else R0 must be less than 10 return 66
\end{verbatim}}}

Only single values can be referenced, so an equation cannot be applied to a vector, e.g., \subcommand{process[Recruit].recruitment\_multipliers\{1974:1980\}} cannot be referenced. More information on which parameters can be included in the equation parser is available (Section~\ref{sec:syntax}). Any subcommand that has a \texttt{type estimable} could be referenced with the equation parser.

\textbf{Note:} the equation parser will not catch all user configuration errors, such as checking whether a parameter that exists in the system has been populated when it is required.

For example, the wrong year could be misspecified in the case of removals in year $y$ which is based on the state of the population in year $y-1$

{\small{\begin{verbatim}
		parameter process[removals].catch
		year 2015
		equation derived_quantity[percent_b0].values{2020}
\end{verbatim}}}

This example is a valid equation but it will have nonsensical results, since a value for 2020 is to be calculated using values for 2015. Although the equation parser adds flexibility, it is easy to incorrectly specify equations.

\subsection{\I{Specifying projections}}\label{sec:Project} \STATUS{Untested}

Given a set of estimated parameter values from a \textit{-e} or a MCMC run,
the model can be projected. Projection years are after the model run years, and are defined in the \command{model} command block using the \subcommand{final\_projection\_year} subcommand, i.e., projection years are \subcommand{final\_year + 1} through to \subcommand{final\_projection\_year}.

Parameter values for the projected years can be specified in a stochastic way or fixed at some value (the default is the estimated value if the parameter is not time-varying) and these are specified in the \command{project} block,

{\small{\begin{verbatim}

@project Future_ycs  # label
type       lognormal_empirical  # which method to use
parameter  process[Recruitment].ycs_values
years      2012:2016
multiplier 1
... # any other parameters
\end{verbatim}}}

The subcommands \subcommand{years} and \subcommand{parameter} are common to all projection methods. Subcommand \subcommand{years} specifies the years to apply the new values to for the parameter in \subcommand{parameter}. Note that the years can be before the \textit{final\_year}, e.g., it is normal to vary the last few recruitment multipliers (YCS) in a projection run because they are usually poorly estimated or they have been set to 1.  The argument \argument{multiplier} is a constant which is multiplied with the projected value after it has been generated. The \subcommand{type} subcommand gives the method to use to generate new parameter values.

\CNAME\ allows any estimable parameter to be specified in a \command{project} block and then varied from the estimated value in a projection. The available projection types for these parameters include:

\begin{itemize}
	\item constant
	\item lognormal
	\item empirical-lognormal
	\item empirical re-sampling
	\item user-defined
\end{itemize}

\CNAME\ has no default projection properties for parameters that are specified by year, e.g., recruitment multiplier parameters, time-varying parameters, and as a special case, future catches. For these, projections  must have a \command{project} command block. For example, \CNAME\ will produce errors if run in projection mode without a \command{project} block for the \subcommand{recruitment\_multipliers} parameter being specified.

DEPRECATED: \textbf{Note for the year class parameters:} the definition of year applies to the \argument{ycs\_years}, not the model years. As defined in Section \ifAgeBased \ref{sec:Process-RecruitmentBevertonHolt}\else \ref{sec:Process-RecruitmentBevertonHolt}\fi, \argument{ycs\_years} are offset between the time of spawning and when individuals are added to the partition.

Future catches are also specified in a \command{project} block, one for each fishery (see \ref{sec:Project-Catch} for examples). Here, a fishery is reference in the \textit{parameter} subcommand with the \textit{method\_} fragment to identify it,

 \textit{process[block label].method\_$[$fisheries label$]$},

For a process called \textit{Fishing} that has three fisheries defined, it would be \textit{process[Fishing].method\_pot} to specify the fishery labelled \textit{pot}.

The \CNAME\ command to run the model in projection mode is \texttt{Casal2 -f 1}. This functionality allows for the exploration of many scenarios with a single set of parameters. The number of projections should be greater than 1 only if applying a projection type that is stochastic.

The \texttt{-{}-tabular} flag should be used when running projections after a Bayesian analysis. This option will output a tabular report (see Section~\ref{sec:Tabular}) which can then be analysed in \R.

An example of the command line evocation is

\textit{casal2 -f 1 -i mcmc.txt --tabular $>$ projection.out.txt}

where \textit{mcmc.txt} is output from a MCMC run, one parameter set per row, which will give one projection per row, and \textit{projection.out.txt} will contain one row for each MCMC run in each of the reports specified in (usually) the \textit{Report.csl2} file (quantities as specified in the \textit{Report.csl2} file).

For a projection run in \CNAME, the model is initialised and run through the model years from \argument{start\_year} to \argument{final\_year}. During this run mode \CNAME\ stores all parameter values so that projection classes can allow parameters before \subcommand{final\_year} to be projected. The model then is re-run from \argument{start\_year} to \argument{projection\_final\_year}, where any parameter can either be fixed or drawn from a stochastic distribution or process.


\subsubsection{\I{Projection methods}\label{sec:ProjectionMethods}}

This section lists all the projections classes available, their functionality, and an example of the syntax.

\paragraph[Constant]{The constant projection type, \argument{constant}}\index{Projections!Constant}\label{sec:Project-Constant}

A parameter can either be fixed during all projection years or specified individually for each projection year. This is a deterministic assumption, where the parameter is assumed to be known without error during projection years.

{\small{\begin{verbatim}
		@project Future_ycs
		type      constant
		parameter process[Recruitment].recruitment_multipliers
		years     2012:2016
		values    1 2 1 2 0.5  # "values 3" means all years use 3
		multiplier 1
		\end{verbatim}}}


\paragraph[Multiple Values]{The multiple values projection type, \argument{multiple\_values}}\index{Project-MultipleValues}\label{sec:Project-MultipleValues}

Users can specify a set of values for each projection year for each row in the \texttt{-i} or \texttt{-I} file using the \subcommand{multiple\_values} projection class. This gives users flexibility in specifying a range of bespoke values during projections and including uncertainty in the projected values. See below for an example on how to configure this class.

{\small{\begin{verbatim}
@project future_disease_rates
type multiple_values
parameter process[dtransition].proportions{disease}  
years 2024:2026
table values
2024 2025 2026
0.1 0.2 0.3
0.3 0.4 0.5
end_table
\end{verbatim}}}

\paragraph[Empirical sampling]{Sampling from a range of years, type  \argument{empirical\_sampling}}\index{Projections!Empirical sampling}\label{sec:Project-EmpiricalSampling}

Parameters that have time components associated with them can be sampled uniformly with replacement over a range of years and used as values for the projected years. The year range to sample from is between \argument{start\_year} and \argument{final\_year}:

{\small{\begin{verbatim}
		@project   Future_ycs
		type       empirical_sampling
		parameter  process[Recruitment].standardised_recruitment_multipliers
		years     2012:2016
		start_year 1988     # re-sample from estimated values
		final_year 2008     # from 1988 to 2008 inclusive
		multiplier 1
		\end{verbatim}}}

Note that when projecting recruitment\_multipliers, the \emph{unstandardised} values are used. This may cause projections to be other than what was intended unless these average one (or some other chosen value). If not using a transformation (e.g., the simplex transformation for recruitments), a vector average penalty should be applied (and the resulting recruitment multipliers checked) to ensure the average recruitment multiplier is equal to one over some pre-defined range. 

\paragraph[Lognormal]{Sampling from a Lognormal distribution, type  \argument{lognormal}}\index{Projections!Lognormal}\label{sec:Project-LogNormal} \STATUS{Untested}

The parameters are drawn from a Gaussian (Normal) distribution in log space and exponentiated  to result in the Lognormal distribution

\begin{equation}\label{eq:lognormal}
X_p = exp(\epsilon_p - \sigma^2 / 2)
\end{equation}

where $\epsilon_p\stackrel{iid}{\sim}N(\mu,\sigma)$ and $X_p$ is the projected value for parameter $X$, and $\mu$ and $\sigma$ are the mean and standard deviation on the log scale.

An example of applying this process to draw future year class parameters from a Lognormal distribution with mean 1 and standard deviation 0.8

{\small{\begin{verbatim}
		@project Future_ycs
		type     lognormal
		parameter process[Recruitment].recruitment_multipliers
		years     2012:2016
		mean      0     # mean 1 on un-transformed scale
		sigma     0.8   # log scale
		multiplier 1
		\end{verbatim}}}

\paragraph[Lognormal-Empirical]{Sampling from a Lognormal distribution where the  variance is estimated from values over a specified year range, type  \argument{lognormal\_empirical} }\index{Projections!Lognormal-Empirical}\label{sec:Project-LogNormalEmpirical} \STATUS{Untested}

This method applies a Lognormal draw as in the \argument{Lognormal} method above and specifies a year range from which the standard deviation of the distribution is calculated. Then equation~\eqref{eq:lognormal} is used to generate future values with a specified $\mu$ and empirically calculated $\sigma$,

{\small{\begin{verbatim}
		@project Future_ycs
		type      lognormal_empirical
		parameter process[Recruitment].recruitment_multipliers
		years     2012:2016
		mean      0
		start_year 1988   # range of years to take the
		final_year 2008   # values for \math{\sigma}
		multiplier 1
		\end{verbatim}}}

%\paragraph[User Defined]{Sample from a user-defined function, \argument{user\_defined}}\index{Projections!User Defined}\label{sec:Project-UserDefined} \STATUS{Untested}

%This method uses the equation parser to calculate the values to use in the projection. This was set up to define and apply harvest control rules (e.g., apply a management action such as changing catch limits based on the current or previous state).

%In fisheries models, this option can be used to calculate the projected catch based on an exploitation rate multiplied by the vulnerable biomass, where the exploitation rate is based on a rule (Figure~\ref{fig:HCR}).

%\begin{figure}[!h]
%	\includegraphics[scale=0.9]{Figures/HarvestControlRules.png}
%	\caption{\textbf{Examples of control rules based on current stock status.}}
%	\label{fig:HCR}
%\end{figure}

%\pagebreak
%{\small{\begin{verbatim}
%		@project HCR_2015
%		type       user_defined
%		parameter process[Instantaneous_Mortality].method_Sub_Ant_F
%		years 2015
%		equation if(derived_quantity[SSB].values{2014} / process[Recruitment].b0 <= 0.1, 0.0,
%		if(derived_quantity[SSB].values{2014} / process[Recruitment].b0 > 0.1 &&
%		derived_quantity[SSB].values{2014} / process[Recruitment].b0 < 0.2,
%		derived_quantity[SSB].values{2014} * derived_quantity[SSB].values{2014}
%		/ process[Recruitment].b0,
%		derived_quantity[SSB].values{2014} * 0.2))
%		\end{verbatim}}}

%Care should be taken when writing user-defined equations. The above equation is: if $\%B_{2014} \leq 0.1$ then set next year's catch to 0.0, else if $\%B_{2014} > 0.1 \text{ } \& \text{ } \%B_{2014} \leq 0.2$ then set next year's catch equal to $\%B_{2014} \times SSB_{2014}$, else set next year's catch to $0.2 SSB_{2014}$.

\paragraph[Catches]{Specifying catch for projections }\index{Projections!Catches}\label{sec:Project-Catch}

Catches are unique in that they are known inputs in a table format. For example, to project catches that are in a table

{\small{\begin{verbatim}
# fishing process
@process Fishing
type mortality_instantaneous_retained
m 0.17*6
time_step_proportions 1
relative_m_by_age One*6   # For age based model. 
                          # For length-based models use relative_m_by_length
categories *
table catches
year\begin{center}
	\begin{center}
		
	\end{center}
\end{center}
hingPot	Recreation
1900	0	0	0
1901	13.2	0	22.9
1902	26.4	0	23.5
1903	39.6	0	24
end_table

# projection block
@project future_catch
type      constant
parameter process[Fishing].method_fishingpot
years     2020:2029
values    4000
\end{verbatim}}}

This uses the syntax \texttt{block\_type[block\_label].method\_fishinglabel}. \textbf{Note:} the fishing label which is defined in the table needs to be lower case form in the \command{projection} block. Notice the use of \textit{method\_} syntax to identify the right fishery.


\clearemptydoublepage{}
\section{\I{The estimation section: estimation methods and parameters}\label{sec:Estimation}}

The command and subcommand syntax for the estimation section is given in Section \ref{syntax:Estimation}.

\subsection{\I{Role of the estimation section}\label{sec:role-of-the-estimation-section}}

The role of the estimation section is to define the tasks carried out by \CNAME:

\begin{enumerate}
  \item Define the objective function (see Section \ref{sec:objective-function})
  \item Define the parameters to be estimated (the free parameters, see Section \ref{sec:FreeParameters})
  \item Calculate a point estimate, i.e., the maximum posterior density estimate (MPD) (see Section \ref{sec:estimate-MPD})\index{Maximum posterior density estimate (MPD)}\index{MPD (Maximum posterior density estimate)}
  \item Calculate a posterior profile on selected parameters, i.e., for each of a series of values of a parameter, minimise the objective function, allowing the other estimated parameters to vary (see Section \ref{sec:Profile}\index{Profiles})
  \item Generate MCMC\index{Markov chain Monte Carlo (MCMC)} samples from the posterior distribution (see Section \ref{sec:MCMC}\index{MCMC})
  \item Calculate the approximate covariance matrix of the parameters as the inverse of the minimizer\textquoteright{}s approximation to the \I{Hessian}, and the corresponding correlation matrix (see Section \ref{sec:estimate-MPD}\index{Covariance matrix})
\end{enumerate}

The estimation section defines the objective function, parameters of the model, and the method of estimation (point estimates, Bayesian posteriors, profiles, etc.). The objective function is based on a goodness-of-fit measure of the model to observations, the assumed priors, and the penalties. See the observation section for a description of the observations, likelihoods, priors, and penalties.

\subsection{\I{The objective function}\label{sec:objective-function}}

In Bayesian estimation, the objective function is a negative log-posterior,

\begin{equation}\label{objective_function}
Objective(p)= - \sum\limits_i {\log \left[ {L\left( {{\bf{p}}|O_i } \right)} \right]}  - \log \left[ {\pi \left( {\bf{p}} \right)} \right]
\end{equation}

where $\pi$ is the joint prior density of the parameters $p$.

The contribution to the objective function from the likelihood components is described in Section \ref{sec:Likelihood}. In addition to likelihoods, priors (see Section \ref{sec:Priors}) and penalties (see Section \ref{sec:Penalty}) are components of the objective function. Note that if the priors are specified as uniform, then the prior contribution is zero and the optimisation is now a penalised likelihood and not Bayesian.

Penalties can be used to ensure that the estimated parameter values and derived quantities meet certain restrictions. For example, exploitation rate constraints on mortality events (i.e., fisheries) that are not violated (otherwise there is nothing to prevent the model from having abundances so low that the recorded catches could not have been taken); penalties on category transitions (to ensure there are enough individuals to move); penalties such that estimated values are similar or smooth, etc.

Equation~\ref{objective_function} can be reduced to a penalised likelihood equation if all priors are assumed to be uniform. This is because uniform priors have no contribution to the objective function so Equation~\ref{objective_function} reduces to the likelihood components plus penalties.

\subsection{\I{Specifying the parameters to be estimated}\label{sec:FreeParameters}}

The parameters to be estimated (estimables) are defined using \command{estimate} commands (see Section \ref{syntax:Estimation}).

For example, a \command{estimate} command block \index{Estimating parameters}

{\small{\begin{verbatim}
  @estimate male.m
  parameter process[NaturalMortality].m{male}
  lower_bound 0.1
  upper_bound 0.4
  type uniform
\end{verbatim}}}

See Section \ref{sec:parameter-names} for information on how to specify the parameter name. At least one parameter is required to be estimated if doing an estimation \texttt{-e}, profile \texttt{-p}, or MCMC \texttt{-m} run. Initial values for the parameters to be estimated are required, and these values are used as the starting values for the minimiser. However, these values may be overwritten if a set of alternative starting values is provided (i.e., using \texttt{\cname\ -i}, see Section \ref{sec:CommandLineArguments}).

All parameters are estimated within the specified bounds\index{Bounds}. For each parameter estimated, the lower and upper bounds and the prior\index{Priors} (\texttt{type}) (Section \ref{sec:Priors}) must be specified. The bounds and the prior should be chosen carefully as they affect the values over which the minimisers search. Some minimisers convert the lower and upper bounds into a minimisation space (for example -1,1 space for the numerical differences algorithm). If estimating only some elements of a vector, either define each element of the vector to be estimated or fix the others by setting the the lower and upper bounds to the same value as the initial value.

\subsection{\I{Point estimation}\label{sec:estimate-MPD}}\label{sec:Minimiser}

Point estimation is invoked with \texttt{\cname\ -e}, which attempts to find a minimum of the objective function\index{Objective function}. \CNAME\ has multiple minimisation algorithms. There are two automatic differentiation (AD) minimisers: ADOL-C, and BetaDiff (the minimiser used in CASAL). There are also three non-automatic differentiation minimisers: numerical differences, DeltaDiff, and the differential evolution minimiser (\subcommand{de\_solver}). Automatic differentiation  minimisers are recommended for more complex models as they are on average much faster and tend to find a more robust minimum when exploring a complex objective surface.

An important input parameter for most minimisers is the \subcommand{tolerance} parameter. This is the gradient of the objective function, and is used as the stopping rule to define the 'solution' (although a solution may be a local minimum and not the global minimum). Evaluating the robustness of a minimum can be tested with different starting values (i.e., using \texttt{-i free\_parameter\_file.txt}).

Start with the default \subcommand{tolerance} parameter value of 1e-5 and decrease it while developing a model. For a given model, the parameter estimates when minimising with different tolerance may be different.

\subsubsection{\I{The numerical differences minimiser}}\label{sec:Minimiser-GammaDiff}

See Section \ref{syntax:Minimiser-GammaDiff} for the command syntax.

The numerical differences minimiser uses a quasi-Newton minimiser which is a slightly modified implementation of the main algorithm of \cite{779}, and uses an $arcsin$ transformation to ensure parameters remain within bounds.

The minimiser has three kinds of (non-error) exit status, depending on the minimiser:

\begin{itemize}
\item Successful convergence (suggests a local minimum has been found, at least).
\item Convergence failure (a local minimum has not been found, although the results may be 'close enough').
\item Convergence unclear (the minimiser halted but was unable to determine if convergence occurred. The result may be a local minimum, although this can be checked by restarting the minimiser at the final values of the estimated parameters).
\end{itemize}

The maximum number of quasi-Newton iterations\index{Quasi-Newton iterations} and objective function evaluations\index{Objective function evaluations} allowed can be specified. If either limit is exceeded, the minimiser exits with a convergence failure\index{Convergence failure}. Set the maximum number of evaluations and iterations to values larger than the defaults of 300 and 1000, unless convergence is reached with fewer. An alternative starting point of the minimiser can be specified using \texttt{\cname\ -i}.

The minimisers are local optimisation algorithms trying to solve a global optimisation problem. What this means is that, even if a 'successful convergence'\index{Successful convergence} is reached, the solution may be only a local minimum\index{Local minimums}, and not a global one. To diagnose this problem, start multiple runs from different starting points and comparing the results, or do profiles of one or more key parameters and seeing if any of the profiled estimates finds a better optimum than than the original point estimate.

The approximate covariance matrix\index{Covariance matrix} of the estimated parameters can be calculated as the inverse of the minimiser's approximation to the \I{Hessian}, and the corresponding correlation matrix\index{Correlation matrix} is also calculated.

Note that

\begin{itemize}
\item the Hessian approximation develops over many minimiser steps, so if the minimiser has only run for a small number of iterations the covariance matrix can be a very poor approximation; and
\item the inverse Hessian is not a good approximation to the covariance matrix of the estimated parameters, and may not be useful to construct, for example, confidence intervals.
\end{itemize}

Also note that if an estimated parameter has equal lower and upper bounds, it will have entries of '0' in the covariance matrix and \texttt{NaN} or \texttt{-1.\#IND} (depending on the operating system) in the correlation matrix.

{\small{\begin{verbatim}
@minimiser numerical_diff
type numerical_differences
tolerance 1e-6
iterations 2500
evaluations 4000
\end{verbatim}}}

\subsubsection{\I{The DeltaDiff numerical differences minimiser}}\label{sec:Minimiser-DeltaDiff}

See Section \ref{syntax:Minimiser-DeltaDiff} for the command syntax.

DeltaDiff applies the same minimiser as Numerical Differences, expect that it uses $tan$ rescaling for the parameters rather than $arcsin$. This minimiser may perform better than the Numerical Differences minimiser when parameters are very close to zero bounds.

\subsubsection{\I{The differential evolution minimiser}}\label{sec:Minimiser-DESolver}

The differential evolution minimiser (\argument{de\_solver}) is a simple population-based, stochastic function minimizer, but is claimed to be quite powerful in solving minimisation problems. It is a method of mathematical optimization of multidimensional functions and belongs to the class of evolution strategy optimizers.

Initially, the procedure randomly generates and evaluates a number of solution vectors (the population size), each with $p$ parameters. Then, for each generation (iteration), the algorithm creates a candidate solution for each existing solution by random mutation and uniform crossover. The random mutation generates a new solution by multiplying the difference between two randomly selected solution vectors by some scale factor, then adding the result to a third vector. Then an element-wise crossover takes place with probability $P_{cr}$, to generate a potential candidate solution. If this is better than the initial solution vector, it replaces it, otherwise the original solution is retained. The algorithm terminates after either a predefined number of generations (\argument{max\_generations}) or when the maximum difference between the scaled individual parameters from the candidate solutions from all populations is less than some predefined amount \argument{tolerance}.

The differential evolution minimiser can be good at finding global minimums in surfaces that may have local minima. However, the speed of the minimiser, and the ability to find a good minima depend on the number of initial 'populations'. Some authors recommend that the number of populations be set at about $10*p$, where $p$ is the number of free parameters. However, depending on the model, this value can be set to a lower value and still find a robust solution.

There is no proof of convergence for the differential evolution solver, but several papers have found it to be an efficient method of solving multidimensional problems. Some results suggest that it can often find a better minima and may be faster or longer (depending on the actual model specification) at finding a solution when compared with the numerical differences minimiser. Comparisons with automatic differentiation minimisers or other more sophisticated algorithms have not been made.

{\small{\begin{verbatim}
		@minimiser DESolver
		type de_solver
		tolerance 1e-6
		iterations 2500
		evaluations 4000
\end{verbatim}}}

\subsubsection{\I{The BetaDiff minimiser}}\label{sec:Minimiser-BetaDiff}

An automatic differentiation minimiser for non-linear models, This is the minimiser from the original CASAL package, based on ADOL-C.

{\small{\begin{verbatim}
		@minimiser beta_diff
		type beta_diff
		tolerance 1e-6
		iterations 2500
		evaluations 4000
\end{verbatim}}}

\subsubsection{\I{The ADOL-C minimiser}}\label{sec:Minimiser-ADOLC}

An automatic differentiation minimiser for non-linear models. See \url{https://projects.coin-or.org/ADOL-C} for more information. Users do have an option of defining what transformation to apply to convert the parameter \(\theta \in [\theta_{LB}, \theta_{UB}]\) to \(X \in [-1, 1]\), for which optimisation is done. The options are $sin$ or $tan$. Initial model runs suggest this assumption will make a difference to convergence, particularly if there are poorly identified parameters which fall at the bounds, we have found the sin transform is more consistent with the betadiff minimiser. The sin transform
\begin{equation}
	X = \frac{asin(2 * (\theta - \theta_{LB}) / (\theta_{UB} - \theta_{LB}) - 1)}{ 1.57079633}
\end{equation}
%
the consequence of this transformation is when \(X\) is back transformed to \(\theta\) there is a penalty which is added to the minimisation to dissuade parameter values close to the bounds. This penalty is hidden from the reported objective function. If you are interested in it, you can run with \texttt{--loglevel medium} and it should be reported. The back transformation follows,
\begin{equation}
\theta = \theta_{LB} + (\theta_{UB} - \theta_{LB}) * (sin(X * 1.57079633) + 1) / 2;
\end{equation}
%
and penalty
\begin{align}
	&if(-0.9999 - X < 0) & penalty += (X + 0.9999)^2\\
	&if(X - 0.9999 < 0) & penalty += (X - 0.9999)^2
\end{align}
%
% This can be seen  \href{https://github.com/NIWAFisheriesModelling/CASAL2/blob/1b1ed731537dc551674c911da3bf387273a97a92/CASAL2/source/Utilities/Math.h#L245}{here}.

The Tan transform uses transformation

\begin{equation}
X = tan(((\theta - \theta_{LB}) / (\theta_{UB} - \theta_{LB}) - 0.5) * \pi)
\end{equation}
%
and back transform
\begin{equation}
\theta = ((atan(X) / \pi) + 0.5) * (\theta_{UB} - \theta_{LB}) + \theta_{LB}
\end{equation}

{\small{\begin{verbatim}
		@minimiser ADOLC
		type adolc
		step_size 1e-6
		iterations 2500
		evaluations 4000
		tolerance 1e-6
		parameter_transformation sin_transform
\end{verbatim}}}

\subsection{\I{Posterior profiles}}\label{sec:Profile}

If profiles are run using the command \texttt{\cname\ -p}, \CNAME\ will first calculate a point estimate. For each scalar parameter or, in the case of vectors or selectivities, the element of the parameter to be profiled, \CNAME\ will fix its value at a sequence of $n$ evenly spaced numbers ($step$) between the specified lower and upper bounds $l$ and $u$, and calculate a point estimate at each value.

By default $step=10$, and $(l, u)=($lower bound on parameter plus $(range/(2n))$, upper bound on parameter less $(range/(2n))$. Each minimisation starts at the final parameter values from the previous resulting value of the parameter being profiled. \CNAME\ will report the objective function for each parameter value. The initial point estimate should be compared with the profile results, to check at least that none of the other points along the profile have a better objective function value than the initial 'minimum'.

The parameters to be profiled are specified, and optionally the number of steps, and lower bound and upper bound, for each parameter. In the case of vector parameters, the element(s) of the vector to be profiled are specified.

The initial starting point for the estimation can also be specified using \texttt{\cname\ -i \emph{file}}, which may improve the minimiser performance for the profiles.

If the profile results are not reasonable, it may be a result of not using enough iterations in the minimiser or a poor choice of minimiser control variables (e.g., the minimiser tolerance). It may also be useful to try other minimisers and compare the results. An example excerpt follows, but also see the syntax at Section~\ref{syntax:Profile}.

{\small{\begin{verbatim}
		@profile B0
		parameter process[Recruitment_east].b0
		steps 10
		lower_bound 10000
		upper_bound 100000
		## you can force other parameters to be the same
		same process[Recruitment_west].b0
		\end{verbatim}}}
	
To run a \CNAME\ you will need to supply the following reports

{\small{\begin{verbatim}
		@report profile
		type profile
		
		@report estimate_values
		type estimate_value
		
		@report objective_scores
		type objective_function
\end{verbatim}}}
 


\subsection{\I{Bayesian estimation}}\label{sec:MCMC}\label{sec:MCMC-RandomWalkMetropolisHastings}

\CNAME\ can use \I{Markov chain Monte Carlo (MCMC)}\index{MCMC} functionality to generate a sample from the posterior distribution of the estimated parameters with command \texttt{\cname\ -m} or \texttt{\cname\ -M MPD\_file} and output the sampled values, optionally keeping only every $n$th set of values.

As \CNAME\ has no post-processing capabilities. \CNAME\ cannot produce MCMC convergence diagnostics. To calculate these diagnostics, use a package such as \href{http://www.public-health.uiowa.edu/boa}{BOA}, plot/summarize the posterior distributions of the output quantities, and/or use a general-purpose statistical package such as \href{http://www.r-project.org}{\R}.

Bayesian methodology\index{Bayesian estimation} and MCMC are both large and complex topics. See Gelman et al. \citeyearpar{823} and Gilks et al. \citeyearpar{143} for details of both Bayesian analysis and MCMC methods. In addition, see Punt \& Hilborn \citeyearpar{828} for an introduction to quantitative fish stock assessment using Bayesian methods.

This section briefly describes the MCMC algorithms used in \CNAME. See Section \ref{syntax:MCMC} for the \CNAME\ commands used in an MCMC Bayesian analysis.

\CNAME\ implements two methods for MCMC. The first is a straightforward implementation of the random walk Metropolis-Hastings algorithm\index{MCMC: Metopolis-Hastings} \citep{823,143}. The Metropolis-Hastings algorithm attempts to draw a sample from a Bayesian posterior distribution, and calculates the posterior density $\pi$, scaled by an unknown constant. The algorithm generates a 'chain' or sequence of values. Typically the beginning of the chain is discarded (the burn-in period) and every $N$th element of the remainder is taken as the posterior sample. The second is Hamiltonian Monte Carlo. This uses similar subcommands as the random walk Metropolis-Hastings algorithm. In both cases, the chain is produced by taking an initial point $x_0$ and repeatedly applying the following rule, where $x_i$ is the current point:

\begin{enumerate}
\item Draw a candidate step s from a proposal distribution J, which should be symmetric i.e., $J(-s)=J(s)$
\item Calculate $r=min(\pi(x_i+s)/\pi(x_i),1)$
\item Let $x_{i+1}=x_i+s$ with probability $r$, or $x_i$ with probability $1-r$
\end{enumerate}

An initial point estimate is produced before the chain starts (although can alternatively be supplied from a previous estimation run), which is done so as to calculate the approximate covariance matrix of the estimated parameters (as the inverse Hessian), and may also be used as the starting point of the chain.

The starting point of the point estimate minimiser can be specified using the command \texttt{\cname\ -i}. Don't start it too close to the actual estimate (either by using \texttt{\cname\ -i}, or by changing the initial parameter values in \config) as it takes a few iterations to determine a reasonable approximation to the Hessian.

There are currently three options for the starting point of the MCMC. It can be from either the MPD, an estimate supplied in a \texttt{-i} file, or from a random 'jump' from the point estimate. In the later case, it is generated from a multivariate normal distribution, centred on the start point, with covariance equal to the inverse Hessian multiplied by a user-specified constant (using the subcommand \subcommand{start}). This may be useful if the chain gets `stuck' at the point estimate, or if you wish to generate multiple chains from for later MCMC diagnostic tests. 

Note that if a number of parameters are at bounds at the start point, then a random jump from this point may fail to find a suitable candidate that is within bounds. A total of 10~000 attempts are made before the algorithm exits with an error. The subcommand  {\subcommand{adjust\_parameters\_at\_bounds}} can be used to identify these parameters and set their start point at a random uniformly generated point between the lower and upper bounds.

The chain moves in natural space, i.e., no transformations are applied to the estimated parameters. The default proposal distribution is a either  a multivariate normal or a multivariate Student's $t$ distribution centred on the current point, with covariance matrix equal to a matrix based on the approximate covariance produced by the minimiser, multiplied by the step-size factor.

The following steps define how the initial covariance matrix of the proposal distribution is calculated:

\begin{enumerate}
\item The covariance matrix is taken as the inverse of the approximate Hessian from the quasi-Newton minimiser.

\item The covariance matrix is modified so as to decrease all correlations greater than \commandsub{mcmc}{max\_correlation} down to \commandsub{mcmc}{max\_correlation}, and similarly to increase all correlations less than -\commandsub{mcmc}{max\_correlation} up to -\commandsub{mcmc}{max\_correlation} (the \commandsub{mcmc}{max\_correlation} parameter defaults to 0.8). This should help to avoid getting 'stuck' in a lower-dimensional subspace.

\item The covariance matrix is then modified either by

\begin{itemize}
\item  \commandsubarg{mcmc}{adjustment\_method}{covariance}: that if the variance of the $i$th parameter is non-zero and less than \commandsub{mcmc}{min\_difference} multiplied by the difference between the parameters' lower and upper bound, then the variance is changed, without changing the associated correlations, to $k=$min\_diff$(upper\_bound_i-lower\_bound_i)$. This is done by setting \[
{\mathop{\rm Cov}\nolimits} \left( {i,j} \right)^\prime   = {{{\mathop{\rm sqrt}\nolimits} \left( k \right){\mathop{\rm Cov}\nolimits} \left( {i,j} \right)} \mathord{\left/
{\vphantom {{{\mathop{\rm sqrt}\nolimits} \left( k \right){\mathop{\rm Cov}\nolimits} \left( {i,j} \right)} {{\mathop{\rm sd}\nolimits} \left( i \right)}}} \right.
\kern-\nulldelimiterspace} {{\mathop{\rm sd}\nolimits} \left( i \right)}}
\]
for $i \ne j$, and ${\mathop{\rm var}} \left( i \right)^\prime   = k$

\item \commandsubarg{mcmc}{adjustment\_method}{correlation}: that if the variance of the $i$th parameter is non-zero and less than \commandsub{mcmc}{min\_difference} multiplied by the difference between the parameters' lower and upper bounds, then its variance is changed to $k=min\_diff(upper\_bound_i-lower\_bound_i)$. This differs from (i) above in that the effect of this option is that it also modifies the resulting correlations between the $i$th parameter and all other parameters.
\end{itemize}

This allows each estimated parameter to move in the MCMC even if its variance is very small according to the inverse Hessian. In both cases, the \commandsub{mcmc}{min\_difference} parameter defaults to $0.0001$.

\item The \commandsub{mcmc}{step\_size} (a scalar factor applied to the covariance matrix to improve the acceptance probability) is set by the user. The default is $2.4d^{-0.5}$ where $d$ is the number of estimated parameters, as recommended by Gelman et al. \citep{823}.
\end{enumerate}

The proposal distribution can also change adaptively during the chain, using two different mechanisms. Both are offered as means of improving the convergence properties of the chain. It is important to note that any adaptive behaviour must finish before the end of the burn-in period, i.e., the proposal distribution must be finalised before the kept portion of the chain starts.

The adaptive mechanisms are:

\begin{itemize}
\item The step-size changes adaptively at one or more sample numbers (See next paragraph for details on the step-size adaptation methods)
\item The entire covariance matrix changes adaptively at one or more sample numbers. At each adaptation, the covariance matrix is replaced with an empirical covariance matrix, derived from the MCMC chain. The idea is that an empirical covariance is a better approximation of the proposal distribution than the inverse of the Hessian matrix, and can improve convergence and mixing of the chain.
\end{itemize}

The two options to adapt the step-size are \texttt{double\_half} or \texttt{ratio}, defined by the subcommand \texttt{adapt\_stepsize\_method}. The \texttt{double\_half} method is the same as the method used in CASAL for adapting the step-size (see \cite{823} for justification). The algorithm for \texttt{double\_half} is, at each adaptation, the step-size is doubled if the acceptance rate since the last adaptation is more than $0.5$, or halved if the acceptance rate is less than $0.2$. The \texttt{ratio} method adapts the current step-size by the acceptance rate since the last adaptation multiplied by 4.1667 so-as to approach an acceptance rate of $\approx$ 0.24. See \cite{mcmc_rate} for justification on the choice of that acceptance rate.

The \texttt{step\_size} parameter is now on a different scale, and must be rescaled. It is set to a user-specified value (which may or may not be the same as the initial step-size). Set the step-size adaptations to occur after this, so that the step-size can be readjusted to an appropriate value which gives good acceptance probabilities with the new matrix.

All modified versions of the covariance matrix are printed to the standard output, but only the initial covariance matrix (inverse Hessian) is saved to the objectives file (see Section \ref{syntax:Report-MCMCCovariance}). \label{sec:Report-MCMCCovariance}  

The variance-covariance matrix of this sub-sample of chain is calculated. As above, correlations greater than \commandsub{mcmc}{max\_correlation} are reduced to \commandsub{mcmc}{max\_correlation}, correlations less than -\commandsub{mcmc}{max\_correlation} are increased to  -\commandsub{mcmc}{max\_correlation}, and very small non-zero variances are increased (\commandsub{mcmc}{covariance\_adjustment} and \commandsub{mcmc}{min\_difference}). The result is the new variance-covariance matrix of the proposal distribution. 

The procedure used to choose the sample of points is that, to start, all points on the chain so far are taken. \TODO{reword this paragraph} All points in an initial user-specified period are discarded. The assumption is that the chain will have started moving during this period. If this is incorrect and the chain has still not moved by the end of this period, it is a fatal error and \CNAME\ stops. The remaining set of points must contain at least some user-specified number of transitions. If this is incorrect and the chain has not had at least this number of transitions, then it is also a fatal error. If this test is passed, the set of points is systematically sub-sampled down to 1000 points (and it must be at least this long to start with).

The probability of acceptance for each jump is $0$ if the jump would move a parameter value outside of its bounds, $1$ if it improves the posterior, or $(new posterior/old posterior)$ otherwise. How often the position of the chain is recorded is specified with the \texttt{keep} parameter. For example, with \texttt{keep 10}, only every $10$th sample is recorded.

The option to specify that some of the estimated parameters are fixed during the MCMC is available via the \subcommand{mcmc\_fixed} in the \command{estimate} block. If the chain starts at the point estimate or at a random location, these fixed parameters are set to their values at the end of estimation during the MCMC phase.

If  the start of the chain is specified with the command \texttt{\cname\ -i}, these fixed parameters are set to the values in the file.

Restarting an MCMC chain: in the case where an MCMC chain was halted or interrupted, the MCMC chain can be restarted from where it finished with

{\small{\begin{verbatim}
casal2 -R MPD_file --objective-file objectives_file --sample-file samples_file
\end{verbatim}}}

where \texttt{Objective\_file\_name} is the file name for the objective function report and \texttt{Sample\_file\_name} is the file name for the sample report from a MCMC chain.

The posterior sample can be used for (projections (Section \ref{sec:Project})) or simulations (see Section \ref{sec:Simulate}) with the values supplied with the command \texttt{\cname\ -i \emph{file} --t}.

A multivariate Student's $t$ distribution can be used as an alternative to the multivariate normal proposal distribution. If you request multivariate Student's $t$ proposals, you can change the degrees of freedom from the default of 4. As the degrees of freedom decreases, the $t$ distribution becomes more heavy tailed. This may lead to better convergence properties. Note the default is the multivariate Student's $t$.

Given a posterior (sub)sample, \CNAME\ can calculate a list of output quantities for each sample point (see Section~\ref{sec:Report} specifically tabular report). These quantities can be output to a file (with the command \texttt{\cname\ -r --tabular}) and read into an external software package where the posterior distributions can be plotted and/or summarised.

The posterior sample can also be used for projections (Section~\ref{sec:Project}). The advantage of this is that the parameter uncertainty, as expressed in the posterior distribution, can be included into the risk estimates.

\CNAME\ will error out if asked to run MCMC for a model that does not contain the following reports,
{\small{\begin{verbatim}
@report mcmc_samples
type mcmc_sample

@report mcmc_objectives
type mcmc_objective
\end{verbatim}}}

The default file name for these reports are \texttt{samples} and \texttt{objectives}. The \subcommand{write\_mode} will default to increment suffix, which means each time you re-run an MCMC in a directory with the same file name, it will increment the extension.

\subsection{Priors\label{sec:Priors}}

In a Bayesian analysis, a prior is required for every parameter that is being estimated. There are no default priors.

When some of these priors are parameterised in terms of mean, c.v., and standard deviation, these refer to the parameters of the distribution before the bounds are applied. The moments of the prior after the bounds are applied may differ.

\CNAME\ has the following priors (expressed in terms of their contribution to the objective function):

\subsubsection{Uniform\index{Uniform prior}\index{Priors ! Uniform}}\label{sec:Prior-Uniform}

\begin{equation}
 - \log \left(\pi \left(p \right) \right) = 0
\end{equation}

\subsubsection{Uniform-log\index{Uniform-log prior}\index{Priors ! Uniform-log}} (i.e., $\log(p) \sim \text{uniform}$)\label{sec:Prior-UniformLog}

\begin{equation}
 - \log \left(\pi \left(p \right) \right) = \log \left( p \right)
\end{equation}

\subsubsection{Normal\index{Normal prior}\index{Priors ! Normal}}\label{sec:Prior-Normal}

The normal distribution with mean $\mu$ and standard deviation with c.v $c$

\begin{equation}
 - \log \left(\pi \left(p \right) \right) = 0.5\left(\frac{p - \mu}{c\mu} \right)^2
\end{equation}

\subsubsection{Normal with standard deviation}\label{sec:Prior-NormalByStdev}

The normal distribution with mean $\mu$ and standard deviation $\sigma$

\begin{equation}
 - \log \left(\pi \left(p \right) \right) = 0.5\left(\frac{p - \mu}
{\sigma }\right)^2
\end{equation}

\subsubsection{Lognormal\index{Lognormal prior}\index{Priors ! Lognormal}}\label{sec:Prior-Lognormal}

The lognormal distribution with mean $\mu$ and c.v. $c$

\begin{equation}
 - \log \left(\pi \left(p \right) \right) = \log \left( p \right) + 0.5 \left( \frac{\log \left( p / \mu \right)}{s} + \frac{s}{2} \right)^2
\end{equation}

where $s$ is the standard deviation of $\log(p)$ and $s= \sqrt{\log \left(1+c^2 \right)}$.

\subsubsection{Normal-log\index{normal-log prior}\index{Priors ! Normal-log}}\label{sec:Prior-NormalLog}

The normal-log distribution  with mean $\mu$ and c.v. $c$

Similar to the lognormal prior, but with the mean ($mu$) and standard deviation ($sigma$) specified in log space.

where $s$ is the standard deviation of $\log(p)$ and $s= \sqrt{\log \left(1+c^2 \right)}$.

\subsubsection{Beta\index{Beta prior}\index{Priors ! Beta}}\label{sec:Prior-Beta}

The Beta distribution with mean $\mu$ and standard deviation $\sigma$, and range parameters $A$ and $B$

\begin{equation}
 - \log \left(\pi \left( p \right) \right) = \left( 1 - m \right) \log \left( p - A \right) + \left( 1 - n \right)\log \left( B - p \right)
\end{equation}

where $\nu  = \frac{\mu  - A}{B - A}$, and $\tau = \frac{\left(\mu -A \right)\left(B - \mu \right)}{\sigma ^2} - 1$ and then $\mu=\tau \nu$ and $n=\tau(1-\nu)$. Note that the beta prior is undefined when $\tau \leq 0$.

\subsubsection{Student's t\index{students\_t prior}\index{Priors ! Student's t}}\label{sec:Prior-Studentst}

The Student's t distribution with location (mean) $\mu$, scale $\sigma$, and degrees of freedom $\nu$ where the pdf is defined as

\begin{equation}
	\begin{aligned}
f(x | \mu, \sigma, \nu) &= \frac{\Gamma\left((\nu + 1)/2\right)} {\Gamma(\nu/2)} \ \frac{1}{\sqrt{\nu \pi} \ \sigma} \ \left( 1 + \frac{1}{\nu} \left(\frac{x - \mu}{\sigma}\right)^2 \right)^{-(\nu + 1)/2} \text{ and} \\
- log \left(\pi \left( p \right) \right) &= -\log \left( f(x | \mu, \sigma, \nu) \right)
	\end{aligned}
\end{equation}

The Student's t prior is Cauchy with $\nu = 1$ and is equivalent to the normal as $\nu \rightarrow \inf$.

Vectors of parameters can be independently (but not necessarily identically) distributed according to any of the above forms, in which case the joint negative-log-prior for the vector is the sum of the negative-log-priors of the components. Values of each parameter need to be specified for each element of the vector. Example of syntax to define the estimation of a parameter and the prior assumed:

{\small{\begin{verbatim}
		## uniform-log example estimate
		@estimate B0
		type uniform_log	# this command "type" defines the prior type.
		parameter process[Recruitment].b0 # "Recruitment" is the label of your process
		upper_bound 20000
		lower_bound 1000

		## Lognormal YCS estimation
		@estimate year_class_strengths_1990_1995
		type lognormal
		parameter process[Recruitment].ycs_values{1990:1995}
		# ycs_year  1990	1991	1992	1993	1994	1995
		mu   		1   	1   	1   	1   	1   	1
		cv 			0.9 	0.9 	0.9 	0.9 	0.9 	0.9
		lower_bound 0.01	0.01	0.01	0.01	0.01	0.01
		upper_bound 9		9		9		9		9		9
\end{verbatim}}}

\subsection{Penalties}\label{sec:Penalty}\label{sec:Penalty-Process}

Penalties are associated with processes and can be used to enforce parameter value or derived quantity restrictions or model outputs that are invalid by adding a penalty to the objective function. For example, estimated parameter values can be restricted so that a known mortality event removes enough individuals from the population within an event mortality process. \CNAME\ requires penalty functions for processes that remove or shift a \emph{number} of individuals between categories or from the partition. Many of the penalties that were available in CASAL are implemented as additional priors in \CNAME (see Section~\ref{sec:AdditionalPriors}).

For penalties, a multiplier is required to be specified, and the objective function is increased by this multiplier multiplied by the penalty value. In some cases the multiplier may need to be quite large to prohibit some model behaviour.

Penalties are implemented for the processes

\begin{itemize}
	\item \commandlabsubarg{process}{type}{event\_mortality},
	\item \commandlabsubarg{process}{type}{mortality\_instantaneous},
	\item \commandlabsubarg{process}{type}{tag\_by\_length},
	\item \commandlabsubarg{process}{type}{tag\_by\_length}, and
	\item \commandlabsubarg{process}{type}{category\_transition}
\end{itemize}

For these processes, two types of penalties can be defined: on the natural scale (the default) and on the log scale. Both of these types add a penalty value of the squared difference between the observed value (e.g., the actual number of individuals to be removed in an event mortality process or the actual number of individuals to shift in a category transition process), and the number that were moved (if less than or equal), multiplied by the penalty multiplier.

The natural scale penalty calculates the squared difference on a natural scale, and the log scale penalty calculates the squared difference of the logged values.

For example:

{\small{\begin{verbatim}
@process Mortality
type mortality_instantaneous
penalty CatchMustBeTaken

# define the penalty in an @penalty block
@penalty CatchMustBeTaken
type process
log_scale True
multiplier 10000
\end{verbatim}}}

Penalties are added to the objective function in the following ways;

\begin{equation}
	Penalty = (X_1 - X_2)^2
\end{equation}

or if \subcommand{log\_scale true}

\begin{equation}
Penalty = (log(X_1) - log(X_2))^2
\end{equation}

where, for example, $X_1$ is observed catch biomass and $X_2$ is the estimated catch biomass. Penalties are usually applied in situations when numbers or weight are known. Another example is for tagging, where the number of individuals that were tagged in a given year is known, so a penalty can be used to restrict the model to estimate reasonable values for the numbers of tagged individuals in that year.

\subsection{Additional priors\label{sec:AdditionalPriors}}\index{Additional priors}

Additional priors are optional additional priors or penalties that can be applied to encourage vectors to be smooth or have some average, or apply a prior to parameter or group of parameters.

The types of additional priors available include smoothing and averaging of vector parameters, a uniform log or lognormal prior on a single parameter, and a prior on the sum of a list of parameters. The additional priors are described below.

\subsubsection{Vector smoothing}\label{sec:AdditionalPrior-VectorSmoothing}\index{Additional prior!Vector smoothing}

The \subcommand{vector\_smoothing} additional prior is applied to a vector parameter. Sum of squares of $r^{th}$ differences, optionally on a log scale. This encourages the vector to be like a polynomial of degree $(r-1)$. A range of the vector to be "smoothed" can be specified (and if not, the smoother is applied to the entire vector). However, this restriction must be specified by an index of the vector and must be between 1 and the length of the vector, inclusive.

\subsubsection{Vector average}\label{sec:AdditionalPrior-VectorAverage}\index{Additional prior!Vector average}

The \subcommand{vector\_average} additional prior is applied to a vector parameter and restricts the vector to average arithmetically with \subcommand{method=k} or \subcommand{method=m}, or geometrically to $exp(k)$ when \subcommand{method=m}. Typically used with $k=1$ with \subcommand{method=k} or \subcommand{method=l}, or $k=0$ with \subcommand{method=m}, to restrict the recruitment\_multipliers to centre on $1$. Optionally, indices can be chosen or excluded outside a given set of bounds.

Methods available for the type \texttt{vector\_average} are \subcommand{l}, \subcommand{k}, \subcommand{m}. For a target vector parameter $\textbf{X}$ and target mean $k$, the contribution to the objective score is
	
\begin{itemize}
	\item \subcommand{method k}
	
	$- \log \left(\pi \left(p \right) \right) = \left(\bar{X} - k\right)^2$
	
	\item \subcommand{method l}
	
	$- \log \left(\pi \left(p \right) \right) = \left(\overline{ln\left(X\right)} - k\right)^2$
	
	\item \subcommand{method m}
	
	$- \log \left(\pi \left(p \right) \right) = \left(ln\left(\bar{X}\right) - k\right)^2$
\end{itemize}

where $\overline{ln\left(X\right)}$ is the mean of the logged values.

\subsubsection{Lognormal}\label{sec:AdditionalPrior-LogNormal}\index{Additional prior!lognormal}

Apply a \subcommand{lognormal} additional prior with mean $\mu$ and c.v. $c$ to a single parameter,

\begin{equation}
- \log \left(\pi \left(p \right) \right) = \log \left( p \right) + 0.5 \left( \frac{\log \left( p / \mu \right)}{s} + \frac{s}{2} \right)^2
\end{equation}

\subsubsection{Uniform}\label{sec:AdditionalPrior-UniformLog}\index{Additional prior!Uniform}
	
Apply a \subcommand{uniform} additional prior to a single parameter,
	
\begin{equation}
- \log \left(\pi \left(p \right) \right) = 0
\end{equation}

\subsubsection{Uniform-log}\label{sec:AdditionalPrior-Uniform}\index{Additional prior!Uniform log}

Apply a \subcommand{uniform\_log} additional prior to a single parameter,

\begin{equation}
	- \log \left(\pi \left(p \right) \right) = \log \left( p \right)
\end{equation}

\subsubsection{Element difference}\label{sec:AdditionalPrior-ElementDifference}\index{Additional prior!Element difference}

Apply an \subcommand{element\_difference} additional prior to encourage two vectors of equal length to be similar, where the value is the difference between the elements of each vector, such that,

\begin{equation}
- \log \left(\pi \left(p_1,p_2 \right) \right) = \sum_{i = 1}^n \left( p_{1,i} - p_{2,i} \right)^2
\end{equation}

\subsubsection{Beta}\label{sec:AdditionalPrior-Beta}\index{Additional prior!Beta}

Apply a \subcommand{beta} additional prior to a single parameter\index{Beta additional prior}\index{Additional Priors ! Beta} with mean $\mu$ and standard deviation $\sigma$, and range parameters $A$ and $B$, for parameter value = $p$

\begin{equation}
- \log \left(\pi \left( p \right) \right) = \left( 1 - m \right) \log \left( p - A \right) + \left( 1 - n \right)\log \left( B - p \right)
\end{equation}

where $\nu  = \frac{\mu  - A}{B - A}$, and $\tau = \frac{\left(\mu -A \right)\left(B - \mu \right)}{\sigma ^2} - 1$ and then $m=\tau \nu$ and $n=\tau(1-\nu)$. 

Note that the beta prior is undefined when $\tau \leq 0$.

\subsubsection{Sum}\label{sec:AdditionalPrior-Sum}\index{Additional prior!Sum}
	
Apply a \subcommand{sum} additional prior on the sum of a list of parameters, using either a normal or lognormal distribution.
	
The normal distribution has mean $\mu$ and standard deviation with c.v $c$

\begin{equation}
	- \log \left(\pi \left(p \right) \right) = 0.5\left(\frac{p - \mu}{c\mu} \right)^2
\end{equation}

And the lognormal distribution has mean $\mu$ and c.v. $c$

\begin{equation}
	- \log \left(\pi \left(p \right) \right) = \log \left( p \right) + 0.5 \left( \frac{\log \left( p / \mu \right)}{s} + \frac{s}{2} \right)^2
\end{equation}
	
where $s$ is the standard deviation of $\log(p)$ and $s= \sqrt{\log \left(1+c^2 \right)}$.

This additional prior can be used to encourage the sum of a list of parameters to have value $mu$.
	
\subsubsection{Applying additional priors}

All parameters that can be estimated can also have an additional prior. For parameters that are not estimated within a specific model run, additional priors can be applied to.

\begin{itemize}
	\item \subcommand{selectivity[Selectivity\_label].values\{i:j\}}.

	This subcommand applies a selectivity to the value by age (for ages $i$ through $j$). This option is available only for certain types of selectivities (\subcommand{all\_values}, \subcommand{all\_values\_bounded}, \subcommand{double\_exponential}). 

	\item \subcommand{catchability[Catchability\_label].q}

	This subcommand is for catchabilities that are of type \subcommand{nuisance} only. Since \subcommand{nuisance} $q$s are not free parameters, additional priors can be applied to replicate CASAL models with \command{estimate} blocks in nuisance $q$s.
\end{itemize}

\subsection{\I{Parameter transformations}\label{sec:Transformation}}
\CNAME\ has multiple methods to transform a parameter into a different \enquote{space}. Transformations are implemented to try and achieve \enquote{better} model optimisation. Complex population models can have highly correlated parameters so transforming them is a method of addressing confounded parameters, and \enquote{help} the minimisers find a \enquote{global} solution faster. To read more about transformations and get a better understanding of why they are used, see \cite{gilks1995markov}, specifically chapter 6.


To transform a parameter the \command{parameter\_transformation} block is used. For example if users wanted to estimate log \(R_0\) instead of \(R_0\), they could do the following,
{\small{\begin{verbatim}
		## define transformation
		@parameter_transformation log_R0
		type log
		parameters process[Recruitment].r0

		## define @estimate for the log parameter
		@estimate log_R0
		type uniform
		parameter parameter_transformation[log_r0].log_parameter
		lower_bound 1
		upper_bound 25
\end{verbatim}}}
%
The available parameter transformations are,
\begin{enumerate}
	\item Log (Univariate transformation) Section~\ref{subsec:Transformation-types} - \ref{sec:Transformation-Log}
	\item Inverse (Univariate transformation) Section~\ref{subsec:Transformation-types} - \ref{sec:Transformation-Inverse}
	\item Difference (Bivariate transformation) Section~\ref{subsec:Transformation-types} - \ref{sec:Transformation-Difference}
	\item Average difference (Bivariate transformation) Section~\ref{subsec:Transformation-types} - \ref{sec:Transformation-AverageDifference}
	\item Log sum (Bivariate transformation) Section~\ref{subsec:Transformation-types} - \ref{sec:Transformation-LogSum}	
	\item Orthogonal (Bivariate transformation) Section~\ref{subsec:Transformation-types} - \ref{sec:Transformation-Orthogonal}
	\item Logistic (Univariate transformation) Section~\ref{subsec:Transformation-types} - \ref{sec:Transformation-Logistic}
	\item Sum to one (Bivariate transformation) Section~\ref{subsec:Transformation-types} - \ref{sec:Transformation-SumToOne}
	\item Simplex (Multivariate transformation) Section~\ref{subsec:Transformation-types} - \ref{sec:Transformation-Simplex}	
	\item Square root (Univariate transformation) Section~\ref{subsec:Transformation-types} - \ref{sec:Transformation-Sqrt}	
\end{enumerate}

To see the parameters that can be used in \command{estimate} block for each estimable transformation see the \subcommand{estimable parameter} description in Section~\ref{subsec:Transformation-types}.

When users estimate a transformed parameter they have the option of defining the prior for the transformed parameter or for the parameter in natural space. An example of when the later has been used. Say a meta-analysis has been done on the catchability parameter, for which an \textit{a priori} assumption can be made, but the user wants to estimate log transformed catchability for optimisation reasons. In this instance users are required to use the subcommand \subcommand{prior\_applies\_to\_restored\_parameters}. If this is true the prior will be applied to the untransformed parameter and a Jacobian will be added (if it is known) to account for the change in variable. If the Jacobian is false then the prior refers to the transformed parameter and no adjustments are needed. If users specify to calculate a Jacobian and the estimate is not a \subcommand{parameter\_transformation} \CNAME\ will print a warning and ignore this input.

\subsubsection{Transform with Jacobian}
The support of a random variable $X$ with density $p_X(x)$ is that subset of values for which it has non-zero density,
\begin{equation}
  supp(X) = \{x|p_X(x) > 0\}
\end{equation}

If $f$ is a transformation function defined on the support of $X$, then $Y = f(X)$ is a new random variable (transformed variable).

This section shows the available transformations in \CNAME\ and the resulting probability density function of $Y$. %%This theory follows the STAN manual \cite{STAN}.

Suppose $X$ is one dimensional and $f$: $supp(X) \to \mathbf{R}$ is a one-to-one, monotonic function with a differentiable inverse $f^{-1}$. Then the density of $Y$ is

\begin{equation}\label{eq:jacobian}
	p_Y(y) = p_X(f^{-1}(y)) \begin{vmatrix} \frac{\partial}{\partial y} f^{-1}(y) \end{vmatrix}
\end{equation}

where $\begin{vmatrix} \frac{\partial}{\partial y} f^{-1}(y) \end{vmatrix}$ si the Jacobian adjustment is the absolute derivative of the transform. The Jacobian measures how the scale of the transformed variable changes with respect to the underlying variable. This can be expanded to the multivariate case where the Jacobian becomes a matrix of partial derivatives.

In equation~\ref{eq:jacobian} the term $p_X(f^{-1}(y)) = p_X(X)$ and in a Bayesian context is the prior of the untransformed variable/parameter. \CNAME\ defines the objective function as the negative log-likelihood. This means \(\begin{vmatrix} \frac{\partial}{\partial y} f^{-1}(y) \end{vmatrix}\) needs to be times by a negative log, as it is currently defined as an adjustment to the density.

\subsubsection{Transformation types}\label{subsec:Transformation-types}
\begin{enumerate}
\item \subcommand{type} \subcommand{log} : natural logarithm transformation\\
\subcommand{Jacobian defined = true}\\
\subcommand{estimable parameter = log\_parameter}\\
$Y = log(X)$\\
$f() = log()$\\
$f^{-1}() = exp()$
\[
log \begin{vmatrix} \frac{\partial}{\partial y}  exp (y) \end{vmatrix} = log \begin{vmatrix}  exp (y) \end{vmatrix} = log(x)
\]
\label{sec:Transformation-Log}
{\small{\begin{verbatim}
@parameter_transformation log_R0
type log
parameters process[Recruitment].r0

@estimate log_R0
type uniform
parameter parameter_transformation[log_r0].log_parameter
lower_bound 1
upper_bound 25
\end{verbatim}}}

\item \subcommand{inverse}\\
\subcommand{Jacobian defined = true}\\
\subcommand{estimable parameter = inverse\_parameter}\\
$Y = X^{-1}$
\[
log \begin{vmatrix} \frac{\partial}{\partial y} \frac{1}{y} \end{vmatrix} = log \begin{vmatrix} y^{-2}\end{vmatrix} = -2log(y)
\]
\label{sec:Transformation-Inverse}
{\small{\begin{verbatim}
		@parameter_transformation inverse_R0
		type inverse
		parameters process[Recruitment].r0
		
		@estimate inverse_R0
		type uniform
		parameter parameter_transformation[inverse_R0].inverse_parameter
		lower_bound 0.001
		upper_bound 1
		\end{verbatim}}}
	
\item \subcommand{difference} : two parameters $X_1$ and $X_2$ are transformed to $X_1$ and $X_1 - d$, where $d$ is the difference between the original parameters.\\
\subcommand{Jacobian defined = true}\\
\subcommand{estimable parameter = difference\_parameter}\\
$Y_1 = X_1$\\
$Y_2 = X_1 - d$\\
Restore transformations\\
$X_1 = Y_1$\\
$X_2 = X_1 - d$\\
\label{sec:Transformation-Difference}
{\small{\begin{verbatim}
			@parameter_transformation diff
			type difference
			parameters process[InstantMortality].m{male} process[InstantMortality].m{female}
			difference_parameter 0.05
			
			@estimate diff_m
			type uniform
			parameter parameter_transformation[diff].difference_parameter
			lower_bound -0.5
			upper_bound 0.5		
\end{verbatim}}}

\item \subcommand{average\_difference} : two parameters $X_1$ and $X_2$ are transformed to $Y_1$ and $Y_2$, where $Y_1$ is the average of the original parameters and $Y_2$ is the difference between the mean and each parameter.\\
\subcommand{Jacobian defined = false}\\
\subcommand{estimable parameter = average\_parameter, difference\_parameter}\\
$Y_1 = \frac{X_1 + X_2}{2}$\\
$Y_2 =  (Y_1 - X_2)^2 $\\
Restore transformations\\
$X_1 = Y_1 + 0.5Y_2$\\
$X_2 = X_1 - 0.5Y_2$\\
$\begin{vmatrix} \frac{\partial}{\partial y} f^{-1}(y) \end{vmatrix}$ Hasn't been assessed (i.e it could exist) \label{sec:Transformation-AverageDifference}
{\small{\begin{verbatim}
		@parameter_transformation avg_diff
		type average_difference
		parameters process[InstantMortality].m{male} process[InstantMortality].m{female}
		
		@estimate avg_m
		type uniform
		parameter parameter_transformation[avg_diff].average_parameter
		lower_bound 0.01
		upper_bound 1
		
		@estimate diff_m
		type uniform
		parameter parameter_transformation[avg_diff].difference_parameter
		lower_bound -0.5
		upper_bound 0.5		
\end{verbatim}}}
	
\item \subcommand{log\_sum} : two parameters $X_1$ and $X_2$ are transformed to $Y_1$ and $Y_2$, where $Y_1$ is the natural logarithm of the sum of $X_1$ and $X_2$. $Y_2$ describes the proportion of the sum with respect to $X_1$\\
\subcommand{Jacobian defined = false}\\
\subcommand{estimable parameter = log\_total\_parameter, total\_proportion\_parameter}\\
$Y_1 = ln(X_1 + X_2)$\\
$Y_2 = X_1 / (X_1 + X_2)$\\
Restore transformations\\
$X_1 = exp(Y_1)Y_2$\\
$X_2 =exp(Y_1)(1 - Y_2)$\\
$\begin{vmatrix} \frac{\partial}{\partial y} f^{-1}(y) \end{vmatrix}$ Hasn't been assessed (i.e it could exist) \TODO{?????}\\
\label{sec:Transformation-LogSum}
{\small{\begin{verbatim}
		@parameter_transformation log_total_r0
		type log_sum
		parameters process[Recruitment_east].r0 process[Recruitment_west].r0
		
		@estimate log_total_r0
		type uniform
		parameter parameter_transformation[log_total_r0].log_total_parameter
		lower_bound 4
		upper_bound 25
		
		@estimate prop_r0_east
		type uniform
		parameter parameter_transformation[log_total_r0].total_proportion_parameter
		lower_bound 0.001
		upper_bound 0.8		
		\end{verbatim}}}
	
\item \subcommand{orthogonal} : two parameters $X_1$ and $X_2$ are transformed to $Y_1$ and $Y_2$, where $Y_1$ is the multiplication of $X_1$ and $X_2$. $Y_2$ is the division of $X_1$ and $X_2$\\
\subcommand{Jacobian defined = true}\\
\subcommand{estimable parameter = product\_parameter, quotient\_parameter}\\
$Y_1 = X_1 X_2$\\
$Y_2 = X_1 / X_2$\\
Restore transformations\\
$X_1 = \sqrt{Y_1 Y_2}$\\
$X_2 = \sqrt{Y_1 / Y_2}$\\
$\begin{vmatrix} \frac{\partial}{\partial y} f^{-1}(y) \end{vmatrix} = 2Y_2$\\\\
\label{sec:Transformation-Orthogonal}

{\small{\begin{verbatim}
			@parameter_transformation orthogonal_trans
			type orthogonal
			parameters process[Recruitment].r0 catchability[CPUEQ].q
			
			@estimate B0_times_q
			type uniform
			parameter parameter_transformation[orthogonal_trans].product_parameter
			lower_bound 0.1
			upper_bound 2500
			
			@estimate B0_divide_q
			type uniform
			parameter parameter_transformation[orthogonal_trans].quotient_parameter
			lower_bound 0.001
			upper_bound 1e8	
\end{verbatim}}}

\item \subcommand{type} \subcommand{logistic} : logistic transformation\\
\subcommand{Jacobian defined = true}\\
\subcommand{estimable parameter = logistic\_parameter}\\
$Y = logit(\frac{X - lb}{ub - lb})$\\
$f^{-1}() = lb +(ub - lb)  logit^{-1}()$
\[
 \begin{vmatrix} \frac{\partial}{\partial y}   lb +(ub - lb)  logit^{-1}(y) \end{vmatrix} =  (ub - lb)logit^{-1}(y)\left(1 - logit^{-1}(y)\right) 
\]
\label{sec:Transformation-Logistic}
{\small{\begin{verbatim}
		@parameter_transformation logistic_R0
		type logistic
		parameters process[Recruitment].r0
		lower_bound 10000
		upper_bound 600000
		
		@estimate logistic_R0
		type uniform
		parameter parameter_transformation[logistic_R0].logistic_parameter
		lower_bound -1000 # theoretically -Inf
		upper_bound 1000 # theoretically Inf
		\end{verbatim}}}
	
\item \subcommand{SumToOne} : given two parameters $X_1$ and $X_2$ that have the constraint $\sum_{i = 1}^2X_i$, estimate $X_1$ only given $X_2 = 1 - X_1$\\
\subcommand{Jacobian defined = false}\\
\label{sec:Transformation-SumToOne}
{\small{\begin{verbatim}
			@parameter_transformation total_r0
			type sum_to_one
			parameters process[Recruitment_east].r0 process[Recruitment_west].r0
			
			@estimate total_r0
			type uniform
			parameter parameter_transformation[log_total_r0].total_parameter
			lower_bound 4
			upper_bound 25
			
			@estimate prop_r0_east
			type uniform
			parameter parameter_transformation[log_total_r0].proportion_parameter
			lower_bound 0.001
			upper_bound 0.8		
\end{verbatim}}}

% useful reference for the simplex https://mc-stan.org/docs/2_27/reference-manual/simplex-transform-section.html#fn15
\item \subcommand{simplex} : given the vector of parameters $\mathbf{X} = (X_1, \dots, X_n)$ which either has the constraint $\sum_{i = 1}^n X_i = 1$ or $\sum_{i = 1}^n X_i = n$. Then the simplex is a suitable transformation. It translates to a new vector parameter \(\mathbf{Y} = (Y_1, \dots, Y_{n - 1})\) which has unconstrained parameter space i.e \(Y_i \in (-\infty, \infty)\). Note that the calculation of the Jacobian for the simplex is still experimental and may not be suitable in all circumstances.\\

This transformation follows the implementation in stan, where an intermediate variable \(Z_i\) is used. The transformation going from $\mathbf{X}$ to $\mathbf{Y}$ follows\\
\[
Z_i = \frac{X_i}{1 - \sum_{j = 1}^{i - 1}X_j}
\]
and 
\[
Y_i = logit(Z_i) - log\left(\frac{1}{n -i}\right)
\]
The inverse transformation going from $\mathbf{Y}$ to $\mathbf{X}$ follows
\[
Z_i = logit^{-1}\left(Y_i +  log\left(\frac{1}{n -i}\right)\right)
\]
and
\begin{align*}
	X_i &= \left( \sum_{j = 1}^{i - 1}X_j\right)Z_i & \text{ for } i < n\\
	X_n &= 1 - \sum_{i = 1}^{n - 1} X_i & \text{ for } i = n
\end{align*}
The Jacobian for the density is evaluated as follows,
\begin{align*}
|det \ J| = \prod_{i = 1}^{n - 1} Z_i \left(1 - Z_i\right) \left(1 - \sum_{j = 1}^{i - 1}X_j\right)
\end{align*}
\subcommand{Jacobian defined = true}\\
\subcommand{estimable parameter = simplex}\\
\label{sec:Transformation-Simplex}
{\small{\begin{verbatim}
		@parameter_transformation simplex_ycs
		type simplex
		sum_to_one false
		parameters process[Recruitment].ycs_values{1950:2018}
		prior_applies_to_restored_parameters true
		
		@estimate simplex_ycs
		type uniform
		parameter parameter_transformation[simplex_ycs].simplex
		lower_bound -10
		upper_bound 10

		\end{verbatim}}}	
	\item \subcommand{type} \subcommand{sqrt} : square root transformation\\
	\subcommand{Jacobian defined = true}\\
	\subcommand{estimable parameter = sqrt\_parameter}\\
	$Y = sqrt(X)$\\
	$f() = sqrt()$\\
	$f^{-1}(x) = x * x$
	\[
	log \begin{vmatrix} \frac{\partial}{\partial y}  (y^2) \end{vmatrix} = sqrt \begin{vmatrix}  (y^2) \end{vmatrix} = sqrt(x)
	\]
	\label{sec:Transformation-Sqrt}
	{\small{\begin{verbatim}
				@parameter_transformation sqrt_R0
				type sqrt
				parameters process[Recruitment].r0
				
				@estimate sqrt_R0
				type uniform
				parameter parameter_transformation[sqrt_r0].sqrt_parameter
				lower_bound 1
				upper_bound 25
	\end{verbatim}}}
	
	
\end{enumerate}


If users want to force other parameters in the system to be the same as an estimated transformation, this can be done by creating multiple \command{parameter\_transformation} blocks. For example if there were multiple categories (spawning and non spawning males and females) and the average difference parametrisation was used to estimate natural mortality. The non-spawning components can be set the same as the spawning values using the following syntax.

{\small{\begin{verbatim}
		@categories
		format sex.maturity
		names male.spawn female.spawn male.nonspawn female.nonspawn
		
		@parameter_transformation avg_diff_spawn
		type average_difference
		parameters process[InstantMortality].m{male.spawn} 
		           process[InstantMortality].m{female.spawn}
		
		@parameter_transformation avg_diff_non_spawn
		type average_difference
		parameters process[InstantMortality].m{male.nonspawn} 
		           process[InstantMortality].m{female.nonspawn}
		
		@estimate avg_m
		type uniform
		parameter parameter_transformation[avg_diff_spawn].average_parameter
		same parameter_transformation[avg_diff_non_spawn].average_parameter
		lower_bound 0.01
		upper_bound 1
		
		@estimate diff_m
		type uniform
		parameter parameter_transformation[avg_diff_spawn].difference_parameter
		same parameter_transformation[avg_diff_non_spawn].average_parameter
		lower_bound -0.5
		upper_bound 0.5		
\end{verbatim}}}

This can be done for any set of parameters in the system, for example if you had multiple recruitment dynamics and wanted to estimate a joint steepness parameter with the log transformation, you would need to create multiple blocks and force them in the same.


\clearemptydoublepage{}
\defComLab{observation}{Define an object of type \emph{Observation}}.
\defRef{sec:Observation}
\label{syntax:Observation}

\defSub{label}{The label of the observation}
\defType{String}
\defDefault{No default}

\defSub{type}{The type of observation}
\defType{String}
\defDefault{No default}

\defSub{likelihood}{The type of likelihood to use}
\defType{String}
\defDefault{No default}

\defSub{categories}{The category labels to use}
\defType{Vector of strings}
\defDefault{true}

\defSub{delta}{The robustification value (delta) for the likelihood}
\defType{Real number (estimable)}
\defDefault{1e-11}
\defLowerBound{0.0 (inclusive)}

\defSub{simulation\_likelihood}{The simulation likelihood to use}
\defType{String}
\defDefault{No default}

\defSub{likelihood\_multiplier}{The likelihood multiplier}
\defType{Real number (estimable)}
\defDefault{1.0}
\defLowerBound{0.0 (inclusive)}

\defSub{error\_value\_multiplier}{The error value multiplier for likelihood}
\defType{Real number (estimable)}
\defDefault{1.0}
\defLowerBound{0.0 (inclusive)}

\defSub{table}{The table of data specifying the observed values}
\defType{Data table with label = obs}
\defDefault{No default}
\defValue{A $n*m$ matrix, where $n=$ the years and $m=$ either the number of ages, lengths, or abundance/biomass observation for each year defined in the model. Each row starts with the year. The table ends with `end\_table'}
\defNote{See \ref{sec:DataTable} for more details each observation may have custom table labels.}

\defSub{table}{The table of data specifying the observed error values}
\defType{Data table with label = error\_values}
\defDefault{No default}
\defValue{A $n*m$ matrix, where $n=$ the years and $m=$ either the number of ages, lengths, or abundance/biomass observation for each year defined in the model. Each row starts with the year. The table ends with `end\_table'}
\defNote{See \ref{sec:DataTable} for more details on specifying data tables.  each observation may have custom table labels.}

\subsubsection{Observation of type Abundance}
\commandlabsubarg{observation}{type}{Abundance}.
\defRef{sec:Observation-Abundance}
\label{syntax:Observation-Abundance}

\defSub{time\_step}{The label of the time step that the observation occurs in}
\defType{String}
\defDefault{No default}

\defSub{catchability}{The label of the catchability coefficient (q)}
\defType{String}
\defDefault{No default}

\defSub{selectivities}{The labels of the selectivities}
\defType{Vector of strings}
\defDefault{true}

\defSub{process\_error}{The process error}
\defType{Real number (estimable)}
\defDefault{0.0}
\defLowerBound{0.0 (inclusive)}

\defSub{years}{The years for which there are observations}
\defType{Vector of non-negative integers}
\defDefault{No default}

\defSub{table obs}{The table of data specifying the observed and error values}
\defType{Data table with label = obs}
\defDefault{No default}
\defValue{A $n*3$ matrix, where $n=$ the years and a column for year, observation and error value. See below for example.}
\defNote{example below}
\begin{verbatim}
table obs 
# year observation error_value
1993 238.2 0.12
1994 170 0.16
1995 216.2 0.18
2004 46.9 0.20
end_table
\end{verbatim}

\subsubsection{Observation of type Biomass}
\commandlabsubarg{observation}{type}{Biomass}.
\defRef{sec:Observation-Biomass}
\label{syntax:Observation-Biomass}

\defSub{time\_step}{The label of the time step that the observation occurs in}
\defType{String}
\defDefault{No default}

\defSub{catchability}{The label of the catchability coefficient (q)}
\defType{String}
\defDefault{No default}

\defSub{selectivities}{The labels of the selectivities}
\defType{Vector of strings}
\defDefault{true}

\defSub{process\_error}{The process error}
\defType{Real number (estimable)}
\defDefault{0.0}
\defLowerBound{0.0 (inclusive)}

\defSub{age\_weight\_labels}{The labels for the \command{$age\_weight$} block which corresponds to each category, to use the weight calculation method for biomass calculations)}
\defType{Vector of strings}
\defDefault{No default}

\defSub{years}{The years of the observed values}
\defType{Vector of non-negative integers}
\defDefault{No default}

\defSub{table obs}{The table of data specifying the observed and error values}
\defType{Data table with label = obs}
\defDefault{No default}
\defValue{A $n*3$ matrix, where $n=$ the years and a column for year, observation and error value. See below for example.}
\defNote{example below}
\begin{verbatim}
table obs 
# year observation error_value
1993 238.2 0.12
1994 170 0.16
1995 216.2 0.18
2004 46.9 0.20
end_table
\end{verbatim}

\subsubsection{Observation of type Process Removals By Age}
\commandlabsubarg{observation}{type}{Process\_Removals\_By\_Age}.
\defRef{sec:Observation-ProcessRemovalsByAge}
\label{syntax:Observation-ProcessRemovalsByAge}

\defSub{min\_age}{The minimum age}
\defType{Non-negative integer}
\defDefault{No default}

\defSub{max\_age}{The maximum age}
\defType{Non-negative integer}
\defDefault{No default}

\defSub{sum\_to\_one}{Scale year (row) observed values by the total so they sum to equal 1}
\defType{Boolean}
\defDefault{false}

\defSub{simulated\_data\_sum\_to\_one}{Whether simulated data is discrete or scaled by totals to be proportions for each year}
\defType{Boolean}
\defDefault{true}


\defSub{plus\_group}{Is the maximum age the age plus group}
\defType{Boolean}
\defDefault{true}

\defSub{time\_step}{The label of time-step that the observation occurs in}
\defType{Vector of strings}
\defDefault{No default}

\defSub{years}{The years for which there are observations}
\defType{Vector of non-negative integers}
\defDefault{No default}

\defSub{process\_errors}{The process errors to use}
\defType{Vector of real numbers (estimable) of length equal to the number of years}
\defDefault{0.0}
\defNote{If only one value is supplied, it will be repeated for all years in the observation}

\defSub{ageing\_error}{The label of the ageing error to use}
\defType{String}
\defDefault{No default}

\defSub{method\_of\_removal}{The label of the observed method of removals}
\defType{Vector of strings}
\defDefault{No default}


\defSub{mortality\_process}{The label of the mortality instantaneous process for the observation}
\defType{String}
\defDefault{No default}
\defNote{Allowed mortality process types are \subcommand{mortality\_instantaneous} and \subcommand{mortality\_hybrid}}

\defSub{table obs}{The table of data specifying the observed values}
\defType{Data table with label = obs}
\defDefault{No default}
\defValue{A $n\times m$ matrix, where $n=$ the years and $m$ is categories \(\times\) length bins. See below for example.}
\defNote{example below}
\begin{verbatim}
table obs 
1993 0.1 0.2 0.3
1994 0.1 0.2 0.3
end_table
\end{verbatim}
\defSub{table error\_values}{The table of data specifying the error values}
\defType{Data table with label = error\_values}
\defDefault{No default}
\defValue{Can be specified two ways either as a $n\times 1$ matrix with an error value for each year. Or a $n\times m$ matrix, where $n=$ the years and $m$ is categories \(\times\) length bins. See below for example.}
\defNote{example below}
\begin{verbatim}
table error_values 
1993 234
1994 343
end_table
\end{verbatim}
\subsubsection{Observation of type Process Removals By Age Retained}
\commandlabsubarg{observation}{type}{Process\_Removals\_By\_Age\_Retained}.
\defRef{sec:Observation-ProcessRemovalsByAgeRetained}
\label{syntax:Observation-ProcessRemovalsByAgeRetained}

\defSub{min\_age}{The minimum age}
\defType{Non-negative integer}
\defDefault{No default}

\defSub{max\_age}{The maximum age}
\defType{Non-negative integer}
\defDefault{No default}

\defSub{plus\_group}{Is the maximum age the age plus group?}
\defType{Boolean}
\defDefault{true}

\defSub{time\_step}{The label of the time step that the observation occurs in}
\defType{Vector of strings}
\defDefault{No default}


\defSub{sum\_to\_one}{Scale the year (row) observed values by the total, so they sum to 1}
\defType{Boolean}
\defDefault{false}

\defSub{simulated\_data\_sum\_to\_one}{Whether simulated data is discrete or scaled by totals to be proportions for each year}
\defType{Boolean}
\defDefault{true}

\defSub{years}{The years for which there are observations}
\defType{Vector of non-negative integers}
\defDefault{No default}

\defSub{process\_errors}{The process errors to use}
\defType{Vector of real numbers (estimable) of length equal to the number of years}
\defDefault{0.0}
\defNote{If only one value is supplied, it will be repeated for all years in the observation}

\defSub{ageing\_error}{The label of the ageing error to use}
\defType{String}
\defDefault{No default}

\defSub{method\_of\_removal}{The label of observed method of removals}
\defType{Vector of strings}
\defDefault{No default}


\defSub{mortality\_process}{The label of the mortality instantaneous process for the observation}
\defType{String}
\defDefault{No default}
\defNote{Allowed mortality process types are \subcommand{mortality\_instantaneous\_retained}}

\defSub{table obs}{The table of data specifying the observed values}
\defType{Data table with label = obs}
\defDefault{No default}
\defValue{A $n\times m$ matrix, where $n=$ the years and $m$ is categories \(\times\) length bins. See below for example.}
\defNote{example below}
\begin{verbatim}
table obs 
1993 0.1 0.2 0.3
1994 0.1 0.2 0.3
end_table
\end{verbatim}
\defSub{table error\_values}{The table of data specifying the error values}
\defType{Data table with label = error\_values}
\defDefault{No default}
\defValue{Can be specified two ways either as a $n\times 1$ matrix with an error value for each year. Or a $n\times m$ matrix, where $n=$ the years and $m$ is categories \(\times\) length bins. See below for example.}
\defNote{example below}
\begin{verbatim}
table error_values 
1993 234
1994 343
end_table
\end{verbatim}
\subsubsection{Observation of type Process Removals By Age Retained Total}
\commandlabsubarg{observation}{type}{Process\_Removals\_By\_Age\_Retained\_Total}.
\defRef{sec:Observation-ProcessRemovalsByAgeRetainedTotal}
\label{syntax:Observation-ProcessRemovalsByAgeRetainedTotal}

\defSub{min\_age}{The minimum age}
\defType{Non-negative integer}
\defDefault{No default}

\defSub{max\_age}{The maximum age}
\defType{Non-negative integer}
\defDefault{No default}

\defSub{plus\_group}{Is the maximum age the age plus group?}
\defType{Boolean}
\defDefault{true}

\defSub{time\_step}{The label of the time step that the observation occurs in}
\defType{Vector of strings}
\defDefault{No default}

\defSub{sum\_to\_one}{Scale the year (row) observed values by the total, so they sum to 1}
\defType{Boolean}
\defDefault{false}

\defSub{simulated\_data\_sum\_to\_one}{Whether simulated data is discrete or scaled by totals to be proportions for each year}
\defType{Boolean}
\defDefault{true}

\defSub{years}{The years for which there are observations}
\defType{Vector of non-negative integers}
\defDefault{No default}

\defSub{process\_errors}{The process errors to use}
\defType{Vector of real numbers (estimable) of length equal to the number of years}
\defDefault{0.0}
\defNote{If only one value is supplied, it will be repeated for all years in the observation}

\defSub{ageing\_error}{The label of the ageing error to use}
\defType{String}
\defDefault{No default}

\defSub{method\_of\_removal}{The label of observed method of removals}
\defType{Vector of strings}
\defDefault{No default}

\defSub{mortality\_process}{The label of the mortality process for this observation}
\defType{String}
\defDefault{No default}
\defNote{Allowed mortality process types are \subcommand{mortality\_instantaneous\_retained}}

\defSub{table obs}{The table of data specifying the observed values}
\defType{Data table with label = obs}
\defDefault{No default}
\defValue{A $n\times m$ matrix, where $n=$ the years and $m$ is categories \(\times\) length bins. See below for example.}
\defNote{example below}
\begin{verbatim}
table obs 
1993 0.1 0.2 0.3
1994 0.1 0.2 0.3
end_table
\end{verbatim}
\defSub{table error\_values}{The table of data specifying the error values}
\defType{Data table with label = error\_values}
\defDefault{No default}
\defValue{Can be specified two ways either as a $n\times 1$ matrix with an error value for each year. Or a $n\times m$ matrix, where $n=$ the years and $m$ is categories \(\times\) length bins. See below for example.}
\defNote{example below}
\begin{verbatim}
table error_values 
1993 234
1994 343
end_table
\end{verbatim}
\subsubsection{Observation of type Process Removals By Length}
\commandlabsubarg{observation}{type}{Process\_Removals\_By\_Length}.
\defRef{sec:Observation-ProcessRemovalsByLength}
\label{syntax:Observation-ProcessRemovalsByLength}

\defSub{time\_step}{The time step to execute in}
\defType{String}
\defDefault{No default}

\defSub{years}{The years for which there are observations}
\defType{Vector of non-negative integers}
\defDefault{No default}

\defSub{process\_errors}{The process errors to use}
\defType{Vector of real numbers (estimable) of length equal to the number of years}
\defDefault{0.0}
\defNote{If only one value is supplied, it will be repeated for all years in the observation}

\defSub{method\_of\_removal}{The label of observed method of removals}
\defType{String}
\defDefault{No default}

\defSub{length\_bins}{The length bins}
\defType{Vector of real numbers (estimable)}
\defDefault{No default}

\defSub{sum\_to\_one}{Scale the year (row) observed values by the total, so they sum to 1}
\defType{Boolean}
\defDefault{false}

\defSub{simulated\_data\_sum\_to\_one}{Whether simulated data is discrete or scaled by totals to be proportions for each year}
\defType{Boolean}
\defDefault{true}

\defSub{plus\_group}{Is the last length bin a plus group? (defaults to @model value)}
\defType{Boolean}
\defDefault{model}

\defSub{mortality\_process}{The label of the mortality instantaneous process for the observation}
\defType{String}
\defDefault{No default}
\defNote{Allowed mortality process types are \subcommand{mortality\_instantaneous} and \subcommand{mortality\_hybrid}}

\defSub{table obs}{The table of data specifying the observed values}
\defType{Data table with label = obs}
\defDefault{No default}
\defValue{A $n\times m$ matrix, where $n=$ the years and $m$ is categories \(\times\) length bins. See below for example.}
\defNote{example below}
\begin{verbatim}
table obs 
1993 0.1 0.2 0.3
1994 0.1 0.2 0.3
end_table
\end{verbatim}
\defSub{table error\_values}{The table of data specifying the error values}
\defType{Data table with label = error\_values}
\defDefault{No default}
\defValue{Can be specified two ways either as a $n\times 1$ matrix with an error value for each year. Or a $n\times m$ matrix, where $n=$ the years and $m$ is categories \(\times\) length bins. See below for example.}
\defNote{example below}
\begin{verbatim}
table error_values 
1993 234
1994 343
end_table
\end{verbatim}
\subsubsection{Observation of type Process Removals By Length Retained}
\commandlabsubarg{observation}{type}{Process\_Removals\_By\_Length\_Retained}.
\defRef{sec:Observation-ProcessRemovalsByLengthRetained}
\label{syntax:Observation-ProcessRemovalsByLengthRetained}

\defSub{time\_step}{The time step to execute in}
\defType{String}
\defDefault{No default}

\defSub{years}{The years for which there are observations}
\defType{Vector of non-negative integers}
\defDefault{No default}

\defSub{process\_errors}{The process errors to use}
\defType{Vector of real numbers (estimable) of length equal to the number of years}
\defDefault{0.0}
\defNote{If only one value is supplied, it will be repeated for all years in the observation}

\defSub{method\_of\_removal}{The label of observed method of removals}
\defType{String}
\defDefault{No default}

\defSub{length\_bins}{The length bins}
\defType{Vector of real numbers (estimable)}
\defDefault{No default}

\defSub{sum\_to\_one}{Scale the year (row) observed values by the total, so they sum to 1}
\defType{Boolean}
\defDefault{false}

\defSub{simulated\_data\_sum\_to\_one}{Whether simulated data is discrete or scaled by totals to be proportions for each year}
\defType{Boolean}
\defDefault{true}

\defSub{plus\_group}{Is the last length bin a plus group? (defaults to @model value)}
\defType{Boolean}
\defDefault{model}

\defSub{mortality\_process}{The label of the mortality instantaneous process for the observation}
\defType{String}
\defDefault{No default}
\defNote{Allowed mortality process types are \subcommand{mortality\_instantaneous\_retained}}

\defSub{table obs}{The table of data specifying the observed values}
\defType{Data table with label = obs}
\defDefault{No default}
\defValue{A $n\times m$ matrix, where $n=$ the years and $m$ is categories \(\times\) length bins. See below for example.}
\defNote{example below}
\begin{verbatim}
table obs 
1993 0.1 0.2 0.3
1994 0.1 0.2 0.3
end_table
\end{verbatim}
\defSub{table error\_values}{The table of data specifying the error values}
\defType{Data table with label = error\_values}
\defDefault{No default}
\defValue{Can be specified two ways either as a $n\times 1$ matrix with an error value for each year. Or a $n\times m$ matrix, where $n=$ the years and $m$ is categories \(\times\) length bins. See below for example.}
\defNote{example below}
\begin{verbatim}
table error_values 
1993 234
1994 343
end_table
\end{verbatim}

\subsubsection{Observation of type Process Removals By Length Retained Total}
\commandlabsubarg{observation}{type}{Process\_Removals\_By\_Length\_Retained\_Total}.
\defRef{sec:Observation-ProcessRemovalsByLengthRetainedTotal}
\label{syntax:Observation-ProcessRemovalsByLengthRetainedTotal}

\defSub{time\_step}{The time step to execute in}
\defType{String}
\defDefault{No default}

\defSub{years}{The years for which there are observations}
\defType{Vector of non-negative integers}
\defDefault{No default}

\defSub{process\_errors}{The process errors to use}
\defType{Vector of real numbers (estimable) of length equal to the number of years}
\defDefault{0.0}
\defNote{If only one value is supplied, it will be repeated for all years in the observation}

\defSub{method\_of\_removal}{The label of observed method of removals}
\defType{String}
\defDefault{No default}

\defSub{length\_bins}{The length bins}
\defType{Vector of real numbers (estimable)}
\defDefault{No default}

\defSub{plus\_group}{Is the last length bin a plus group? (defaults to @model value)}
\defType{Boolean}
\defDefault{model}

\defSub{sum\_to\_one}{Scale the year (row) observed values by the total, so they sum to 1}
\defType{Boolean}
\defDefault{false}

\defSub{simulated\_data\_sum\_to\_one}{Whether simulated data is discrete or scaled by totals to be proportions for each year}
\defType{Boolean}
\defDefault{true}

\defSub{mortality\_process}{The label of the mortality instantaneous process for the observation}
\defType{String}
\defDefault{No default}
\defNote{Allowed mortality process types are \subcommand{mortality\_instantaneous\_retained}}

\defSub{table obs}{The table of data specifying the observed values}
\defType{Data table with label = obs}
\defDefault{No default}
\defValue{A $n\times m$ matrix, where $n=$ the years and $m$ is categories \(\times\) length bins. See below for example.}
\defNote{example below}
\begin{verbatim}
table obs 
1993 0.1 0.2 0.3
1994 0.1 0.2 0.3
end_table
\end{verbatim}
\defSub{table error\_values}{The table of data specifying the error values}
\defType{Data table with label = error\_values}
\defDefault{No default}
\defValue{Can be specified two ways either as a $n\times 1$ matrix with an error value for each year. Or a $n\times m$ matrix, where $n=$ the years and $m$ is categories \(\times\) length bins. See below for example.}
\defNote{example below}
\begin{verbatim}
table error_values 
1993 234
1994 343
end_table
\end{verbatim}

\subsubsection{Observation of type Process Removals By Weight}
\commandlabsubarg{observation}{type}{Process\_Removals\_By\_Weight}.
%\defRef{sec:Observation-ProcessRemovalsByWeight}
\label{syntax:Observation-ProcessRemovalsByWeight}

\defSub{mortality\_process}{The label of the mortality instantaneous process for the observation}
\defType{String}
\defDefault{No default}
\defNote{Allowed mortality process types are \subcommand{mortality\_instantaneous}}

\defSub{method\_of\_removal}{The label of observed method of removals}
\defType{String}
\defDefault{No default}

\defSub{time\_step}{The time step to execute in}
\defType{String}
\defDefault{No default}

\defSub{years}{The years for which there are observations}
\defType{Vector of non-negative integers}
\defDefault{No default}

\defSub{process\_errors}{The process errors to use}
\defType{Vector of real numbers (estimable) of length equal to the number of years}
\defDefault{0.0}
\defNote{If only one value is supplied, it will be repeated for all years in the observation}

\defSub{length\_weight\_cv}{The CV for the length-weight relationship}
\defType{Real number (estimable)}
\defDefault{0.10}
\defLowerBound{0.0 (exclusive)}

\defSub{length\_weight\_distribution}{The distribution of the length-weight relationship}
\defType{String}
\defDefault{normal}

\defSub{length\_bins}{The length bins}
\defType{Vector of real numbers (estimable)}
\defDefault{No default}

\defSub{length\_bins\_n}{The average number in each length bin}
\defType{Vector of real numbers (estimable)}
\defDefault{No default}

\defSub{units}{The units for the weight bins (grams, kilograms (kgs), or tonnes)}
\defType{String}
\defDefault{kgs}

\defSub{fishbox\_weight}{The target weight of each box}
\defType{Real number (estimable)}
\defDefault{20.0}
\defLowerBound{0.0 (exclusive)}

\defSub{weight\_bins}{The weight bins}
\defType{Vector of real numbers (estimable)}
\defDefault{No default}

\subsubsection{Observation of type Proportions At Age}
\commandlabsubarg{observation}{type}{Proportions\_At\_Age}.
\defRef{sec:Observation-ProportionsAtAge}
\label{syntax:Observation-ProportionsAtAge}

\defSub{min\_age}{The minimum age}
\defType{Non-negative integer}
\defDefault{No default}

\defSub{max\_age}{The maximum age}
\defType{Non-negative integer}
\defDefault{No default}

\defSub{plus\_group}{Is the maximum age the age plus group?}
\defType{Boolean}
\defDefault{true}

\defSub{time\_step}{The label of the time step that the observation occurs in}
\defType{String}
\defDefault{No default}

\defSub{years}{The years of the observed values}
\defType{Vector of non-negative integers}
\defDefault{No default}

\defSub{selectivities}{The labels of the selectivities}
\defType{Vector of strings}
\defDefault{true}

\defSub{process\_errors}{The process errors to use}
\defType{Vector of real numbers (estimable) of length equal to the number of years}
\defDefault{0.0}
\defNote{If only one value is supplied, it will be repeated for all years in the observation}

\defSub{ageing\_error}{The label of ageing error to use}
\defType{String}
\defDefault{No default}

\defSub{sum\_to\_one}{Scale the year (row) observed values by the total, so they sum to 1}
\defType{Boolean}
\defDefault{false}

\defSub{simulated\_data\_sum\_to\_one}{Whether simulated data is discrete or scaled by totals to be proportions for each year}
\defType{Boolean}
\defDefault{true}

\defSub{table obs}{The table of data specifying the observed values}
\defType{Data table with label = obs}
\defDefault{No default}
\defValue{A $n\times m$ matrix, where $n=$ the years and $m$ is categories \(\times\) length bins. See below for example.}
\defNote{example below}
\begin{verbatim}
table obs 
1993 0.1 0.2 0.3
1994 0.1 0.2 0.3
end_table
\end{verbatim}
\defSub{table error\_values}{The table of data specifying the error values}
\defType{Data table with label = error\_values}
\defDefault{No default}
\defValue{Can be specified two ways either as a $n\times 1$ matrix with an error value for each year. Or a $n\times m$ matrix, where $n=$ the years and $m$ is categories \(\times\) length bins. See below for example.}
\defNote{example below}
\begin{verbatim}
table error_values 
1993 234
1994 343
end_table
\end{verbatim}

\subsubsection{Observation of type Proportions At Length}
\commandlabsubarg{observation}{type}{Proportions\_At\_Length}.
\defRef{sec:Observation-ProportionsAtLength}
\label{syntax:Observation-ProportionsAtLength}

\defSub{time\_step}{The label of the time step that the observation occurs in}
\defType{String}
\defDefault{No default}

\defSub{years}{The years for which there are observations}
\defType{Vector of non-negative integers}
\defDefault{No default}

\defSub{selectivities}{The labels of the selectivities}
\defType{Vector of strings}
\defDefault{true}

\defSub{process\_errors}{The process errors to use}
\defType{Vector of real numbers (estimable) of length equal to the number of years}
\defDefault{0.0}
\defNote{If only one value is supplied, it will be repeated for all years in the observation}

\defSub{length\_bins}{The length bins}
\defType{Vector of real numbers (estimable)}
\defDefault{true}

\defSub{plus\_group}{Is the last length bin a plus group?}
\defType{Boolean}
\defDefault{true if the value of \commandsub{model}{length\_plus} is true, otherwise false}

\defSub{sum\_to\_one}{Scale the year (row) observed values by the total, so they sum to 1}
\defType{Boolean}
\defDefault{false}

\defSub{simulated\_data\_sum\_to\_one}{Whether simulated data is discrete or scaled by totals to be proportions for each year}
\defType{Boolean}
\defDefault{true}

\defSub{table obs}{The table of data specifying the observed values}
\defType{Data table with label = obs}
\defDefault{No default}
\defValue{A $n\times m$ matrix, where $n=$ the years and $m$ is categories \(\times\) length bins. See below for example.}
\defNote{example below}
\begin{verbatim}
table obs 
1993 0.1 0.2 0.3
1994 0.1 0.2 0.3
end_table
\end{verbatim}
\defSub{table error\_values}{The table of data specifying the error values}
\defType{Data table with label = error\_values}
\defDefault{No default}
\defValue{Can be specified two ways either as a $n\times 1$ matrix with an error value for each year. Or a $n\times m$ matrix, where $n=$ the years and $m$ is categories \(\times\) length bins. See below for example.}
\defNote{example below}
\begin{verbatim}
table error_values 
1993 234
1994 343
end_table
\end{verbatim}

\subsubsection{Observation of type Proportions By Category}
\commandlabsubarg{observation}{type}{Proportions\_By\_Category}.
\defRef{sec:Observation-ProportionsByCategory}
\label{syntax:Observation-ProportionsByCategory}

\defSub{min\_age}{The minimum age}
\defType{Non-negative integer}
\defDefault{No default}

\defSub{max\_age}{The maximum age}
\defType{Non-negative integer}
\defDefault{No default}

\defSub{time\_step}{The label of the time step that the observation occurs in}
\defType{String}
\defDefault{No default}

\defSub{plus\_group}{Use the age plus group?}
\defType{Boolean}
\defDefault{true}

\defSub{years}{The years for which there are observations}
\defType{Vector of non-negative integers}
\defDefault{No default}

\defSub{selectivities}{The labels of the selectivities}
\defType{Vector of strings}
\defDefault{true}

\defSub{categories2}{The target categories}
\defType{Vector of strings}
\defDefault{No default}

\defSub{selectivities2}{The target selectivities}
\defType{Vector of strings}
\defDefault{No default}

\subsubsection{Observation of type Proportions Mature By Age}
\commandlabsubarg{observation}{type}{Proportions\_Mature\_By\_Age}.
%\defRef{sec:Observation-ProportionsMatureByAge}
\label{syntax:Observation-ProportionsMatureByAge}

\defSub{min\_age}{The minimum age}
\defType{Non-negative integer}
\defDefault{No default}

\defSub{max\_age}{The maximum age}
\defType{Non-negative integer}
\defDefault{No default}

\defSub{time\_step}{The label of time-step that the observation occurs in}
\defType{String}
\defDefault{No default}

\defSub{plus\_group}{Use the age plus group?}
\defType{Boolean}
\defDefault{true}

\defSub{years}{The years for which there are observations}
\defType{Vector of non-negative integers}
\defDefault{No default}

\defSub{ageing\_error}{The label of ageing error to use}
\defType{String}
\defDefault{No default}

\defSub{total\_categories}{All category labels that were vulnerable to sampling at the time of this observation (not including the categories already given)}
\defType{Vector of strings}
\defDefault{true}

\defSub{time\_step\_proportion}{The proportion through the mortality block of the time step when the observation is evaluated}
\defType{Real number (estimable)}
\defDefault{0.5}
\defLowerBound{0.0 (inclusive)}
\defUpperBound{1.0 (inclusive)}

\subsubsection{Observation of type Proportions Migrating}
\commandlabsubarg{observation}{type}{Proportions\_Migrating}.
\defRef{sec:Observation-ProportionsMigrating}
\label{syntax:Observation-ProportionsMigrating}

\defSub{min\_age}{The minimum age}
\defType{Non-negative integer}
\defDefault{No default}

\defSub{max\_age}{The maximum age}
\defType{Non-negative integer}
\defDefault{No default}

\defSub{time\_step}{The label of the time step that the observation occurs in}
\defType{String}
\defDefault{No default}

\defSub{plus\_group}{Is the maximum age the age plus group?}
\defType{Boolean}
\defDefault{true}

\defSub{years}{The years for which there are observations}
\defType{Vector of non-negative integers}
\defDefault{No default}

\defSub{process\_errors}{The process errors to use}
\defType{Vector of real numbers (estimable) of length equal to the number of years}
\defDefault{0.0}
\defNote{If only one value is supplied, it will be repeated for all years in the observation}

\defSub{ageing\_error}{The label of the ageing error to use}
\defType{String}
\defDefault{No default}

\defSub{process}{The process label}
\defType{String}
\defDefault{No default}

\subsubsection{Observation of type Tag Recapture by Fishery}
\commandlabsubarg{observation}{type}{tag\_recapture\_by\_fishery}.
\defRef{sec:Observation-TagRecaptureByFishery}
\label{syntax:Observation-TagRecaptureByFishery}

\defSub{tagged\_categories}{The tagged categories that we want to generate recaptures for. Categories need to be space separated no use of the '+' category syntax.}
\defType{Vector of strings}
\defDefault{No default}

\defSub{time\_step}{The label of time-step that the observation occurs in}
\defType{Vector of strings}
\defDefault{No default}

\defSub{reporting\_rate}{The reporting rate for this observation}
\defType{Real number (estimable)}
\defDefault{No default}
\defLowerBound{0.0 (inclusive)}
\defUpperBound{1.0 (inclusive)}


\defSub{years}{The years for which there are observations}
\defType{Vector of non-negative integers}
\defDefault{No default}

\defSub{method\_of\_removal}{The label of the observed method of removals}
\defType{Vector of strings}
\defDefault{No default}


\defSub{mortality\_process}{The label of the mortality instantaneous process for the observation}
\defType{String}
\defDefault{No default}
\defNote{Allowed mortality process types are \subcommand{mortality\_instantaneous} and \subcommand{mortality\_hybrid}}

\defSub{table recaptured}{The table of recaptures in each year}
\defType{Data table with label = \subcommand{recaptured}}
\defDefault{No default}
\defValue{A $n\times 2$ matrix, where $n=$ the years and the first column specifies the year and the second column specifies the observed tag recaptures.}
\defNote{example below}
\begin{verbatim}
table recaptured
2000 10120
2001 90123
end_table
\end{verbatim}

\subsubsection{Observation of type Tag Recapture By Age}
\commandlabsubarg{observation}{type}{Tag\_Recapture\_By\_Age}.
\defRef{sec:Observation-TagRecaptureByAge}
\label{syntax:Observation-TagRecaptureByAge}

\defSub{min\_age}{The minimum age}
\defType{Non-negative integer}
\defDefault{No default}

\defSub{max\_age}{The maximum age}
\defType{Non-negative integer}
\defDefault{No default}

\defSub{plus\_group}{Is the maximum age the age plus group?}
\defType{Boolean}
\defDefault{true}

\defSub{years}{The years for which there are observations}
\defType{Vector of non-negative integers}
\defDefault{No default}

\defSub{time\_step}{The label of the time step that the observation occurs in}
\defType{String}
\defDefault{No default}

\defSub{selectivities}{The labels of the selectivities used for untagged categories}
\defType{Vector of strings}
\defDefault{true}

\defSub{tagged\_selectivities}{The labels of the tag category selectivities}
\defType{Vector of strings}
\defDefault{true}

\defSub{tagged\_categories}{The  categories of tagged individuals}
\defType{Vector of strings}
\defDefault{No default}

\defSub{detection}{The probability of detecting a recaptured individual}
\defType{Real number (estimable)}
\defDefault{No default}
\defLowerBound{0.0 (inclusive)}
\defUpperBound{1.0 (inclusive)}

\defSub{dispersion}{The over-dispersion parameter, $\phi$}
\defType{Vector of real numbers, one for each year of recaptures}
\defDefault{No default}
\defLowerBound{0.0}

\defSub{overlap\_scalar}{The overlap\_scalar parameter, $k$}
\defType{Vector of real numbers, one for each year of recaptures (if only one value is supplied, it is repeated for each year of recaptures)}
\defDefault{1.0}
\defLowerBound{0.0 (inclusive)}
\defNote{See Section \ref{sec:Observation-TagRecaptures} for more information}

\defSub{time\_step\_proportion}{The proportion through the mortality block of the time step when the observation is evaluated}
\defType{Real number (estimable)}
\defDefault{0.5}
\defLowerBound{0.0 (inclusive)}
\defUpperBound{1.0 (inclusive)}

\defSub{table recaptured}{The table of data specifying the recaptures}
\defType{Data table with label = recaptured}
\defDefault{No default}
\defValue{A $n\times m$ matrix, where $n=$ the years and $m$ is categories \(\times\) length bins. See below for example.}
\defNote{example below}
\begin{verbatim}
table recaptured 
1993 1 32 25
1994 3 4 43
end_table
\end{verbatim}

\defSub{table scanned}{The table of data specifying the scanned fish}
\defType{Data table with label = scanned}
\defDefault{No default}
\defValue{A $n\times m$ matrix, where $n=$ the years and $m$ is categories \(\times\) length bins. See below for example.}
\defNote{example below}
\begin{verbatim}
table scanned 
1993 1 32 25
1994 3 4 43
end_table
\end{verbatim}

\subsubsection{Observation of type Tag Recapture By Length}
\commandlabsubarg{observation}{type}{Tag\_Recapture\_By\_Length}.
\defRef{sec:Observation-TagRecaptureByLength}
\label{syntax:Observation-TagRecaptureByLength}

\defSub{years}{The years for which there are observations}
\defType{Vector of non-negative integers}
\defDefault{No default}

\defSub{time\_step}{The time step to execute in}
\defType{String}
\defDefault{No default}

\defSub{length\_bins}{The length bins}
\defType{Vector of real numbers (estimable)}
\defDefault{true}

\defSub{selectivities}{The labels of the selectivities used for untagged categories}
\defType{Vector of strings}
\defDefault{true}

\defSub{tagged\_selectivities}{The labels of the tag category selectivities}
\defType{Vector of strings}
\defDefault{No default}

\defSub{tagged\_categories}{The  categories of tagged individuals}
\defType{Vector of strings}
\defDefault{No default}

\defSub{detection}{The probability of detecting a recaptured individual}
\defType{Real number (estimable)}
\defDefault{No default}
\defLowerBound{0.0 (inclusive)}
\defUpperBound{1.0 (inclusive)}

\defSub{dispersion}{The over-dispersion parameter, $\phi$}
\defType{Vector of real numbers, one for each year of recaptures}
\defDefault{No default}
\defLowerBound{0.0}

\defSub{overlap\_scalar}{The overlap\_scalar parameter, $k$}
\defType{Vector of real numbers, one for each year of recaptures (if only one value is supplied, it is repeated for each year of recaptures)}
\defDefault{1.0}
\defLowerBound{0.0 (inclusive)}
\defNote{See Section \ref{sec:Observation-TagRecaptures} for more information}

\defSub{time\_step\_proportion}{The proportion through the mortality block of the time step when the observation is evaluated}
\defType{Real number (estimable)}
\defDefault{0.5}
\defLowerBound{0.0 (inclusive)}
\defUpperBound{1.0 (inclusive)}

\defSub{table recaptured}{The table of data specifying the recaptures}
\defType{Data table with label = recaptured}
\defDefault{No default}
\defValue{A $n\times m$ matrix, where $n=$ the years and $m$ is categories \(\times\) length bins. See below for example.}
\defNote{example below}
\begin{verbatim}
table recaptured 
1993 1 32 25
1994 3 4 43
end_table
\end{verbatim}

\defSub{table scanned}{The table of data specifying the scanned fish}
\defType{Data table with label = scanned}
\defDefault{No default}
\defValue{A $n\times m$ matrix, where $n=$ the years and $m$ is categories \(\times\) length bins. See below for example.}
\defNote{example below}
\begin{verbatim}
table scanned 
1993 1 32 25
1994 3 4 43
end_table
\end{verbatim}

\subsubsection{Observation of type Age Length}
\commandlabsubarg{observation}{type}{age\_length}.
\defRef{sec:Observation-AgeSize}
\label{syntax:Observation-AgeLength}

\defSub{time\_step}{The label of the time step that the observation occurs in}
\defType{String}
\defDefault{No default}

\defSub{selectivities}{The labels of the selectivities, one for each combined category}
\defType{Vector of strings}
\defDefault{true}

\defSub{numerator\_categories}{A combined category label that defines categories that make up the numerator}
\defType{Vector of strings}
\defNote{These categories are required to have the same age-length definition and have the same selectivity.}
\defDefault{the values defined in categories}

\defSub{year}{The year this observation occurred in}
\defType{Vector of non-negative integers}
\defDefault{No default}

\defSub{sample\_type}{The sample type}
\defType{string}
\defDefault{length}
\defallowed{age, length, random}

\defSub{ages}{vector of observed ages}
\defType{Vector of positive integers}
\defNote{Needs to be integers, with model age definition, and same number of elements as lengths}
\defDefault{No default}

\defSub{lengths}{vector of observed lengths}
\defType{Vector of real numbers}
\defNote{same number of elements as ages}
\defDefault{No default}

\defSub{ageing\_error}{The label of ageing error to use}
\defType{String}
\defDefault{No ageing error}



\clearemptydoublepage{}
\defComLab{report}{Define an object of type \emph{Report}}.
\defRef{sec:Report}
\label{syntax:Report}

\defSub{label}{The report label}
\defType{String}
\defDefault{No default}

\defSub{type}{The report type}
\defType{String}
\defDefault{No default}

\defSub{file\_name}{The file name. If not supplied, then output is directed to standard out}
\defType{String}
\defDefault{No default}

\defSub{write\_mode}{Specify if any previous file with the same name should be overwritten, appended to, or is generated using a sequential suffix}
\defType{String}
\defDefault{overwrite}
\defValue{valid options are \subcommand{append}, \subcommand{overwrite}, \subcommand{incremental\_suffix}}

\defSub{format}{Report output format}
\defType{String}
\defDefault{r}
\defValue{Either \R\ for formatting for reading into \R\, or \texttt{none} for no formatting}


\subsubsection{Report of type Default}
\commandlabsubarg{report}{type}{Default}.
\defRef{sec:Report-Default}
\label{syntax:Report-Default}

\defSub{catchabilities}{Report catchabilities}
\defType{Boolean}
\defDefault{false}
\defNote{Reports all valid catchabilities}

\defSub{derived\_quantities}{Report derived quantities}
\defType{Boolean}
\defDefault{false}
\defNote{Reports all valid derived quantities}

\defSub{observations}{Report observations}
\defType{Boolean}
\defDefault{false}
\defNote{Reports all valid observations}

\defSub{processes}{Report processes}
\defType{Boolean}
\defDefault{false}
\defNote{Reports all valid processes}

\defSub{projects}{Report projects}
\defType{Boolean}
\defDefault{false}
\defNote{Reports all valid projections}

\defSub{selectivities}{Report selectivities}
\defType{Boolean}
\defDefault{false}
\defNote{Reports all valid selectivities}

\defSub{time\_varying}{Report time-varying parameters}
\defType{Boolean}
\defDefault{false}
\defNote{Reports all valid time-varying parameters}

\defSub{parameter\_transformations}{Report all parameter transformations}
\defType{Boolean}
\defDefault{false}
\defNote{Reports all valid parameter transformations}

\subsubsection{Report of type Addressable}
\commandlabsubarg{report}{type}{Addressable}.
\defRef{sec:Report-Addressable}
\label{syntax:Report-Addressable}

\defSub{parameter}{The addressable parameter name}
\defType{String}
\defDefault{No default}

\defSub{years}{Define the years that the report is generated for}
\defType{Vector of non-negative integers}
\defDefault{No default}

\defSub{time\_step}{Defines the time-step that the report applies to}
\defType{String}
\defDefault{No default}
\defValue{A valid time step label}

\ifAgeBased
\subsubsection{Report of type Age Length}
\commandlabsubarg{report}{type}{Age\_Length}.
\defRef{sec:Report-AgeLength}
\label{syntax:Report-AgeLength}

\defSub{time\_step}{The time step label}
\defType{String}
\defDefault{No default}
\defValue{A valid time step label}

\defSub{years}{The years for the report}
\defType{Vector of non-negative integers}
\defDefault{All years}

\defSub{age\_length}{The age-length label}
\defType{String}
\defDefault{No default}
\else
\subsubsection{Report of type Growth Increment model}
\commandlabsubarg{report}{type}{growth\_increment}.
\defRef{sec:Report-GrowthIncrement}
\label{syntax:Report-GrowthIncrement}

\defSub{time\_step}{The time step label}
\defType{String}
\defDefault{No default}
\defValue{A valid time step label}

\defSub{years}{The years for the report}
\defType{Vector of non-negative integers}
\defDefault{All years}

\defSub{growth\_increment}{The growth-increment label}
\defType{String}
\defDefault{No default}
\fi

\ifAgeBased
\subsubsection{Report of type Ageing Error Matrix}
\commandlabsubarg{report}{type}{Ageing\_Error\_Matrix}.
\defRef{sec:Report-AgeingErrorMatrix}
\label{syntax:Report-AgeingErrorMatrix}

\defSub{ageing\_error}{The ageing error label}
\defType{String}
\defDefault{No default}
\fi

\subsubsection{Report of type Catchability}
\commandlabsubarg{report}{type}{Catchability}.
\defRef{sec:Report-Catchability}
\label{syntax:Report-Catchability}

\defSub{catchability}{The catchability label}
\defType{String}
\defDefault{No default}
\defValue{If not specified, then the label of the report is assumed to be the category label}

%\subsubsection{Report of type Category List}
%\commandlabsubarg{report}{type}{Category\_List}.
%\defRef{sec:Report-CategoryList}
%\label{syntax:Report-CategoryList}

%The Category\_List report has no additional subcommands.

\subsubsection{Report of type Correlation Matrix}
\commandlabsubarg{report}{type}{Correlation\_Matrix}.
\defRef{sec:Report-CorrelationMatrix}
\label{syntax:Report-CorrelationMatrix}

The Correlation\_Matrix report has no additional subcommands.

\subsubsection{Report of type Covariance Matrix}
\commandlabsubarg{report}{type}{Covariance\_Matrix}.
\defRef{sec:Report-CovarianceMatrix}
\label{syntax:Report-CovarianceMatrix}

The Covariance\_Matrix type has no additional subcommands.

\subsubsection{Report of type Derived Quantity}
\commandlabsubarg{report}{type}{Derived\_Quantity}.
\defRef{sec:Report-DerivedQuantity}
\label{syntax:Report-DerivedQuantity}

\defSub{derived\_quantity}{The derived quantity label}
\defType{String}
\defDefault{No default}
\defValue{If not specified, then the label of the report is assumed to be the derived quantity label}
 
\subsubsection{Report of type Equation Test}
\commandlabsubarg{report}{type}{Equation\_Test}.
\defRef{sec:eq_parser}
\label{syntax:Report-EquationTest}

\defSub{equation}{The equation to do a test run of}
\defType{Vector of strings}
\defDefault{No default}

\subsubsection{Report of type Estimate Summary}
\commandlabsubarg{report}{type}{Estimate\_Summary}.
\defRef{sec:Report-EstimateSummary}
\label{syntax:Report-EstimateSummary}

\defValue{A summary of the estimated (free parameters)}

The Estimate\_Summary type has no additional subcommands.

\subsubsection{Report of type Estimate Value}
\commandlabsubarg{report}{type}{Estimate\_Value}.
\defRef{sec:Report-EstimateValue}
\label{syntax:Report-EstimateValue}

\defValue{The free parameters and their values, in a format suitable for use with \texttt{-i}}

The Estimate\_Value report has no additional subcommands.

\subsubsection{Report of type Estimation Result}
\commandlabsubarg{report}{type}{Estimation\_Result}.
\defRef{sec:Report-EstimationResult}
\label{syntax:Report-EstimationResult}

\defValue{A summary of the results of the minimisation}

The Estimation\_Result report has no additional subcommands.

\subsubsection{Report of type Hessian Matrix}
\commandlabsubarg{report}{type}{Hessian\_Matrix}.
\defRef{sec:Report-HessianMatrix}
\label{syntax:Report-HessianMatrix}

The Hessian\_Matrix report has no additional subcommands.

\subsubsection{Report of type Initialisation}
\commandlabsubarg{report}{type}{Initialisation}.
\defRef{sec:Report-Initialisation}
\label{syntax:Report-Initialisation}

The Initialisation report has no additional subcommands.

\subsubsection{Report of type Initialisation\_Partition}
\commandlabsubarg{report}{type}{Initialisation\_Partition}.
\defRef{sec:Report-InitialisationPartition}
\label{syntax:Report-InitialisationPartition}

\subsubsection{Report of type MCMC Covariance}
\commandlabsubarg{report}{type}{MCMC\_Covariance}.
\defRef{sec:Report-MCMCCovariance}
\label{syntax:Report-MCMCCovariance}\\
\defValue{This will output the covariance matrices (the initial covariance matrix and the covariance matrix if adapted ) used for the MCMC chain.}

The MCMC\_Covariance report has no additional subcommands.  

\subsubsection{Report of type MCMC Objective}
\commandlabsubarg{report}{type}{MCMC\_Objective}.
\defRef{sec:Report-MCMCObjective}
\label{syntax:Report-MCMCObjective}

The MCMC\_Objective report has no additional subcommands.

\defSub{file\_name}{The file name. If not supplied the default filename is used}
\defType{string}
\defDefault{objectives}

\defSub{write\_mode}{Has a different default to the rest of the reports.}
\defType{String}
\defDefault{\subcommand{incremental\_suffix}}
\defValue{valid options are \subcommand{append}, \subcommand{overwrite}, \subcommand{incremental\_suffix}}


\subsubsection{Report of type MCMC Sample}
\commandlabsubarg{report}{type}{MCMC\_Sample}.
\defRef{sec:Report-MCMCSample}
\label{syntax:Report-MCMCSample}


\defSub{file\_name}{The file name. If not supplied the default filename is used}
\defType{string}
\defDefault{samples}

\defSub{write\_mode}{Has a different default to the rest of the reports.}
\defType{String}
\defDefault{\subcommand{incremental\_suffix}}
\defValue{valid options are \subcommand{append}, \subcommand{overwrite}, \subcommand{incremental\_suffix}}

The MCMC\_Sample report has no additional subcommands.

%\subsubsection{Report of type MPD}
%\commandlabsubarg{report}{type}{MPD}.
%\defRef{sec:Report-MPD}
%\label{syntax:Report-MPD}\\
%\defValue{An MPD report, consisting of the free parameters and the covariance matrix}

%The MPD report has no additional subcommands.

\subsubsection{Report of type Objective Function}
\commandlabsubarg{report}{type}{Objective\_Function}.
\defRef{sec:Report-ObjectiveFunction}
\label{syntax:Report-ObjectiveFunction}

The Objective\_Function type has no additional subcommands.

\subsubsection{Report of type Observation}
\commandlabsubarg{report}{type}{Observation}.
\defRef{sec:Report-Observation}
\label{syntax:Report-Observation}

\defSub{observation}{The observation label}
\defType{String}
\defDefault{No default}

\defSub{normalised\_residuals}{Print Normalised Residuals?}
\defType{Boolean}
\defDefault{true}
\defNote{Only generated if valid for associated likelihood}

\defSub{pearsons\_residuals}{Print Pearsons Residuals?}
\defType{Boolean}
\defDefault{true}
\defNote{Only generated if valid for associated likelihood}

\subsubsection{Report of type Output Parameters}
\commandlabsubarg{report}{type}{Output\_Parameters}.
\defRef{sec:Report-OutputParameters}
\label{syntax:Report-OutputParameters}

The Output\_Parameters report has no additional subcommands.

\subsubsection{Report of Parameter transformations}
\commandlabsubarg{report}{type}{parameter\_transformation}.
\defRef{sec:Report-ParameterTransformations}
\label{syntax:Report-ParameterTransformation}

\defSub{parameter\_transformation}{label of parameter transformation block}
\defType{String}
\defDefault{No default}


\subsubsection{Report of type Partition}
\commandlabsubarg{report}{type}{Partition}.
\defRef{sec:Report-Partition}
\label{syntax:Report-Partition}

\defSub{time\_step}{Time Step label}
\defType{String}
\defDefault{No default}

\defSub{years}{Years}
\defType{Vector of non-negative integers}
\defDefault{All years}

\subsubsection{Report of type Partition Biomass}
\commandlabsubarg{report}{type}{Partition\_Biomass}.
\defRef{sec:Report-PartitionBiomass}
\label{syntax:Report-PartitionBiomass}

\defSub{time\_step}{The time step label}
\defType{String}
\defDefault{No default}

\defSub{years}{The years for the report}
\defType{Vector of non-negative integers}
\defDefault{All years}

\subsubsection{Report of type Process}
\commandlabsubarg{report}{type}{Process}.
\defRef{sec:Report-Process}
\label{syntax:Report-Process}

\defSub{process}{The process label that is reported}
\defType{String}
\defDefault{No default}
\defValue{A valid process label}
\defValue{If not specified, then the label of the report is assumed to be the process label}


\subsubsection{Report of type Profile}
\commandlabsubarg{report}{type}{Profile}.
\defRef{sec:Profile}
\label{syntax:Report-Profile}

\subsubsection{Report of type Project}
\commandlabsubarg{report}{type}{Project}.
\defRef{sec:Project}
\label{syntax:Report-Project}

\defSub{project}{The project label that is reported}
\defType{String}
\defDefault{No default}
\defValue{If not specified, then the label of the report is assumed to be the projection label}

\subsubsection{Report of type Random Number Seed}
\commandlabsubarg{report}{type}{Random\_Number\_Seed}.
\defRef{sec:Report-RandomNumberSeed}
\label{syntax:Report-RandomNumberSeed}

The Random\_Number\_Seed type has no additional subcommands.

\subsubsection{Report of type Selectivity}
\commandlabsubarg{report}{type}{Selectivity}.
\defRef{sec:Report-Selectivity}
\label{syntax:Report-Selectivity}

\defSub{selectivity}{Selectivity name}
\defType{String}
\defDefault{No default}
\defValue{If not specified, then the label of the report is assumed to be the selectivity label}

\defSub{length\_values}{Length bins for reporting if a length-based selectivity in an age-based model}
\defType{Vector of real numbers}
\defDefault{If not specified and this is a length-based selectivity in an age-based model, then length bins specified for the model will be used}
\defNote{It is a fatal error if this is a report for a length-based selectivity in an age-based model, but neither the length values or \command{model.length\_bins} were supplied}

\subsubsection{Report of type Selectivity By Year}
\commandlabsubarg{report}{type}{selectivity\_by\_year}.
\defRef{sec:Report-SelectivityByYear}
\label{syntax:Report-SelectivityByYear}

\defSub{selectivity}{Selectivity name}
\defType{String}
\defDefault{No default}
\defValue{If not specified, then the label of the report is assumed to be the selectivity label}

\defSub{years}{years to report the selectivity in}
\defType{String}
\defDefault{true}
\defValue{If not specified will print for all years in of the model}

\defSub{time\_step}{Time step label}
\defType{String}
\defDefault{No default}
\defNote{This should not matter, but is required in order to identify the time step for each year when values are printed.}

\subsubsection{Report of type Simulated Observation}
\commandlabsubarg{report}{type}{Simulated\_Observation}.
\defRef{sec:Report-SimulatedObservation}
\label{syntax:Report-SimulatedObservation}

\defSub{observation}{The observation label}
\defType{String}
\defDefault{No default}
\defValue{If not specified, then the label of the report is assumed to be the observation label}

\subsubsection{Report of type Time Varying}
\commandlabsubarg{report}{type}{Time\_Varying}.
\defRef{sec:Report-TimeVarying}
\label{syntax:Report-TimeVarying}

\defSub{time\_varying}{The time varying label that is reported}
\defType{String}
\defDefault{No default}
\defValue{If not specified, then the label of the report is assumed to be the time varying label}



\clearemptydoublepage{}
\section{\I{Population command and subcommand syntax}\label{syntax:Population}}

The description of the methods for the population section is given in Section \ref{sec:Population}.

In the following section, the sub-section headers use a notation of the form "\textbf {@observation[label].type=abundance}" which, in this case, represents the input command fragment
{\small{\begin{verbatim}
@observation label # where label is a unique label for that observation
type=abundance
...
\end{verbatim}}}
The specific subcommands for a command are given within each command.

\subsection{\I{Model structure}}
\ifAgeBased
\defComLab{Model}{Define an object of type \emph{Model}}.
\defRef{sec:Model}
\label{syntax:Model}

\defSub{type}{Type of model (only type=age is currently implemented)}
\defType{String}
\defDefault{age}

\defSub{base\_weight\_units}{Define the units for the base weight measurement unit (grams, kilograms (kgs), or tonnes). This will be the default unit of any weight input values}
\defType{String}
\defDefault{tonnes}

\defSub{threads}{The number of threads to use for this model}
\defType{Non-negative integer}
\defDefault{1}
\defLowerBound{1 (inclusive)}

\subsubsection{Model of type Age}
\commandlabsubarg{Model}{type}{Age}.
\defRef{sec:Model-Age}
\label{syntax:Model-Age}

\defSub{final\_year}{Define the final year of the model, excluding years in the projection period}
\defType{Non-negative integer}
\defDefault{No default}
\defValue{Defines the last year of the model, i.e., the model is run from start\_year to final\_year}

\defSub{min\_age}{Minimum age of individuals in the population}
\defType{Non-negative integer}
\defDefault{0}
\defValue{$0 \le$ age\textlow{min} $\le$ age\textlow{max}}

\defSub{max\_age}{Maximum age of individuals in the population}
\defType{Non-negative integer}
\defDefault{0}
\defValue{$0 \le$ age\textlow{min} $\le$ age\textlow{max}}

\defSub{age\_plus}{Define the oldest age or extra length midpoint (plus group size) as a plus group}
\defType{Boolean}
\defDefault{true}
\defValue{true, false}

\defSub{initialisation\_phases}{Define the labels of the phases of the initialisation}
\defType{Vector of strings}
\defDefault{true}
\defValue{A list of valid labels defined by \texttt{@initialisation\_phase}}

\defSub{time\_steps}{Define the labels of the time steps, in the order that they are applied, to form the annual cycle}
\defType{Vector of strings}
\defDefault{No default}
\defValue{A list of valid labels defined by \texttt{@time\_step}}

\defSub{projection\_final\_year}{Define the final year of the model when running projections}
\defType{Non-negative integer}
\defDefault{0}
\defValue{A value greater than final\_year}

\defSub{length\_bins}{The minimum length in each length bin}
\defType{Vector of real numbers}
\defDefault{true}
\defValue{$0 \le$ length\textlow{min} $\le$ length\textlow{max}}

\defSub{length\_plus}{Specify whether there is a length plus group or not}
\defType{Boolean}
\defDefault{true}
\defValue{true, false}

\defSub{length\_plus\_group}{Mean length of length plus group}
\defType{Real number}
\defDefault{0}
\defValue{length\textlow{max} $<$ length\_plus\_group}


\else
\input{IncludedSyntax/ModelLength}
\fi

\subsection{\I{Initialisation}}
\input{IncludedSyntax/InitialisationPhase}

\subsection{\I{Categories}}
\ifAgeBased
\defComLab{Categories}{Define an object of type \emph{Categories}}.
\defRef{sec:Categories}
\label{syntax:Categories}

\defSub{format}{The format that the category names use}
\defType{String}
\defDefault{No default}

\defSub{names}{The names of the categories}
\defType{Vector of strings}
\defDefault{No default}

\defSub{age\_lengths}{The age-length relationship labels for each category}
\defType{Vector of strings}
\defDefault{true}

\defSub{growth\_increment}{The growth increment model label for each category}
\defType{Vector of strings}
\defDefault{No default}


\else
\input{IncludedSyntax/CategoriesLength}
\fi

\subsection{\I{Time-steps}}
\input{IncludedSyntax/TimeStep}

\subsection{\I{Processes}}
\ifAgeBased
\defComLab{process}{Define an object of type \emph{Process}}.
\defRef{sec:Process}
\label{syntax:Process}

\defSub{label}{The label of the process}
\defType{String}
\defDefault{No default}

\defSub{type}{The type of process}
\defType{String}
\defDefault{No default}

\subsubsection{Process of type Ageing}
\commandlabsubarg{process}{type}{Ageing}.
\defRef{sec:Process-Ageing}
\label{syntax:Process-Ageing}

\defSub{categories}{The labels of the categories to age}
\defType{Vector of strings}
\defDefault{No default}

%\subsubsection{Process of type Load Partition}
%\commandlabsubarg{process}{type}{Load\_Partition}.
%\defRef{sec:Process-LoadPartition}
%\label{syntax:Process-LoadPartition}
%
%\defSub{table}{The table of data specifying the n in each partition category and age}
%\defType{Data table with label = data}
%\defDefault{No default}
%\defValue{}
%\defNote{See \ref{sec:DataTable} for more details on specifying data tables}
%
\subsubsection{Process of type Maturation}
\commandlabsubarg{process}{type}{Maturation}.
\defRef{sec:Process-Maturation}
\label{syntax:Process-Maturation}

\defSub{from}{The list of categories to mature from}
\defType{Vector of strings}
\defDefault{No default}

\defSub{to}{The list of categories to mature to}
\defType{Vector of strings}
\defDefault{No default}

\defSub{selectivities}{The list of selectivities to use for maturation}
\defType{Vector of strings}
\defDefault{No default}

\defSub{years}{The years to be associated with the maturity rates}
\defType{Vector of non-negative integers}
\defDefault{No default}

\defSub{rates}{The rates to mature for each year}
\defType{Vector of real numbers (estimable)}
\defDefault{No default}

\subsubsection{Process of type Mortality Constant Rate}
\commandlabsubarg{process}{type}{Mortality\_Constant\_Rate}.
\defRef{sec:Process-MortalityConstantRate}
\label{syntax:Process-MortalityConstantRate}

\defSub{categories}{The list of category labels}
\defType{Vector of strings}
\defDefault{No default}

\defSub{m}{The mortality rates}
\defType{Real number (estimable)}
\defDefault{No default}
\defLowerBound{0.0 (inclusive)}

\defSub{time\_step\_proportions}{The time step proportions for the mortality rates}
\defType{Vector of real numbers}
\defDefault{false}
\defLowerBound{0.0 (inclusive)}
\defUpperBound{1.0 (inclusive)}
\defValue{The time step proportions must sum to one. If only one value is supplied, then the each time step is allocated an equal proportion. Otherwise the number of values must equal the number of time steps}

\defSub{relative\_m\_by\_age}{The list of mortality by age ogive labels for the categories}
\defType{Vector of strings}
\defDefault{No default}

\subsubsection{Process of type Mortality Constant Exploitation}
\commandlabsubarg{process}{type}{Mortality\_Constant\_Exploitation}.
\defRef{sec:Process-MortalityConstantExploitation}
\label{syntax:Process-MortalityConstantExploitation}

\defSub{categories}{The list of category labels}
\defType{Vector of strings}
\defDefault{No default}

\defSub{u}{The exploitation rates}
\defType{Real number (estimable)}
\defDefault{No default}
\defLowerBound{0.0 (inclusive)}
\defUpperBound{1.0 (inclusive)}

\defSub{time\_step\_proportions}{The time step proportions for the exploitation rates}
\defType{Vector of real numbers}
\defDefault{false}
\defLowerBound{0.0 (inclusive)}
\defUpperBound{1.0 (inclusive)}
\defValue{The time step proportions must sum to one. If only one value is supplied, then the each time step is allocated an equal proportion. Otherwise the number of values must equal the number of time steps}

\defSub{relative\_u\_by\_age}{The list of exploitation by age ogive labels for the categories}
\defType{Vector of strings}
\defDefault{No default}

\subsubsection{Process of type Mortality Disease Rate}
\commandlabsubarg{process}{type}{Mortality\_Disease\_Rate}.
\defRef{sec:Process-MortalityDiseaseRate}
\label{syntax:Process-MortalityDiseaseRate}

\defSub{categories}{The list of category labels}
\defType{Vector of strings}
\defDefault{No default}

\defSub{disease\_mortality\_rate}{The disease mortality rates}
\defType{Real number (estimable)}
\defDefault{No default}
\defLowerBound{0.0 (inclusive)}
\defUpperBound{10.0 (inclusive)}

\defSub{year\_effects}{Annual deviations around the disease mortality rate}
\defType{Vector of real numbers (estimable)}
\defDefault{No default}
\defLowerBound{0.0 (inclusive)}

\defSub{selectivities}{The list of selectivities}
\defType{Vector of strings}
\defDefault{No default}

\defSub{years}{Years in which to apply the disease mortality in}
\defType{Vector of non-negative integers}
\defDefault{No default}

\subsubsection{Process of type Mortality Event}
\commandlabsubarg{process}{type}{Mortality\_Event}.
\defRef{sec:Process-MortalityEvent}
\label{syntax:Process-MortalityEvent}

\defSub{categories}{The categories}
\defType{Vector of strings}
\defDefault{No default}

\defSub{years}{The years in which to apply the mortality process}
\defType{Vector of non-negative integers}
\defDefault{No default}

\defSub{catches}{The number of removals (catches) to apply for each year}
\defType{Vector of real numbers (estimable)}
\defDefault{No default}

\defSub{u\_max}{The maximum exploitation rate ($U\_{max}$)}
\defType{Real number}
\defDefault{0.99}
\defLowerBound{0.0 (inclusive)}
\defUpperBound{1.0 (inclusive)}

\defSub{selectivities}{The list of selectivities}
\defType{Vector of strings}
\defDefault{No default}

\defSub{penalty}{The label of the penalty to apply if the total number of removals cannot be taken}
\defType{String}
\defDefault{No default}

\subsubsection{Process of type Mortality Event Biomass}
\commandlabsubarg{process}{type}{Mortality\_Event\_Biomass}.
\defRef{sec:Process-MortalityEventBiomass}
\label{syntax:Process-MortalityEventBiomass}

\defSub{categories}{The category labels}
\defType{Vector of strings}
\defDefault{No default}

\defSub{selectivities}{The labels of the selectivities for each of the categories}
\defType{Vector of strings}
\defDefault{No default}

\defSub{years}{The years in which to apply the mortality process}
\defType{Vector of non-negative integers}
\defDefault{No default}

\defSub{catches}{The biomass of removals (catches) to apply for each year}
\defType{Vector of real numbers (estimable)}
\defDefault{No default}

\defSub{u\_max}{The maximum exploitation rate ($U\_{max}$)}
\defType{Real number (estimable)}
\defDefault{0.99}
\defLowerBound{0.0 (inclusive)}
\defUpperBound{1.0 (inclusive)}

\defSub{penalty}{The label of the penalty to apply if the total biomass of removals cannot be taken}
\defType{String}
\defDefault{No default}

\subsubsection{Process of type Mortality Holling Rate}
\commandlabsubarg{process}{type}{Mortality\_Holling\_Rate}.
\defRef{sec:Process-MortalityHollingRate}
\label{syntax:Process-MortalityHollingRate}

\defSub{prey\_categories}{The prey categories labels}
\defType{Vector of strings}
\defDefault{No default}

\defSub{predator\_categories}{The predator categories labels}
\defType{Vector of strings}
\defDefault{No default}

\defSub{is\_abundance}{Is vulnerable amount of prey and predator an abundance [true] or biomass [false]}
\defType{Boolean}
\defDefault{true}

\defSub{a}{Parameter a}
\defType{Real number (estimable)}
\defDefault{No default}
\defLowerBound{0.0 (inclusive)}

\defSub{b}{Parameter b}
\defType{Real number (estimable)}
\defDefault{No default}
\defLowerBound{0.0 (inclusive)}

\defSub{x}{This parameter controls the functional form: Holling function type 2 (x=2) or 3 (x=3), or generalised (Michaelis Menten, x>=1)}
\defType{Real number (estimable)}
\defDefault{No default}
\defLowerBound{1.0 (inclusive)}

\defSub{u\_max}{The maximum exploitation rate ($U\_{max}$)}
\defType{Real number}
\defDefault{0.99}
\defLowerBound{0.0 (inclusive)}
\defUpperBound{1.0 (exclusive)}

\defSub{prey\_selectivities}{The selectivities for prey categories}
\defType{Vector of strings}
\defDefault{true}

\defSub{predator\_selectivities}{The selectivities for predator categories}
\defType{Vector of strings}
\defDefault{true}

\defSub{penalty}{The label of penalty}
\defType{String}
\defDefault{No default}

\defSub{years}{The years in which to apply the mortality process}
\defType{Vector of non-negative integers}
\defDefault{No default}

\defSub{table}{The table of data specifying the predator selectivities}
\defType{Data table with label = \TODO{I don't know how this one works?}}
\defDefault{No default}
\defValue{}
\defNote{See \ref{sec:DataTable} for more details on specifying data tables}

\defSub{table}{The table of data specifying the prey selectivities}
\defType{Data table with label = \TODO{I don't know how this one works?}}
\defDefault{No default}
\defValue{}
\defNote{See \ref{sec:DataTable} for more details on specifying data tables}

\subsubsection{Process of type Mortality Initialisation Event}
\commandlabsubarg{process}{type}{mortality\_initialisation\_event}.
\defRef{sec:Process-MortalityInitialisationEvent}
\label{syntax:Process-MortalityInitialisationEvent}

\defSub{categories}{The categories}
\defType{Vector of strings}
\defDefault{No default}

\defSub{catch}{The number of removals (catches) to apply for each year}
\defType{Real number (estimable)}
\defDefault{No default}

\defSub{u\_max}{The maximum exploitation rate ($U\_{max}$)}
\defType{Real number}
\defDefault{0.99}
\defLowerBound{0.0 (inclusive)}
\defUpperBound{1.0 (exclusive)}

\defSub{selectivities}{The list of selectivities}
\defType{Vector of strings}
\defDefault{No default}

\defSub{penalty}{The label of the penalty to apply if the total number of removals cannot be taken}
\defType{String}
\defDefault{No default}

\defSub{table}{The table of data specifying the catches for each fishery, the categories, years, and the $U_max$}
\defType{Data table with label = \TODO{Craig? Can you specify this table?}}
\defDefault{No default}
\defNote{See \ref{sec:DataTable} for more details on specifying data tables}

\subsubsection{Process of type Mortality Initialisation Event Biomass}
\commandlabsubarg{process}{type}{Mortality\_Initialisation\_Event\_Biomass}.
\defRef{sec:Process-MortalityInitialisationEventBiomass}
\label{syntax:Process-MortalityInitialisationEventBiomass}

\defSub{categories}{The categories}
\defType{Vector of strings}
\defDefault{No default}

\defSub{catch}{The number of removals (catches) to apply for each year}
\defType{Real number (estimable)}
\defDefault{No default}

\defSub{u\_max}{The maximum exploitation rate ($U\_{max}$)}
\defType{Real number}
\defDefault{0.99}
\defLowerBound{0.0 (inclusive)}
\defUpperBound{1.0 (inclusive)}

\defSub{selectivities}{The list of selectivities}
\defType{Vector of strings}
\defDefault{No default}

\defSub{penalty}{The label of the penalty to apply if the total number of removals cannot be taken}
\defType{String}
\defDefault{No default}


\subsubsection{Process of type Mortality Initialisation Baranov}
\commandlabsubarg{process}{type}{mortality\_initialisation\_baranov}.
\defRef{sec:Process-MortalityInitialisationEventBiomass}
\label{syntax:Process-MortalityInitialisationBaranov}

\defSub{categories}{The categories}
\defType{Vector of strings}
\defDefault{No default}

\defSub{fishing\_mortality}{The fishing mortality to apply}
\defType{Real number (estimable)}
\defDefault{No default}

\defSub{selectivities}{The list of selectivities for each category}
\defType{Vector of strings}
\defDefault{No default}

\subsubsection{Process of type Mortality Hybrid}
\commandlabsubarg{process}{type}{mortality\_hybrid}.
\defRef{sec:Process-MortalityHybrid}
\label{syntax:Process-MortalityHybrid}

\defSub{categories}{The categories for to apply natural mortality to}
\defType{Vector of strings}
\defDefault{No default}

\defSub{m}{The natural mortality rates for each category}
\defType{Real number (estimable)}
\defDefault{No default}
\defLowerBound{0.0 (inclusive)}

\defSub{time\_step\_proportions}{The time step proportions for natural mortality}
\defType{Vector of real numbers}
\defDefault{true}
\defLowerBound{0.0 (inclusive)}
\defUpperBound{1.0 (inclusive)}
\defValue{Proportions must sum to one}

\defSub{biomass}{Switch to indicate if the catches are biomasses or abundances}
\defType{Boolean}
\defDefault{True}

\defSub{relative\_m\_by\_age}{The M-by-age selectivities to apply to each of the categories for natural mortality}
\defType{Vector of strings}
\defDefault{No default}

\defSub{max\_f}{Maximum \(F\) allowed}
\defType{Real number}
\defDefault{4.0}
\defLowerBound{0.0 (inclusive)}

\defSub{f\_iterations}{The number of tuning iterations to solve \(F\)}
\defType{integer}
\defDefault{4}
\defLowerBound{0 (inclusive)}

\defSub{table catches}{The table of data specifying the catches for each fishery and year}
\defDefault{No default}
\defNote{See below for example}
\begin{verbatim}
	table catches
	year Fishery1_label Fishery2_label
	1993 34 34 
	1994 23 34 
	end_table
\end{verbatim}

\defSub{table method}{The table of data specifying which fishery interacts with which category, selectivities, penalty, and time-step for each fishery and annual duration of the fishery.}
\defDefault{No default}
\defNote{See below for example}
\begin{verbatim}
	table method
	method         category selectivity annual_duration time_step penalty
	Fishery1_label male     fish_sel    1               step1     none
	Fishery2_label male     fish_sel    1               step1     none
	end_table
\end{verbatim}

\subsubsection{Process of type Mortality Instantaneous}
\commandlabsubarg{process}{type}{Mortality\_Instantaneous}.
\defRef{sec:Process-MortalityInstantaneous}
\label{syntax:Process-MortalityInstantaneous}

\defSub{categories}{The categories for instantaneous mortality}
\defType{Vector of strings}
\defDefault{No default}

\defSub{m}{The natural mortality rates for each category}
\defType{Real number (estimable)}
\defDefault{No default}
\defLowerBound{0.0 (inclusive)}

\defSub{time\_step\_proportions}{The time step proportions for natural mortality}
\defType{Vector of real numbers}
\defDefault{true}
\defLowerBound{0.0 (inclusive)}
\defUpperBound{1.0 (inclusive)}
\defValue{Proportions must sum to one}

\defSub{relative\_m\_by\_age}{The M-by-age selectivities to apply to each of the categories for natural mortality}
\defType{Vector of strings}
\defDefault{No default}

\defSub{table catches}{The table of data specifying the catches for each fishery and year}
\defDefault{No default}
\defValue{A column of catches as biomass or abundance; or a column of exploitation rates of biomass or abundance}
\defLowerBound{0 (inclusive)}
\defNote{See below for example}
\begin{verbatim}
	table catches
	year Fishery1_label Fishery2_label
	1993 34 34 
	1994 23 34 
	end_table
\end{verbatim}

\defSub{table method}{The table of data specifying which fishery interacts with which category, selectivities, u\_max, time-step, and penalty for each fishery and annual duration of the fishery.Optional columns can indicate the catches are in abundance and if the values are exploitation rates}
\defDefault{No default}
\defValue{Valid column names are: method, category, selectivity, u\_max, time\_step, and penalty; and optionally biomass, u, and age\_weight\_label}
\defNote{See below for example}
\begin{verbatim}
	table method
	method         category selectivity u_max time_step penalty
	Fishery1_label male     fish_sel    1     step1     CatchMustBeTaken
	Fishery2_label male     fish_sel    1     step1     CatchMustBeTaken
	end_table
\end{verbatim}

\subsubsection{Process of type Mortality Instantaneous Retained}
\commandlabsubarg{process}{type}{Mortality\_Instantaneous\_Retained}.
\defRef{sec:Process-MortalityInstantaneousRetained}
\label{syntax:Process-MortalityInstantaneousRetained}

\defSub{categories}{The categories for instantaneous mortality}
\defType{Vector of strings}
\defDefault{No default}

\defSub{m}{The natural mortality rates for each category}
\defType{Real number (estimable)}
\defDefault{No default}
\defLowerBound{0.0 (inclusive)}

\defSub{time\_step\_proportions}{The time step proportions for natural mortality}
\defType{Vector of real numbers}
\defDefault{No default}
\defLowerBound{0.0 (inclusive)}
\defUpperBound{1.0 (inclusive)}
\defValue{Proportions must sum to one}

\defSub{relative\_m\_by\_age}{The M-by-age selectivities to apply on the categories for natural mortality}
\defType{Vector of strings}
\defDefault{No default}

\defSub{table}{The table of data specifying the catches for each fishery, the categories, years, and the $U_max$}
\defType{Data table with label = \TODO{Craig? Can you specify this table?}}
\defDefault{No default}
\defValue{}
\defNote{See \ref{sec:DataTable} for more details on specifying data tables}

\subsubsection{Process of type Mortality Prey Suitability}
\commandlabsubarg{process}{type}{Mortality\_Prey\_Suitability}.
\defRef{sec:Process-MortalityPreySuitability}
\label{syntax:Process-MortalityPreySuitability}

\defSub{prey\_categories}{The prey categories labels}
\defType{Vector of strings}
\defDefault{No default}

\defSub{predator\_categories}{The predator categories labels}
\defType{Vector of strings}
\defDefault{No default}

\defSub{consumption\_rate}{The predator consumption rate}
\defType{Real number (estimable)}
\defDefault{No default}
\defLowerBound{0.0 (inclusive)}
\defUpperBound{1.0 (inclusive)}

\defSub{electivities}{The prey electivities}
\defType{Vector of real numbers (estimable)}
\defDefault{No default}
\defLowerBound{0.0 (inclusive)}
\defUpperBound{1.0 (inclusive)}

\defSub{u\_max}{The maximum exploitation rate ($U\_{max}$)}
\defType{Real number}
\defDefault{0.99}
\defLowerBound{0.0 (inclusive)}
\defUpperBound{1.0 (exclusive)}

\defSub{prey\_selectivities}{The selectivities for prey categories}
\defType{Vector of strings}
\defDefault{No default}

\defSub{predator\_selectivities}{The selectivities for predator categories}
\defType{Vector of strings}
\defDefault{No default}

\defSub{penalty}{The label of the penalty}
\defType{String}
\defDefault{No default}

\defSub{years}{The year that process occurs}
\defType{Vector of non-negative integers}
\defDefault{No default}
\subsubsection{Markovian Movement}
\commandlabsubarg{process}{type}{markovian\_movement}.
\defRef{sec:Process-MarkovianMovement}
\label{syntax:Process-MarkovianMovement}

\defSub{from}{The categories to transition from}
\defType{Vector of strings}
\defDefault{No default}
\defValue{Valid category labels}

\defSub{to}{The categories to transition to}
\defType{Vector of strings}
\defDefault{No default}
\defValue{Valid category labels}

\defSub{proportions}{The proportions to transition for each category}
\defType{Real number (estimable)}
\defDefault{No default}
\defLowerBound{0.0 (inclusive)}
\defUpperBound{1.0 (inclusive)}

\defSub{selectivities}{The selectivities to apply to each proportion}
\defType{Vector of strings}
\defDefault{No default}
\defValue{Valid selectivity labels}

\subsubsection{Process of type null process}
\commandlabsubarg{process}{type}{null\_process}.
\label{syntax:Process-Null}

The null\_process  type has no additional subcommands. Note that this process does nothing. It is included primarily as a means of replacing other processes with "no action" to allow for testing of alternative model structures.

\subsubsection{Process of type Recruitment Beverton Holt}
\commandlabsubarg{process}{type}{Recruitment\_Beverton\_Holt}.
\defRef{sec:Process-RecruitmentBevertonHolt}
\label{syntax:Process-RecruitmentBevertonHolt}

\defSub{categories}{The category labels}
\defType{Vector of strings}
\defDefault{No default}

\defSub{r0}{R0, the mean recruitment used to scale annual recruits or initialise the model}
\defType{Real number (estimable)}
\defDefault{No default}
\defLowerBound{0.0 (inclusive)}
\defValue{Use either R0 or B0, but not both}

\defSub{b0}{B0, the SSB corresponding to R0, and used to scale annual recruits or initialise the model}
\defType{Real number (estimable)}
\defDefault{No default}
\defLowerBound{0.0 (inclusive)}
\defValue{Use either R0 or B0, but not both}

\defSub{proportions}{The proportion for each category}
\defType{Real number (estimable)}
\defDefault{No default}

\defSub{age}{The age at recruitment}
\defType{Non-negative integer}
\defDefault{No default}

\defSub{ssb\_offset}{The spawning biomass year offset}
\defType{Non-negative integer}
\defDefault{No default}

\defSub{steepness}{Steepness (h)}
\defType{Real number (estimable)}
\defDefault{1.0}
\defLowerBound{0.2 (inclusive)}
\defUpperBound{1.0 (inclusive)}

\defSub{ssb}{The SSB label (i.e., the derived quantity label)}
\defType{String}
\defDefault{No default}

\defSub{b0\_initialisation\_phase}{The initialisation phase label that B0 is from}
\defType{String}
\defDefault{No default}

\defSub{ycs\_values}{Deprecated}
\defSub{ycs\_years}{Deprecated}
\defSub{standardise\_ycs\_years}{Deprecated}

\defSub{recruitment\_multipliers}{The recruitment values also termed year class strengths.}
\defType{Vector of real numbers (estimable)}
\defDefault{No default}

\defSub{standardise\_years}{The years that are included for year class standardisation, they refer to the recruited year not spawning or year class year.}
\defType{Vector of non-negative integers}
\defDefault{true}

\subsubsection{Process of type Recruitment Beverton Holt With Deviations}
\commandlabsubarg{process}{type}{Recruitment\_Beverton\_Holt\_With\_Deviations}.
\defRef{sec:Process-RecruitmentBevertonHoltWithDeviations}
\label{syntax:Process-RecruitmentBevertonHoltWithDeviations}

\defSub{categories}{The category labels}
\defType{Vector of strings}
\defDefault{No default}

\defSub{r0}{R0, the mean recruitment used to scale annual recruits or initialise the model}
\defType{Real number (estimable)}
\defDefault{No default}
\defLowerBound{0.0 (inclusive)}
\defValue{Use either R0 or B0, but not both}

\defSub{b0}{B0, the SSB corresponding to R0, and used to scale annual recruits or initialise the model}
\defType{Real number (estimable)}
\defDefault{No default}
\defLowerBound{0.0 (inclusive)}
\defValue{Use either R0 or B0, but not both}

\defSub{proportions}{The proportion for each category}
\defType{Real number (estimable)}
\defDefault{No default}

\defSub{age}{The age at recruitment}
\defType{Non-negative integer}
\defDefault{true}

\defSub{ssb\_offset}{The spawning biomass year offset}
\defType{Non-negative integer}
\defDefault{No default}

\defSub{steepness}{Steepness (h)}
\defType{Real number (estimable)}
\defDefault{1.0}
\defLowerBound{0.2 (inclusive)}
\defUpperBound{1.0 (inclusive)}

\defSub{ssb}{The SSB label (i.e., the derived quantity label)}
\defType{String}
\defDefault{No default}
\defValue(A valid derived quantity)

\defSub{sigma\_r}{The standard deviation of recruitment, $\sigma_R$}
\defType{Real number (estimable)}
\defDefault{No default}
\defLowerBound{0.0 (inclusive)}

\defSub{b\_max}{The maximum bias adjustment}
\defType{Real number (estimable)}
\defDefault{0.85}
\defLowerBound{0.0 (inclusive)}
\defUpperBound{1.0 (inclusive)}

\defSub{last\_year\_with\_no\_bias}{The last year with no bias adjustment}
\defType{Non-negative integer}
\defDefault{false}

\defSub{first\_year\_with\_bias}{The first year with full bias adjustment}
\defType{Non-negative integer}
\defDefault{false}

\defSub{last\_year\_with\_bias}{The last year with full bias adjustment}
\defType{Non-negative integer}
\defDefault{false}

\defSub{first\_recent\_year\_with\_no\_bias}{The first recent year with no bias adjustment}
\defType{Non-negative integer}
\defDefault{false}

\defSub{b0\_initialisation\_phase}{The initialisation phase label that B0 is from}
\defType{String}
\defDefault{No default}

\defSub{deviation\_values}{The recruitment deviation values}
\defType{Vector of real numbers (estimable)}
\defDefault{No default}

\defSub{deviation\_years}{Deprecated}

\subsubsection{Process of type Recruitment Ricker}
\commandlabsubarg{process}{type}{Recruitment\_Ricker}.
\defRef{sec:Process-RecruitmentRicker}
\label{syntax:Process-RecruitmentRicker}

\defSub{categories}{The category labels}
\defType{Vector of strings}
\defDefault{No default}

\defSub{r0}{R0, the mean recruitment used to scale annual recruits or initialise the model}
\defType{Real number (estimable)}
\defDefault{No default}
\defLowerBound{0.0 (inclusive)}
\defValue{Use either R0 or B0, but not both}

\defSub{b0}{B0, the SSB corresponding to R0, and used to scale annual recruits or initialise the model}
\defType{Real number (estimable)}
\defDefault{No default}
\defLowerBound{0.0 (inclusive)}
\defValue{Use either R0 or B0, but not both}

\defSub{proportions}{The proportion for each category}
\defType{Real number (estimable)}
\defDefault{No default}

\defSub{age}{The age at recruitment}
\defType{Non-negative integer}
\defDefault{No default}

\defSub{ssb\_offset}{The spawning biomass year offset}
\defType{Non-negative integer}
\defDefault{No default}

\defSub{steepness}{Steepness (h)}
\defType{Real number (estimable)}
\defDefault{1.0}
\defLowerBound{0.2 (inclusive)}
\defUpperBound{1.0 (inclusive)}

\defSub{ssb}{The SSB label (i.e., the derived quantity label)}
\defType{String}
\defDefault{No default}

\defSub{b0\_initialisation\_phase}{The initialisation phase label that B0 is from}
\defType{String}
\defDefault{No default}

\defSub{ycs\_values}{Deprecated}
\defSub{ycs\_years}{Deprecated}
\defSub{standardise\_ycs\_years}{Deprecated}

\defSub{recruitment\_multipliers}{The recruitment values also termed year class strengths.}
\defType{Vector of real numbers (estimable)}
\defDefault{No default}

\defSub{standardise\_years}{The years that are included for year class standardisation, they refer to the recruited year not spawning or year class year.}
\defType{Vector of non-negative integers}
\defDefault{true}

\subsubsection{Process of type Recruitment Constant}
\commandlabsubarg{process}{type}{Recruitment\_Constant}.
\defRef{sec:Process-RecruitmentConstant}
\label{syntax:Process-RecruitmentConstant}

\defSub{categories}{The categories}
\defType{Vector of strings}
\defDefault{No default}

\defSub{proportions}{The proportion for each category}
\defType{Real number (estimable)}
\defDefault{true}

\defSub{age}{The age at recruitment}
\defType{Non-negative integer}
\defDefault{No default}

\defSub{r0}{R0, the recruitment used for annual recruits and initialise the model}
\defType{Real number (estimable)}
\defDefault{No default}
\defLowerBound{0.0 (inclusive)}

\subsubsection{Process of type Survival Constant Rate}
\commandlabsubarg{process}{type}{Survival\_Constant\_Rate}.
\defRef{sec:Process-SurvivalConstantRate}
\label{syntax:Process-SurvivalConstantRate}

\defSub{categories}{The list of categories}
\defType{Vector of strings}
\defDefault{No default}

\defSub{s}{The survival rates}
\defType{Real number (estimable)}
\defDefault{No default}
\defLowerBound{0.0 (inclusive)}
\defUpperBound{1.0 (inclusive)}

\defSub{time\_step\_proportions}{The time step proportions for the survival rate $S$}
\defType{Vector of real numbers}
\defDefault{true}
\defLowerBound{0.0 (exclusive)}
\defUpperBound{1.0 (inclusive)}
\defValue{The proportions must sum to one}

\defSub{selectivities}{The selectivity labels for each category}
\defType{Vector of strings}
\defDefault{No default}

\subsubsection{Process of type Tag By Age}
\commandlabsubarg{process}{type}{Tag\_By\_Age}.
\defRef{sec:Process-TagByAge}
\label{syntax:Process-TagByAge}

\defSub{from}{The categories that are selected for tagging (i.e, transition from)}
\defType{Vector of strings}
\defDefault{No default}

\defSub{to}{The categories that have tags (i.e., transition to)}
\defType{Vector of strings}
\defDefault{No default}

\defSub{min\_age}{The minimum age tagged}
\defType{Non-negative integer}
\defDefault{No default}

\defSub{max\_age}{The maximum age tagged}
\defType{Non-negative integer}
\defDefault{No default}

\defSub{penalty}{The penalty label}
\defType{String}
\defDefault{No default}

\defSub{u\_max}{The maximum exploitation rate ($U\_{max}$)}
\defType{Real number}
\defDefault{0.99}
\defLowerBound{0.0 (inclusive)}
\defUpperBound{1.0 (exclusive)}

\defSub{years}{The years to execute the tagging in}
\defType{Vector of non-negative integers}
\defDefault{No default}
\defNote{This should list the years of data in the table numbers or table proportions described below}

\defSub{initial\_mortality}{The initial mortality value (as a instantaneous proportion)}
\defType{Real number (estimable)}
\defDefault{0.0}
\defLowerBound{0.0 (inclusive)}
\defUpperBound{1.0 (inclusive)}

\defSub{initial\_mortality\_selectivity}{The initial mortality selectivity label}
\defType{String}
\defDefault{No default}
\defValue{Valid selectivity labels}

\defSub{selectivities}{The selectivity labels}
\defType{Vector of strings}
\defDefault{No default}
\defValue{Valid selectivity labels}

\defSub{n}{the total number of tagged fish}
\defType{Vector of real numbers (estimable)}
\defDefault{No default}
\defNote{Only required if table proportions are also supplied}

\defSub{table}{The table of data specifying the either the numbers or proportions to tag from and to each category and year}
\defType{Data table with either label = numbers or label = proportions}
\defDefault{No default}
\defNote{If table proportions, then the total number (@process[label].n) should also be specified. See \ref{sec:DataTable} for more details on specifying data tables}

\defSub{table numbers}{The table of releases as numbers for the process}
\defType{Data table with label = numbers}
\defDefault{Can be replaced with 'table proportions' -- see below}
\defValue{A \(n_y \times (n_l\times n_c) + 1\) matrix, where $n_y=$ is the number of years, \(n_a\) are the ages and \(n_c\) are the number of categories. The first column is the year value for that row. See below for an example}
\defNote{example below}
\begin{verbatim}
	table numbers 
	1993 34 34 23 43
	1994 23 34 23 43
	end_table
\end{verbatim}

\defSub{table proportions}{The table of releases as numbers for the process}
\defType{Data table with label = proportions}
\defDefault{Can be replaced with table numbers -- see above}
\defValue{A \(n_y \times (n_l\times n_c) + 1\) matrix, where $n_y=$ is the number of years, \(n_a\) are the ages and \(n_c\) are the number of categories. The first column is the year value for that row. See below for an example}
\defNote{example below}
\begin{verbatim}
	n 200 300 ## need to specify n if you give proportions
	table proportions 
	1993 0.1 0.2 0.7
	1994 0.1 0.2 0.7
	end_table
\end{verbatim}

\defSub{tolerance}{Tolerance for checking the specified proportions sum to one}
\defType{Real number}
\defDefault{1e-5}
\defLowerBound{0 (inclusive)}
\defUpperBound{1.0 (inclusive)}

\subsubsection{Process of type Tag By Length}
\commandlabsubarg{process}{type}{Tag\_By\_Length}.
\defRef{sec:Process-TagByLength}
\label{syntax:Process-TagByLength}

\defSub{from}{The categories that are selected for tagging (i.e, transition from)}
\defType{Vector of strings}
\defDefault{No default}

\defSub{to}{The categories that have tags (i.e., transition to)}
\defType{Vector of strings}
\defDefault{No default}

\defSub{penalty}{The penalty label}
\defType{String}
\defDefault{No default}

\defSub{u\_max}{The maximum exploitation rate ($U\_{max}$)}
\defType{Real number}
\defDefault{0.99}
\defLowerBound{0.0 (inclusive)}
\defUpperBound{1.0 (exclusive)}

\defSub{compatibility\_option}{Backwards compatibility option: either casal2 (the default) or casal. This affects the penalty and age-length calculations}
\defType{string}
\defDefault{casal2}
\defValue{Valid options are \subcommand{casal2} \& \subcommand{casal}}

\defSub{years}{The years to execute the tagging events in}
\defType{Vector of non-negative integers}
\defDefault{No default}

\defSub{initial\_mortality}{The initial mortality value (as a proportion)}
\defType{Real number (estimable)}
\defDefault{0.0}
\defLowerBound{0.0 (inclusive)}
\defUpperBound{1.0 (inclusive)}

\defSub{initial\_mortality\_selectivity}{The initial mortality selectivity label}
\defType{String}
\defDefault{No default}
\defValue{A valid selectivity label}

\defSub{selectivities}{The selectivity labels}
\defType{Vector of strings}
\defDefault{No default}
\defValue{Valid selectivity labels}

\defSub{n}{the total number of tagged fish}
\defType{Vector of real numbers (estimable)}
\defDefault{No default}
\defNote{Only required if table proportions are also supplied}

\defSub{table}{The table of data specifying the either the numbers or proportions to tag from and to each category and year}
\defType{Data table with either label = numbers or label = proportions}
\defDefault{No default}
\defNote{If table proportions, then the total number (@process[label].n) should also be specified. See \ref{sec:DataTable} for more details on specifying data tables}

\defSub{table numbers}{The table of releases as numbers for the process}
\defType{Data table with label = numbers}
\defDefault{Can be replaced with 'table proportions' -- see below}
\defValue{A \(n_y \times (n_l\times n_c) + 1\) matrix, where $n_y=$ is the number of years, \(n_a\) are the number of ages ages and \(n_c\) are the number of categories. The first column is the year value for that row. See below for an example}
\defNote{example below}
\begin{verbatim}
	table numbers 
	1993 34 34 23 43
	1994 23 34 23 43
	end_table
\end{verbatim}

\defSub{table proportions}{The table of releases as numbers for the process}
\defType{Data table with label = proportions}
\defDefault{Can be replaced with table numbers -- see above}
\defValue{A \(n_y \times (n_l\times n_c) + 1\) matrix, where $n_y=$ is the number of years, \(n_a\) are the number of ages and \(n_c\) are the number of categories. The first column is the year value for that row. See below for an example}
\defNote{example below}
\begin{verbatim}
	n 200 300 ## need to specify n if you give proportions
	table proportions 
	1993 0.1 0.2 0.7
	1994 0.1 0.2 0.7
	end_table
\end{verbatim}

\defSub{tolerance}{Tolerance for checking the specified proportions sum to one}
\defType{Real number}
\defDefault{1e-5}
\defLowerBound{0 (inclusive)}
\defUpperBound{1.0 (inclusive)}

\subsubsection{Process of type Tag Loss}
\commandlabsubarg{process}{type}{Tag\_Loss}.
\defRef{sec:Process-TagLoss}
\label{syntax:Process-TagLoss}

\defSub{categories}{The list of categories}
\defType{Vector of strings}
\defDefault{No default}
\defValue{Valid category labels}

\defSub{tag\_loss\_rate}{The instantaneous tag loss rates}
\defType{Vector of real numbers (estimable)}
\defDefault{No default}
\defLowerBound{0.0 (inclusive)}
\defValue{The instantaneous rate of tag loss (supplied as either a single value that is applied to all categories, or a vector of length equal to the number of categories defined for this process)}

\defSub{time\_step\_proportions}{The time step proportions for tag loss}
\defType{Vector of real numbers (estimable)}
\defDefault{true}
\defLowerBound{0.0 (inclusive)}
\defUpperBound{1.0 (inclusive)}
\defNote{The sum of the values of time\_step\_proportions must equal 1.0}

\defSub{tag\_loss\_type}{The type of tag loss}
\defType{String}
\defDefault{single}
\defValue{Valid options are \subcommand{single} \& \subcommand{double}}

\defSub{selectivities}{The selectivities}
\defType{Vector of strings}
\defDefault{No default}

\defSub{year}{The year the first tagging release process was executed}
\defType{Non-negative integer}
\defDefault{No default}
\defNote{For the double tag loss rate, this is also assumed to be the first year in the calculation of the annual loss rates}

\subsubsection{Process of type Tag Loss Empirical}
\commandlabsubarg{process}{type}{Tag\_Loss\_Empirical}.
\defRef{sec:Process-TagLossEmpirical}
\label{syntax:Process-TagLossEmpirical}

\defSub{categories}{The list of categories}
\defType{Vector of strings}
\defDefault{No default}
\defValue{Valid category labels}

\defSub{tag\_loss\_rate}{The instantaneous tag loss rates}
\defType{Vector of real numbers (estimable)}
\defDefault{No default}
\defLowerBound{0.0 (inclusive)}
\defValue{The instantaneous rate of tag loss (supplied as either a single value that is applied to all years at liberty, or a vector of length equal to the number of years at liberty defined for this process)}

\defSub{time\_step\_proportions}{The time step proportions for tag loss}
\defType{Vector of real numbers (estimable)}
\defDefault{true}
\defLowerBound{0.0 (inclusive)}
\defUpperBound{1.0 (inclusive)}
\defNote{The sum of the values of time\_step\_proportions must equal 1.0}

\defSub{selectivities}{The selectivities}
\defType{Vector of strings}
\defDefault{No default}

\defSub{year}{The year the first tagging release process was executed}
\defType{Non-negative integer}
\defDefault{No default}
\defNote{For the tag loss rate, this is also assumed to be the first year at liberty in the calculation of the annual loss rates}

\defSub{years\_at\_liberty}{The years at liberty that the tag\_loss\_rate applies to}
\defType{Non-negative integer}
\defDefault{No default}
\defLowerBound{0 (inclusive)}
\defNote{Years at liberty must be a valid value between 0 and the maximum number of years in the model. The tag\_loss\_rate is not applied for years at liberty that are not specified}

\subsubsection{Process of type Transition Category}
\commandlabsubarg{process}{type}{Transition\_Category}.
\defRef{sec:Process-TransitionCategory}
\label{syntax:Process-TransitionCategory}

\defSub{from}{The categories to transition from}
\defType{Vector of strings}
\defDefault{No default}
\defValue{Valid category labels}

\defSub{to}{The categories to transition to}
\defType{Vector of strings}
\defDefault{No default}
\defValue{Valid category labels}

\defSub{proportions}{The proportions to transition for each category}
\defType{Real number (estimable)}
\defDefault{No default}
\defLowerBound{0.0 (inclusive)}
\defUpperBound{1.0 (inclusive)}

\defSub{selectivities}{The selectivities to apply to each proportion}
\defType{Vector of strings}
\defDefault{No default}
\defValue{Valid selectivity labels}

\defSub{include\_in\_mortality\_block}{Include this process within the mortality block}
\defType{Boolean}
\defDefault{false}
\defValue{Either true or false}
\defNote{Warning, if true, this may adversely effect derived quantities and observations that interpolate over the mortality block - see \defRef{sec:Process-TransitionCategory}}

\subsubsection{Process of type Transition Category By Age}
\commandlabsubarg{process}{type}{Transition\_Category\_By\_Age}.
\defRef{sec:Process-TransitionCategoryByAge}
\label{syntax:Process-TransitionCategoryByAge}

\defSub{from}{The categories to transition from}
\defType{Vector of strings}
\defDefault{No default}
\defValue{Valid category labels}

\defSub{to}{The categories to transition to}
\defType{Vector of strings}
\defDefault{No default}
\defValue{Valid category labels}

\defSub{min\_age}{The minimum age to transition}
\defType{Non-negative integer}
\defDefault{No default}
\defValue{Valid category labels}

\defSub{max\_age}{The maximum age to transition}
\defType{Non-negative integer}
\defDefault{No default}

\defSub{penalty}{The penalty label}
\defType{String}
\defDefault{No default}
\defValue{A valid penalty label}

\defSub{u\_max}{The maximum exploitation rate ($U\_{max}$)}
\defType{Real number (estimable)}
\defDefault{0.99}
\defLowerBound{0.0 (inclusive)}
\defUpperBound{1.0 (exclusive)}

\defSub{years}{The years to execute the transition in}
\defType{Vector of non-negative integers}
\defDefault{No default}

\defSub{table n}{The table of numbers at age to transition from and to each category}
\defType{See below for example}
\defDefault{No default}
\defValue{}
\defNote{See below for example}
\begin{verbatim}
table n
year 3 4 5 6
2008 1000 2000 3000 4000
end_table
\end{verbatim}

\else
\input{IncludedSyntax/ProcessLength}
\fi 

\subsection{\I{Time varying parameters}}
\input{IncludedSyntax/TimeVarying}

\subsection{\I{Derived quantities}}
\defComLab{Derived\_Quantity}{Define an object of type \emph{Derived\_Quantity}}.
\defRef{sec:DerivedQuantity}
\label{syntax:DerivedQuantity}

\defSub{type}{The type of derived quantity}
\defType{String}
\defDefault{No default}

\defSub{time\_step}{The time step in which to calculate the derived quantity}
\defType{String}
\defDefault{No default}

\defSub{categories}{The list of categories to use when calculating the derived quantity}
\defType{Vector of strings}
\defDefault{No default}

\defSub{selectivities}{The list of selectivities to use when calculating the derived quantity}
\defType{Vector of strings}
\defDefault{No default}

\defSub{time\_step\_proportion}{The proportion through the mortality block of the time step when the derived quantity is calculated}
\defType{Real number (estimable)}
\defDefault{0.5}
\defLowerBound{0.0 (inclusive)}
\defUpperBound{1.0 (inclusive)}

\defSub{time\_step\_proportion\_method}{The method for interpolating for the proportion through the mortality block}
\defType{String}
\defDefault{weighted\_sum}

\subsubsection{Derived\_Quantity of type Abundance}
\commandlabsubarg{Derived\_Quantity}{type}{Abundance}.
\defRef{sec:DerivedQuantity-Abundance}
\label{syntax:DerivedQuantity-Abundance}

The Abundance type has no additional subcommands.
\subsubsection{Derived\_Quantity of type Biomass}
\commandlabsubarg{Derived\_Quantity}{type}{Biomass}.
\defRef{sec:DerivedQuantity-Biomass}
\label{syntax:DerivedQuantity-Biomass}

\defSub{age\_weight\_labels}{The labels for the age-weights that correspond to each category for the biomass calculation}
\defType{Vector of strings}
\defDefault{No default}

\subsubsection{Derived\_Quantity of type Biomass\_Index}
\commandlabsubarg{Derived\_Quantity}{type}{Biomass\_Index}.
\defRef{sec:DerivedQuantity-BiomassIndex}
\label{syntax:DerivedQuantity-BiomassIndex}

\defSub{distribution}{The type of distribution for the biomass index}
\defType{String}
\defDefault{"lognormal"}

\defSub{cv}{The cv for the uncertainty for the distribution when generating the biomass index}
\defType{Real number}
\defDefault{0.2}
\defLowerBound{0.0 (inclusive)}

\defSub{bias}{The bias (a positive or negative proportion) when generating the biomass index}
\defType{Real number}
\defDefault{0.0}

\defSub{rho}{The autocorrelation in annual values when generating the biomass index}
\defType{Real number}
\defDefault{0.0}
\defLowerBound{0.0 (inclusive)}

\defSub{catchability}{The catchability to use when generating the biomass index}
\defType{String}
\defDefault{No default}



\ifAgeBased
\subsection{\I{Age-length relationship}}
\input{IncludedSyntax/AgeLength}
\subsection{\I{Age-weight}}
\input{IncludedSyntax/AgeWeight}
\else
\subsection{\I{Growth-Increment}}
\input{IncludedSyntax/GrowthIncrement}
\fi 

\subsection{\I{Length-weight}}
\input{IncludedSyntax/LengthWeight}

\subsection{\I{Selectivities}}
\defComLab{selectivity}{Define an object of type \emph{Selectivity}}.
\defRef{sec:Selectivity}
\label{syntax:Selectivity}

\defSub{label}{The label for the selectivity}
\defType{String}
\defDefault{No default}

\defSub{type}{The type of selectivity}
\defType{String}
\defDefault{No default}

\defSub{length\_based}{Is the selectivity length based?}
\defType{Boolean}
\defDefault{false}

\defSub{intervals}{The number of quantiles to evaluate a length-based selectivity over the age-length distribution}
\defType{Non-negative integer}
\defDefault{5}

\defSub{values}{}
\defType{Vector of addressables}
\defDefault{No default}

\defSub{length\_values}{}
\defType{Vector of addressables}
\defDefault{No default}

\subsubsection{Selectivity of type All Values}
\commandlabsubarg{selectivity}{type}{All\_Values}.
\defRef{sec:Selectivity-AllValues}
\label{syntax:Selectivity-AllValues}

\defSub{v}{The v parameter}
\defType{Vector of real numbers (estimable)}
\defDefault{No default}

\subsubsection{Selectivity of type All Values Bounded}
\commandlabsubarg{selectivity}{type}{All\_Values\_Bounded}.
\defRef{sec:Selectivity-AllValuesBounded}
\label{syntax:Selectivity-AllValuesBounded}

\defSub{l}{The low value (L)}
\defType{Non-negative integer}
\defDefault{No default}

\defSub{h}{The high value (H)}
\defType{Non-negative integer}
\defDefault{No default}

\defSub{v}{The v parameter}
\defType{Vector of real numbers (estimable)}
\defDefault{No default}

\subsubsection{Selectivity of type Constant}
\commandlabsubarg{selectivity}{type}{Constant}.
\defRef{sec:Selectivity-Constant}
\label{syntax:Selectivity-Constant}

\defSub{a}{The $a$ value in $ax^b + c$}
\defType{Real number (estimable)}
\defDefault{0.0}
\defNote{The defaults result in a simple linear constant where $x=1$ for all values of $x$}

\defSub{b}{The $b$ value in $ax^b + c$}
\defType{Real number (estimable)}
\defDefault{0.0}
\defLowerBound{0.0 (inclusive)}
\defNote{The defaults result in a simple linear constant where $x=1$ for all values of $x$}

\defSub{c}{The $c$ value in $ax^b + c$}
\defType{Real number (estimable)}
\defDefault{1.0}
\defLowerBound{0.0 (inclusive)}
\defNote{The defaults result in a simple linear constant where $x=1$ for all values of $x$}

\defSub{beta}{The minimum age/length for which the selectivity applies}
\defType{Real number (constant)}
\defDefault{0.0}
\defLowerBound{0.0 (inclusive)}

\subsubsection{Selectivity of type Double Exponential}
\commandlabsubarg{selectivity}{type}{Double\_Exponential}.
\defRef{sec:Selectivity-DoubleExponential}
\label{syntax:Selectivity-DoubleExponential}

\defSub{x0}{The $x0$ parameter}
\defType{Real number (estimable)}
\defDefault{No default}

\defSub{x1}{The $x1$ parameter}
\defType{Real number (estimable)}
\defDefault{No default}

\defSub{x2}{The $x2$ parameter}
\defType{Real number (estimable)}
\defDefault{No default}

\defSub{y0}{The $y0$ parameter}
\defType{Real number (estimable)}
\defDefault{No default}
\defLowerBound{0.0 (inclusive)}

\defSub{y1}{The $y1$ parameter}
\defType{Real number (estimable)}
\defDefault{No default}
\defLowerBound{0.0 (inclusive)}

\defSub{y2}{The $y2$ parameter}
\defType{Real number (estimable)}
\defDefault{No default}
\defLowerBound{0.0 (inclusive)}

\defSub{alpha}{The maximum value of the selectivity}
\defType{Real number (estimable)}
\defDefault{1.0}
\defLowerBound{0.0 (exclusive)}

\defSub{beta}{The minimum age/length for which the selectivity applies}
\defType{Real number (constant)}
\defDefault{0}
\defLowerBound{0.0 (inclusive)}

\subsubsection{Selectivity of type Double Normal}
\commandlabsubarg{selectivity}{type}{Double\_Normal}.
\defRef{sec:Selectivity-DoubleNormal}
\label{syntax:Selectivity-DoubleNormal}

\defSub{mu}{The mean ($\mu$}
\defType{Real number (estimable)}
\defDefault{No default}

\defSub{sigma\_l}{The left-hand variance (sigma\_l) parameter}
\defType{Real number (estimable)}
\defDefault{No default}
\defLowerBound{0.0 (exclusive)}

\defSub{sigma\_r}{The right-hand variance (sigma\_r) parameter}
\defType{Real number (estimable)}
\defDefault{No default}
\defLowerBound{0.0 (exclusive)}

\defSub{alpha}{The maximum value of the selectivity}
\defType{Real number (estimable)}
\defDefault{1.0}
\defLowerBound{0.0 (exclusive)}

\defSub{beta}{The minimum age/length for which the selectivity applies}
\defType{Real number (constant)}
\defDefault{0}
\defLowerBound{0.0 (inclusive)}

\subsubsection{Selectivity of type Double Normal Plateau}
\commandlabsubarg{selectivity}{type}{Double\_Normal\_Plateau}.
\defRef{sec:Selectivity-DoubleNormalPlateau}
\label{syntax:Selectivity-DoubleNormalPlateau}

\defSub{a1}{The a1 ($a1$}
\defType{Real number (estimable)}
\defDefault{No default}

\defSub{a2}{The a2 ($a2$}
\defType{Real number (estimable)}
\defDefault{No default}

\defSub{sigma\_l}{The left-hand variance (sigma\_l) parameter}
\defType{Real number (estimable)}
\defDefault{No default}
\defLowerBound{0.0 (exclusive)}

\defSub{sigma\_r}{The right-hand variance (sigma\_r) parameter}
\defType{Real number (estimable)}
\defDefault{No default}
\defLowerBound{0.0 (exclusive)}

\defSub{alpha}{The maximum value of the selectivity}
\defType{Real number (estimable)}
\defDefault{1.0}
\defLowerBound{0.0 (exclusive)}

\defSub{beta}{The minimum age/length for which the selectivity applies}
\defType{Real number (constant)}
\defDefault{0}
\defLowerBound{0.0 (inclusive)}

\subsubsection{Selectivity of type Double Normal Stock Synthesis}
\commandlabsubarg{selectivity}{type}{Double\_Normal\_Stock\_Synthesis}.
\defRef{sec:Selectivity-DoubleNormalStockSynthesis}
\label{syntax:Selectivity-DoubleNormalStockSynthesis}

\defSub{peak}{Age or length of plateau (max selectivity)}
\defType{Real number (estimable)}
\defLowerBound{0.0 (exclusive)}

\defSub{y0}{Transformed selectivity for the first age or length bin}
\defType{Real number (estimable)}
\defLowerBound{-20}
\defUpperBound{0}

\defSub{y1}{Transformed selectivity for the last age or length bins}
\defType{Real number (estimable)}
\defLowerBound{-20}
\defUpperBound{10}

\defSub{descending}{The shape of descending limb in either ages or lengths}
\defType{Real number (estimable)}
\defDefault{No default}

\defSub{ascending}{The shape of ascending limb in either ages or lengths}
\defType{Real number (estimable)}
\defDefault{No default}

\defSub{width}{width of plateau how many ages or lengths are in the plateau}
\defType{Real number (estimable)}
\defDefault{No default}
\defLowerBound{0.0 (exclusive)}

\defSub{l}{min age or first length bin}
\defType{Real number}
\defDefault{No default}
\defLowerBound{0.0 (exclusive)}

\defSub{l}{max age or last length bin}
\defType{Real number}
\defDefault{No default}
\defLowerBound{0.0 (exclusive)}

\defSub{alpha}{The maximum value of the selectivity}
\defType{Real number (estimable)}
\defDefault{1.0}
\defLowerBound{0.0 (exclusive)}

\subsubsection{Selectivity of type Increasing}
\commandlabsubarg{selectivity}{type}{Increasing}.
\defRef{sec:Selectivity-Increasing}
\label{syntax:Selectivity-Increasing}

\defSub{l}{The low value (L)}
\defType{Non-negative integer}
\defDefault{No default}

\defSub{h}{The high value (H)}
\defType{Non-negative integer}
\defDefault{No default}

\defSub{v}{The v parameter}
\defType{Vector of real numbers (estimable)}
\defDefault{No default}

\defSub{alpha}{The maximum value of the selectivity }
\defType{Real number (estimable)}
\defDefault{1.0}
\defLowerBound{0.0 (exclusive)}

\subsubsection{Selectivity of type Inverse Logistic}
\commandlabsubarg{selectivity}{type}{Inverse\_Logistic}.
\defRef{sec:Selectivity-InverseLogistic}
\label{syntax:Selectivity-InverseLogistic}

\defSub{a50}{The age or length where the selectivity is \(50\%\)}
\defType{Real number (estimable)}
\defDefault{No default}

\defSub{ato95}{The age or length between \(50\%\) and \(95\%\) selective}
\defType{Real number (estimable)}
\defDefault{No default}
\defLowerBound{0.0 (exclusive)}

\defSub{alpha}{The maximum value of the selectivity }
\defType{Real number (estimable)}
\defDefault{1.0}
\defLowerBound{0.0 (exclusive)}

\defSub{beta}{The minimum age/length for which the selectivity applies}
\defType{Real number (constant)}
\defDefault{0}
\defLowerBound{0.0 (inclusive)}

\subsubsection{Selectivity of type Knife Edge}
\commandlabsubarg{selectivity}{type}{Knife\_Edge}.
\defRef{sec:Selectivity-KnifeEdge}
\label{syntax:Selectivity-KnifeEdge}

\defSub{e}{The edge value}
\defType{Real number (estimable)}
\defDefault{No default}

\defSub{alpha}{The maximum value of the selectivity }
\defType{Real number (estimable)}
\defDefault{1.0}

\subsubsection{Selectivity of type Logistic}
\commandlabsubarg{selectivity}{type}{Logistic}.
\defRef{sec:Selectivity-Logistic}
\label{syntax:Selectivity-Logistic}

\defSub{a50}{The age or length where the selectivity is \(50\%\)}
\defType{Real number (estimable)}
\defDefault{No default}

\defSub{ato95}{The age or length between \(50\%\) and \(95\%\) selective}
\defType{Real number (estimable)}
\defDefault{No default}
\defLowerBound{0.0 (exclusive)}

\defSub{alpha}{The maximum value of the selectivity}
\defType{Real number (estimable)}
\defDefault{1.0}
\defLowerBound{0.0 (exclusive)}

\defSub{beta}{The minimum age/length for which the selectivity applies}
\defType{Real number (constant)}
\defDefault{0}
\defLowerBound{0.0 (inclusive)}

\subsubsection{Selectivity of type Logistic Producing}
\commandlabsubarg{selectivity}{type}{Logistic\_Producing}.
\defRef{sec:Selectivity-LogisticProducing}
\label{syntax:Selectivity-LogisticProducing}

\defSub{l}{The low value (L)}
\defType{Non-negative integer}
\defDefault{No default}

\defSub{h}{The high value (H)}
\defType{Non-negative integer}
\defDefault{No default}

\defSub{a50}{The a50 parameter}
\defType{Real number (estimable)}
\defDefault{No default}

\defSub{ato95}{the ato95 parameter}
\defType{Real number (estimable)}
\defDefault{No default}
\defLowerBound{0.0 (exclusive)}

\defSub{alpha}{The maximum value of the selectivity}
\defType{Real number (estimable)}
\defDefault{1.0}
\defLowerBound{0.0 (exclusive)}

\subsubsection{Selectivity of type Compound Left }
\commandlabsubarg{selectivity}{type}{compound\_left}.
\defRef{sec:Selectivity-CompoundLeft}
\label{syntax:Selectivity-CompoundLeft}

\defSub{a50}{The a50 ($a50$}
\defType{Real number (estimable)}
\defDefault{No default}

\defSub{ato95}{The age or length between \(50\%\) and \(95\%\) selective}
\defType{Real number (estimable)}
\defDefault{No default}
\defLowerBound{0.0 (exclusive)}

\defSub{a\_min}{The (a\_min) parameter}
\defType{Real number (estimable)}
\defDefault{No default}
\defLowerBound{0.0 (exclusive)}

\defSub{left\_mean}{The left\_mean parameter}
\defType{Real number (estimable)}
\defDefault{1.0}
\defLowerBound{0.0 (exclusive)}

\defSub{sigma}{The sigma parameter}
\defType{Real number (estimable)}
\defDefault{1.0}
\defLowerBound{0.0 (exclusive)}


\subsubsection{Selectivity of type Compound Right }
\commandlabsubarg{selectivity}{type}{compound\_right}.
\defRef{sec:Selectivity-CompoundRight}
\label{syntax:Selectivity-CompoundRight}

\defSub{a50}{The a50 ($a50$}
\defType{Real number (estimable)}
\defDefault{No default}

\defSub{ato95}{The age or length between \(50\%\) and \(95\%\) selective}
\defType{Real number (estimable)}
\defDefault{No default}
\defLowerBound{0.0 (exclusive)}

\defSub{a\_min}{The (a\_min) parameter}
\defType{Real number (estimable)}
\defDefault{No default}
\defLowerBound{0.0 (exclusive)}

\defSub{left\_mean}{The left\_mean parameter}
\defType{Real number (estimable)}
\defDefault{1.0}
\defLowerBound{0.0 (exclusive)}

\defSub{to\_right\_mean}{The to\_right\_mean parameter}
\defType{Real number (estimable)}
\defDefault{1.0}
\defLowerBound{0.0 (exclusive)}

\defSub{sigma}{The sigma parameter}
\defType{Real number (estimable)}
\defDefault{1.0}
\defLowerBound{0.0 (exclusive)}

\subsubsection{Selectivity of type Compound Middle }
\commandlabsubarg{selectivity}{type}{compound\_middle}.
%\defRef{sec:Selectivity-CompoundMiddle}
\label{syntax:Selectivity-CompoundMiddle}

\defSub{a50}{The a50 ($a50$}
\defType{Real number (estimable)}
\defDefault{No default}

\defSub{ato95}{The age or length between \(50\%\) and \(95\%\) selective}
\defType{Real number (estimable)}
\defDefault{No default}
\defLowerBound{0.0 (exclusive)}

\defSub{a\_min}{The (a\_min) parameter}
\defType{Real number (estimable)}
\defDefault{No default}
\defLowerBound{0.0 (exclusive)}

\defSub{left\_mean}{The left\_mean parameter}
\defType{Real number (estimable)}
\defDefault{1.0}
\defLowerBound{0.0 (exclusive)}

\defSub{to\_right\_mean}{The to\_right\_mean parameter}
\defType{Real number (estimable)}
\defDefault{1.0}
\defLowerBound{0.0 (exclusive)}

\defSub{sigma}{The sigma parameter}
\defType{Real number (estimable)}
\defDefault{1.0}
\defLowerBound{0.0 (exclusive)}

\subsubsection{Selectivity of type Compound All }
\commandlabsubarg{selectivity}{type}{compound\_all}.
\defRef{sec:Selectivity-CompoundAll}
\label{syntax:Selectivity-CompoundAll}

\defSub{a50}{The a50 ($a50$}
\defType{Real number (estimable)}
\defDefault{No default}

\defSub{ato95}{The age or length between \(50\%\) and \(95\%\) selective}
\defType{Real number (estimable)}
\defDefault{No default}
\defLowerBound{0.0 (exclusive)}

\defSub{a\_min}{The (a\_min) parameter}
\defType{Real number (estimable)}
\defDefault{No default}
\defLowerBound{0.0 (exclusive)}

\defSub{sigma}{The sigma parameter}
\defType{Real number (estimable)}
\defDefault{1.0}
\defLowerBound{0.0 (exclusive)}

\subsubsection{Selectivity of type Multi-Selectivity}
\commandlabsubarg{Selectivity}{type}{multi\_selectivity}.
\defRef{sec:Selectivity-MultiSelectivities}
\label{syntax:Selectivity-MultiSelectivities}

\defSub{years}{The years for which we want to apply the corresponding selectivity in}
\defType{Vector of integer for all model years to apply corresponding selectivity.}
\defDefault{No default}

\defSub{selectivity\_labels}{The labels of the selectivities, one for each year}
\defType{Vector of strings defining the labels of the selectivities to be used for each year}
\defDefault{No default}

\defSub{default\_selectivity}{The selectivity used in missing years}
\defType{string}
\defDefault{No default}

\defSub{projection\_selectivity}{The selectivity used in missing years in projections}
\defType{string}
\defDefault{Defaults to \argument{default\_selectivity} if not supplied}


\subsection{\I{Projections}}
\defComLab{project}{Define an object of type \emph{Project}}.
\defRef{sec:Project}
\label{syntax:Project}

\defSub{label}{Label}
\defType{String}
\defDefault{No default}

\defSub{type}{Type}
\defType{String}
\defDefault{No default}

\defSub{years}{Years to recalculate the values}
\defType{Vector of non-negative integers}
\defDefault{No default}

\defSub{parameter}{Parameter to project}
\defType{String}
\defDefault{No default}

\subsubsection{Project of type Constant}
\commandlabsubarg{project}{type}{Constant}.
\defRef{sec:Project-Constant}
\label{syntax:Project-Constant}

\defSub{values}{The values to assign to the addressable}
\defType{Vector of real numbers (estimable)}
\defDefault{No default}

\defSub{multiplier}{Multiplier that is applied to the projected value}
\defType{Real number (estimable)}
\defDefault{1.0}
\defLowerBound{0.0 (exclusive)}
\defValue{A vector of length 1 (for a constant value for all years), or a vector of length years (for a specific value for each year)}

\subsubsection{Project of type Multiple Values}
\commandlabsubarg{project}{type}{multiple\_values}.
\defRef{sec:Project-MultipleValues}
\label{syntax:Multiple-Values}

\defSub{multiplier}{Multiplier that is applied to the projected value}
\defType{Real number (estimable)}
\defDefault{1.0}
\defLowerBound{0.0 (exclusive)}
\defValue{A vector of length 1 (for a constant value for all years), or a vector of length years (for a specific value for each year)}

\defSub{table values}{The table of data specifying the projected values to use for each row of the supplied free parameter file (\texttt{-i} or \texttt{-I}) with one column of data for each projected year}
\defType{table with label = values}
\defDefault{No default}
\defValue{A $n_i \times n_y$ matrix. Where \(n_i\) is the number of rows (parameter sets) in the free parameter file. And \(n_y\) is the number of projection years defined by the input \texttt{years}}
\defNote{example below}
\begin{verbatim}
	@project future_disease_rates
	type multiple_values
	parameter process[dtransition].proportions{disease}  	
	years 2024:2026
	table values
	# 2024 2025 2026
	   0.1  0.2  0.3
	   0.3  0.4  0.5
	end_table
\end{verbatim}

\subsubsection{Project of type Empirical Sampling}
\commandlabsubarg{project}{type}{Empirical\_Sampling}.
\defRef{sec:Project-EmpiricalSampling}
\label{syntax:Project-EmpiricalSampling}

\defSub{start\_year}{The start year of sampling}
\defType{Non-negative integer}
\defDefault{No default}

\defSub{final\_year}{The final year of sampling}
\defType{Non-negative integer}
\defDefault{No default}

\defSub{multiplier}{Multiplier that is applied to the projected value}
\defType{Real number (estimable)}
\defDefault{1.0}
\defLowerBound{0.0 (exclusive)}
\defValue{A vector of length 1 (for a constant value for all years), or a vector of length years (for a specific value for each year)}

\subsubsection{Project of type Lognormal}
\commandlabsubarg{project}{type}{Lognormal}.
\defRef{sec:Project-LogNormal}
\label{syntax:Project-LogNormal}

\defSub{mean}{The mean of the lognormal process}
\defType{Real number (estimable)}
\defDefault{0.0}

\defSub{sigma}{The standard deviation (sigma) of the lognormal sampling}
\defType{Real number (estimable)}
\defDefault{No default}
\defLowerBound{0.0 (inclusive)}

\defSub{multiplier}{Multiplier that is applied to the projected value}
\defType{Real number (estimable)}
\defDefault{1.0}
\defLowerBound{0.0 (exclusive)}
\defValue{A vector of length 1 (for a constant value for all years), or a vector of length years (for a specific value for each year)}

\subsubsection{Project of type Lognormal Empirical}
\commandlabsubarg{project}{type}{Lognormal\_Empirical}.
\defRef{sec:Project-LogNormalEmpirical}
\label{syntax:Project-LogNormalEmpirical}

\defSub{mean}{The mean of the Gaussian process}
\defType{Real number (estimable)}
\defDefault{0.0}

\defSub{start\_year}{The start year of sampling}
\defType{Non-negative integer}
\defDefault{No default}

\defSub{final\_year}{The final year of sampling}
\defType{Non-negative integer}
\defDefault{No default}

\defSub{multiplier}{Multiplier that is applied to the projected value}
\defType{Real number (estimable)}
\defDefault{1.0}
\defLowerBound{0.0 (exclusive)}
\defValue{A vector of length 1 (for a constant value for all years), or a vector of length years (for a specific value for each year)}

\subsubsection{Project of type Harvest Strategy Constant Catch}
\commandlabsubarg{Project}{type}{harvest\_strategy\_constant\_catch}.
\defRef{sec:Project-HarvestStrategyConstantCatch}
\label{syntax:Project-HarvestStrategyConstantCatch}

\defSub{catch}{The catch to apply}
\defType{Real number (estimable)}
\defDefault{0.0}
\defLowerBound{0 (inclusive)}

\defSub{alpha}{The multiplier on the proportional change in biomass applied to the catch}
\defType{Real number (estimable)}
\defDefault{1.0}
\defLowerBound{0 (exclusive)}

\defSub{min\_delta}{The minimum difference (proportion) in catch required before it is updated}
\defType{Real number (estimable)}
\defDefault{0.0}
\defLowerBound{0 (inclusive)}

\defSub{max\_delta}{The maximum difference (proportion) in catch that can be applied}
\defType{Real number (estimable)}
\defDefault{0.0}
\defLowerBound{0 (inclusive)}
\defNote(Use max\_delta = 0 to have no maximum)

\defSub{update\_frequency\_years}{The number of years between updates}
\defType{Non-negative integer}
\defDefault{1}
\defLowerBound{1 (inclusive)}

\defSub{biomass\_index\_lag\_years}{The lag (years) of the derived\_quantity that is used for the calculation of the catch}
\defType{Non-negative integer}
\defDefault{1}
\defLowerBound{1 (inclusive)}

\defSub{current\_catch}{The current catch to apply at the start of the projections (applied until first\_year)}
\defType{Real number (estimable)}
\defDefault{0}
\defLowerBound{0 (inclusive)}

\defSub{multiplier}{Multiplier that is applied to the projected value}
\defType{Vector of real numbers (estimable)}
\defDefault{1.0}
\defLowerBound{0 (exclusive)}

\defSub{first\_year}{The first year in which to consider an update using the harvest strategy rule}
\defType{Non-negative integer}
\defDefault{0}

\subsubsection{Project of type Harvest Strategy Constant U}
\commandlabsubarg{Project}{type}{Harvest\_Strategy\_Constant\_U}.
\defRef{sec:Project-HarvestStrategyConstantU}
\label{syntax:Project-HarvestStrategyConstantU}

\defSub{u}{The exploitation rate to apply}
\defType{Real number (estimable)}
\defDefault{0.0}
\defLowerBound{0 (inclusive)}

\defSub{min\_delta}{The minimum difference (proportion) in catch required before it is updated}
\defType{Real number (estimable)}
\defDefault{0.0}
\defLowerBound{0 (inclusive)}

\defSub{max\_delta}{The maximum difference (proportion) in catch that can be applied}
\defType{Real number (estimable)}
\defDefault{0.0}
\defLowerBound{0 (inclusive)}
\defNote(Use max\_delta = 0 to have no maximum)

\defSub{update\_frequency\_years}{The number of years between updates}
\defType{Non-negative integer}
\defDefault{1}
\defLowerBound{1 (inclusive)}

\defSub{biomass\_index\_lag\_years}{The lag (years) of the derived\_quantity that is used for the calculation of the catch}
\defType{Non-negative integer}
\defDefault{1}
\defLowerBound{1 (inclusive)}

\defSub{current\_catch}{The current catch to apply at the start of the projections (applied until first\_year)}
\defType{Real number (estimable)}
\defDefault{0}
\defLowerBound{0 (inclusive)}

\defSub{multiplier}{Multiplier that is applied to the projected value}
\defType{Vector of real numbers (estimable)}
\defDefault{1.0}
\defLowerBound{0 (exclusive)}

\defSub{first\_year}{The first year in which to consider an update using the harvest strategy rule}
\defType{Non-negative integer}
\defDefault{0}

\subsubsection{Project of type Harvest Strategy Ramp U}
\commandlabsubarg{Project}{type}{harvest\_strategy\_ramp\_u}.
\defRef{sec:Project-HarvestStrategyRampU}
\label{syntax:Project-HarvestStrategyRampU}

\defSub{u}{The exploitation rates to apply}
\defType{Vector of real numbers (estimable)}
\defDefault{0.0}
\defLowerBound{0 (inclusive)}

\defSub{reference\_points}{The reference points for each exploitation rate}
\defType{Vector of real numbers (estimable)}
\defDefault{0.0}
\defLowerBound{0 (inclusive)}

\defSub{reference\_index}{The biomass index for reference points (i.e., the derived quantity label for the calculation of reference points)}
\defType{String}
\defDefault{No default}

\defSub{min\_delta}{The minimum difference (proportion) in catch required before it is updated}
\defType{Real number (estimable)}
\defDefault{0.0}
\defLowerBound{0 (inclusive)}

\defSub{max\_delta}{The maximum difference (proportion) in catch that can be applied}
\defType{Real number (estimable)}
\defDefault{0.0}
\defLowerBound{0 (inclusive)}
\defNote(Use max\_delta = 0 to have no maximum)

\defSub{update\_frequency\_years}{The number of years between updates}
\defType{Non-negative integer}
\defDefault{1}
\defLowerBound{1 (inclusive)}

\defSub{biomass\_index\_lag\_years}{The lag (years) of the derived\_quantity that is used for the calculation of the catch}
\defType{Non-negative integer}
\defDefault{1}
\defLowerBound{1 (inclusive)}

\defSub{current\_catch}{The current catch to apply at the start of the projections (applied until first\_year)}
\defType{Real number (estimable)}
\defDefault{0}
\defLowerBound{0 (inclusive)}

\defSub{multiplier}{Multiplier that is applied to the projected value}
\defType{Vector of real numbers (estimable)}
\defDefault{1.0}
\defLowerBound{0 (exclusive)}

\defSub{first\_year}{The first year in which to consider an update using the harvest strategy rule}
\defType{Non-negative integer}
\defDefault{0}

%\subsubsection{Project of type User Defined}
%\commandlabsubarg{project}{type}{User\_Defined}.
%\defRef{sec:Project-UserDefined}
%\label{syntax:Project-UserDefined}
%
%\defSub{equation}{The equation to do a test run of}
%\defType{Vector of strings}
%\defDefault{No default}
%


 \pagebreak 
 \section{\I{Estimation command and subcommand syntax}\label{syntax:Estimation}}

The description of methods for the estimation section is given in Section \ref{sec:Estimation}.

In the following section, the sub-section headers use a notation of the form "\textbf {@observation[label].type=abundance}" which, in this case, represents the input command fragment
{\small{\begin{verbatim}
@observation label # where label is a unique label for that observation
type=abundance
...
\end{verbatim}}}
The specific subcommands for a command are given within each command.

\subsection{\I{Estimation methods}}
\defComLab{estimate}{Define an object of type \emph{Estimate}}.
\defRef{sec:Estimation}
\label{syntax:Estimate}

\defSub{label}{The label of the estimate}
\defType{String}
\defDefault{No default}

\defSub{type}{The type of prior for the estimate}
\defType{String}
\defDefault{No default}

\defSub{parameter}{The name of the parameter to estimate}
\defType{String}
\defDefault{No default}

\defSub{lower\_bound}{The lower bound for the parameter}
\defType{Real number (estimable)}
\defDefault{No default}

\defSub{upper\_bound}{The upper bound for the parameter}
\defType{Real number (estimable)}
\defDefault{No default}

\defSub{same}{List of other parameters that are constrained to have the same value as this parameter}
\defType{Vector of strings}
\defDefault{No default}

\defSub{estimation\_phase}{The first estimation phase to allow this to be estimated}
\defType{Non-negative integer}
\defDefault{1}
\defValue{Phases must be number sequentially and start at one}

\defSub{mcmc\_fixed}{Indicates if this parameter is estimated at the point estimate but fixed during MCMC estimation run}
\defType{Boolean}
\defDefault{false}

\subsubsection{Estimate of type Uniform}
\commandlabsubarg{estimate}{type}{Uniform}.
\defRef{sec:Prior-Uniform}
\label{syntax:Estimate-Uniform}

The Uniform type has no additional subcommands.
\subsubsection{Estimate of type Uniform-Log}
\commandlabsubarg{estimate}{type}{Uniform\_Log}.
\defRef{sec:Prior-UniformLog}
\label{syntax:Estimate-UniformLog}

The Uniform\_Log type has no additional subcommands.

\subsubsection{Estimate of prior type Normal}
\commandlabsubarg{estimate}{type}{Normal}.
\defRef{sec:Prior-Normal}
\label{syntax:Estimate-Normal}

\defSub{mu}{The normal prior mean (mu) parameter}
\defType{Real number (estimable)}
\defDefault{No default}

\defSub{cv}{The normal standard deviation (sigma) parameter}
\defType{Real number (estimable)}
\defDefault{No default}
\defLowerBound{0.0 (exclusive)}

\subsubsection{Estimate of prior type Normal By Stdev}
\commandlabsubarg{estimate}{type}{Normal\_By\_Stdev}.
\defRef{sec:Prior-NormalByStdev}
\label{syntax:Estimate-NormalByStdev}

\defSub{mu}{The normal prior mean (mu) parameter}
\defType{Real number (estimable)}
\defDefault{No default}

\defSub{sigma}{The normal standard deviation (sigma) parameter}
\defType{Real number (estimable)}
\defDefault{No default}
\defLowerBound{0.0 (exclusive)}

\defSub{lognormal\_transformation}{Add a Jacobian if the derived outcome of the estimate is assumed to be lognormal, e.g., used for recruitment deviations in the recruitment process. See the User Manual for more information}
\defType{Boolean}
\defDefault{false}

\subsubsection{Estimate of prior type Lognormal}
\commandlabsubarg{estimate}{type}{Lognormal}.
\defRef{sec:Prior-Lognormal}
\label{syntax:Estimate-Lognormal}

\defSub{mu}{The lognormal prior mean (mu) parameter}
\defType{Real number (estimable)}
\defDefault{No default}
\defLowerBound{0.0 (exclusive)}

\defSub{cv}{The lognormal variance (cv) parameter}
\defType{Real number (estimable)}
\defDefault{No default}
\defLowerBound{0.0 (exclusive)}

\subsubsection{Estimate of prior type Normal-Log}
\commandlabsubarg{estimate}{type}{Normal\_Log}.
\defRef{sec:Prior-NormalLog}
\label{syntax:Estimate-NormalLog}

\defSub{mu}{The normal-log prior mean (mu) parameter}
\defType{Real number (estimable)}
\defDefault{No default}

\defSub{sigma}{The normal-log prior standard deviation (sigma) parameter}
\defType{Real number (estimable)}
\defDefault{No default}
\defLowerBound{0.0 (exclusive)}

\subsubsection{Estimate of prior type Beta}
\commandlabsubarg{estimate}{type}{Beta}.
\defRef{sec:Prior-Beta}
\label{syntax:Estimate-Beta}

\defSub{mu}{Beta prior  mean (mu) parameter}
\defType{Real number (estimable)}
\defDefault{No default}

\defSub{sigma}{Beta prior standard deviation (sigma) parameter}
\defType{Real number (estimable)}
\defDefault{No default}
\defLowerBound{0.0 (exclusive)}

\defSub{a}{Beta prior lower bound of the range (A) parameter}
\defType{Real number (estimable)}
\defDefault{No default}

\defSub{b}{Beta prior upper bound of the range (B) parameter}
\defType{Real number (estimable)}
\defDefault{No default}

\subsubsection{Estimate of prior type Student's t}
\commandlabsubarg{estimate}{type}{students\_t}.
\defRef{sec:Prior-Studentst}
\label{syntax:Estimate-Studentst}

\defSub{mu}{The Student's t prior location (mu) parameter}
\defType{Real number (estimable)}
\defDefault{No default}

\defSub{sigma}{The Student's t scale (sigma) parameter}
\defType{Real number (estimable)}
\defDefault{No default}
\defLowerBound{0.0 (exclusive)}

\defSub{df}{The Student's t degrees of freedom (df) parameter}
\defType{Real number (constant)}
\defDefault{No default}
\defLowerBound{0.0 (exclusive)}


\subsection{\I{Point estimation}}
\defComLab{minimiser}{Define an object of type \emph{Minimiser}}.
\defRef{sec:Minimiser}
\label{syntax:Minimiser}

\defSub{label}{The minimiser label}
\defType{String}
\defDefault{No default}

\defSub{type}{The type of minimiser to use}
\defType{String}
\defDefault{No default}

\defSub{active}{Indicates if this minimiser is active}
\defType{Boolean}
\defDefault{false}

\defSub{covariance}{Indicates if a covariance matrix should be generated}
\defType{Boolean}
\defDefault{true}

\subsubsection{Minimiser of type ADOLC}
\commandlabsubarg{minimiser}{type}{ADOLC}.
\defRef{sec:Minimiser-ADOLC}
\label{syntax:Minimiser-ADOLC}

\defSub{iterations}{The maximum number of iterations}
\defType{Integer}
\defDefault{1000}
\defLowerBound{1 (inclusive)}

\defSub{evaluations}{The maximum number of evaluations}
\defType{Integer}
\defDefault{4000}
\defLowerBound{1 (inclusive)}

\defSub{tolerance}{The tolerance of the gradient for convergence}
\defType{Real number}
\defDefault{1e-5}
\defLowerBound{0.0 (exclusive)}

\defSub{step\_size}{The minimum step-size before minimisation fails}
\defType{Real number}
\defDefault{1e-7}
\defLowerBound{0.0 (exclusive)}

\defSub{parameter\_transformation}{The choice of parametrisation used to scale the parameters for ADOLC}
\defType{string}
\defDefault{sin\_transformation}
\defValue{Either \subcommand{sin\_transform} or \subcommand{tan\_transform}. See \ref{sec:Minimiser-ADOLC} for more information}

\subsubsection{Minimiser of type Betadiff}
\commandlabsubarg{minimiser}{type}{Betadiff}.
\defRef{sec:Minimiser-BetaDiff}
\label{syntax:Minimiser-BetaDiff}

\defSub{iterations}{The maximum number of iterations}
\defType{Integer}
\defDefault{1000}
\defLowerBound{1 (inclusive)}

\defSub{evaluations}{The maximum number of evaluations}
\defType{Integer}
\defDefault{4000}
\defLowerBound{1 (inclusive)}

\defSub{tolerance}{The tolerance of the gradient for convergence}
\defType{Real number}
\defDefault{1e-5}
\defLowerBound{0.0 (exclusive)}

\subsubsection{Minimiser of type DESolver}
\commandlabsubarg{minimiser}{type}{de\_solver}.
\defRef{sec:Minimiser-DESolver}
\label{syntax:Minimiser-DESolver}

\defSub{population\_size}{The number of candidate solutions to have in the population}
\defType{Non-negative integer}
\defDefault{25}
\defLowerBound{1 (inclusive)}

\defSub{crossover\_probability}{The minimiser's crossover probability}
\defType{Real number}
\defDefault{0.9}
\defLowerBound{0.0 (inclusive)}
\defUpperBound{1.0 (inclusive)}

\defSub{difference\_scale}{The scale to apply to new solutions when comparing candidates}
\defType{Real number}
\defDefault{0.02}

\defSub{max\_generations}{The maximum number of iterations to run}
\defType{1000}
\defDefault{No default}

\defSub{tolerance}{The total variance between the population and best candidate before acceptance}
\defType{Real number}
\defDefault{1e-5}
\defLowerBound{0.0 (exclusive)}

\subsubsection{Minimiser of type Deltadiff}
\commandlabsubarg{minimiser}{type}{Deltadiff}.
\defRef{sec:Minimiser-DeltaDiff}
\label{syntax:Minimiser-DeltaDiff}

\defSub{iterations}{Maximum number of iterations}
\defType{Integer}
\defDefault{1000}
\defLowerBound{1 (inclusive)}

\defSub{evaluations}{Maximum number of evaluations}
\defType{Integer}
\defDefault{4000}
\defLowerBound{1 (inclusive)}

\defSub{tolerance}{Tolerance of the gradient for convergence}
\defType{Real number}
\defDefault{1e-5}
\defLowerBound{0 (exclusive)}

\defSub{step\_size}{Minimum Step-size before minimisation fails}
\defType{Real number}
\defDefault{1e-7}
\defLowerBound{0 (exclusive)}

\subsubsection{Minimiser of type Numerical Differences}
\commandlabsubarg{minimiser}{type}{Numerical\_Differences}.
\defRef{sec:Minimiser-GammaDiff}
\label{syntax:Minimiser-GammaDiff}

\defSub{iterations}{The maximum number of iterations}
\defType{Integer}
\defDefault{1000}
\defLowerBound{1 (inclusive)}

\defSub{evaluations}{The maximum number of evaluations}
\defType{Integer}
\defDefault{4000}
\defLowerBound{1 (inclusive)}

\defSub{tolerance}{The tolerance of the gradient for convergence}
\defType{Real number}
\defDefault{1e-5}
\defLowerBound{0.0 (exclusive)}

\defSub{step\_size}{The minimum step-size before minimisation fails}
\defType{Real number}
\defDefault{1e-7}
\defLowerBound{0.0 (exclusive)}



\subsection{\I{Markov chain Monte Carlo (MCMC)}}
\defComLab{mcmc}{Define an object of type \emph{MCMC}}.
\defRef{sec:MCMC}
\label{syntax:MCMC}

\defSub{label}{The label of the MCMC}
\defType{String}
\defDefault{No default}

\defSub{type}{The MCMC method}
\defType{String}
\defDefault{No default}

\defSub{length}{The number of iterations for the MCMC (including the burn in period)}
\defType{Non-negative integer}
\defDefault{No default}
\defLowerBound{1 (inclusive)}

\defSub{burn\_in}{The number of iterations for the burn\_in period of the MCMC}
\defType{Non-negative integer}
\defDefault{0}
\defLowerBound{0 (inclusive)}

\defSub{active}{Indicates if this is the active MCMC algorithm}
\defType{Boolean}
\defDefault{true}

\defSub{step\_size}{Initial step-size (as a multiplier of the approximate covariance matrix)}
\defType{Real number}
\defDefault{The default is $2.4d^{-0.5}$}
\defLowerBound{0 (inclusive)}
\defNote{If the value is set to zero or the subcommand is omitted, the default value is used instead}

\defSub{start}{The covariance multiplier for the starting point of the MCMC}
\defType{Real number}
\defDefault{0.0}
\defLowerBound{0.0 (inclusive)}
\defValue{If zero, then the MCMC starts at the point estimate (i.e., the MPD). Otherwise a random (multivariate normal) jump from the point estimate with \subcommand{start} used as the standard deviation multiplier}

\defSub{adjust\_parameters\_at\_bounds}{Adjust the start point for parameters at bounds}
\defType{Boolean}
\defDefault{false}
\defValue{If true, then the MCMC will adjust the start point of any parameters at a bound to a random uniform location between the lower and upper bound}

\defSub{keep}{The spacing between recorded values in the MCMC}
\defType{Non-negative integer}
\defDefault{1}
\defLowerBound{1 (inclusive)}

\defSub{max\_correlation}{The maximum absolute correlation in the covariance matrix of the proposal distribution}
\defType{Real number}
\defDefault{0.8}
\defLowerBound{0.0 (exclusive)}
\defUpperBound{1.0 (inclusive)}

\defSub{covariance\_adjustment\_method}{The method for adjusting small variances in the covariance proposal matrix}
\defType{String}
\defDefault{correlation}
\defValue{Either covariance, correlation, or none}

\defSub{correlation\_adjustment\_diff}{The minimum non-zero variance times the range of the bounds in the covariance matrix of the proposal distribution}
\defType{Real number}
\defDefault{0.0001}
\defLowerBound{0.0 (exclusive)}

\defSub{proposal\_distribution}{The shape of the proposal distribution (either the t or the normal distribution)}
\defType{String}
\defDefault{t}
\defValue{Either t or normal}

\defSub{df}{The degrees of freedom of the multivariate t proposal distribution}
\defType{Non-negative integer}
\defDefault{4}
\defLowerBound{1}

\defSub{adapt\_stepsize\_at}{The iteration numbers in which to check and resize the MCMC step-size}
\defType{Vector of non-negative integers}
\defDefault{true}
\defLowerBound{0 (inclusive)}
\defValue{If zero, then no step-size adaption is applied. Otherwise must be a positive integer less than less than the burn\_in}

\defSub{adapt\_stepsize\_method}{The method to use to adapt the step-size}
\defType{String}
\defDefault{ratio}
\defValue{Either \subcommand{double\_half} or \subcommand{ratio}}

\defSub{adapt\_covariance\_matrix\_at}{The iteration number in which to adapt the covariance matrix}
\defType{Non-negative integer}
\defDefault{0}
\defLowerBound{0 (inclusive)}
\defValue{If zero, then no covariance matrix adaption is applied. Otherwise must be a positive integer that is less than the burn\_in}

\subsubsection{MCMC of type Hamiltonian Monte Carlo}
\commandlabsubarg{mcmc}{type}{Hamiltonian}.
\label{syntax:MCMC-HamiltonianMonteCarlo}

\defSub{leapfrog\_steps}{Number of leapfrog steps}
\defType{Non-negative integer}
\defDefault{1}
\defLowerBound{0 (exclusive)}

\defSub{leapfrog\_delta}{Amount to leapfrog per step}
\defType{Real number}
\defDefault{1e-7}
\defLowerBound{0 (exclusive)}

\defSub{gradient\_step\_size}{Step-size to use when calculating gradient}
\defType{Real number}
\defDefault{1e-7}
\defLowerBound{1e-13 (inclusive)}

\subsubsection{MCMC of type Random Walk Metropolis Hastings}
\commandlabsubarg{mcmc}{type}{Random\_Walk}.
\defRef{sec:MCMC-RandomWalkMetropolisHastings}
\label{syntax:MCMC-RandomWalkMetropolisHastings}

The Random\_Walk type has no additional subcommands.


\subsection{\I{Posterior profiles}}
\defComLab{profile}{Define an object of type \emph{Profile}}.
\defRef{sec:Profile}
\label{syntax:Profile}

\defSub{label}{The label of the profile}
\defType{String}
\defDefault{No default}

\defSub{steps}{The number of steps between the lower and upper bound}
\defType{Non-negative integer}
\defDefault{No default}
\defValue{A positive integer $\ge 2$}

\defSub{lower\_bound}{The lower value of the range}
\defType{Real number (estimable)}
\defDefault{No default}

\defSub{upper\_bound}{The upper value of the range}
\defType{Real number (estimable)}
\defDefault{No default}

\defSub{parameter}{The free parameter to profile}
\defType{String}
\defDefault{No default}

\defSub{same}{A free parameter that is constrained to have the same value as the parameter being profiled}
\defType{String}
\defDefault{No default}

\defSub{transformation}{The transformation to apply to the upper and lower bounds}
\defType{String}
\defDefault{'none'}
\defValue{'none', 'log', 'square\_root', and 'inverse'}
\defNote{This specifies that the upper and lower bounds should be transformed from natural space into the transformed space before being evaluated and used for the profile}


\subsection{\I{Catchability constants}}
\input{IncludedSyntax/Catchability}

\subsection{\I{Penalties}}
\input{IncludedSyntax/Penalty}

\subsection{\I{Additional priors}}
\defComLab{Additional\_Prior}{Define an object of type \emph{Additional\_Prior}}.
\defRef{sec:AdditionalPrior}
\label{syntax:AdditionalPrior}

\defSub{label}{The label for the additional prior}
\defType{String}
\defDefault{No default}

\defSub{type}{The additional prior type}
\defType{String}
\defDefault{No default}

\subsubsection{Additional\_Prior of type Beta}
\commandlabsubarg{Additional\_Prior}{type}{Beta}.
\defRef{sec:AdditionalPrior-Beta}
\label{syntax:AdditionalPrior-Beta}

\defSub{parameter}{The name of the parameter for the additional prior}
\defType{String}
\defDefault{No default}

\defSub{mu}{Beta distribution mean (mu) parameter}
\defType{Real number (estimable)}
\defDefault{No default}

\defSub{sigma}{Beta distribution variance (sigma) parameter}
\defType{Real number (estimable)}
\defDefault{No default}
\defLowerBound{0.0 (exclusive)}

\defSub{a}{Beta distribution lower bound, of the range (A) parameter}
\defType{Real number (estimable)}
\defDefault{No default}

\defSub{b}{Beta distribution upper bound of the range (B) parameter}
\defType{Real number (estimable)}
\defDefault{No default}

\subsubsection{Additional\_Prior of type Element\_Difference}
\commandlabsubarg{Additional\_Prior}{type}{Element\_Difference}.
\defRef{sec:AdditionalPrior-ElementDifference}
\label{syntax:AdditionalPrior-ElementDifference}

\defSub{parameter}{The name of the parameter for the additional prior}
\defType{String}
\defDefault{No default}

\defSub{second\_parameter}{The name of the second parameter for comparing}
\defType{String}
\defDefault{No default}

\defSub{multiplier}{Multiply the penalty by this factor}
\defType{Real number (estimable)}
\defDefault{1}

\subsubsection{Additional\_Prior of type Log\_Normal}
\commandlabsubarg{Additional\_Prior}{type}{Log\_Normal}.
\defRef{sec:AdditionalPrior-LogNormal}
\label{syntax:AdditionalPrior-LogNormal}

\defSub{parameter}{The name of the parameter for the additional prior}
\defType{String}
\defDefault{No default}

\defSub{mu}{The lognormal prior mean (mu) parameter}
\defType{Real number (estimable)}
\defDefault{No default}
\defLowerBound{0.0 (exclusive)}

\defSub{cv}{The lognormal CV parameter}
\defType{Real number (estimable)}
\defDefault{No default}
\defLowerBound{0.0 (exclusive)}

\subsubsection{Additional\_Prior of type Ratio}
\commandlabsubarg{Additional\_Prior}{type}{Ratio}.
\defRef{sec:AdditionalPrior-Ratio}
\label{syntax:AdditionalPrior-Ratio}

\defSub{parameter}{The name of the parameter for the additional prior}
\defType{String}
\defDefault{No default}

\defSub{second\_parameter}{The name of the parameter on the denominator}
\defType{String}
\defDefault{No default}

\defSub{mu}{The lognormal prior mean (mu) of the ratio}
\defType{Real number (estimable)}
\defDefault{No default}
\defLowerBound{0.0 (exclusive)}

\defSub{cv}{The lognormal CV parameter for the ratio}
\defType{Real number (estimable)}
\defDefault{No default}
\defLowerBound{0.0 (exclusive)}

\subsubsection{Additional\_Prior of type Sum}
\commandlabsubarg{Additional\_Prior}{type}{Sum}.
\defRef{sec:AdditionalPrior-Sum}
\label{syntax:AdditionalPrior-Sum}

\defSub{parameters}{The names of the parameters for summing}
\defType{Vector of strings}
\defDefault{No default}

\defSub{distribution}{The additional prior distribution to apply}
\defType{String}
\defDefault{lognormal}

\defSub{mu}{Mean of the distribution}
\defType{Real number (estimable)}
\defDefault{1.0}
\defLowerBound{0.0 (exclusive)}

\defSub{cv}{cv of the distribution}
\defType{Real number (estimable)}
\defDefault{No default}
\defLowerBound{0.0 (exclusive)}

\subsubsection{Additional\_Prior of type Uniform}
\commandlabsubarg{Additional\_Prior}{type}{Uniform}.
\defRef{sec:AdditionalPrior-Uniform}
\label{syntax:AdditionalPrior-Uniform}

\defSub{parameter}{The name of the parameter for the additional prior}
\defType{String}
\defDefault{No default}

\subsubsection{Additional\_Prior of type Uniform\_Log}
\commandlabsubarg{Additional\_Prior}{type}{Uniform\_Log}.
\defRef{sec:AdditionalPrior-UniformLog}
\label{syntax:AdditionalPrior-UniformLog}

\defSub{parameter}{The name of the parameter for the additional prior}
\defType{String}
\defDefault{No default}

\subsubsection{Additional\_Prior of type Vector\_Average}
\commandlabsubarg{Additional\_Prior}{type}{Vector\_Average}.
\defRef{sec:AdditionalPrior-VectorAverage}
\label{syntax:AdditionalPrior-VectorAverage}

\defSub{parameter}{The name of the parameter for the additional prior}
\defType{String}
\defDefault{No default}

\defSub{method}{Which calculation method to use: k, l, or m}
\defType{String}
\defDefault{k}

\defSub{k}{The k value to use in the calculation}
\defType{Real number (estimable)}
\defDefault{No default}

\defSub{multiplier}{Multiplier for the penalty amount}
\defType{Real number (estimable)}
\defDefault{1}

\subsubsection{Additional\_Prior of type Vector\_Smoothing}
\commandlabsubarg{Additional\_Prior}{type}{Vector\_Smoothing}.
\defRef{sec:AdditionalPrior-VectorSmoothing}
\label{syntax:AdditionalPrior-VectorSmoothing}

\defSub{parameter}{The name of the parameter for the additional prior}
\defType{String}
\defDefault{No default}

\defSub{log\_scale}{Should the sums of squares be calculated on the log scale?}
\defType{Boolean}
\defDefault{false}

\defSub{multiplier}{Multiply the penalty by this factor}
\defType{Real number (estimable)}
\defDefault{1}

\defSub{lower\_bound}{The first element to apply the penalty to in the vector}
\defType{Non-negative integer}
\defDefault{0}

\defSub{upper\_bound}{The last element to apply the penalty to in the vector}
\defType{Non-negative integer}
\defDefault{0}

\defSub{r}{The rth difference that the penalty is applied to}
\defType{Non-negative integer}
\defDefault{2}



\subsection{\I{Parameter transformations}}
\input{IncludedSyntax/ParameterTransformation.tex}

 \pagebreak 
 \section{\I{Observation command and subcommand syntax}\label{syntax:Observations}}

The description of methods for the observation section is given in Section \ref{sec:Observation}.

In the following section, the sub-section headers use a notation of the form "\textbf {@observation[label].type=abundance}" which, in this case, represents the input command fragment
{\small{\begin{verbatim}
@observation label # where label is a unique label for that observation
type=abundance
...
\end{verbatim}}}
The specific subcommands for a command are given within each command.

\subsection{\I{Observation types}}\label{syntax:ObservationTypes}

The description of the observations is given in Section \ref{sec:Observation}. The observation types available are:

\begin{description}
  \item Observations of proportions of individuals by age class
  \item Observations of proportions of individuals by category and age class
  \item Relative and absolute abundance observations
  \item Relative and absolute biomass observations
\end{description}

Each type of observation requires a set of subcommands and arguments specific to that process.

\ifAgeBased
\defComLab{observation}{Define an object of type \emph{Observation}}.
\defRef{sec:Observation}
\label{syntax:Observation}

\defSub{label}{The label of the observation}
\defType{String}
\defDefault{No default}

\defSub{type}{The type of observation}
\defType{String}
\defDefault{No default}

\defSub{likelihood}{The type of likelihood to use}
\defType{String}
\defDefault{No default}

\defSub{categories}{The category labels to use}
\defType{Vector of strings}
\defDefault{true}

\defSub{delta}{The robustification value (delta) for the likelihood}
\defType{Real number (estimable)}
\defDefault{1e-11}
\defLowerBound{0.0 (inclusive)}

\defSub{simulation\_likelihood}{The simulation likelihood to use}
\defType{String}
\defDefault{No default}

\defSub{likelihood\_multiplier}{The likelihood multiplier}
\defType{Real number (estimable)}
\defDefault{1.0}
\defLowerBound{0.0 (inclusive)}

\defSub{error\_value\_multiplier}{The error value multiplier for likelihood}
\defType{Real number (estimable)}
\defDefault{1.0}
\defLowerBound{0.0 (inclusive)}

\defSub{table}{The table of data specifying the observed values}
\defType{Data table with label = obs}
\defDefault{No default}
\defValue{A $n*m$ matrix, where $n=$ the years and $m=$ either the number of ages, lengths, or abundance/biomass observation for each year defined in the model. Each row starts with the year. The table ends with `end\_table'}
\defNote{See \ref{sec:DataTable} for more details each observation may have custom table labels.}

\defSub{table}{The table of data specifying the observed error values}
\defType{Data table with label = error\_values}
\defDefault{No default}
\defValue{A $n*m$ matrix, where $n=$ the years and $m=$ either the number of ages, lengths, or abundance/biomass observation for each year defined in the model. Each row starts with the year. The table ends with `end\_table'}
\defNote{See \ref{sec:DataTable} for more details on specifying data tables.  each observation may have custom table labels.}

\subsubsection{Observation of type Abundance}
\commandlabsubarg{observation}{type}{Abundance}.
\defRef{sec:Observation-Abundance}
\label{syntax:Observation-Abundance}

\defSub{time\_step}{The label of the time step that the observation occurs in}
\defType{String}
\defDefault{No default}

\defSub{catchability}{The label of the catchability coefficient (q)}
\defType{String}
\defDefault{No default}

\defSub{selectivities}{The labels of the selectivities}
\defType{Vector of strings}
\defDefault{true}

\defSub{process\_error}{The process error}
\defType{Real number (estimable)}
\defDefault{0.0}
\defLowerBound{0.0 (inclusive)}

\defSub{years}{The years for which there are observations}
\defType{Vector of non-negative integers}
\defDefault{No default}

\defSub{table obs}{The table of data specifying the observed and error values}
\defType{Data table with label = obs}
\defDefault{No default}
\defValue{A $n*3$ matrix, where $n=$ the years and a column for year, observation and error value. See below for example.}
\defNote{example below}
\begin{verbatim}
table obs 
# year observation error_value
1993 238.2 0.12
1994 170 0.16
1995 216.2 0.18
2004 46.9 0.20
end_table
\end{verbatim}

\subsubsection{Observation of type Biomass}
\commandlabsubarg{observation}{type}{Biomass}.
\defRef{sec:Observation-Biomass}
\label{syntax:Observation-Biomass}

\defSub{time\_step}{The label of the time step that the observation occurs in}
\defType{String}
\defDefault{No default}

\defSub{catchability}{The label of the catchability coefficient (q)}
\defType{String}
\defDefault{No default}

\defSub{selectivities}{The labels of the selectivities}
\defType{Vector of strings}
\defDefault{true}

\defSub{process\_error}{The process error}
\defType{Real number (estimable)}
\defDefault{0.0}
\defLowerBound{0.0 (inclusive)}

\defSub{age\_weight\_labels}{The labels for the \command{$age\_weight$} block which corresponds to each category, to use the weight calculation method for biomass calculations)}
\defType{Vector of strings}
\defDefault{No default}

\defSub{years}{The years of the observed values}
\defType{Vector of non-negative integers}
\defDefault{No default}

\defSub{table obs}{The table of data specifying the observed and error values}
\defType{Data table with label = obs}
\defDefault{No default}
\defValue{A $n*3$ matrix, where $n=$ the years and a column for year, observation and error value. See below for example.}
\defNote{example below}
\begin{verbatim}
table obs 
# year observation error_value
1993 238.2 0.12
1994 170 0.16
1995 216.2 0.18
2004 46.9 0.20
end_table
\end{verbatim}

\subsubsection{Observation of type Process Removals By Age}
\commandlabsubarg{observation}{type}{Process\_Removals\_By\_Age}.
\defRef{sec:Observation-ProcessRemovalsByAge}
\label{syntax:Observation-ProcessRemovalsByAge}

\defSub{min\_age}{The minimum age}
\defType{Non-negative integer}
\defDefault{No default}

\defSub{max\_age}{The maximum age}
\defType{Non-negative integer}
\defDefault{No default}

\defSub{sum\_to\_one}{Scale year (row) observed values by the total so they sum to equal 1}
\defType{Boolean}
\defDefault{false}

\defSub{simulated\_data\_sum\_to\_one}{Whether simulated data is discrete or scaled by totals to be proportions for each year}
\defType{Boolean}
\defDefault{true}


\defSub{plus\_group}{Is the maximum age the age plus group}
\defType{Boolean}
\defDefault{true}

\defSub{time\_step}{The label of time-step that the observation occurs in}
\defType{Vector of strings}
\defDefault{No default}

\defSub{years}{The years for which there are observations}
\defType{Vector of non-negative integers}
\defDefault{No default}

\defSub{process\_errors}{The process errors to use}
\defType{Vector of real numbers (estimable) of length equal to the number of years}
\defDefault{0.0}
\defNote{If only one value is supplied, it will be repeated for all years in the observation}

\defSub{ageing\_error}{The label of the ageing error to use}
\defType{String}
\defDefault{No default}

\defSub{method\_of\_removal}{The label of the observed method of removals}
\defType{Vector of strings}
\defDefault{No default}


\defSub{mortality\_process}{The label of the mortality instantaneous process for the observation}
\defType{String}
\defDefault{No default}
\defNote{Allowed mortality process types are \subcommand{mortality\_instantaneous} and \subcommand{mortality\_hybrid}}

\defSub{table obs}{The table of data specifying the observed values}
\defType{Data table with label = obs}
\defDefault{No default}
\defValue{A $n\times m$ matrix, where $n=$ the years and $m$ is categories \(\times\) length bins. See below for example.}
\defNote{example below}
\begin{verbatim}
table obs 
1993 0.1 0.2 0.3
1994 0.1 0.2 0.3
end_table
\end{verbatim}
\defSub{table error\_values}{The table of data specifying the error values}
\defType{Data table with label = error\_values}
\defDefault{No default}
\defValue{Can be specified two ways either as a $n\times 1$ matrix with an error value for each year. Or a $n\times m$ matrix, where $n=$ the years and $m$ is categories \(\times\) length bins. See below for example.}
\defNote{example below}
\begin{verbatim}
table error_values 
1993 234
1994 343
end_table
\end{verbatim}
\subsubsection{Observation of type Process Removals By Age Retained}
\commandlabsubarg{observation}{type}{Process\_Removals\_By\_Age\_Retained}.
\defRef{sec:Observation-ProcessRemovalsByAgeRetained}
\label{syntax:Observation-ProcessRemovalsByAgeRetained}

\defSub{min\_age}{The minimum age}
\defType{Non-negative integer}
\defDefault{No default}

\defSub{max\_age}{The maximum age}
\defType{Non-negative integer}
\defDefault{No default}

\defSub{plus\_group}{Is the maximum age the age plus group?}
\defType{Boolean}
\defDefault{true}

\defSub{time\_step}{The label of the time step that the observation occurs in}
\defType{Vector of strings}
\defDefault{No default}


\defSub{sum\_to\_one}{Scale the year (row) observed values by the total, so they sum to 1}
\defType{Boolean}
\defDefault{false}

\defSub{simulated\_data\_sum\_to\_one}{Whether simulated data is discrete or scaled by totals to be proportions for each year}
\defType{Boolean}
\defDefault{true}

\defSub{years}{The years for which there are observations}
\defType{Vector of non-negative integers}
\defDefault{No default}

\defSub{process\_errors}{The process errors to use}
\defType{Vector of real numbers (estimable) of length equal to the number of years}
\defDefault{0.0}
\defNote{If only one value is supplied, it will be repeated for all years in the observation}

\defSub{ageing\_error}{The label of the ageing error to use}
\defType{String}
\defDefault{No default}

\defSub{method\_of\_removal}{The label of observed method of removals}
\defType{Vector of strings}
\defDefault{No default}


\defSub{mortality\_process}{The label of the mortality instantaneous process for the observation}
\defType{String}
\defDefault{No default}
\defNote{Allowed mortality process types are \subcommand{mortality\_instantaneous\_retained}}

\defSub{table obs}{The table of data specifying the observed values}
\defType{Data table with label = obs}
\defDefault{No default}
\defValue{A $n\times m$ matrix, where $n=$ the years and $m$ is categories \(\times\) length bins. See below for example.}
\defNote{example below}
\begin{verbatim}
table obs 
1993 0.1 0.2 0.3
1994 0.1 0.2 0.3
end_table
\end{verbatim}
\defSub{table error\_values}{The table of data specifying the error values}
\defType{Data table with label = error\_values}
\defDefault{No default}
\defValue{Can be specified two ways either as a $n\times 1$ matrix with an error value for each year. Or a $n\times m$ matrix, where $n=$ the years and $m$ is categories \(\times\) length bins. See below for example.}
\defNote{example below}
\begin{verbatim}
table error_values 
1993 234
1994 343
end_table
\end{verbatim}
\subsubsection{Observation of type Process Removals By Age Retained Total}
\commandlabsubarg{observation}{type}{Process\_Removals\_By\_Age\_Retained\_Total}.
\defRef{sec:Observation-ProcessRemovalsByAgeRetainedTotal}
\label{syntax:Observation-ProcessRemovalsByAgeRetainedTotal}

\defSub{min\_age}{The minimum age}
\defType{Non-negative integer}
\defDefault{No default}

\defSub{max\_age}{The maximum age}
\defType{Non-negative integer}
\defDefault{No default}

\defSub{plus\_group}{Is the maximum age the age plus group?}
\defType{Boolean}
\defDefault{true}

\defSub{time\_step}{The label of the time step that the observation occurs in}
\defType{Vector of strings}
\defDefault{No default}

\defSub{sum\_to\_one}{Scale the year (row) observed values by the total, so they sum to 1}
\defType{Boolean}
\defDefault{false}

\defSub{simulated\_data\_sum\_to\_one}{Whether simulated data is discrete or scaled by totals to be proportions for each year}
\defType{Boolean}
\defDefault{true}

\defSub{years}{The years for which there are observations}
\defType{Vector of non-negative integers}
\defDefault{No default}

\defSub{process\_errors}{The process errors to use}
\defType{Vector of real numbers (estimable) of length equal to the number of years}
\defDefault{0.0}
\defNote{If only one value is supplied, it will be repeated for all years in the observation}

\defSub{ageing\_error}{The label of the ageing error to use}
\defType{String}
\defDefault{No default}

\defSub{method\_of\_removal}{The label of observed method of removals}
\defType{Vector of strings}
\defDefault{No default}

\defSub{mortality\_process}{The label of the mortality process for this observation}
\defType{String}
\defDefault{No default}
\defNote{Allowed mortality process types are \subcommand{mortality\_instantaneous\_retained}}

\defSub{table obs}{The table of data specifying the observed values}
\defType{Data table with label = obs}
\defDefault{No default}
\defValue{A $n\times m$ matrix, where $n=$ the years and $m$ is categories \(\times\) length bins. See below for example.}
\defNote{example below}
\begin{verbatim}
table obs 
1993 0.1 0.2 0.3
1994 0.1 0.2 0.3
end_table
\end{verbatim}
\defSub{table error\_values}{The table of data specifying the error values}
\defType{Data table with label = error\_values}
\defDefault{No default}
\defValue{Can be specified two ways either as a $n\times 1$ matrix with an error value for each year. Or a $n\times m$ matrix, where $n=$ the years and $m$ is categories \(\times\) length bins. See below for example.}
\defNote{example below}
\begin{verbatim}
table error_values 
1993 234
1994 343
end_table
\end{verbatim}
\subsubsection{Observation of type Process Removals By Length}
\commandlabsubarg{observation}{type}{Process\_Removals\_By\_Length}.
\defRef{sec:Observation-ProcessRemovalsByLength}
\label{syntax:Observation-ProcessRemovalsByLength}

\defSub{time\_step}{The time step to execute in}
\defType{String}
\defDefault{No default}

\defSub{years}{The years for which there are observations}
\defType{Vector of non-negative integers}
\defDefault{No default}

\defSub{process\_errors}{The process errors to use}
\defType{Vector of real numbers (estimable) of length equal to the number of years}
\defDefault{0.0}
\defNote{If only one value is supplied, it will be repeated for all years in the observation}

\defSub{method\_of\_removal}{The label of observed method of removals}
\defType{String}
\defDefault{No default}

\defSub{length\_bins}{The length bins}
\defType{Vector of real numbers (estimable)}
\defDefault{No default}

\defSub{sum\_to\_one}{Scale the year (row) observed values by the total, so they sum to 1}
\defType{Boolean}
\defDefault{false}

\defSub{simulated\_data\_sum\_to\_one}{Whether simulated data is discrete or scaled by totals to be proportions for each year}
\defType{Boolean}
\defDefault{true}

\defSub{plus\_group}{Is the last length bin a plus group? (defaults to @model value)}
\defType{Boolean}
\defDefault{model}

\defSub{mortality\_process}{The label of the mortality instantaneous process for the observation}
\defType{String}
\defDefault{No default}
\defNote{Allowed mortality process types are \subcommand{mortality\_instantaneous} and \subcommand{mortality\_hybrid}}

\defSub{table obs}{The table of data specifying the observed values}
\defType{Data table with label = obs}
\defDefault{No default}
\defValue{A $n\times m$ matrix, where $n=$ the years and $m$ is categories \(\times\) length bins. See below for example.}
\defNote{example below}
\begin{verbatim}
table obs 
1993 0.1 0.2 0.3
1994 0.1 0.2 0.3
end_table
\end{verbatim}
\defSub{table error\_values}{The table of data specifying the error values}
\defType{Data table with label = error\_values}
\defDefault{No default}
\defValue{Can be specified two ways either as a $n\times 1$ matrix with an error value for each year. Or a $n\times m$ matrix, where $n=$ the years and $m$ is categories \(\times\) length bins. See below for example.}
\defNote{example below}
\begin{verbatim}
table error_values 
1993 234
1994 343
end_table
\end{verbatim}
\subsubsection{Observation of type Process Removals By Length Retained}
\commandlabsubarg{observation}{type}{Process\_Removals\_By\_Length\_Retained}.
\defRef{sec:Observation-ProcessRemovalsByLengthRetained}
\label{syntax:Observation-ProcessRemovalsByLengthRetained}

\defSub{time\_step}{The time step to execute in}
\defType{String}
\defDefault{No default}

\defSub{years}{The years for which there are observations}
\defType{Vector of non-negative integers}
\defDefault{No default}

\defSub{process\_errors}{The process errors to use}
\defType{Vector of real numbers (estimable) of length equal to the number of years}
\defDefault{0.0}
\defNote{If only one value is supplied, it will be repeated for all years in the observation}

\defSub{method\_of\_removal}{The label of observed method of removals}
\defType{String}
\defDefault{No default}

\defSub{length\_bins}{The length bins}
\defType{Vector of real numbers (estimable)}
\defDefault{No default}

\defSub{sum\_to\_one}{Scale the year (row) observed values by the total, so they sum to 1}
\defType{Boolean}
\defDefault{false}

\defSub{simulated\_data\_sum\_to\_one}{Whether simulated data is discrete or scaled by totals to be proportions for each year}
\defType{Boolean}
\defDefault{true}

\defSub{plus\_group}{Is the last length bin a plus group? (defaults to @model value)}
\defType{Boolean}
\defDefault{model}

\defSub{mortality\_process}{The label of the mortality instantaneous process for the observation}
\defType{String}
\defDefault{No default}
\defNote{Allowed mortality process types are \subcommand{mortality\_instantaneous\_retained}}

\defSub{table obs}{The table of data specifying the observed values}
\defType{Data table with label = obs}
\defDefault{No default}
\defValue{A $n\times m$ matrix, where $n=$ the years and $m$ is categories \(\times\) length bins. See below for example.}
\defNote{example below}
\begin{verbatim}
table obs 
1993 0.1 0.2 0.3
1994 0.1 0.2 0.3
end_table
\end{verbatim}
\defSub{table error\_values}{The table of data specifying the error values}
\defType{Data table with label = error\_values}
\defDefault{No default}
\defValue{Can be specified two ways either as a $n\times 1$ matrix with an error value for each year. Or a $n\times m$ matrix, where $n=$ the years and $m$ is categories \(\times\) length bins. See below for example.}
\defNote{example below}
\begin{verbatim}
table error_values 
1993 234
1994 343
end_table
\end{verbatim}

\subsubsection{Observation of type Process Removals By Length Retained Total}
\commandlabsubarg{observation}{type}{Process\_Removals\_By\_Length\_Retained\_Total}.
\defRef{sec:Observation-ProcessRemovalsByLengthRetainedTotal}
\label{syntax:Observation-ProcessRemovalsByLengthRetainedTotal}

\defSub{time\_step}{The time step to execute in}
\defType{String}
\defDefault{No default}

\defSub{years}{The years for which there are observations}
\defType{Vector of non-negative integers}
\defDefault{No default}

\defSub{process\_errors}{The process errors to use}
\defType{Vector of real numbers (estimable) of length equal to the number of years}
\defDefault{0.0}
\defNote{If only one value is supplied, it will be repeated for all years in the observation}

\defSub{method\_of\_removal}{The label of observed method of removals}
\defType{String}
\defDefault{No default}

\defSub{length\_bins}{The length bins}
\defType{Vector of real numbers (estimable)}
\defDefault{No default}

\defSub{plus\_group}{Is the last length bin a plus group? (defaults to @model value)}
\defType{Boolean}
\defDefault{model}

\defSub{sum\_to\_one}{Scale the year (row) observed values by the total, so they sum to 1}
\defType{Boolean}
\defDefault{false}

\defSub{simulated\_data\_sum\_to\_one}{Whether simulated data is discrete or scaled by totals to be proportions for each year}
\defType{Boolean}
\defDefault{true}

\defSub{mortality\_process}{The label of the mortality instantaneous process for the observation}
\defType{String}
\defDefault{No default}
\defNote{Allowed mortality process types are \subcommand{mortality\_instantaneous\_retained}}

\defSub{table obs}{The table of data specifying the observed values}
\defType{Data table with label = obs}
\defDefault{No default}
\defValue{A $n\times m$ matrix, where $n=$ the years and $m$ is categories \(\times\) length bins. See below for example.}
\defNote{example below}
\begin{verbatim}
table obs 
1993 0.1 0.2 0.3
1994 0.1 0.2 0.3
end_table
\end{verbatim}
\defSub{table error\_values}{The table of data specifying the error values}
\defType{Data table with label = error\_values}
\defDefault{No default}
\defValue{Can be specified two ways either as a $n\times 1$ matrix with an error value for each year. Or a $n\times m$ matrix, where $n=$ the years and $m$ is categories \(\times\) length bins. See below for example.}
\defNote{example below}
\begin{verbatim}
table error_values 
1993 234
1994 343
end_table
\end{verbatim}

\subsubsection{Observation of type Process Removals By Weight}
\commandlabsubarg{observation}{type}{Process\_Removals\_By\_Weight}.
%\defRef{sec:Observation-ProcessRemovalsByWeight}
\label{syntax:Observation-ProcessRemovalsByWeight}

\defSub{mortality\_process}{The label of the mortality instantaneous process for the observation}
\defType{String}
\defDefault{No default}
\defNote{Allowed mortality process types are \subcommand{mortality\_instantaneous}}

\defSub{method\_of\_removal}{The label of observed method of removals}
\defType{String}
\defDefault{No default}

\defSub{time\_step}{The time step to execute in}
\defType{String}
\defDefault{No default}

\defSub{years}{The years for which there are observations}
\defType{Vector of non-negative integers}
\defDefault{No default}

\defSub{process\_errors}{The process errors to use}
\defType{Vector of real numbers (estimable) of length equal to the number of years}
\defDefault{0.0}
\defNote{If only one value is supplied, it will be repeated for all years in the observation}

\defSub{length\_weight\_cv}{The CV for the length-weight relationship}
\defType{Real number (estimable)}
\defDefault{0.10}
\defLowerBound{0.0 (exclusive)}

\defSub{length\_weight\_distribution}{The distribution of the length-weight relationship}
\defType{String}
\defDefault{normal}

\defSub{length\_bins}{The length bins}
\defType{Vector of real numbers (estimable)}
\defDefault{No default}

\defSub{length\_bins\_n}{The average number in each length bin}
\defType{Vector of real numbers (estimable)}
\defDefault{No default}

\defSub{units}{The units for the weight bins (grams, kilograms (kgs), or tonnes)}
\defType{String}
\defDefault{kgs}

\defSub{fishbox\_weight}{The target weight of each box}
\defType{Real number (estimable)}
\defDefault{20.0}
\defLowerBound{0.0 (exclusive)}

\defSub{weight\_bins}{The weight bins}
\defType{Vector of real numbers (estimable)}
\defDefault{No default}

\subsubsection{Observation of type Proportions At Age}
\commandlabsubarg{observation}{type}{Proportions\_At\_Age}.
\defRef{sec:Observation-ProportionsAtAge}
\label{syntax:Observation-ProportionsAtAge}

\defSub{min\_age}{The minimum age}
\defType{Non-negative integer}
\defDefault{No default}

\defSub{max\_age}{The maximum age}
\defType{Non-negative integer}
\defDefault{No default}

\defSub{plus\_group}{Is the maximum age the age plus group?}
\defType{Boolean}
\defDefault{true}

\defSub{time\_step}{The label of the time step that the observation occurs in}
\defType{String}
\defDefault{No default}

\defSub{years}{The years of the observed values}
\defType{Vector of non-negative integers}
\defDefault{No default}

\defSub{selectivities}{The labels of the selectivities}
\defType{Vector of strings}
\defDefault{true}

\defSub{process\_errors}{The process errors to use}
\defType{Vector of real numbers (estimable) of length equal to the number of years}
\defDefault{0.0}
\defNote{If only one value is supplied, it will be repeated for all years in the observation}

\defSub{ageing\_error}{The label of ageing error to use}
\defType{String}
\defDefault{No default}

\defSub{sum\_to\_one}{Scale the year (row) observed values by the total, so they sum to 1}
\defType{Boolean}
\defDefault{false}

\defSub{simulated\_data\_sum\_to\_one}{Whether simulated data is discrete or scaled by totals to be proportions for each year}
\defType{Boolean}
\defDefault{true}

\defSub{table obs}{The table of data specifying the observed values}
\defType{Data table with label = obs}
\defDefault{No default}
\defValue{A $n\times m$ matrix, where $n=$ the years and $m$ is categories \(\times\) length bins. See below for example.}
\defNote{example below}
\begin{verbatim}
table obs 
1993 0.1 0.2 0.3
1994 0.1 0.2 0.3
end_table
\end{verbatim}
\defSub{table error\_values}{The table of data specifying the error values}
\defType{Data table with label = error\_values}
\defDefault{No default}
\defValue{Can be specified two ways either as a $n\times 1$ matrix with an error value for each year. Or a $n\times m$ matrix, where $n=$ the years and $m$ is categories \(\times\) length bins. See below for example.}
\defNote{example below}
\begin{verbatim}
table error_values 
1993 234
1994 343
end_table
\end{verbatim}

\subsubsection{Observation of type Proportions At Length}
\commandlabsubarg{observation}{type}{Proportions\_At\_Length}.
\defRef{sec:Observation-ProportionsAtLength}
\label{syntax:Observation-ProportionsAtLength}

\defSub{time\_step}{The label of the time step that the observation occurs in}
\defType{String}
\defDefault{No default}

\defSub{years}{The years for which there are observations}
\defType{Vector of non-negative integers}
\defDefault{No default}

\defSub{selectivities}{The labels of the selectivities}
\defType{Vector of strings}
\defDefault{true}

\defSub{process\_errors}{The process errors to use}
\defType{Vector of real numbers (estimable) of length equal to the number of years}
\defDefault{0.0}
\defNote{If only one value is supplied, it will be repeated for all years in the observation}

\defSub{length\_bins}{The length bins}
\defType{Vector of real numbers (estimable)}
\defDefault{true}

\defSub{plus\_group}{Is the last length bin a plus group?}
\defType{Boolean}
\defDefault{true if the value of \commandsub{model}{length\_plus} is true, otherwise false}

\defSub{sum\_to\_one}{Scale the year (row) observed values by the total, so they sum to 1}
\defType{Boolean}
\defDefault{false}

\defSub{simulated\_data\_sum\_to\_one}{Whether simulated data is discrete or scaled by totals to be proportions for each year}
\defType{Boolean}
\defDefault{true}

\defSub{table obs}{The table of data specifying the observed values}
\defType{Data table with label = obs}
\defDefault{No default}
\defValue{A $n\times m$ matrix, where $n=$ the years and $m$ is categories \(\times\) length bins. See below for example.}
\defNote{example below}
\begin{verbatim}
table obs 
1993 0.1 0.2 0.3
1994 0.1 0.2 0.3
end_table
\end{verbatim}
\defSub{table error\_values}{The table of data specifying the error values}
\defType{Data table with label = error\_values}
\defDefault{No default}
\defValue{Can be specified two ways either as a $n\times 1$ matrix with an error value for each year. Or a $n\times m$ matrix, where $n=$ the years and $m$ is categories \(\times\) length bins. See below for example.}
\defNote{example below}
\begin{verbatim}
table error_values 
1993 234
1994 343
end_table
\end{verbatim}

\subsubsection{Observation of type Proportions By Category}
\commandlabsubarg{observation}{type}{Proportions\_By\_Category}.
\defRef{sec:Observation-ProportionsByCategory}
\label{syntax:Observation-ProportionsByCategory}

\defSub{min\_age}{The minimum age}
\defType{Non-negative integer}
\defDefault{No default}

\defSub{max\_age}{The maximum age}
\defType{Non-negative integer}
\defDefault{No default}

\defSub{time\_step}{The label of the time step that the observation occurs in}
\defType{String}
\defDefault{No default}

\defSub{plus\_group}{Use the age plus group?}
\defType{Boolean}
\defDefault{true}

\defSub{years}{The years for which there are observations}
\defType{Vector of non-negative integers}
\defDefault{No default}

\defSub{selectivities}{The labels of the selectivities}
\defType{Vector of strings}
\defDefault{true}

\defSub{categories2}{The target categories}
\defType{Vector of strings}
\defDefault{No default}

\defSub{selectivities2}{The target selectivities}
\defType{Vector of strings}
\defDefault{No default}

\subsubsection{Observation of type Proportions Mature By Age}
\commandlabsubarg{observation}{type}{Proportions\_Mature\_By\_Age}.
%\defRef{sec:Observation-ProportionsMatureByAge}
\label{syntax:Observation-ProportionsMatureByAge}

\defSub{min\_age}{The minimum age}
\defType{Non-negative integer}
\defDefault{No default}

\defSub{max\_age}{The maximum age}
\defType{Non-negative integer}
\defDefault{No default}

\defSub{time\_step}{The label of time-step that the observation occurs in}
\defType{String}
\defDefault{No default}

\defSub{plus\_group}{Use the age plus group?}
\defType{Boolean}
\defDefault{true}

\defSub{years}{The years for which there are observations}
\defType{Vector of non-negative integers}
\defDefault{No default}

\defSub{ageing\_error}{The label of ageing error to use}
\defType{String}
\defDefault{No default}

\defSub{total\_categories}{All category labels that were vulnerable to sampling at the time of this observation (not including the categories already given)}
\defType{Vector of strings}
\defDefault{true}

\defSub{time\_step\_proportion}{The proportion through the mortality block of the time step when the observation is evaluated}
\defType{Real number (estimable)}
\defDefault{0.5}
\defLowerBound{0.0 (inclusive)}
\defUpperBound{1.0 (inclusive)}

\subsubsection{Observation of type Proportions Migrating}
\commandlabsubarg{observation}{type}{Proportions\_Migrating}.
\defRef{sec:Observation-ProportionsMigrating}
\label{syntax:Observation-ProportionsMigrating}

\defSub{min\_age}{The minimum age}
\defType{Non-negative integer}
\defDefault{No default}

\defSub{max\_age}{The maximum age}
\defType{Non-negative integer}
\defDefault{No default}

\defSub{time\_step}{The label of the time step that the observation occurs in}
\defType{String}
\defDefault{No default}

\defSub{plus\_group}{Is the maximum age the age plus group?}
\defType{Boolean}
\defDefault{true}

\defSub{years}{The years for which there are observations}
\defType{Vector of non-negative integers}
\defDefault{No default}

\defSub{process\_errors}{The process errors to use}
\defType{Vector of real numbers (estimable) of length equal to the number of years}
\defDefault{0.0}
\defNote{If only one value is supplied, it will be repeated for all years in the observation}

\defSub{ageing\_error}{The label of the ageing error to use}
\defType{String}
\defDefault{No default}

\defSub{process}{The process label}
\defType{String}
\defDefault{No default}

\subsubsection{Observation of type Tag Recapture by Fishery}
\commandlabsubarg{observation}{type}{tag\_recapture\_by\_fishery}.
\defRef{sec:Observation-TagRecaptureByFishery}
\label{syntax:Observation-TagRecaptureByFishery}

\defSub{tagged\_categories}{The tagged categories that we want to generate recaptures for. Categories need to be space separated no use of the '+' category syntax.}
\defType{Vector of strings}
\defDefault{No default}

\defSub{time\_step}{The label of time-step that the observation occurs in}
\defType{Vector of strings}
\defDefault{No default}

\defSub{reporting\_rate}{The reporting rate for this observation}
\defType{Real number (estimable)}
\defDefault{No default}
\defLowerBound{0.0 (inclusive)}
\defUpperBound{1.0 (inclusive)}


\defSub{years}{The years for which there are observations}
\defType{Vector of non-negative integers}
\defDefault{No default}

\defSub{method\_of\_removal}{The label of the observed method of removals}
\defType{Vector of strings}
\defDefault{No default}


\defSub{mortality\_process}{The label of the mortality instantaneous process for the observation}
\defType{String}
\defDefault{No default}
\defNote{Allowed mortality process types are \subcommand{mortality\_instantaneous} and \subcommand{mortality\_hybrid}}

\defSub{table recaptured}{The table of recaptures in each year}
\defType{Data table with label = \subcommand{recaptured}}
\defDefault{No default}
\defValue{A $n\times 2$ matrix, where $n=$ the years and the first column specifies the year and the second column specifies the observed tag recaptures.}
\defNote{example below}
\begin{verbatim}
table recaptured
2000 10120
2001 90123
end_table
\end{verbatim}

\subsubsection{Observation of type Tag Recapture By Age}
\commandlabsubarg{observation}{type}{Tag\_Recapture\_By\_Age}.
\defRef{sec:Observation-TagRecaptureByAge}
\label{syntax:Observation-TagRecaptureByAge}

\defSub{min\_age}{The minimum age}
\defType{Non-negative integer}
\defDefault{No default}

\defSub{max\_age}{The maximum age}
\defType{Non-negative integer}
\defDefault{No default}

\defSub{plus\_group}{Is the maximum age the age plus group?}
\defType{Boolean}
\defDefault{true}

\defSub{years}{The years for which there are observations}
\defType{Vector of non-negative integers}
\defDefault{No default}

\defSub{time\_step}{The label of the time step that the observation occurs in}
\defType{String}
\defDefault{No default}

\defSub{selectivities}{The labels of the selectivities used for untagged categories}
\defType{Vector of strings}
\defDefault{true}

\defSub{tagged\_selectivities}{The labels of the tag category selectivities}
\defType{Vector of strings}
\defDefault{true}

\defSub{tagged\_categories}{The  categories of tagged individuals}
\defType{Vector of strings}
\defDefault{No default}

\defSub{detection}{The probability of detecting a recaptured individual}
\defType{Real number (estimable)}
\defDefault{No default}
\defLowerBound{0.0 (inclusive)}
\defUpperBound{1.0 (inclusive)}

\defSub{dispersion}{The over-dispersion parameter, $\phi$}
\defType{Vector of real numbers, one for each year of recaptures}
\defDefault{No default}
\defLowerBound{0.0}

\defSub{overlap\_scalar}{The overlap\_scalar parameter, $k$}
\defType{Vector of real numbers, one for each year of recaptures (if only one value is supplied, it is repeated for each year of recaptures)}
\defDefault{1.0}
\defLowerBound{0.0 (inclusive)}
\defNote{See Section \ref{sec:Observation-TagRecaptures} for more information}

\defSub{time\_step\_proportion}{The proportion through the mortality block of the time step when the observation is evaluated}
\defType{Real number (estimable)}
\defDefault{0.5}
\defLowerBound{0.0 (inclusive)}
\defUpperBound{1.0 (inclusive)}

\defSub{table recaptured}{The table of data specifying the recaptures}
\defType{Data table with label = recaptured}
\defDefault{No default}
\defValue{A $n\times m$ matrix, where $n=$ the years and $m$ is categories \(\times\) length bins. See below for example.}
\defNote{example below}
\begin{verbatim}
table recaptured 
1993 1 32 25
1994 3 4 43
end_table
\end{verbatim}

\defSub{table scanned}{The table of data specifying the scanned fish}
\defType{Data table with label = scanned}
\defDefault{No default}
\defValue{A $n\times m$ matrix, where $n=$ the years and $m$ is categories \(\times\) length bins. See below for example.}
\defNote{example below}
\begin{verbatim}
table scanned 
1993 1 32 25
1994 3 4 43
end_table
\end{verbatim}

\subsubsection{Observation of type Tag Recapture By Length}
\commandlabsubarg{observation}{type}{Tag\_Recapture\_By\_Length}.
\defRef{sec:Observation-TagRecaptureByLength}
\label{syntax:Observation-TagRecaptureByLength}

\defSub{years}{The years for which there are observations}
\defType{Vector of non-negative integers}
\defDefault{No default}

\defSub{time\_step}{The time step to execute in}
\defType{String}
\defDefault{No default}

\defSub{length\_bins}{The length bins}
\defType{Vector of real numbers (estimable)}
\defDefault{true}

\defSub{selectivities}{The labels of the selectivities used for untagged categories}
\defType{Vector of strings}
\defDefault{true}

\defSub{tagged\_selectivities}{The labels of the tag category selectivities}
\defType{Vector of strings}
\defDefault{No default}

\defSub{tagged\_categories}{The  categories of tagged individuals}
\defType{Vector of strings}
\defDefault{No default}

\defSub{detection}{The probability of detecting a recaptured individual}
\defType{Real number (estimable)}
\defDefault{No default}
\defLowerBound{0.0 (inclusive)}
\defUpperBound{1.0 (inclusive)}

\defSub{dispersion}{The over-dispersion parameter, $\phi$}
\defType{Vector of real numbers, one for each year of recaptures}
\defDefault{No default}
\defLowerBound{0.0}

\defSub{overlap\_scalar}{The overlap\_scalar parameter, $k$}
\defType{Vector of real numbers, one for each year of recaptures (if only one value is supplied, it is repeated for each year of recaptures)}
\defDefault{1.0}
\defLowerBound{0.0 (inclusive)}
\defNote{See Section \ref{sec:Observation-TagRecaptures} for more information}

\defSub{time\_step\_proportion}{The proportion through the mortality block of the time step when the observation is evaluated}
\defType{Real number (estimable)}
\defDefault{0.5}
\defLowerBound{0.0 (inclusive)}
\defUpperBound{1.0 (inclusive)}

\defSub{table recaptured}{The table of data specifying the recaptures}
\defType{Data table with label = recaptured}
\defDefault{No default}
\defValue{A $n\times m$ matrix, where $n=$ the years and $m$ is categories \(\times\) length bins. See below for example.}
\defNote{example below}
\begin{verbatim}
table recaptured 
1993 1 32 25
1994 3 4 43
end_table
\end{verbatim}

\defSub{table scanned}{The table of data specifying the scanned fish}
\defType{Data table with label = scanned}
\defDefault{No default}
\defValue{A $n\times m$ matrix, where $n=$ the years and $m$ is categories \(\times\) length bins. See below for example.}
\defNote{example below}
\begin{verbatim}
table scanned 
1993 1 32 25
1994 3 4 43
end_table
\end{verbatim}

\subsubsection{Observation of type Age Length}
\commandlabsubarg{observation}{type}{age\_length}.
\defRef{sec:Observation-AgeSize}
\label{syntax:Observation-AgeLength}

\defSub{time\_step}{The label of the time step that the observation occurs in}
\defType{String}
\defDefault{No default}

\defSub{selectivities}{The labels of the selectivities, one for each combined category}
\defType{Vector of strings}
\defDefault{true}

\defSub{numerator\_categories}{A combined category label that defines categories that make up the numerator}
\defType{Vector of strings}
\defNote{These categories are required to have the same age-length definition and have the same selectivity.}
\defDefault{the values defined in categories}

\defSub{year}{The year this observation occurred in}
\defType{Vector of non-negative integers}
\defDefault{No default}

\defSub{sample\_type}{The sample type}
\defType{string}
\defDefault{length}
\defallowed{age, length, random}

\defSub{ages}{vector of observed ages}
\defType{Vector of positive integers}
\defNote{Needs to be integers, with model age definition, and same number of elements as lengths}
\defDefault{No default}

\defSub{lengths}{vector of observed lengths}
\defType{Vector of real numbers}
\defNote{same number of elements as ages}
\defDefault{No default}

\defSub{ageing\_error}{The label of ageing error to use}
\defType{String}
\defDefault{No ageing error}


\else
\input{IncludedSyntax/ObservationLength}
\fi 

\subsection{\I{Likelihoods}}
\defComLab{Likelihood}{Define an object of type \emph{Likelihood}}.
\defRef{sec:Likelihood}
\label{syntax:Likelihood}

\subsubsection{Likelihood of type Bernoulli}
\commandlabsubarg{Likelihood}{type}{Bernoulli}.
\defRef{sec:Likelihood-Bernoulli}
\label{syntax:Likelihood-Bernoulli}

The Bernoulli type has no additional subcommands.
\subsubsection{Likelihood of type Binomial}
\commandlabsubarg{Likelihood}{type}{Binomial}.
\defRef{sec:Likelihood-Binomial}
\label{syntax:Likelihood-Binomial}

The Binomial type has no additional subcommands.
\subsubsection{Likelihood of type Binomial\_Approx}
\commandlabsubarg{Likelihood}{type}{Binomial\_Approx}.
\defRef{sec:Likelihood-BinomialApprox}
\label{syntax:Likelihood-BinomialApprox}

The Binomial\_Approx type has no additional subcommands.
\subsubsection{Likelihood of type Dirichlet}
\commandlabsubarg{Likelihood}{type}{Dirichlet}.
\defRef{sec:Likelihood-Dirichlet}
\label{syntax:Likelihood-Dirichlet}

The Dirichlet type has no additional subcommands.
\subsubsection{Likelihood of type Dirichlet\_Multinomial}
\commandlabsubarg{Likelihood}{type}{Dirichlet\_Multinomial}.
\defRef{sec:Likelihood-DirichletMultinomial}
\label{syntax:Likelihood-DirichletMultinomial}

\defSub{label}{Label of the Dirichlet-multinomial distribution}
\defType{String}
\defDefault{No default}

\defSub{type}{Type of likelihood}
\defType{String}
\defDefault{No default}

\defSub{theta}{Theta parameter (account for overdispersion)}
\defType{Real number (estimable)}
\defDefault{false}
\defLowerBound{0.0 (inclusive)}

\subsubsection{Likelihood of type Log\_Normal}
\commandlabsubarg{Likelihood}{type}{Log\_Normal}.
\defRef{sec:Likelihood-LogNormal}
\label{syntax:Likelihood-LogNormal}

The Log\_Normal type has no additional subcommands.
\subsubsection{Likelihood of type Log\_Normal\_With\_Q}
\commandlabsubarg{Likelihood}{type}{Log\_Normal\_With\_Q}.
\defRef{sec:Likelihood-LogNormalWithQ}
\label{syntax:Likelihood-LogNormalWithQ}

The Log\_Normal\_With\_Q type has no additional subcommands.
\subsubsection{Likelihood of type Multinomial}
\commandlabsubarg{Likelihood}{type}{Multinomial}.
\defRef{sec:Likelihood-Multinomial}
\label{syntax:Likelihood-Multinomial}

The Multinomial type has no additional subcommands.
\subsubsection{Likelihood of type Normal}
\commandlabsubarg{Likelihood}{type}{Normal}.
\defRef{sec:Likelihood-Normal}
\label{syntax:Likelihood-Normal}

The Normal type has no additional subcommands.
\subsubsection{Likelihood of type Poisson}
\commandlabsubarg{Likelihood}{type}{Poisson}.
\defRef{sec:Likelihood-Poisson}
\label{syntax:Likelihood-Poisson}

The Poisson type has no additional subcommands.
\subsubsection{Likelihood of type Pseudo}
\commandlabsubarg{Likelihood}{type}{Pseudo}.
\defRef{sec:Likelihood-Pseudo}
\label{syntax:Likelihood-Pseudo}

The Pseudo type has no additional subcommands.


\ifAgeBased
\subsection{\I{Defining ageing error}}\label{syntax:AgeingError}

The methods for including ageing error into estimation for observations are:

\begin{itemize}
	\item None
	\item Data
	\item Normal
	\item Off-by-one
\end{itemize}

Each type of ageing error has a set of subcommands and arguments specific to its type.

\input{IncludedSyntax/AgeingError}
\fi % end if
%\subsection{\I{Simulating observations}}
%\input{IncludedSyntax/Simulate}

 \pagebreak 
 \section{\I{Report command and subcommand syntax}\label{syntax:Reports}}

The description of each report is given in Section \ref{sec:Report}.

\subsection{\I{Report commands and subcommands}}

\defComLab{report}{Define an object of type \emph{Report}}.
\defRef{sec:Report}
\label{syntax:Report}

\defSub{label}{The report label}
\defType{String}
\defDefault{No default}

\defSub{type}{The report type}
\defType{String}
\defDefault{No default}

\defSub{file\_name}{The file name. If not supplied, then output is directed to standard out}
\defType{String}
\defDefault{No default}

\defSub{write\_mode}{Specify if any previous file with the same name should be overwritten, appended to, or is generated using a sequential suffix}
\defType{String}
\defDefault{overwrite}
\defValue{valid options are \subcommand{append}, \subcommand{overwrite}, \subcommand{incremental\_suffix}}

\defSub{format}{Report output format}
\defType{String}
\defDefault{r}
\defValue{Either \R\ for formatting for reading into \R\, or \texttt{none} for no formatting}


\subsubsection{Report of type Default}
\commandlabsubarg{report}{type}{Default}.
\defRef{sec:Report-Default}
\label{syntax:Report-Default}

\defSub{catchabilities}{Report catchabilities}
\defType{Boolean}
\defDefault{false}
\defNote{Reports all valid catchabilities}

\defSub{derived\_quantities}{Report derived quantities}
\defType{Boolean}
\defDefault{false}
\defNote{Reports all valid derived quantities}

\defSub{observations}{Report observations}
\defType{Boolean}
\defDefault{false}
\defNote{Reports all valid observations}

\defSub{processes}{Report processes}
\defType{Boolean}
\defDefault{false}
\defNote{Reports all valid processes}

\defSub{projects}{Report projects}
\defType{Boolean}
\defDefault{false}
\defNote{Reports all valid projections}

\defSub{selectivities}{Report selectivities}
\defType{Boolean}
\defDefault{false}
\defNote{Reports all valid selectivities}

\defSub{time\_varying}{Report time-varying parameters}
\defType{Boolean}
\defDefault{false}
\defNote{Reports all valid time-varying parameters}

\defSub{parameter\_transformations}{Report all parameter transformations}
\defType{Boolean}
\defDefault{false}
\defNote{Reports all valid parameter transformations}

\subsubsection{Report of type Addressable}
\commandlabsubarg{report}{type}{Addressable}.
\defRef{sec:Report-Addressable}
\label{syntax:Report-Addressable}

\defSub{parameter}{The addressable parameter name}
\defType{String}
\defDefault{No default}

\defSub{years}{Define the years that the report is generated for}
\defType{Vector of non-negative integers}
\defDefault{No default}

\defSub{time\_step}{Defines the time-step that the report applies to}
\defType{String}
\defDefault{No default}
\defValue{A valid time step label}

\ifAgeBased
\subsubsection{Report of type Age Length}
\commandlabsubarg{report}{type}{Age\_Length}.
\defRef{sec:Report-AgeLength}
\label{syntax:Report-AgeLength}

\defSub{time\_step}{The time step label}
\defType{String}
\defDefault{No default}
\defValue{A valid time step label}

\defSub{years}{The years for the report}
\defType{Vector of non-negative integers}
\defDefault{All years}

\defSub{age\_length}{The age-length label}
\defType{String}
\defDefault{No default}
\else
\subsubsection{Report of type Growth Increment model}
\commandlabsubarg{report}{type}{growth\_increment}.
\defRef{sec:Report-GrowthIncrement}
\label{syntax:Report-GrowthIncrement}

\defSub{time\_step}{The time step label}
\defType{String}
\defDefault{No default}
\defValue{A valid time step label}

\defSub{years}{The years for the report}
\defType{Vector of non-negative integers}
\defDefault{All years}

\defSub{growth\_increment}{The growth-increment label}
\defType{String}
\defDefault{No default}
\fi

\ifAgeBased
\subsubsection{Report of type Ageing Error Matrix}
\commandlabsubarg{report}{type}{Ageing\_Error\_Matrix}.
\defRef{sec:Report-AgeingErrorMatrix}
\label{syntax:Report-AgeingErrorMatrix}

\defSub{ageing\_error}{The ageing error label}
\defType{String}
\defDefault{No default}
\fi

\subsubsection{Report of type Catchability}
\commandlabsubarg{report}{type}{Catchability}.
\defRef{sec:Report-Catchability}
\label{syntax:Report-Catchability}

\defSub{catchability}{The catchability label}
\defType{String}
\defDefault{No default}
\defValue{If not specified, then the label of the report is assumed to be the category label}

%\subsubsection{Report of type Category List}
%\commandlabsubarg{report}{type}{Category\_List}.
%\defRef{sec:Report-CategoryList}
%\label{syntax:Report-CategoryList}

%The Category\_List report has no additional subcommands.

\subsubsection{Report of type Correlation Matrix}
\commandlabsubarg{report}{type}{Correlation\_Matrix}.
\defRef{sec:Report-CorrelationMatrix}
\label{syntax:Report-CorrelationMatrix}

The Correlation\_Matrix report has no additional subcommands.

\subsubsection{Report of type Covariance Matrix}
\commandlabsubarg{report}{type}{Covariance\_Matrix}.
\defRef{sec:Report-CovarianceMatrix}
\label{syntax:Report-CovarianceMatrix}

The Covariance\_Matrix type has no additional subcommands.

\subsubsection{Report of type Derived Quantity}
\commandlabsubarg{report}{type}{Derived\_Quantity}.
\defRef{sec:Report-DerivedQuantity}
\label{syntax:Report-DerivedQuantity}

\defSub{derived\_quantity}{The derived quantity label}
\defType{String}
\defDefault{No default}
\defValue{If not specified, then the label of the report is assumed to be the derived quantity label}
 
\subsubsection{Report of type Equation Test}
\commandlabsubarg{report}{type}{Equation\_Test}.
\defRef{sec:eq_parser}
\label{syntax:Report-EquationTest}

\defSub{equation}{The equation to do a test run of}
\defType{Vector of strings}
\defDefault{No default}

\subsubsection{Report of type Estimate Summary}
\commandlabsubarg{report}{type}{Estimate\_Summary}.
\defRef{sec:Report-EstimateSummary}
\label{syntax:Report-EstimateSummary}

\defValue{A summary of the estimated (free parameters)}

The Estimate\_Summary type has no additional subcommands.

\subsubsection{Report of type Estimate Value}
\commandlabsubarg{report}{type}{Estimate\_Value}.
\defRef{sec:Report-EstimateValue}
\label{syntax:Report-EstimateValue}

\defValue{The free parameters and their values, in a format suitable for use with \texttt{-i}}

The Estimate\_Value report has no additional subcommands.

\subsubsection{Report of type Estimation Result}
\commandlabsubarg{report}{type}{Estimation\_Result}.
\defRef{sec:Report-EstimationResult}
\label{syntax:Report-EstimationResult}

\defValue{A summary of the results of the minimisation}

The Estimation\_Result report has no additional subcommands.

\subsubsection{Report of type Hessian Matrix}
\commandlabsubarg{report}{type}{Hessian\_Matrix}.
\defRef{sec:Report-HessianMatrix}
\label{syntax:Report-HessianMatrix}

The Hessian\_Matrix report has no additional subcommands.

\subsubsection{Report of type Initialisation}
\commandlabsubarg{report}{type}{Initialisation}.
\defRef{sec:Report-Initialisation}
\label{syntax:Report-Initialisation}

The Initialisation report has no additional subcommands.

\subsubsection{Report of type Initialisation\_Partition}
\commandlabsubarg{report}{type}{Initialisation\_Partition}.
\defRef{sec:Report-InitialisationPartition}
\label{syntax:Report-InitialisationPartition}

\subsubsection{Report of type MCMC Covariance}
\commandlabsubarg{report}{type}{MCMC\_Covariance}.
\defRef{sec:Report-MCMCCovariance}
\label{syntax:Report-MCMCCovariance}\\
\defValue{This will output the covariance matrices (the initial covariance matrix and the covariance matrix if adapted ) used for the MCMC chain.}

The MCMC\_Covariance report has no additional subcommands.  

\subsubsection{Report of type MCMC Objective}
\commandlabsubarg{report}{type}{MCMC\_Objective}.
\defRef{sec:Report-MCMCObjective}
\label{syntax:Report-MCMCObjective}

The MCMC\_Objective report has no additional subcommands.

\defSub{file\_name}{The file name. If not supplied the default filename is used}
\defType{string}
\defDefault{objectives}

\defSub{write\_mode}{Has a different default to the rest of the reports.}
\defType{String}
\defDefault{\subcommand{incremental\_suffix}}
\defValue{valid options are \subcommand{append}, \subcommand{overwrite}, \subcommand{incremental\_suffix}}


\subsubsection{Report of type MCMC Sample}
\commandlabsubarg{report}{type}{MCMC\_Sample}.
\defRef{sec:Report-MCMCSample}
\label{syntax:Report-MCMCSample}


\defSub{file\_name}{The file name. If not supplied the default filename is used}
\defType{string}
\defDefault{samples}

\defSub{write\_mode}{Has a different default to the rest of the reports.}
\defType{String}
\defDefault{\subcommand{incremental\_suffix}}
\defValue{valid options are \subcommand{append}, \subcommand{overwrite}, \subcommand{incremental\_suffix}}

The MCMC\_Sample report has no additional subcommands.

%\subsubsection{Report of type MPD}
%\commandlabsubarg{report}{type}{MPD}.
%\defRef{sec:Report-MPD}
%\label{syntax:Report-MPD}\\
%\defValue{An MPD report, consisting of the free parameters and the covariance matrix}

%The MPD report has no additional subcommands.

\subsubsection{Report of type Objective Function}
\commandlabsubarg{report}{type}{Objective\_Function}.
\defRef{sec:Report-ObjectiveFunction}
\label{syntax:Report-ObjectiveFunction}

The Objective\_Function type has no additional subcommands.

\subsubsection{Report of type Observation}
\commandlabsubarg{report}{type}{Observation}.
\defRef{sec:Report-Observation}
\label{syntax:Report-Observation}

\defSub{observation}{The observation label}
\defType{String}
\defDefault{No default}

\defSub{normalised\_residuals}{Print Normalised Residuals?}
\defType{Boolean}
\defDefault{true}
\defNote{Only generated if valid for associated likelihood}

\defSub{pearsons\_residuals}{Print Pearsons Residuals?}
\defType{Boolean}
\defDefault{true}
\defNote{Only generated if valid for associated likelihood}

\subsubsection{Report of type Output Parameters}
\commandlabsubarg{report}{type}{Output\_Parameters}.
\defRef{sec:Report-OutputParameters}
\label{syntax:Report-OutputParameters}

The Output\_Parameters report has no additional subcommands.

\subsubsection{Report of Parameter transformations}
\commandlabsubarg{report}{type}{parameter\_transformation}.
\defRef{sec:Report-ParameterTransformations}
\label{syntax:Report-ParameterTransformation}

\defSub{parameter\_transformation}{label of parameter transformation block}
\defType{String}
\defDefault{No default}


\subsubsection{Report of type Partition}
\commandlabsubarg{report}{type}{Partition}.
\defRef{sec:Report-Partition}
\label{syntax:Report-Partition}

\defSub{time\_step}{Time Step label}
\defType{String}
\defDefault{No default}

\defSub{years}{Years}
\defType{Vector of non-negative integers}
\defDefault{All years}

\subsubsection{Report of type Partition Biomass}
\commandlabsubarg{report}{type}{Partition\_Biomass}.
\defRef{sec:Report-PartitionBiomass}
\label{syntax:Report-PartitionBiomass}

\defSub{time\_step}{The time step label}
\defType{String}
\defDefault{No default}

\defSub{years}{The years for the report}
\defType{Vector of non-negative integers}
\defDefault{All years}

\subsubsection{Report of type Process}
\commandlabsubarg{report}{type}{Process}.
\defRef{sec:Report-Process}
\label{syntax:Report-Process}

\defSub{process}{The process label that is reported}
\defType{String}
\defDefault{No default}
\defValue{A valid process label}
\defValue{If not specified, then the label of the report is assumed to be the process label}


\subsubsection{Report of type Profile}
\commandlabsubarg{report}{type}{Profile}.
\defRef{sec:Profile}
\label{syntax:Report-Profile}

\subsubsection{Report of type Project}
\commandlabsubarg{report}{type}{Project}.
\defRef{sec:Project}
\label{syntax:Report-Project}

\defSub{project}{The project label that is reported}
\defType{String}
\defDefault{No default}
\defValue{If not specified, then the label of the report is assumed to be the projection label}

\subsubsection{Report of type Random Number Seed}
\commandlabsubarg{report}{type}{Random\_Number\_Seed}.
\defRef{sec:Report-RandomNumberSeed}
\label{syntax:Report-RandomNumberSeed}

The Random\_Number\_Seed type has no additional subcommands.

\subsubsection{Report of type Selectivity}
\commandlabsubarg{report}{type}{Selectivity}.
\defRef{sec:Report-Selectivity}
\label{syntax:Report-Selectivity}

\defSub{selectivity}{Selectivity name}
\defType{String}
\defDefault{No default}
\defValue{If not specified, then the label of the report is assumed to be the selectivity label}

\defSub{length\_values}{Length bins for reporting if a length-based selectivity in an age-based model}
\defType{Vector of real numbers}
\defDefault{If not specified and this is a length-based selectivity in an age-based model, then length bins specified for the model will be used}
\defNote{It is a fatal error if this is a report for a length-based selectivity in an age-based model, but neither the length values or \command{model.length\_bins} were supplied}

\subsubsection{Report of type Selectivity By Year}
\commandlabsubarg{report}{type}{selectivity\_by\_year}.
\defRef{sec:Report-SelectivityByYear}
\label{syntax:Report-SelectivityByYear}

\defSub{selectivity}{Selectivity name}
\defType{String}
\defDefault{No default}
\defValue{If not specified, then the label of the report is assumed to be the selectivity label}

\defSub{years}{years to report the selectivity in}
\defType{String}
\defDefault{true}
\defValue{If not specified will print for all years in of the model}

\defSub{time\_step}{Time step label}
\defType{String}
\defDefault{No default}
\defNote{This should not matter, but is required in order to identify the time step for each year when values are printed.}

\subsubsection{Report of type Simulated Observation}
\commandlabsubarg{report}{type}{Simulated\_Observation}.
\defRef{sec:Report-SimulatedObservation}
\label{syntax:Report-SimulatedObservation}

\defSub{observation}{The observation label}
\defType{String}
\defDefault{No default}
\defValue{If not specified, then the label of the report is assumed to be the observation label}

\subsubsection{Report of type Time Varying}
\commandlabsubarg{report}{type}{Time\_Varying}.
\defRef{sec:Report-TimeVarying}
\label{syntax:Report-TimeVarying}

\defSub{time\_varying}{The time varying label that is reported}
\defType{String}
\defDefault{No default}
\defValue{If not specified, then the label of the report is assumed to be the time varying label}



 \pagebreak 
 \section{\I{Including commands from other files}\label{syntax:General}}

\defComArg{include}{file}{\I{Include an external file}}

\defArg{file}{The name of the external \config\ to include}
\defType{string}
\defDefault{No default}
\defValue{A valid \config}
\defNote{If \texttt{file\_name} includes a space character, then it must be enclosed in quotes, for example \command{include} \argument{\ "my file.csl2"}. Also note that the \command{include} does not denote the end of the previous command block as is the case for all other commands}

 \pagebreak 
 \section{\I{Validating model values using asserts}}\label{syntax:Assert}

\CNAME\ can validate or check certain addressables parameters as a part of testing and validation with the assert command. Asserts check the value of a specific addressables (for example, and observations, parameters, or the objective function). Asserts are one aspect of the internal tests \CNAME\ uses to ensure accuracy across versions and revisions (see Section \ref{sec:Assert})

\subsection{Assert syntax}

\input{IncludedSyntax/Assert}

\label{sec:syntax}

\clearemptydoublepage{}
\section{\I{Tips for setting up \CNAME\ model based on an existing CASAL model}\label{sec:setupCasal2}}

Many users of \CNAME\ may be be starting with a functioning CASAL model. This section focuses on transitioning from CASAL to \CNAME.

There are a range of reasons why \CNAME\ will output different values when comparing model output to CASAL models. There are also reasons why values will differ that are not so obvious such as, reasons caused from using different compilers on different machines where over/underflow might occur. It is assumed that the latter reasons  should be rare, and the 'overall' behaviour when it comes to estimation will be the same between CASAL and \CNAME.

Reasons why there may be different values reported between CASAL and \CNAME\ include:

\begin{itemize}
	\item Report rounding. There are settings with respect to output in CASAL that set the number of significant figures for writing to files. So if values look truncated, this might be the reason.

	
	\item Priors on parameters that are turned off with \subcommand{upper\_bound} = \subcommand{lower\_bound}. In both CASAL and \CNAME\ the estimation of parameters can be turned off by setting the bounds equal. CASAL will evaluate the prior value and add this to the objective function.  This contribution is a constant value so it will not affect parameter inference. It may however be confusing when comparing output between the two models.

	\item Default values. There are a lot of switches in these programs, and options like the \subcommand{delta} in \CNAME\ or \subcommand{r} parameter in CASAL for robustifying likelihoods can cause differences.

	\item The order of processes. CASAL has a predefined sequence in which it executes processes within a time step (i.e., ageing, recruitment, maturation, migration, growth, natural and fishing mortality, disease mortality, tag release events, tag shedding rate, and semelparous mortality), where as \CNAME\ is completely user defined.

	\item Length-based processes or observations. \CNAME\ has updated the cumulative normal distribution calculation (CASAL used the approximated no closed form solution) with better approximations.

	\item Compositional observations. CASAL will only normalise (scale by the total) if the sum of proportions for a year are greater than \(1.01\) or less than \(0.99\). \CNAME\ will re normalise the prooprtions for a year even if they sum to one. If the observations are within those bounds technically CASAL and \CNAME\ will have slightly different observations and will generate small differences in likelihoods.

	\item Tag penalties. CASAL applies a penalty to the sum of squares on total tagged fish in a 'tagging episode' from the model compared to observed number of tagged fish. \CNAME\ applies a penalty on the transition rate by length. If tags are applied in a length bin that does not have individuals, e.g., a model configuration which tags 2 individuals of length $l$ when there are no individuals in that length bin will include a penalty.
\end{itemize}

Many of the flags and options in CASAL and \CNAME\ are the same or similar. The syntax section of this document (Section~\ref{syntax:Population}) provides more details about the \CNAME\ functionality and behaviour. Check that the programs produce the same results with a \textbf{range} of parameter values using the deterministic run command (\texttt{casal2 -r}), before doing an estimation run (\texttt{casal2 -e}).

The first outputs to check when comparing \CNAME\ and CASAL versions of the same model are the stock dynamics outputs, ignoring the fits to observations. That is, check the initial age structure, the SSB and YCS values and patterns, R0, B0, etc. If these outputs differ, then the fits to the observations will likely also be different.

There are a few linkages with certain stock dynamics outputs to check to determine if processes are misspecified. Differences between the proportions in the initial age structure, assuming an equilibrium state, are due to $M$, natural mortality. Differences in the initial equilibrium recruitment value, $R_0$, are due to growth (\command{age\_length} or \command{length\_weight}). Many models estimate $B_0$ so that $R_0$ is a back calculation through the growth curve.

If the initial age structure is the same, next check the derived quantities such as the SSB values. Differences in these values are generally caused by how fishing and recruitment processes are specified. Check which $YCS$ values are estimated or standardised, the definition and designation of selectivities, etc.

Once the stock dynamics outputs match, check the results with a few different sets of starting parameter values by using the \texttt{-i} command line option. Next, check the fits to the observation data by comparing the expected values.  Assuming the observations in both models match, the differences in the objective function value come from the expected values and the likelihood configurations. This is where subcommands such as the robustification values and the default values may differ between CASAL and \CNAME.

Once the stock dynamics outputs and the fits to the observation data are the same, do an estimation run (\texttt{casal2 -e}). If CASAL and \CNAME\ do not optimise to the same parameter values, then use the parameter values from CASAL and do a deterministic run with \CNAME\ using the CASAL estimated parameter values (\texttt{casal2 -r -i CASAL\_mpd\_pars.txt}). Then check the stock dynamics outputs and the fits to the observation data and determine where the differences in the parameter estimates and outputs are.

The next question is, how close do the parameter estimates, expected values, and objective function values have to be to say that the models are equivalent? This is an ongoing topic of discussion.  Previously, subjective qualitative measures have been used to decide whether the models are equivalent. A recorded comparison for the hake stock assessment can be found at Appendix B in \cite{horn2017stock}.









\clearemptydoublepage{}
\include{Examples}

\clearemptydoublepage{}
\section{\I{Post-processing output using \R} \label{sec:PostProcessing}}\index{Post processing}\index{Post-processing section}

\R\ (\url{https://www.r-project.org/}) is the main application used to process and visualise output from a \CNAME\ model. \R\ is free and can be downloaded from \url{https://cran.r-project.org/}. Once you have installed \R\ you can install the \cname\ \R\ package from the file (\texttt{Casal2\_1.0.tar.gz}) which is part of the \CNAME\ download.

\CNAME\ has two \R\ packages, a base library which is bundled with \CNAME\ application and a post processing package \texttt{r4Casal2} for plotting and model comparisons \url{https://github.com/Casal2/Casal2_contrib}. The base \R\ package is made to read and write output from \CNAME\ where as the post-processing package is more generalisable. 

There are three types of output that \CNAME\ can produce, depending on the type of analysis run. These outputs are: Standard, MCMC, and Derived Quantity.

The Standard outputs are the reports that are produced in most \CNAME\ run modes, with the exception of \texttt{-s} and \texttt{-m}. The Standard output can be split into two additional categories, a single parameter run (\texttt{casal2 -r}) or a multi-parameter run (\texttt{casal2 -r -i many\_pars.out}), or running in projection mode (\texttt{-f 1}). The Standard outputs can be read into \R\ using the \texttt{extract.mpd()} function.

The second type of output is generated when doing an MCMC analysis (\texttt{casal2 -m}), which can generate two files, \texttt{mcmc\_objective.out} and \texttt{mcmc\_samples.out}. The MCMC outputs can be used to summarise convergence properties or chain behaviour, and can also be used to view marginal posteriors and quantify parameter uncertainty.

The third output type is the Derived Quantity outputs, also referred to as tabular output. The Derived Quantity output can be generated after an MCMC analysis is done, to produce the marginal posteriors for derived quantities. A commonly reported derived quantity in fisheries stock assessment modelling is the time series of spawning stock biomass. To get the posterior distributions for these derived quantities use the \texttt{-{}-tabular} flag (e.g., \texttt{casal2 -r -i mcmc\_samples.out -{}-tabular > Tabular\_report.out}). This output can then be read into \R\ using the \texttt{extract.tabular()} function.

\CNAME's reported output is written so that each \command{report} will start with a '*' and end with `*end'. This format can be used as the basis to construct functions that read \CNAME\ output to identify and read individual reports for post-processing.

The \CNAME\ \R\ \texttt{extract()} functions differ by how the expected output is structured and they each create a different \cname\ object. The \texttt{summary()} and \texttt{plot()} functions will generate different plots for the different \cname\ objects. Objects produced by the \texttt{extract()} function can be queried with \texttt{class(object)}.

The list of \cname\ \R\ functions include:

\begin{itemize}
	\item \texttt{extract.mpd()}, which parses the \CNAME\ default output into a list
	\item \texttt{extract.mcmc()}, which parses the \CNAME\ MCMC output into a list
	\item \texttt{extract.tabular()}, which parses the \CNAME\ tabular output into a list
	\item \texttt{extract.parameters()}, which parses the \CNAME\ parameter files into a list
	\item \texttt{generate.starting.pars()}, which reads in a file that contains the \command{estimate} blocks and generates 'N' starting values to test convergence
	\item \texttt{burn.in.tabular()}, which omits the first 'N' rows from a \subcommand{casal2TAB} object
	\item \texttt{extract.csl2.file()}, which reads a \CNAME\ .csl2 (configuration) file into a list
	\item \texttt{write.csl2.file()}, which writes a \CNAME\ .csl2 (configuration) file to a file
	\item \texttt{ReadSimulatedData()}, which parses \CNAME\ output from a \texttt{casal2 -s} run
	\item \texttt{Method.TA1.8()}, which returns a weighting factor for age or length composition data. See \cite{francis2011data} for more detail
	\item \texttt{apply.dataweighting.to.csl2()}, which parses a \CNAME\ .csl2 (configuration) file that contains \command{observation} blocks, applies a weighting factor to an age or length composition data set, and generates a new \subcommand{.csl2} file with modified effective sample size values
\end{itemize}

The required and optional arguments for these functions can be queried after loading the \CNAME\ \R\ library with \texttt{library(Casal2)} and using the standard \R\ help syntax \texttt{?} (e.g., \texttt{?param.profile()}). Many of the help files have example code and data to demonstrate function syntax.

\paragraph*{Data weighting}

An important component of fisheries stock assessment modelling is addressing data conflicts through the use of data weighting. There are a range of methods that can be used (\cite{francis2011data}). The \CNAME\ \R\ function is \texttt{Method.TA1.8()}. An additional function \subcommand{apply.dataweighting.to.csl2()} automatically applies a weighting factor to a specific age or length composition data in an \command{observation} block, and generates a new \subcommand{.csl2} file with modified effective sample size values.

\begin{lstlisting}
library(casal2)

## read in the reported output from a "casal2 -e" run
## ensure there is a @report block for the observation of interest.
mpd <- extract.mpd(file = "estimate.log")

## calculate weighting factor from Francis method
WeightingFactor <- Method.TA1.8(model = mpd, observation_labels = "chatTANage")

## Apply the weighting factor to the block in the Observation.csl2 file
## this call generates a new file (Observation.csl2.0) with the re-weighted effective sample sizes
apply.dataweighting.to.csl2(weighting_factor = WeightingFactor,
                                Observation_csl2_file = "Observations.csl2",
                                Observation_label = "chatTANage",
                                Observation_out_filename = "Observation.csl2.0")
\end{lstlisting}


Automating the data weighting process:

\begin{lstlisting}
library(Casal2)

mpd <- extract.mpd(file = "estimate.log")

ModelFactor <- Method.TA1.8(mpd, observation_labels = c("ObserverProportionsAtAge"))

## make a back-up copy of the file Observation.csl2 before running this section

while(abs(ModelFactor - 1) > 0.01) {
	shell("betadiff & casal2 -e > estimate.log 2> log.out")

	new_mpd <- extract.mpd(file = "estimate.log")

	ModelFactor <- Method.TA1.8(new_mpd, observation_labels = c("ObserverProportionsAtAge"))

	apply.dataweighting.to.csl2(weighting_factor = ModelFactor,
                                  Observation_csl2_file = "Observation.csl2",
                                  Observation_out_filename = "Observation.csl2",
                                  Observation_label = c("ObserverProportionsAtAge"))
	print(ModelFactor)
}
\end{lstlisting}

\paragraph*{Troubleshooting the \cname\ \R\ package}

If you get this error when using one of the \texttt{extract()} functions

\begin{lstlisting}
Read 1 item
Warning messages:
1: In scan(filename, what = "", sep = "\n", fileEncoding = fileEncoding) :
embedded nul(s) found in input
2: In extract.mpd(file = "results.txt", fileEncoding = "") :
File is empty, no reports found
\end{lstlisting}

You may be able to resolve this issue by using an alternative UTF format by specifying this format with the \subcommand{fileEncoding} parameter

\begin{lstlisting}
MyOutput <- extract.mpd(file = "Estimate.log", path = getwd(), fileEncoding = "UTF-16LE")
\end{lstlisting}


\clearemptydoublepage{}
\section{Troubleshooting\label{sec:TroubleShooting}}

\CNAME\ can implement complex models that provide many opportunities for error --- either because the parameter files do not correctly specify the model, or because the model specified does not appear to work as expected. When in doubt, ask an experienced user. Debugging versions of \CNAME\ are available that can help to track down cryptic errors.

If you cannot resolve an issue using these guidelines then please contact the development team. To report an issue please follow the guidelines described in Section~\ref{sec:ErrorGuidelines}.

For most issues, \CNAME\ attempts to produce informative error messages. There are optional command line arguments that will give more verbose reporting, and should enable additional information to help resolve a problem. However, when \CNAME\ generates an error and the error message does not make sense, please let the Development Team know. Even if you manage to fix the problem yourself, we may be able to implement a more helpful error message or modify the user manual, and make life easier for the next person to encounter the problem. You can do this by submitting an issue in the GitHub repository at \url{https://github.com/Casal2/CASAL2/issues}.

\subsection{Logging}\label{sec:TroubleShooting-logging}

\CNAME's internal logging system can be invoked at the command line with argument  \argument{--loglevel} followed by one of the options: \argument{trace}, \argument{finest}, \argument{fine}, \argument{medium}.

The optimal level of logging will depend on what run mode you are using and the granularity of information that you would like to see. The ordering for the options is that \argument{medium} is the most coarse, and \argument{trace} being the finest level, with \argument{fine} and \argument{finest} in-between. We suggest that if you are running \CNAME\ in an iterative state such as for estimation (\argument{casal2 -e}) or MCMC you use \argument{medium} level. This is because the logging can print a lot of information for a single model run, so an estimation which could comprise thousands of model runs can produce very large text files with the finer logging option specified. For a single iteration run such as \argument{casal2 -r} each of the logging options can be useful during different phases of model development.

For example, to enable logging with trace level output:

\begin{itemize}
	\item On Windows: \argument{casal2 -r --loglevel trace > run.log 2> run.err}
	\item On Linux: \argument{casal2 -r --loglevel trace > run.log 2\&> run.err}
\end{itemize}

This argument will output \CNAME's reports to the file "run.log", and the "\argument{2>}" or "\argument{2\&>}" syntax will print the error logged information to the file "run.err". You should be able to see where \CNAME\ failed, and is exited by going to the end of the "run.err" file and looking at the last few messages.

\subsection{Reporting errors\label{sec:ReportingErrors}}

If you find a bug or error in \CNAME, please submit an issue in the GitHub repository at \url{https://github.com/Casal2/CASAL2/issues}.

Please follow the guidelines below so that the bug or error can be reproduced. It is helpful to be as detailed and specific as possible when describing the observed behaviour as well as the expected behaviour. If possible, try to reduce the \config\ to demonstrate the error with a reduced set of commands and model structure, and aim to have as little else going on in the model as possible. This will make it faster to isolate the problem and provide a solution or fix.

\subsubsection{Guidelines for reporting an error with \CNAME\label{sec:ErrorGuidelines}}

\begin{enumerate}
\item Ensure you are using the most recent version of \CNAME, as the bug or error you are having may have already been resolved.
\item Provide the version of \CNAME\ you are using, e.g., "\CNAME\ \VER". The version is output by \CNAME\ with the command \argument{casal2 -v}.
\item Provide the system  you are using, e.g., "x64 Intel CPU with Microsoft Windows 10".
\item Provide a brief description of the problem, e.g., "a segmentation fault was produced".
\item If the problem is reproducible, please describe in detail the steps required to cause it, and include the \CNAME\ configuration files, other input files, and any output files generated. Specify the \emph{exact} command line arguments that were used, e.g., "Using the command \texttt{casal2 -e} produced a segmentation fault. The \config s are attached."
\item If the problem is not reproducible (it happened only once, or occasionally for no apparent reason), please describe in detail the circumstances in which it occurred and the behaviour observed, e.g., "\CNAME\ crashed, but I have not been able to reproduce the issue. It seemed to be related to a local network crash but I cannot be sure."
\item If the problem produced any error messages, please give the \emph{exact} text displayed, e.g., "\texttt{segmentation fault (core dumped)}".
\item Attach all relevant input and output files so that the problem can be reproduced; these files can be compressed into a single file e.g., a zip file, and uploaded to GitHub.
\end{enumerate}


\clearemptydoublepage{}
\section{\I{\CNAME\ software license}}\label{sec:License}

\begin{center}
{\parindent 0in

\large{GNU GENERAL PUBLIC LICENSE}
\date{Version 2, June 1991}

Copyright \copyright\ 1989, 1991 Free Software Foundation, Inc.

\bigskip

51 Franklin Street, Fifth Floor, Boston, MA  02110-1301, USA

\bigskip

Everyone is permitted to copy and distribute verbatim copies
of this license document, but changing it is not allowed.
}
\end{center}

\begin{center}
{\bf\large Preamble}
\end{center}


The licenses for most software are designed to take away your freedom to
share and change it.  By contrast, the GNU General Public License is
intended to guarantee your freedom to share and change free software---to
make sure the software is free for all its users.  This General Public
License applies to most of the Free Software Foundation's software and to
any other program whose authors commit to using it.  (Some other Free
Software Foundation software is covered by the GNU Library General Public
License instead.)  You can apply it to your programs, too.

When we speak of free software, we are referring to freedom, not price.
Our General Public Licenses are designed to make sure that you have the
freedom to distribute copies of free software (and charge for this service
if you wish), that you receive source code or can get it if you want it,
that you can change the software or use pieces of it in new free programs;
and that you know you can do these things.

To protect your rights, we need to make restrictions that forbid anyone to
deny you these rights or to ask you to surrender the rights.  These
restrictions translate to certain responsibilities for you if you
distribute copies of the software, or if you modify it.

For example, if you distribute copies of such a program, whether gratis or
for a fee, you must give the recipients all the rights that you have.  You
must make sure that they, too, receive or can get the source code.  And
you must show them these terms so they know their rights.

We protect your rights with two steps: (1) copyright the software, and (2)
offer you this license which gives you legal permission to copy,
distribute and/or modify the software.

Also, for each author's protection and ours, we want to make certain that
everyone understands that there is no warranty for this free software.  If
the software is modified by someone else and passed on, we want its
recipients to know that what they have is not the original, so that any
problems introduced by others will not reflect on the original authors'
reputations.

Finally, any free program is threatened constantly by software patents.
We wish to avoid the danger that redistributors of a free program will
individually obtain patent licenses, in effect making the program
proprietary.  To prevent this, we have made it clear that any patent must
be licensed for everyone's free use or not licensed at all.

The precise terms and conditions for copying, distribution and
modification follow.

\begin{center}
{\Large \sc Terms and Conditions For Copying, Distribution and
  Modification}
\end{center}


%\renewcommand{\theenumi}{\alpha{enumi}}
\begin{enumerate}

\addtocounter{enumi}{-1}

\item

This License applies to any program or other work which contains a notice
placed by the copyright holder saying it may be distributed under the
terms of this General Public License.  The ``Program'', below, refers to
any such program or work, and a ``work based on the Program'' means either
the Program or any derivative work under copyright law: that is to say, a
work containing the Program or a portion of it, either verbatim or with
modifications and/or translated into another language.  (Hereinafter,
translation is included without limitation in the term ``modification''.)
Each licensee is addressed as ``you''.

Activities other than copying, distribution and modification are not
covered by this License; they are outside its scope.  The act of
running the Program is not restricted, and the output from the Program
is covered only if its contents constitute a work based on the
Program (independent of having been made by running the Program).
Whether that is true depends on what the Program does.

\item You may copy and distribute verbatim copies of the Program's source
  code as you receive it, in any medium, provided that you conspicuously
  and appropriately publish on each copy an appropriate copyright notice
  and disclaimer of warranty; keep intact all the notices that refer to
  this License and to the absence of any warranty; and give any other
  recipients of the Program a copy of this License along with the Program.

You may charge a fee for the physical act of transferring a copy, and you
may at your option offer warranty protection in exchange for a fee.

\item

You may modify your copy or copies of the Program or any portion
of it, thus forming a work based on the Program, and copy and
distribute such modifications or work under the terms of Section 1
above, provided that you also meet all of these conditions:

\begin{enumerate}

\item

You must cause the modified files to carry prominent notices stating that
you changed the files and the date of any change.

\item

You must cause any work that you distribute or publish, that in
whole or in part contains or is derived from the Program or any
part thereof, to be licensed as a whole at no charge to all third
parties under the terms of this License.

\item
If the modified program normally reads commands interactively
when run, you must cause it, when started running for such
interactive use in the most ordinary way, to print or display an
announcement including an appropriate copyright notice and a
notice that there is no warranty (or else, saying that you provide
a warranty) and that users may redistribute the program under
these conditions, and telling the user how to view a copy of this
License.  (Exception: if the Program itself is interactive but
does not normally print such an announcement, your work based on
the Program is not required to print an announcement.)

\end{enumerate}


These requirements apply to the modified work as a whole.  If
identifiable sections of that work are not derived from the Program,
and can be reasonably considered independent and separate works in
themselves, then this License, and its terms, do not apply to those
sections when you distribute them as separate works.  But when you
distribute the same sections as part of a whole which is a work based
on the Program, the distribution of the whole must be on the terms of
this License, whose permissions for other licensees extend to the
entire whole, and thus to each and every part regardless of who wrote it.

Thus, it is not the intent of this section to claim rights or contest
your rights to work written entirely by you; rather, the intent is to
exercise the right to control the distribution of derivative or
collective works based on the Program.

In addition, mere aggregation of another work not based on the Program
with the Program (or with a work based on the Program) on a volume of
a storage or distribution medium does not bring the other work under
the scope of this License.

\item
You may copy and distribute the Program (or a work based on it,
under Section 2) in object code or executable form under the terms of
Sections 1 and 2 above provided that you also do one of the following:

\begin{enumerate}

\item

Accompany it with the complete corresponding machine-readable
source code, which must be distributed under the terms of Sections
1 and 2 above on a medium customarily used for software interchange; or,

\item

Accompany it with a written offer, valid for at least three
years, to give any third party, for a charge no more than your
cost of physically performing source distribution, a complete
machine-readable copy of the corresponding source code, to be
distributed under the terms of Sections 1 and 2 above on a medium
customarily used for software interchange; or,

\item

Accompany it with the information you received as to the offer
to distribute corresponding source code.  (This alternative is
allowed only for noncommercial distribution and only if you
received the program in object code or executable form with such
an offer, in accord with Subsection b above.)

\end{enumerate}


The source code for a work means the preferred form of the work for
making modifications to it.  For an executable work, complete source
code means all the source code for all modules it contains, plus any
associated interface definition files, plus the scripts used to
control compilation and installation of the executable.  However, as a
special exception, the source code distributed need not include
anything that is normally distributed (in either source or binary
form) with the major components (compiler, kernel, and so on) of the
operating system on which the executable runs, unless that component
itself accompanies the executable.

If distribution of executable or object code is made by offering
access to copy from a designated place, then offering equivalent
access to copy the source code from the same place counts as
distribution of the source code, even though third parties are not
compelled to copy the source along with the object code.

\item
You may not copy, modify, sublicense, or distribute the Program
except as expressly provided under this License.  Any attempt
otherwise to copy, modify, sublicense or distribute the Program is
void, and will automatically terminate your rights under this License.
However, parties who have received copies, or rights, from you under
this License will not have their licenses terminated so long as such
parties remain in full compliance.

\item
You are not required to accept this License, since you have not
signed it.  However, nothing else grants you permission to modify or
distribute the Program or its derivative works.  These actions are
prohibited by law if you do not accept this License.  Therefore, by
modifying or distributing the Program (or any work based on the
Program), you indicate your acceptance of this License to do so, and
all its terms and conditions for copying, distributing or modifying
the Program or works based on it.

\item
Each time you redistribute the Program (or any work based on the
Program), the recipient automatically receives a license from the
original licensor to copy, distribute or modify the Program subject to
these terms and conditions.  You may not impose any further
restrictions on the recipients' exercise of the rights granted herein.
You are not responsible for enforcing compliance by third parties to
this License.

\item
If, as a consequence of a court judgment or allegation of patent
infringement or for any other reason (not limited to patent issues),
conditions are imposed on you (whether by court order, agreement or
otherwise) that contradict the conditions of this License, they do not
excuse you from the conditions of this License.  If you cannot
distribute so as to satisfy simultaneously your obligations under this
License and any other pertinent obligations, then as a consequence you
may not distribute the Program at all.  For example, if a patent
license would not permit royalty-free redistribution of the Program by
all those who receive copies directly or indirectly through you, then
the only way you could satisfy both it and this License would be to
refrain entirely from distribution of the Program.

If any portion of this section is held invalid or unenforceable under
any particular circumstance, the balance of the section is intended to
apply and the section as a whole is intended to apply in other
circumstances.

It is not the purpose of this section to induce you to infringe any
patents or other property right claims or to contest validity of any
such claims; this section has the sole purpose of protecting the
integrity of the free software distribution system, which is
implemented by public license practices.  Many people have made
generous contributions to the wide range of software distributed
through that system in reliance on consistent application of that
system; it is up to the author/donor to decide if he or she is willing
to distribute software through any other system and a licensee cannot
impose that choice.

This section is intended to make thoroughly clear what is believed to
be a consequence of the rest of this License.

\item
If the distribution and/or use of the Program is restricted in
certain countries either by patents or by copyrighted interfaces, the
original copyright holder who places the Program under this License
may add an explicit geographical distribution limitation excluding
those countries, so that distribution is permitted only in or among
countries not thus excluded.  In such case, this License incorporates
the limitation as if written in the body of this License.

\item
The Free Software Foundation may publish revised and/or new versions
of the General Public License from time to time.  Such new versions will
be similar in spirit to the present version, but may differ in detail to
address new problems or concerns.

Each version is given a distinguishing version number.  If the Program
specifies a version number of this License which applies to it and ``any
later version'', you have the option of following the terms and conditions
either of that version or of any later version published by the Free
Software Foundation.  If the Program does not specify a version number of
this License, you may choose any version ever published by the Free Software
Foundation.

\item
If you wish to incorporate parts of the Program into other free
programs whose distribution conditions are different, write to the author
to ask for permission.  For software which is copyrighted by the Free
Software Foundation, write to the Free Software Foundation; we sometimes
make exceptions for this.  Our decision will be guided by the two goals
of preserving the free status of all derivatives of our free software and
of promoting the sharing and reuse of software generally.

\begin{center}
{\Large\sc
No Warranty
}
\end{center}

\item
{\sc Because the program is licensed free of charge, there is no warranty
for the program, to the extent permitted by applicable law.  Except when
otherwise stated in writing the copyright holders and/or other parties
provide the program ``as is'' without warranty of any kind, either expressed
or implied, including, but not limited to, the implied warranties of
merchantability and fitness for a particular purpose.  The entire risk as
to the quality and performance of the program is with you.  Should the
program prove defective, you assume the cost of all necessary servicing,
repair or correction.}

\item
{\sc In no event unless required by applicable law or agreed to in writing
will any copyright holder, or any other party who may modify and/or
redistribute the program as permitted above, be liable to you for damages,
including any general, special, incidental or consequential damages arising
out of the use or inability to use the program (including but not limited
to loss of data or data being rendered inaccurate or losses sustained by
you or third parties or a failure of the program to operate with any other
programs), even if such holder or other party has been advised of the
possibility of such damages.}

\end{enumerate}


\clearemptydoublepage{}
\section{\I{Acknowledgements}\label{sec:acknowledgements}}

We thank the developers of CASAL \citep{1388} for their ideas that led to the development of \CNAME. The \CNAME\ logo was designed by Ian Doonan and Erika Mackay \href{http://www.niwa.co.nz}{(NIWA)}.

Much of the structure of \CNAME, the methods and equations, and documentation draw heavily on similar components of the fisheries population modelling application CASAL \citep{1388} and  the spatial population model SPM \citep{SPM}. We thank the authors of CASAL and SPM for their permission to use their work as the basis for parts of \CNAME\ and allow the use of the definitions, concepts, and documentation.

The development of \CNAME\ was funded by the New Zealand \href{http://www.mpi.govt.nz}{Ministry for Primary Industries} and the \href{http://www.niwa.co.nz}{National Institute of Water \& Atmospheric Research Ltd. (NIWA)}. More recent developments of this version were funded by Ocean Environmental Ltd.


\clearemptydoublepage{}
% Referencing
\bibliographystyle{plainnat}
\renewcommand{\bibsection}{%
  \section{References}}
  \setcitestyle{round,aysep={},yysep={;}%
}
\include{References}

\clearemptydoublepage{}

%\clearemptydoublepage{}
%\section{Quick reference}\label{sec:QuickReference}
\small

\defComLab{additional\_prior}{Define an object of type \emph{Additional\_Prior}}

\defSub{label}{The label for the additional prior}
\defSub{parameter}{The name of the parameter for the additional prior}
\defSub{type}{The additional prior type}

\commandlabsubarg{additional\_prior}{type}{Beta}

\defSub{mu}{Beta distribution mean $\mu$ parameter}
\defSub{sigma}{Beta distribution variance $\sigma$ parameter}
\defSub{a}{Beta distribution lower bound, of the range $A$ parameter}
\defSub{b}{Beta distribution upper bound of the range $B$ parameter}

\commandlabsubarg{additional\_prior}{type}{Element\_Difference}

\defSub{second\_parameter}{The name of the second parameter for comparing}
\defSub{multiplier}{Multiply the penalty by this factor}

\commandlabsubarg{additional\_prior}{type}{Log\_Normal}

\defSub{mu}{The lognormal prior mean (mu) parameter}
\defSub{cv}{The lognormal CV parameter}

\commandlabsubarg{additional\_prior}{type}{Uniform\_Log}


\commandlabsubarg{additional\_prior}{type}{Vector\_Average}

\defSub{method}{Which calculation method to use: k, l, or m}
\defSub{k}{The k value to use in the calculation}
\defSub{multiplier}{Multiplier for the penalty amount}

\commandlabsubarg{additional\_prior}{type}{Vector\_Smoothing}

\defSub{log\_scale}{Should the sums of squares be calculated on the log scale?}
\defSub{multiplier}{Multiply the penalty by this factor}
\defSub{lower\_bound}{The first element to apply the penalty to in the vector}
\defSub{upper\_bound}{The last element to apply the penalty to in the vector}
\defSub{r}{The rth difference that the penalty is applied to}

\defComLab{ageing\_error}{Define an object of type \emph{Ageing\_Error}}

\defSub{label}{The label of the ageing error}
\defSub{type}{The type of ageing error}

\commandlabsubarg{ageing\_error}{type}{Data}

\defSub{data}{The table of data specifying the ageing misclassification matrix}
\defSub{table data}{The table of data specifying the ageing misclassification matrix}

\commandlabsubarg{ageing\_error}{type}{None}


\commandlabsubarg{ageing\_error}{type}{Normal}

\defSub{cv}{CV of the misclassification matrix}
\defSub{k}{k defines the minimum age of individuals which can be misclassified, i.e., individuals of age less than k have no ageing error}

\commandlabsubarg{ageing\_error}{type}{Off\_By\_One}

\defSub{p1}{The proportion misclassified as one year younger, e.g., the proportion of age k individuals that were misclassified as age (k-1)}
\defSub{p2}{The proportion misclassified as one year older, e.g., the proportion of age k individuals that were misclassified as age (k+1)}
\defSub{k}{The minimum age of animals which can be misclassified, i.e., animals of age less than k are assumed to be correctly classified}

\defComLab{age\_length}{Define an object of type \emph{Age\_Length}}

\defSub{label}{The label of the age length relationship}
\defSub{type}{The type of age length relationship}
\defSub{time\_step\_proportions}{The fraction of the year applied in each time step that is added to the age for the purposes of evaluating the length, i.e., a value of 0.5 for a time step will evaluate the length of individuals at age+0.5 in that time step}
\defSub{distribution}{The assumed distribution for the growth curve}
\defSub{cv\_first}{The CV for the first age class}
\defSub{cv\_last}{The CV for last age class}
\defSub{compatibility\_option}{Backwards compatibility option: either casal2 (the default) or casal to use the (less accurate) cumulative normal function from CASAL}
\defSub{by\_length}{Specifies if the linear interpolation of CVs is a linear function of mean length at age, or at age}

\commandlabsubarg{age\_length}{type}{Data}

\defSub{external\_gaps}{The method to use for external data gaps}
\defSub{internal\_gaps}{The method to use for internal data gaps}
\defSub{length\_weight}{The label from an associated length-weight block}
\defSub{time\_step\_measurements\_were\_made}{The time step label for which size-at-age data are provided}
\defSub{table}{The table of data specifying the length at age values}

\commandlabsubarg{age\_length}{type}{None}


\commandlabsubarg{age\_length}{type}{Schnute}

\defSub{y1}{The $y\_1$ parameter}
\defSub{y2}{The $y\_2$ parameter}
\defSub{tau1}{The $\tau\_1$ parameter}
\defSub{tau2}{The $\tau\_2$ parameter}
\defSub{a}{The $a$ parameter}
\defSub{b}{The $b$ parameter}
\defSub{length\_weight}{The label of the associated length-weight relationship}

\commandlabsubarg{age\_length}{type}{Von\_Bertalanffy}

\defSub{linf}{The $L\_{infinity}$ parameter}
\defSub{k}{The $k$ parameter}
\defSub{t0}{The $t\_0$ parameter}
\defSub{length\_weight}{The label of the associated length-weight relationship}

\defComLab{age\_weight}{Define an object of type \emph{Age\_Weight}}

\defSub{label}{Label of the age weight relationship}
\defSub{type}{The type of age weight}

\commandlabsubarg{age\_weight}{type}{Data}

\defSub{equilibrium\_method}{If used in an SSB calculation, what is the method to calculate equilibrium SSB}
\defSub{units}{The units of measure (grams, kilograms (kgs), or tonnes)}
\defSub{table}{The table of data specifying the age at weight values}

\commandlabsubarg{age\_weight}{type}{None}


\defComLab{assert}{Define an object of type \emph{Assert}}

\defSub{label}{The label for the assert}
\defSub{type}{The type of the assert}

\commandlabsubarg{assert}{type}{Addressable}

\defSub{parameter}{The addressable to check}
\defSub{years}{The years to check addressable}
\defSub{time\_step}{The time step to execute after}
\defSub{values}{The values to check against the addressable}
\defSub{tolerance}{The tolerance of the difference test}
\defSub{error\_type}{Report assert failures as either an error or warning}

\commandlabsubarg{assert}{type}{Objective\_Function}

\defSub{value}{Expected value of the objective function}
\defSub{tolerance}{The tolerance of the difference test}
\defSub{error\_type}{Report assert failures as either an error or warning}

\commandlabsubarg{assert}{type}{Partition}

\defSub{category}{Category to check population values for}
\defSub{values}{Values expected in the partition}
\defSub{tolerance}{The tolerance of the difference test}
\defSub{error\_type}{Report assert failures as either an error or warning}

\defComLab{catchability}{Define an object of type \emph{Catchability}}

\defSub{label}{Label of the catchability}
\defSub{type}{The type of catchability}

\commandlabsubarg{catchability}{type}{Free}

\defSub{q}{The value of the catchability}

\commandlabsubarg{catchability}{type}{Nuisance}

\defSub{lower\_bound}{The upper bound for nuisance catchability}
\defSub{upper\_bound}{The lower bound for nuisance catchability}
\defSub{q}{The value of the catchability}

\defComLab{categories}{Define an object of type \emph{Categories}}

\defSub{format}{The format that the category names use}
\defSub{names}{The names of the categories}
\defSub{age\_lengths}{The age-length relationship labels for each category}
\defSub{age\_weight}{The age-weight relationships labels for each category}

\defComLab{derived\_quantity}{Define an object of type \emph{Derived\_Quantity}}

\defSub{label}{The label of the derived quantity}
\defSub{type}{The type of derived quantity}
\defSub{time\_step}{The time step in which to calculate the derived quantity}
\defSub{categories}{The list of categories to use when calculating the derived quantity}
\defSub{selectivities}{The list of selectivities to use when calculating the derived quantity}
\defSub{time\_step\_proportion}{The proportion through the mortality block of the time step when the derived quantity is calculated}
\defSub{time\_step\_proportion\_method}{The method for interpolating for the proportion through the mortality block}

\commandlabsubarg{derived\_quantity}{type}{Abundance}


\commandlabsubarg{derived\_quantity}{type}{Biomass}

\defSub{age\_weight\_labels}{The labels for the age-weights that correspond to each category for the biomass calculation}

\defComLab{estimate}{Define an object of type \emph{Estimate}}

\defSub{label}{The label of the estimate}
\defSub{type}{The type of prior for the estimate}
\defSub{parameter}{The name of the parameter to estimate}
\defSub{lower\_bound}{The lower bound for the parameter}
\defSub{upper\_bound}{The upper bound for the parameter}
\defSub{same}{List of other parameters that are constrained to have the same value as this parameter}
\defSub{estimation\_phase}{The first estimation phase to allow this to be estimated}
\defSub{mcmc\_fixed}{Indicates if this parameter is estimated at the point estimate but fixed during MCMC estimation run}

\commandlabsubarg{estimate}{type}{Beta}

\defSub{mu}{Beta prior  mean (mu) parameter}
\defSub{sigma}{Beta prior standard deviation (sigma) parameter}
\defSub{a}{Beta prior lower bound of the range (A) parameter}
\defSub{b}{Beta prior upper bound of the range (B) parameter}

\commandlabsubarg{estimate}{type}{Lognormal}

\defSub{mu}{The lognormal prior mean (mu) parameter}
\defSub{cv}{The lognormal variance (cv) parameter}

\commandlabsubarg{estimate}{type}{Normal}

\defSub{mu}{The normal prior mean (mu) parameter}
\defSub{cv}{The normal standard deviation (sigma) parameter}

\commandlabsubarg{estimate}{type}{Normal\_By\_Stdev}

\defSub{mu}{The normal prior mean (mu) parameter}
\defSub{sigma}{The normal standard deviation (sigma) parameter}
\defSub{lognormal\_transformation}{Add a Jacobian if the derived outcome of the estimate is assumed to be lognormal, e.g., used for recruitment deviations in the recruitment process. See the User Manual for more information}

\commandlabsubarg{estimate}{type}{Normal\_Log}

\defSub{mu}{The normal-log prior mean (mu) parameter}
\defSub{sigma}{The normal-log prior standard deviation (sigma) parameter}

\commandlabsubarg{estimate}{type}{Uniform}


\commandlabsubarg{estimate}{type}{Uniform\_Log}


\defComLab{growth\_increment}{Define an object of type \emph{growth\_increment}}

\defSub{label}{The label of the growth increment model}
\defSub{type}{The type of growth increment model}
\defSub{time\_step\_proportions}{The proportion of annual increment to apply in each time-step. Must sum = 1.0}
\defSub{compatibility\_option}{Backwards compatibility option: either casal2 (the default) or casal to use the (less accurate) cumulative normal function from CASAL}
\defSub{distribution}{The assumed distribution for the growth curve}
\defSub{length\_weight}{The label from an associated length-weight block}
\defSub{cv}{The $cv$ for the growth model.}
\defSub{min\_sigma}{The minimum standard deviation for the growth model.}

\commandlabsubarg{growth\_increment}{type}{basic}

\defSub{g\_alpha}{The $g_{\alpha}$ parameter}
\defSub{g\_beta}{The $g_{\beta}$ parameter}
\defSub{l\_alpha}{The $l_{\alpha}$ parameter}
\defSub{l\_beta}{The $l_{\beta}$ parameter}

\commandlabsubarg{growth\_increment}{type}{exponential}

\defSub{g\_alpha}{The $g_{\alpha}$ parameter}
\defSub{g\_beta}{The $g_{\beta}$ parameter}
\defSub{l\_alpha}{The $l_{\alpha}$ parameter}
\defSub{l\_beta}{The $l_{\beta}$ parameter}

\commandlabsubarg{growth\_increment}{type}{none}


\defComLab{initialisation\_phase}{Define an object of type \emph{Initialisation\_Phase}}

\defSub{label}{The label of the initialisation phase}
\defSub{type}{The type of initialisation}

\commandlabsubarg{initialisation\_phase}{type}{Iterative}

\defSub{years}{The number of iterations (years) over which to execute this initialisation phase}
\defSub{insert\_processes}{The processes in the annual cycle to be include in this initialisation phase}
\defSub{exclude\_processes}{The processes in the annual cycle to be excluded from this initialisation phase}
\defSub{convergence\_years}{The iteration (year) when the test for convergence ($\lambda$) is evaluated}
\defSub{lambda}{The maximum value of the proportional summed difference between the partition at year and year+1 that indicates successful convergence}
\defSub{plus\_group}{Indicates if the convergence check applies only to the plus\_group of the partition}

\commandlabsubarg{initialisation\_phase}{type}{Cinitial}

\defSub{categories}{The list of categories for the Cinitial initialisation}
\defSub{table}{The table of data specifying the initial values by age}

\commandlabsubarg{initialisation\_phase}{type}{Derived}

\defSub{insert\_processes}{Specifies the additional processes that are not in the annual cycle, but should be be inserted into this initialisation phase}
\defSub{exclude\_processes}{Specifies the processes in the annual cycle that should be excluded from this initialisation phase}

\commandlabsubarg{initialisation\_phase}{type}{State\_Category\_By\_Age}

\defSub{categories}{The list of categories for the category state initialisation}
\defSub{min\_age}{The minimum age of values supplied in the definition of the category state}
\defSub{max\_age}{The maximum age of values supplied in the definition of the category state}
\defSub{table}{The table of data specifying the initial values by age}

\defComLab{length\_weight}{Define an object of type \emph{Length\_Weight}}

\defSub{label}{The label of the length-weight relationship}
\defSub{type}{The type of the length-weight relationship}

\commandlabsubarg{length\_weight}{type}{Basic}

\defSub{a}{The $a$ parameter ($W = a L^b$)}
\defSub{b}{The $b$ parameter ($W = a L^b$)}
\defSub{units}{The units for weights (grams, kilograms (kgs), or tonnes)}

\commandlabsubarg{length\_weight}{type}{None}


\defComLab{likelihood}{Define an object of type \emph{Likelihood}}


\commandlabsubarg{likelihood}{type}{Binomial}


\commandlabsubarg{likelihood}{type}{Binomial\_Approx}


\commandlabsubarg{likelihood}{type}{Dirichlet}


\commandlabsubarg{likelihood}{type}{Log\_Normal}


\commandlabsubarg{likelihood}{type}{Log\_Normal\_With\_Q}


\commandlabsubarg{likelihood}{type}{Multinomial}


\commandlabsubarg{likelihood}{type}{Normal}


\commandlabsubarg{likelihood}{type}{none}


\commandlabsubarg{likelihood}{type}{bernoulli}


\defComLab{mcmc}{Define an object of type \emph{MCMC}}

\defSub{label}{The label of the MCMC}
\defSub{type}{The MCMC method}
\defSub{length}{The number of iterations for the MCMC (including the burn in period)}
\defSub{burn\_in}{The number of iterations for the burn\_in period of the MCMC}
\defSub{active}{Indicates if this is the active MCMC algorithm}
\defSub{step\_size}{Initial step-size (as a multiplier of the approximate covariance matrix)}
\defSub{start}{The covariance multiplier for the starting point of the MCMC}
\defSub{adjust\_parameters\_at\_bounds}{Adjust the start point for parameters at bounds}
\defSub{keep}{The spacing between recorded values in the MCMC}
\defSub{max\_correlation}{The maximum absolute correlation in the covariance matrix of the proposal distribution}
\defSub{covariance\_adjustment\_method}{The method for adjusting small variances in the covariance proposal matrix}
\defSub{correlation\_adjustment\_diff}{The minimum non-zero variance times the range of the bounds in the covariance matrix of the proposal distribution}
\defSub{proposal\_distribution}{The shape of the proposal distribution (either the t or the normal distribution)}
\defSub{df}{The degrees of freedom of the multivariate t proposal distribution}
\defSub{adapt\_stepsize\_at}{The iteration numbers in which to check and resize the MCMC stepsize}
\defSub{adapt\_stepsize\_method}{The method to use to adapt the step size}
\defSub{adapt\_covariance\_matrix\_at}{The iteration number in which to adapt the covariance matrix}

\commandlabsubarg{mcmc}{type}{Hamiltonian}

\defSub{leapfrog\_steps}{Number of leapfrog steps}
\defSub{leapfrog\_delta}{Amount to leapfrog per step}
\defSub{gradient\_step\_size}{Step size to use when calculating gradient}

\commandlabsubarg{mcmc}{type}{Random\_Walk}


\defComLab{minimiser}{Define an object of type \emph{Minimiser}}

\defSub{label}{The minimiser label}
\defSub{type}{The type of minimiser to use}
\defSub{active}{Indicates if this minimiser is active}
\defSub{covariance}{Indicates if a covariance matrix should be generated}

\commandlabsubarg{minimiser}{type}{ADOLC}

\defSub{iterations}{The maximum number of iterations}
\defSub{evaluations}{The maximum number of evaluations}
\defSub{tolerance}{The tolerance of the gradient for convergence}
\defSub{step\_size}{The minimum step size before minimisation fails}
\defSub{parameter\_transformation}{The choice of parametrisation used to scale the parameters for ADOLC}

\commandlabsubarg{minimiser}{type}{Betadiff}

\defSub{iterations}{The maximum number of iterations}
\defSub{evaluations}{The maximum number of evaluations}
\defSub{tolerance}{The tolerance of the gradient for convergence}

\commandlabsubarg{minimiser}{type}{de\_solver}

\defSub{population\_size}{The number of candidate solutions to have in the population}
\defSub{crossover\_probability}{The minimiser's crossover probability}
\defSub{difference\_scale}{The scale to apply to new solutions when comparing candidates}
\defSub{max\_generations}{The maximum number of iterations to run}
\defSub{tolerance}{The total variance between the population and best candidate before acceptance}

\commandlabsubarg{minimiser}{type}{Deltadiff}

\defSub{iterations}{Maximum number of iterations}
\defSub{evaluations}{Maximum number of evaluations}
\defSub{tolerance}{Tolerance of the gradient for convergence}
\defSub{step\_size}{Minimum Step-size before minimisation fails}

\commandlabsubarg{minimiser}{type}{Numerical\_Differences}

\defSub{iterations}{The maximum number of iterations}
\defSub{evaluations}{The maximum number of evaluations}
\defSub{tolerance}{The tolerance of the gradient for convergence}
\defSub{step\_size}{The minimum step size before minimisation fails}

\defComLab{model}{Define an object of type \emph{Model}}

\defSub{type}{Type of model (either type=age or type=length)}
\defSub{base\_weight\_units}{Define the units for the base weight measurement unit (grams, kilograms (kgs), or tonnes). This will be the default unit of any weight input values}
\defSub{threads}{The number of threads to use for this model}

\commandlabsubarg{model}{type}{Age}

\defSub{start\_year}{Define the first year of the model, immediately following initialisation}
\defSub{final\_year}{Define the final year of the model, excluding years in the projection period}
\defSub{min\_age}{Minimum age of individuals in the population}
\defSub{max\_age}{Maximum age of individuals in the population}
\defSub{age\_plus}{Define the oldest age as a plus group}
\defSub{initialisation\_phases}{Define the labels of the phases of the initialisation}
\defSub{time\_steps}{Define the labels of the time steps, in the order that they are applied, to form the annual cycle}
\defSub{projection\_final\_year}{Define the final year of the model when running projections}
\defSub{length\_bins}{The minimum length in each length bin}
\defSub{length\_plus}{Specify whether there is a length plus group or not}
\defSub{length\_plus\_group}{Mean length of length plus group}

\defComLab{observation}{Define an object of type \emph{Observation}}

\defSub{label}{The label of the observation}
\defSub{type}{The type of observation}
\defSub{likelihood}{The type of likelihood to use}
\defSub{categories}{The category labels to use}
\defSub{delta}{The robustification value (delta) for the likelihood}
\defSub{simulation\_likelihood}{The simulation likelihood to use}
\defSub{likelihood\_multiplier}{The likelihood multiplier}
\defSub{error\_value\_multiplier}{The error value multiplier for likelihood}
\defSub{table}{The table of data specifying the observed values}
\defSub{table}{The table of data specifying the observed error values}

\commandlabsubarg{observation}{type}{Abundance}

\defSub{time\_step}{The label of the time step that the observation occurs in}
\defSub{catchability}{The label of the catchability coefficient (q)}
\defSub{selectivities}{The labels of the selectivities}
\defSub{process\_error}{The process error}
\defSub{years}{The years for which there are observations}
\defSub{table obs}{The table of data specifying the observed and error values}

\commandlabsubarg{observation}{type}{Biomass}

\defSub{time\_step}{The label of the time step that the observation occurs in}
\defSub{catchability}{The label of the catchability coefficient (q)}
\defSub{selectivities}{The labels of the selectivities}
\defSub{process\_error}{The process error}
\defSub{age\_weight\_labels}{The labels for the \command{$age\_weight$} block which corresponds to each category, to use the weight calculation method for biomass calculations)}
\defSub{years}{The years of the observed values}
\defSub{table obs}{The table of data specifying the observed and error values}

\commandlabsubarg{observation}{type}{Process\_Removals\_By\_Age}

\defSub{min\_age}{The minimum age}
\defSub{max\_age}{The maximum age}
\defSub{sum\_to\_one}{Scale year (row) observed values by the total so they sum to equal 1}
\defSub{simulated\_data\_sum\_to\_one}{Whether simulated data is discrete or scaled by totals to be proportions for each year}
\defSub{plus\_group}{Is the maximum age the age plus group}
\defSub{time\_step}{The label of time-step that the observation occurs in}
\defSub{years}{The years for which there are observations}
\defSub{process\_errors}{The process errors to use}
\defSub{ageing\_error}{The label of the ageing error to use}
\defSub{method\_of\_removal}{The label of the observed method of removals}
\defSub{mortality\_process}{The label of the mortality instantaneous process for the observation}
\defSub{table obs}{The table of data specifying the observed values}
\defSub{table error\_values}{The table of data specifying the error values}

\commandlabsubarg{observation}{type}{Process\_Removals\_By\_Age\_Retained}

\defSub{min\_age}{The minimum age}
\defSub{max\_age}{The maximum age}
\defSub{plus\_group}{Is the maximum age the age plus group?}
\defSub{time\_step}{The label of the time step that the observation occurs in}
\defSub{sum\_to\_one}{Scale the year (row) observed values by the total, so they sum to 1}
\defSub{simulated\_data\_sum\_to\_one}{Whether simulated data is discrete or scaled by totals to be proportions for each year}
\defSub{years}{The years for which there are observations}
\defSub{process\_errors}{The process errors to use}
\defSub{ageing\_error}{The label of the ageing error to use}
\defSub{method\_of\_removal}{The label of observed method of removals}
\defSub{mortality\_process}{The label of the mortality instantaneous process for the observation}
\defSub{table obs}{The table of data specifying the observed values}
\defSub{table error\_values}{The table of data specifying the error values}

\commandlabsubarg{observation}{type}{Process\_Removals\_By\_Age\_Retained\_Total}

\defSub{min\_age}{The minimum age}
\defSub{max\_age}{The maximum age}
\defSub{plus\_group}{Is the maximum age the age plus group?}
\defSub{time\_step}{The label of the time step that the observation occurs in}
\defSub{sum\_to\_one}{Scale the year (row) observed values by the total, so they sum to 1}
\defSub{simulated\_data\_sum\_to\_one}{Whether simulated data is discrete or scaled by totals to be proportions for each year}
\defSub{years}{The years for which there are observations}
\defSub{process\_errors}{The process errors to use}
\defSub{ageing\_error}{The label of the ageing error to use}
\defSub{method\_of\_removal}{The label of observed method of removals}
\defSub{mortality\_process}{The label of the mortality process for this observation}
\defSub{table obs}{The table of data specifying the observed values}
\defSub{table error\_values}{The table of data specifying the error values}

\commandlabsubarg{observation}{type}{Process\_Removals\_By\_Length}

\defSub{time\_step}{The time step to execute in}
\defSub{years}{The years for which there are observations}
\defSub{process\_errors}{The process errors to use}
\defSub{method\_of\_removal}{The label of observed method of removals}
\defSub{length\_bins}{The length bins}
\defSub{sum\_to\_one}{Scale the year (row) observed values by the total, so they sum to 1}
\defSub{simulated\_data\_sum\_to\_one}{Whether simulated data is discrete or scaled by totals to be proportions for each year}
\defSub{plus\_group}{Is the last length bin a plus group? (defaults to @model value)}
\defSub{mortality\_process}{The label of the mortality instantaneous process for the observation}
\defSub{table obs}{The table of data specifying the observed values}
\defSub{table error\_values}{The table of data specifying the error values}

\commandlabsubarg{observation}{type}{Process\_Removals\_By\_Length\_Retained}

\defSub{time\_step}{The time step to execute in}
\defSub{years}{The years for which there are observations}
\defSub{process\_errors}{The process errors to use}
\defSub{method\_of\_removal}{The label of observed method of removals}
\defSub{length\_bins}{The length bins}
\defSub{sum\_to\_one}{Scale the year (row) observed values by the total, so they sum to 1}
\defSub{simulated\_data\_sum\_to\_one}{Whether simulated data is discrete or scaled by totals to be proportions for each year}
\defSub{plus\_group}{Is the last length bin a plus group? (defaults to @model value)}
\defSub{mortality\_process}{The label of the mortality instantaneous process for the observation}
\defSub{table obs}{The table of data specifying the observed values}
\defSub{table error\_values}{The table of data specifying the error values}

\commandlabsubarg{observation}{type}{Process\_Removals\_By\_Length\_Retained\_Total}

\defSub{time\_step}{The time step to execute in}
\defSub{years}{The years for which there are observations}
\defSub{process\_errors}{The process errors to use}
\defSub{method\_of\_removal}{The label of observed method of removals}
\defSub{length\_bins}{The length bins}
\defSub{plus\_group}{Is the last length bin a plus group? (defaults to @model value)}
\defSub{sum\_to\_one}{Scale the year (row) observed values by the total, so they sum to 1}
\defSub{simulated\_data\_sum\_to\_one}{Whether simulated data is discrete or scaled by totals to be proportions for each year}
\defSub{mortality\_process}{The label of the mortality instantaneous process for the observation}
\defSub{table obs}{The table of data specifying the observed values}
\defSub{table error\_values}{The table of data specifying the error values}

\commandlabsubarg{observation}{type}{Process\_Removals\_By\_Weight}

\defSub{mortality\_process}{The label of the mortality instantaneous process for the observation}
\defSub{method\_of\_removal}{The label of observed method of removals}
\defSub{time\_step}{The time step to execute in}
\defSub{years}{The years for which there are observations}
\defSub{process\_errors}{The process errors to use}
\defSub{length\_weight\_cv}{The CV for the length-weight relationship}
\defSub{length\_weight\_distribution}{The distribution of the length-weight relationship}
\defSub{length\_bins}{The length bins}
\defSub{length\_bins\_n}{The average number in each length bin}
\defSub{units}{The units for the weight bins (grams, kilograms (kgs), or tonnes)}
\defSub{fishbox\_weight}{The target weight of each box}
\defSub{weight\_bins}{The weight bins}

\commandlabsubarg{observation}{type}{Proportions\_At\_Age}

\defSub{min\_age}{The minimum age}
\defSub{max\_age}{The maximum age}
\defSub{plus\_group}{Is the maximum age the age plus group?}
\defSub{time\_step}{The label of the time step that the observation occurs in}
\defSub{years}{The years of the observed values}
\defSub{selectivities}{The labels of the selectivities}
\defSub{process\_errors}{The process errors to use}
\defSub{ageing\_error}{The label of ageing error to use}
\defSub{sum\_to\_one}{Scale the year (row) observed values by the total, so they sum to 1}
\defSub{simulated\_data\_sum\_to\_one}{Whether simulated data is discrete or scaled by totals to be proportions for each year}
\defSub{table obs}{The table of data specifying the observed values}
\defSub{table error\_values}{The table of data specifying the error values}

\commandlabsubarg{observation}{type}{Proportions\_At\_Length}

\defSub{time\_step}{The label of the time step that the observation occurs in}
\defSub{years}{The years for which there are observations}
\defSub{selectivities}{The labels of the selectivities}
\defSub{process\_errors}{The process errors to use}
\defSub{length\_bins}{The length bins}
\defSub{plus\_group}{Is the last length bin a plus group?}
\defSub{sum\_to\_one}{Scale the year (row) observed values by the total, so they sum to 1}
\defSub{simulated\_data\_sum\_to\_one}{Whether simulated data is discrete or scaled by totals to be proportions for each year}
\defSub{table obs}{The table of data specifying the observed values}
\defSub{table error\_values}{The table of data specifying the error values}

\commandlabsubarg{observation}{type}{Proportions\_By\_Category}

\defSub{min\_age}{The minimum age}
\defSub{max\_age}{The maximum age}
\defSub{time\_step}{The label of the time step that the observation occurs in}
\defSub{plus\_group}{Use the age plus group?}
\defSub{years}{The years for which there are observations}
\defSub{selectivities}{The labels of the selectivities}
\defSub{categories2}{The target categories}
\defSub{selectivities2}{The target selectivities}

\commandlabsubarg{observation}{type}{Proportions\_Mature\_By\_Age}

\defSub{min\_age}{The minimum age}
\defSub{max\_age}{The maximum age}
\defSub{time\_step}{The label of time-step that the observation occurs in}
\defSub{plus\_group}{Use the age plus group?}
\defSub{years}{The years for which there are observations}
\defSub{ageing\_error}{The label of ageing error to use}
\defSub{total\_categories}{All category labels that were vulnerable to sampling at the time of this observation (not including the categories already given)}
\defSub{time\_step\_proportion}{The proportion through the mortality block of the time step when the observation is evaluated}

\commandlabsubarg{observation}{type}{Proportions\_Migrating}

\defSub{min\_age}{The minimum age}
\defSub{max\_age}{The maximum age}
\defSub{time\_step}{The label of the time step that the observation occurs in}
\defSub{plus\_group}{Is the maximum age the age plus group?}
\defSub{years}{The years for which there are observations}
\defSub{process\_errors}{The process errors to use}
\defSub{ageing\_error}{The label of the ageing error to use}
\defSub{process}{The process label}

\commandlabsubarg{observation}{type}{tag\_recapture\_by\_fishery}

\defSub{tagged\_categories}{The tagged categories that we want to generate recaptures for. Categories need to be space separated no use of the '+' category syntax.}
\defSub{time\_step}{The label of time-step that the observation occurs in}
\defSub{reporting\_rate}{The reporting rate for this observation}
\defSub{years}{The years for which there are observations}
\defSub{method\_of\_removal}{The label of the observed method of removals}
\defSub{mortality\_process}{The label of the mortality instantaneous process for the observation}
\defSub{table recaptured}{The table of recaptures in each year}

\commandlabsubarg{observation}{type}{Tag\_Recapture\_By\_Age}

\defSub{min\_age}{The minimum age}
\defSub{max\_age}{The maximum age}
\defSub{plus\_group}{Is the maximum age the age plus group?}
\defSub{years}{The years for which there are observations}
\defSub{time\_step}{The label of the time step that the observation occurs in}
\defSub{selectivities}{The labels of the selectivities used for untagged categories}
\defSub{tagged\_selectivities}{The labels of the tag category selectivities}
\defSub{tagged\_categories}{The  categories of tagged individuals}
\defSub{detection}{The probability of detecting a recaptured individual}
\defSub{dispersion}{The over-dispersion parameter, $\phi$}
\defSub{overlap\_scalar}{The overlap\_scalar parameter, $k$}
\defSub{time\_step\_proportion}{The proportion through the mortality block of the time step when the observation is evaluated}
\defSub{table recaptured}{The table of data specifying the recaptures}
\defSub{table scanned}{The table of data specifying the scanned fish}

\commandlabsubarg{observation}{type}{Tag\_Recapture\_By\_Length}

\defSub{years}{The years for which there are observations}
\defSub{time\_step}{The time step to execute in}
\defSub{length\_bins}{The length bins}
\defSub{selectivities}{The labels of the selectivities used for untagged categories}
\defSub{tagged\_selectivities}{The labels of the tag category selectivities}
\defSub{tagged\_categories}{The  categories of tagged individuals}
\defSub{detection}{The probability of detecting a recaptured individual}
\defSub{dispersion}{The over-dispersion parameter, $\phi$}
\defSub{overlap\_scalar}{The overlap\_scalar parameter, $k$}
\defSub{time\_step\_proportion}{The proportion through the mortality block of the time step when the observation is evaluated}
\defSub{table recaptured}{The table of data specifying the recaptures}
\defSub{table scanned}{The table of data specifying the scanned fish}

\commandlabsubarg{observation}{type}{age\_length}

\defSub{time\_step}{The label of the time step that the observation occurs in}
\defSub{selectivities}{The labels of the selectivities, one for each combined category}
\defSub{numerator\_categories}{A combined category label that defines categories that make up the numerator}
\defSub{year}{The year this observation occurred in}
\defSub{sample\_type}{The sample type}
\defSub{ages}{vector of observed ages}
\defSub{lengths}{vector of observed lengths}
\defSub{ageing\_error}{The label of ageing error to use}

\defComLab{observation}{Define an object of type \emph{Observation}}

\defSub{label}{The label of the observation}
\defSub{type}{The type of observation}
\defSub{likelihood}{The type of likelihood to use}
\defSub{categories}{The category labels to use}
\defSub{delta}{The robustification value (delta) for the likelihood}
\defSub{simulation\_likelihood}{The simulation likelihood to use}
\defSub{likelihood\_multiplier}{The likelihood multiplier}
\defSub{error\_value\_multiplier}{The error value multiplier for likelihood}

\commandlabsubarg{observation}{type}{Abundance}

\defSub{time\_step}{The label of the time step that the observation occurs in}
\defSub{catchability}{The label of the catchability coefficient (q)}
\defSub{selectivities}{The labels of the selectivities}
\defSub{process\_error}{The process error}
\defSub{years}{The years for which there are observations}
\defSub{table obs}{The table of data specifying the observed and error values}

\commandlabsubarg{observation}{type}{Biomass}

\defSub{time\_step}{The label of the time step that the observation occurs in}
\defSub{catchability}{The label of the catchability coefficient (q)}
\defSub{selectivities}{The labels of the selectivities}
\defSub{process\_error}{The process error}
\defSub{age\_weight\_labels}{The labels for the \command{$age\_weight$} block which corresponds to each category, to use the weight calculation method for biomass calculations)}
\defSub{years}{The years of the observed values}
\defSub{table obs}{The table of data specifying the observed and error values}

\commandlabsubarg{observation}{type}{Process\_Removals\_By\_Length}

\defSub{length\_bins}{Bespoke length bins for the observation. They need to be a subset of the \command{model} length bins.}
\defSub{sum\_to\_one}{Scale the year (row) observed values by the total, so they sum to 1}
\defSub{simulated\_data\_sum\_to\_one}{Whether simulated data are discrete or scaled by totals to be proportions for each year}
\defSub{plus\_group}{Is the maximum length bin a plus group}
\defSub{time\_step}{The label of time-step that the observation occurs in}
\defSub{years}{The years for which there are observations}
\defSub{process\_errors}{The label of process error to use}
\defSub{method\_of\_removal}{The label of the observed method of removals}
\defSub{mortality\_instantaneous\_process}{The label of the mortality instantaneous process for the observation}
\defSub{table obs}{The table of data specifying the observed values}
\defSub{table error\_values}{The table of data specifying the error values}

\commandlabsubarg{observation}{type}{Proportions\_At\_Length}

\defSub{length\_bins}{Bespoke length bins for the observation. They need to be a subset of the \command{model}{length\_bins}.}
\defSub{plus\_group}{Is the maximum length bin a plus group}
\defSub{time\_step}{The label of the time step that the observation occurs in}
\defSub{years}{The years of the observed values}
\defSub{selectivities}{The labels of the selectivities}
\defSub{process\_errors}{The process error}
\defSub{sum\_to\_one}{Scale the year (row) observed values by the total, so they sum to 1}
\defSub{simulated\_data\_sum\_to\_one}{Whether simulated data is discrete or scaled by totals to be proportions for each year}
\defSub{table obs}{The table of data specifying the observed values}
\defSub{table error\_values}{The table of data specifying the error values}

\commandlabsubarg{observation}{type}{Proportions\_By\_Category}

\defSub{length\_bins}{Bespoke length bins for the observation. They need to be a subset of the \command{model}{length\_bins}.}
\defSub{time\_step}{The label of the time step that the observation occurs in}
\defSub{plus\_group}{Use the age plus group?}
\defSub{years}{The years for which there are observations}
\defSub{selectivities}{The labels of the selectivities}
\defSub{total\_categories2}{Categories in the denominator}
\defSub{total\_selectivities}{Selectivities to apply to the total categories}

\commandlabsubarg{observation}{type}{Tag\_Recapture\_By\_Length}

\defSub{years}{The years for which there are observations}
\defSub{time\_step}{The time step to execute in}
\defSub{length\_bins}{The length bins}
\defSub{selectivities}{The labels of the selectivities used for untagged categories}
\defSub{tagged\_selectivities}{The labels of the tag category selectivities}
\defSub{detection}{The probability of detecting a recaptured individual}
\defSub{dispersion}{The overdispersion parameter (phi)}
\defSub{time\_step\_proportion}{The proportion through the mortality block of the time step when the observation is evaluated}
\defSub{table recaptured}{The table of data specifying the recaptures}
\defSub{table scanned}{The table of data specifying the scanned fish}

\commandlabsubarg{observation}{type}{tag\_recapture\_by\_length\_for\_growth}

\defSub{years}{The years for which there are observations}
\defSub{time\_step}{The time step to execute in}
\defSub{length\_bins}{The length bins}
\defSub{selectivities}{The labels of the selectivities used for untagged categories}
\defSub{tagged\_selectivities}{The labels of the tag category selectivities}
\defSub{detection}{The probability of detecting a recaptured individual}
\defSub{dispersion}{The overdispersion parameter (phi)}
\defSub{time\_step\_proportion}{The proportion through the mortality block of the time step when the observation is evaluated}
\defSub{table recaptured}{The table of data specifying the recaptures}

\defComLab{parameter\_transformation}{Define an object of type \emph{parameter\_transformation}}

\defSub{label}{Label for the transformation block}
\defSub{type}{The type of transformation}
\defSub{prior\_applies\_to\_restored\_parameters}{If the prior applies to the parameters (true) with jacobian (if it exists) or prior applies to transformed\_parameter (false) with no jacobian}
\defSub{parameters}{The label of the parameters used in the transformation}

\commandlabsubarg{parameter\_transformation}{type}{log}


\commandlabsubarg{parameter\_transformation}{type}{logistic}

\defSub{lower\_bound}{Lower bound for the transformation}
\defSub{upper\_bound}{Upper bound for the transformation}

\commandlabsubarg{parameter\_transformation}{type}{inverse}


\commandlabsubarg{parameter\_transformation}{type}{difference\_parameter}


\commandlabsubarg{parameter\_transformation}{type}{average\_difference}


\commandlabsubarg{parameter\_transformation}{type}{log\_sum}


\commandlabsubarg{parameter\_transformation}{type}{orthogonal}


\commandlabsubarg{parameter\_transformation}{type}{sum\_to\_one}


\commandlabsubarg{parameter\_transformation}{type}{simplex}

\defSub{sum\_to\_one}{Apply the sum\_to\_one constraint}

\commandlabsubarg{parameter\_transformation}{type}{sqrt}


\defComLab{penalty}{Define an object of type \emph{Penalty}}

\defSub{label}{The label of the penalty}
\defSub{type}{The type of penalty}

\commandlabsubarg{penalty}{type}{Process}

\defSub{multiplier}{The penalty multiplier}
\defSub{log\_scale}{Indicates if the sums of squares will be calculated on the log scale}

\defComLab{process}{Define an object of type \emph{Process}}

\defSub{label}{The label of the process}
\defSub{type}{The type of process}

\commandlabsubarg{process}{type}{Ageing}

\defSub{categories}{The labels of the categories to age}

\commandlabsubarg{process}{type}{Maturation}

\defSub{from}{The list of categories to mature from}
\defSub{to}{The list of categories to mature to}
\defSub{selectivities}{The list of selectivities to use for maturation}
\defSub{years}{The years to be associated with the maturity rates}
\defSub{rates}{The rates to mature for each year}

\commandlabsubarg{process}{type}{Mortality\_Constant\_Rate}

\defSub{categories}{The list of category labels}
\defSub{m}{The mortality rates}
\defSub{time\_step\_proportions}{The time step proportions for the mortality rates}
\defSub{relative\_m\_by\_age}{The list of mortality by age ogive labels for the categories}

\commandlabsubarg{process}{type}{Mortality\_Constant\_Exploitation}

\defSub{categories}{The list of category labels}
\defSub{u}{The exploitation rates}
\defSub{time\_step\_proportions}{The time step proportions for the exploitation rates}
\defSub{relative\_u\_by\_age}{The list of exploitation by age ogive labels for the categories}

\commandlabsubarg{process}{type}{Mortality\_Disease\_Rate}

\defSub{categories}{The list of category labels}
\defSub{disease\_mortality\_rate}{The disease mortality rates}
\defSub{year\_effects}{Annual deviations around the disease mortality rate}
\defSub{selectivities}{The list of selectivities}
\defSub{years}{Years in which to apply the disease mortality in}

\commandlabsubarg{process}{type}{Mortality\_Event}

\defSub{categories}{The categories}
\defSub{years}{The years in which to apply the mortality process}
\defSub{catches}{The number of removals (catches) to apply for each year}
\defSub{u\_max}{The maximum exploitation rate ($U\_{max}$)}
\defSub{selectivities}{The list of selectivities}
\defSub{penalty}{The label of the penalty to apply if the total number of removals cannot be taken}

\commandlabsubarg{process}{type}{Mortality\_Event\_Biomass}

\defSub{categories}{The category labels}
\defSub{selectivities}{The labels of the selectivities for each of the categories}
\defSub{years}{The years in which to apply the mortality process}
\defSub{catches}{The biomass of removals (catches) to apply for each year}
\defSub{u\_max}{The maximum exploitation rate ($U\_{max}$)}
\defSub{penalty}{The label of the penalty to apply if the total biomass of removals cannot be taken}

\commandlabsubarg{process}{type}{Mortality\_Holling\_Rate}

\defSub{prey\_categories}{The prey categories labels}
\defSub{predator\_categories}{The predator categories labels}
\defSub{is\_abundance}{Is vulnerable amount of prey and predator an abundance [true] or biomass [false]}
\defSub{a}{Parameter a}
\defSub{b}{Parameter b}
\defSub{x}{This parameter controls the functional form: Holling function type 2 (x=2) or 3 (x=3), or generalised (Michaelis Menten, x>=1)}
\defSub{u\_max}{The maximum exploitation rate ($U\_{max}$)}
\defSub{prey\_selectivities}{The selectivities for prey categories}
\defSub{predator\_selectivities}{The selectivities for predator categories}
\defSub{penalty}{The label of penalty}
\defSub{years}{The years in which to apply the mortality process}
\defSub{table}{The table of data specifying the predator selectivities}
\defSub{table}{The table of data specifying the prey selectivities}

\commandlabsubarg{process}{type}{mortality\_initialisation\_event}

\defSub{categories}{The categories}
\defSub{catch}{The number of removals (catches) to apply for each year}
\defSub{u\_max}{The maximum exploitation rate ($U\_{max}$)}
\defSub{selectivities}{The list of selectivities}
\defSub{penalty}{The label of the penalty to apply if the total number of removals cannot be taken}
\defSub{table}{The table of data specifying the catches for each fishery, the categories, years, and the $U_max$}

\commandlabsubarg{process}{type}{Mortality\_Initialisation\_Event\_Biomass}

\defSub{categories}{The categories}
\defSub{catch}{The number of removals (catches) to apply for each year}
\defSub{u\_max}{The maximum exploitation rate ($U\_{max}$)}
\defSub{selectivities}{The list of selectivities}
\defSub{penalty}{The label of the penalty to apply if the total number of removals cannot be taken}

\commandlabsubarg{process}{type}{mortality\_initialisation\_baranov}

\defSub{categories}{The categories}
\defSub{fishing\_mortality}{The fishing mortality to apply}
\defSub{selectivities}{The list of selectivities for each category}

\commandlabsubarg{process}{type}{mortality\_hybrid}

\defSub{categories}{The categories for to apply natural mortality to}
\defSub{m}{The natural mortality rates for each category}
\defSub{time\_step\_proportions}{The time step proportions for natural mortality}
\defSub{biomass}{Switch to indicate if the catches are biomasses or abundances}
\defSub{relative\_m\_by\_age}{The M-by-age selectivities to apply to each of the categories for natural mortality}
\defSub{max\_f}{Maximum \(F\) allowed}
\defSub{f\_iterations}{The number of tuning iterations to solve \(F\)}
\defSub{table catches}{The table of data specifying the catches for each fishery and year}
\defSub{table method}{The table of data specifying which fishery interacts with which category, selectivities, penalty, and time-step for each fishery and annual duration of the fishery.}

\commandlabsubarg{process}{type}{Mortality\_Instantaneous}

\defSub{categories}{The categories for instantaneous mortality}
\defSub{m}{The natural mortality rates for each category}
\defSub{time\_step\_proportions}{The time step proportions for natural mortality}
\defSub{biomass}{Switch to indicate if the catches are biomasses or abundances}
\defSub{relative\_m\_by\_age}{The M-by-age selectivities to apply to each of the categories for natural mortality}
\defSub{table catches}{The table of data specifying the catches for each fishery and year}
\defSub{table method}{The table of data specifying which fishery interacts with which category, selectivities, time-step, penalties and \(u_{max}\) for each fishery and year.}

\commandlabsubarg{process}{type}{Mortality\_Instantaneous\_Retained}

\defSub{categories}{The categories for instantaneous mortality}
\defSub{m}{The natural mortality rates for each category}
\defSub{time\_step\_proportions}{The time step proportions for natural mortality}
\defSub{relative\_m\_by\_age}{The M-by-age selectivities to apply on the categories for natural mortality}
\defSub{table}{The table of data specifying the catches for each fishery, the categories, years, and the $U_max$}

\commandlabsubarg{process}{type}{Mortality\_Prey\_Suitability}

\defSub{prey\_categories}{The prey categories labels}
\defSub{predator\_categories}{The predator categories labels}
\defSub{consumption\_rate}{The predator consumption rate}
\defSub{electivities}{The prey electivities}
\defSub{u\_max}{The maximum exploitation rate ($U\_{max}$)}
\defSub{prey\_selectivities}{The selectivities for prey categories}
\defSub{predator\_selectivities}{The selectivities for predator categories}
\defSub{penalty}{The label of the penalty}
\defSub{years}{The year that process occurs}

\commandlabsubarg{process}{type}{markovian\_movement}

\defSub{from}{The categories to transition from}
\defSub{to}{The categories to transition to}
\defSub{proportions}{The proportions to transition for each category}
\defSub{selectivities}{The selectivities to apply to each proportion}

\commandlabsubarg{process}{type}{null\_process}


\commandlabsubarg{process}{type}{Recruitment\_Beverton\_Holt}

\defSub{categories}{The category labels}
\defSub{r0}{R0, the mean recruitment used to scale annual recruits or initialise the model}
\defSub{b0}{B0, the SSB corresponding to R0, and used to scale annual recruits or initialise the model}
\defSub{proportions}{The proportion for each category}
\defSub{age}{The age at recruitment}
\defSub{ssb\_offset}{The spawning biomass year offset}
\defSub{steepness}{Steepness (h)}
\defSub{ssb}{The SSB label (i.e., the derived quantity label)}
\defSub{b0\_initialisation\_phase}{The initialisation phase label that B0 is from}
\defSub{ycs\_values}{Deprecated}
\defSub{ycs\_years}{Deprecated}
\defSub{standardise\_ycs\_years}{Deprecated}
\defSub{recruitment\_multipliers}{The recruitment values also termed year class strengths.}
\defSub{standardise\_years}{The years that are included for year class standardisation, they refer to the recruited year not spawning or year class year.}

\commandlabsubarg{process}{type}{Recruitment\_Beverton\_Holt\_With\_Deviations}

\defSub{categories}{The category labels}
\defSub{r0}{R0, the mean recruitment used to scale annual recruits or initialise the model}
\defSub{b0}{B0, the SSB corresponding to R0, and used to scale annual recruits or initialise the model}
\defSub{proportions}{The proportion for each category}
\defSub{age}{The age at recruitment}
\defSub{ssb\_offset}{The spawning biomass year offset}
\defSub{steepness}{Steepness (h)}
\defSub{ssb}{The SSB label (i.e., the derived quantity label)}
\defSub{sigma\_r}{The standard deviation of recruitment, $\sigma_R$}
\defSub{b\_max}{The maximum bias adjustment}
\defSub{last\_year\_with\_no\_bias}{The last year with no bias adjustment}
\defSub{first\_year\_with\_bias}{The first year with full bias adjustment}
\defSub{last\_year\_with\_bias}{The last year with full bias adjustment}
\defSub{first\_recent\_year\_with\_no\_bias}{The first recent year with no bias adjustment}
\defSub{b0\_initialisation\_phase}{The initialisation phase label that B0 is from}
\defSub{deviation\_values}{The recruitment deviation values}
\defSub{deviation\_years}{Deprecated}

\commandlabsubarg{process}{type}{Recruitment\_Ricker}

\defSub{categories}{The category labels}
\defSub{r0}{R0, the mean recruitment used to scale annual recruits or initialise the model}
\defSub{b0}{B0, the SSB corresponding to R0, and used to scale annual recruits or initialise the model}
\defSub{proportions}{The proportion for each category}
\defSub{age}{The age at recruitment}
\defSub{ssb\_offset}{The spawning biomass year offset}
\defSub{steepness}{Steepness (h)}
\defSub{ssb}{The SSB label (i.e., the derived quantity label)}
\defSub{b0\_initialisation\_phase}{The initialisation phase label that B0 is from}
\defSub{ycs\_values}{Deprecated}
\defSub{ycs\_years}{Deprecated}
\defSub{standardise\_ycs\_years}{Deprecated}
\defSub{recruitment\_multipliers}{The recruitment values also termed year class strengths.}
\defSub{standardise\_years}{The years that are included for year class standardisation, they refer to the recruited year not spawning or year class year.}

\commandlabsubarg{process}{type}{Recruitment\_Constant}

\defSub{categories}{The categories}
\defSub{proportions}{The proportion for each category}
\defSub{age}{The age at recruitment}
\defSub{r0}{R0, the recruitment used for annual recruits and initialise the model}

\commandlabsubarg{process}{type}{Survival\_Constant\_Rate}

\defSub{categories}{The list of categories}
\defSub{s}{The survival rates}
\defSub{time\_step\_proportions}{The time step proportions for the survival rate $S$}
\defSub{selectivities}{The selectivity labels for each category}

\commandlabsubarg{process}{type}{Tag\_By\_Age}

\defSub{from}{The categories that are selected for tagging (i.e, transition from)}
\defSub{to}{The categories that have tags (i.e., transition to)}
\defSub{min\_age}{The minimum age tagged}
\defSub{max\_age}{The maximum age tagged}
\defSub{penalty}{The penalty label}
\defSub{u\_max}{The maximum exploitation rate ($U\_{max}$)}
\defSub{years}{The years to execute the tagging in}
\defSub{initial\_mortality}{The initial mortality value (as a instantaneous proportion)}
\defSub{initial\_mortality\_selectivity}{The initial mortality selectivity label}
\defSub{selectivities}{The selectivity labels}
\defSub{n}{the total number of tagged fish}
\defSub{table}{The table of data specifying the either the numbers or proportions to tag from and to each category and year}
\defSub{table numbers}{The table of releases as numbers for the process}
\defSub{table proportions}{The table of releases as numbers for the process}
\defSub{tolerance}{Tolerance for checking the specified proportions sum to one}

\commandlabsubarg{process}{type}{Tag\_By\_Length}

\defSub{from}{The categories that are selected for tagging (i.e, transition from)}
\defSub{to}{The categories that have tags (i.e., transition to)}
\defSub{penalty}{The penalty label}
\defSub{u\_max}{The maximum exploitation rate ($U\_{max}$)}
\defSub{compatibility\_option}{Backwards compatibility option: either casal2 (the default) or casal. This affects the penalty and age-length calculations}
\defSub{years}{The years to execute the tagging events in}
\defSub{initial\_mortality}{The initial mortality value (as a proportion)}
\defSub{initial\_mortality\_selectivity}{The initial mortality selectivity label}
\defSub{selectivities}{The selectivity labels}
\defSub{n}{the total number of tagged fish}
\defSub{table}{The table of data specifying the either the numbers or proportions to tag from and to each category and year}
\defSub{table numbers}{The table of releases as numbers for the process}
\defSub{table proportions}{The table of releases as numbers for the process}
\defSub{tolerance}{Tolerance for checking the specified proportions sum to one}

\commandlabsubarg{process}{type}{Tag\_Loss}

\defSub{categories}{The list of categories}
\defSub{tag\_loss\_rate}{The instantaneous tag loss rates}
\defSub{time\_step\_proportions}{The time step proportions for tag loss}
\defSub{tag\_loss\_type}{The type of tag loss}
\defSub{selectivities}{The selectivities}
\defSub{year}{The year the first tagging release process was executed}

\commandlabsubarg{process}{type}{Tag\_Loss\_Empirical}

\defSub{categories}{The list of categories}
\defSub{tag\_loss\_rate}{The instantaneous tag loss rates}
\defSub{time\_step\_proportions}{The time step proportions for tag loss}
\defSub{selectivities}{The selectivities}
\defSub{year}{The year the first tagging release process was executed}
\defSub{years\_at\_liberty}{The years at liberty that the tag\_loss\_rate applies to}

\commandlabsubarg{process}{type}{Transition\_Category}

\defSub{from}{The categories to transition from}
\defSub{to}{The categories to transition to}
\defSub{proportions}{The proportions to transition for each category}
\defSub{selectivities}{The selectivities to apply to each proportion}
\defSub{include\_in\_mortality\_block}{Include this process within the mortality block}

\commandlabsubarg{process}{type}{Transition\_Category\_By\_Age}

\defSub{from}{The categories to transition from}
\defSub{to}{The categories to transition to}
\defSub{min\_age}{The minimum age to transition}
\defSub{max\_age}{The maximum age to transition}
\defSub{penalty}{The penalty label}
\defSub{u\_max}{The maximum exploitation rate ($U\_{max}$)}
\defSub{years}{The years to execute the transition in}
\defSub{table n}{The table of numbers at age to transition from and to each category}

\defComLab{process}{Define an object of type \emph{Process}}

\defSub{label}{The label of the process}
\defSub{type}{The type of process}

\commandlabsubarg{process}{type}{Mortality\_Constant\_Rate}

\defSub{categories}{The list of category labels}
\defSub{m}{The mortality rates}
\defSub{time\_step\_proportions}{The time step proportions for the mortality rates}
\defSub{relative\_m\_by\_length}{The list of mortality by length ogive labels for the categories}

\defSub{categories}{The list of category labels}
\defSub{u}{The exploitation rates}
\defSub{time\_step\_proportions}{The time step proportions for the exploitation rates}
\defSub{relative\_u\_by\_length}{The list of exploitation by length ogive labels for the categories}

\commandlabsubarg{process}{type}{Mortality\_Disease\_Rate}

\defSub{categories}{The list of category labels}
\defSub{disease\_mortality\_rate}{The disease mortality rates}
\defSub{year\_effects}{Annual deviations around the disease mortality rate}
\defSub{selectivities}{The list of selectivities}
\defSub{years}{Years in which to apply the disease mortality in}

\commandlabsubarg{process}{type}{Mortality\_Instantaneous}

\defSub{categories}{The categories for instantaneous mortality}
\defSub{m}{The natural mortality rates for each category}
\defSub{time\_step\_proportions}{The time step proportions for natural mortality}
\defSub{biomass}{Switch to indicate if the catches are biomasses or abundances}
\defSub{relative\_m\_by\_length}{The M-by-length selectivities to apply to each of the categories for natural mortality}
\defSub{table}{The table of data specifying the catches for each fishery, the categories, years, and the $U_max$}

\commandlabsubarg{process}{type}{null\_process}


\commandlabsubarg{process}{type}{Recruitment\_Beverton\_Holt}

\defSub{categories}{The category labels}
\defSub{r0}{R0, the mean recruitment used to scale annual recruits or initialise the model}
\defSub{b0}{B0, the SSB corresponding to R0, and used to scale annual recruits or initialise the model}
\defSub{proportions}{The proportion for each category}
\defSub{initial\_mean\_length}{The initial mean length at recruitment}
\defSub{initial\_length\_cv}{The initial length cv at recruitment}
\defSub{ssb\_offset}{The spawning biomass year offset}
\defSub{steepness}{Steepness (h)}
\defSub{ssb}{The SSB label (i.e., the derived quantity label)}
\defSub{b0\_initialisation\_phase}{The initialisation phase label that B0 is from}
\defSub{ycs\_values}{Deprecated}
\defSub{ycs\_years}{Deprecated}
\defSub{standardise\_ycs\_years}{Deprecated}
\defSub{recruitment\_multipliers}{The recruitment values also termed year class strengths.}
\defSub{standardise\_years}{The years that are included for year class standardisation, they refer to the recruited year not spawning or year class year.}

\commandlabsubarg{process}{type}{Recruitment\_Constant}

\defSub{categories}{The categories}
\defSub{proportions}{The proportion for each category}
\defSub{length\_bins}{The length bin that recruits are uniformly distributed over at the time of recruitment}
\defSub{r0}{R0, the recruitment used for annual recruits and initialise the model}

\commandlabsubarg{process}{type}{Tagging}

\defSub{from}{The categories that are selected for tagging (i.e, transition from)}
\defSub{to}{The categories that have tags (i.e., transition to)}
\defSub{penalty}{The penalty label}
\defSub{u\_max}{The maximum exploitation rate ($U\_{max}$)}
\defSub{compatibility\_option}{Backwards compatibility option: either casal2 (the default) or casal. This affects the penalty and age-length calculations}
\defSub{years}{The years to execute the tagging events in}
\defSub{initial\_mortality}{The initial mortality to apply to tags as a proportion}
\defSub{initial\_mortality\_selectivity}{The initial mortality selectivity label}
\defSub{selectivities}{The selectivity labels}
\defSub{n}{The total number of tags to apply}
\defSub{table}{The table of data specifying the either the numbers or proportions to tag from and to each category and year}
\defSub{table numbers}{The table of releases as numbers for the process}
\defSub{table proportions}{The table of releases as numbers for the process}
\defSub{tolerance}{Tolerance for checking the specified proportions sum to one}

\commandlabsubarg{process}{type}{Transition\_Category}

\defSub{from}{The categories to transition from}
\defSub{to}{The categories to transition to}
\defSub{proportions}{The proportions to transition for each category}
\defSub{selectivities}{The selectivities to apply to each proportion}
\defSub{include\_in\_mortality\_block}{Include this process within the mortality block}

\defComLab{profile}{Define an object of type \emph{Profile}}

\defSub{label}{The label of the profile}
\defSub{steps}{The number of steps between the lower and upper bound}
\defSub{lower\_bound}{The lower value of the range}
\defSub{upper\_bound}{The upper value of the range}
\defSub{parameter}{The free parameter to profile}
\defSub{same}{A free parameter that is constrained to have the same value as the parameter being profiled}

\defComLab{project}{Define an object of type \emph{Project}}

\defSub{label}{Label}
\defSub{type}{Type}
\defSub{years}{Years to recalculate the values}
\defSub{parameter}{Parameter to project}
\defSub{multiplier}{Multiplier that is applied to the projected value}

\commandlabsubarg{project}{type}{Constant}

\defSub{values}{The values to assign to the addressable}

\commandlabsubarg{project}{type}{multiple\_values}

\defSub{table values}{The table of data specifying the projected values to use for each row of the \texttt{-i} or \texttt{-I} file}

\commandlabsubarg{project}{type}{Empirical\_Sampling}

\defSub{start\_year}{The start year of sampling}
\defSub{final\_year}{The final year of sampling}

\commandlabsubarg{project}{type}{Lognormal}

\defSub{mean}{The mean of the lognormal process}
\defSub{sigma}{The standard deviation (sigma) of the lognormal sampling}

\commandlabsubarg{project}{type}{Lognormal\_Empirical}

\defSub{mean}{The mean of the Gaussian process}
\defSub{start\_year}{The start year of sampling}
\defSub{final\_year}{The final year of sampling}

\defComLab{report}{Define an object of type \emph{Report}}

\defSub{label}{The report label}
\defSub{type}{The report type}
\defSub{file\_name}{The file name. If not supplied, then output is directed to standard out}
\defSub{write\_mode}{Specify if any previous file with the same name should be overwritten, appended to, or is generated using a sequential suffix}
\defSub{format}{Report output format}

\commandlabsubarg{report}{type}{Default}

\defSub{catchabilities}{Report catchabilities}
\defSub{derived\_quantities}{Report derived quantities}
\defSub{observations}{Report observations}
\defSub{processes}{Report processes}
\defSub{projects}{Report projects}
\defSub{selectivities}{Report selectivities}
\defSub{time\_varying}{Report time-varying parameters}
\defSub{parameter\_transformations}{Report all parameter transformations}

\commandlabsubarg{report}{type}{Addressable}

\defSub{parameter}{The addressable parameter name}
\defSub{years}{Define the years that the report is generated for}
\defSub{time\_step}{Defines the time-step that the report applies to}

\commandlabsubarg{report}{type}{Age\_Length}

\defSub{time\_step}{The time step label}
\defSub{years}{The years for the report}
\defSub{age\_length}{The age-length label}

\commandlabsubarg{report}{type}{growth\_increment}

\defSub{time\_step}{The time step label}
\defSub{years}{The years for the report}
\defSub{growth\_increment}{The growth-increment label}

\commandlabsubarg{report}{type}{Ageing\_Error\_Matrix}

\defSub{ageing\_error}{The ageing error label}

\commandlabsubarg{report}{type}{Catchability}

\defSub{catchability}{The catchability label}

\commandlabsubarg{report}{type}{Correlation\_Matrix}


\commandlabsubarg{report}{type}{Covariance\_Matrix}


\commandlabsubarg{report}{type}{Derived\_Quantity}

\defSub{derived\_quantity}{The derived quantity label}

\commandlabsubarg{report}{type}{Equation\_Test}

\defSub{equation}{The equation to do a test run of}

\commandlabsubarg{report}{type}{Estimate\_Summary}


\commandlabsubarg{report}{type}{Estimate\_Value}


\commandlabsubarg{report}{type}{Estimation\_Result}


\commandlabsubarg{report}{type}{Hessian\_Matrix}


\commandlabsubarg{report}{type}{Initialisation}


\commandlabsubarg{report}{type}{Initialisation\_Partition}


\commandlabsubarg{report}{type}{MCMC\_Covariance}


\commandlabsubarg{report}{type}{MCMC\_Objective}

\defSub{file\_name}{The file name. If not supplied the default filename is used}
\defSub{write\_mode}{Has a different default to the rest of the reports.}

\commandlabsubarg{report}{type}{MCMC\_Sample}

\defSub{file\_name}{The file name. If not supplied the default filename is used}
\defSub{write\_mode}{Has a different default to the rest of the reports.}

\commandlabsubarg{report}{type}{Objective\_Function}


\commandlabsubarg{report}{type}{Observation}

\defSub{observation}{The observation label}
\defSub{normalised\_residuals}{Print Normalised Residuals?}
\defSub{pearsons\_residuals}{Print Pearsons Residuals?}

\commandlabsubarg{report}{type}{Output\_Parameters}


\commandlabsubarg{report}{type}{parameter\_transformation}

\defSub{parameter\_transformation}{label of parameter transformation block}

\commandlabsubarg{report}{type}{Partition}

\defSub{time\_step}{Time Step label}
\defSub{years}{Years}

\commandlabsubarg{report}{type}{Partition\_Biomass}

\defSub{time\_step}{The time step label}
\defSub{years}{The years for the report}

\commandlabsubarg{report}{type}{Process}

\defSub{process}{The process label that is reported}

\commandlabsubarg{report}{type}{Profile}


\commandlabsubarg{report}{type}{Project}

\defSub{project}{The project label that is reported}

\commandlabsubarg{report}{type}{Random\_Number\_Seed}


\commandlabsubarg{report}{type}{Selectivity}

\defSub{selectivity}{Selectivity name}
\defSub{length\_values}{Length bins for reporting if a length-based selectivity in an age-based model}

\commandlabsubarg{report}{type}{selectivity\_by\_year}

\defSub{selectivity}{Selectivity name}
\defSub{years}{years to report the selectivity in}
\defSub{time\_step}{Time step label}

\commandlabsubarg{report}{type}{Simulated\_Observation}

\defSub{observation}{The observation label}

\commandlabsubarg{report}{type}{Time\_Varying}

\defSub{time\_varying}{The time varying label that is reported}

\defComLab{selectivity}{Define an object of type \emph{Selectivity}}

\defSub{label}{The label for the selectivity}
\defSub{type}{The type of selectivity}
\defSub{length\_based}{Is the selectivity length based?}
\defSub{intervals}{The number of quantiles to evaluate a length-based selectivity over the age-length distribution}
\defSub{values}{}
\defSub{length\_values}{}

\commandlabsubarg{selectivity}{type}{All\_Values}

\defSub{v}{The v parameter}

\commandlabsubarg{selectivity}{type}{All\_Values\_Bounded}

\defSub{l}{The low value (L)}
\defSub{h}{The high value (H)}
\defSub{v}{The v parameter}

\commandlabsubarg{selectivity}{type}{Constant}

\defSub{a}{The $a$ value in $ax^b + c$}
\defSub{b}{The $b$ value in $ax^b + c$}
\defSub{c}{The $c$ value in $ax^b + c$}
\defSub{beta}{The minimum age/length for which the selectivity applies}

\commandlabsubarg{selectivity}{type}{Double\_Exponential}

\defSub{x0}{The $x0$ parameter}
\defSub{x1}{The $x1$ parameter}
\defSub{x2}{The $x2$ parameter}
\defSub{y0}{The $y0$ parameter}
\defSub{y1}{The $y1$ parameter}
\defSub{y2}{The $y2$ parameter}
\defSub{alpha}{The maximum value of the selectivity}
\defSub{beta}{The minimum age/length for which the selectivity applies}

\commandlabsubarg{selectivity}{type}{Double\_Normal}

\defSub{mu}{The mean ($\mu$}
\defSub{sigma\_l}{The left-hand variance (sigma\_l) parameter}
\defSub{sigma\_r}{The right-hand variance (sigma\_r) parameter}
\defSub{alpha}{The maximum value of the selectivity}
\defSub{beta}{The minimum age/length for which the selectivity applies}

\commandlabsubarg{selectivity}{type}{Double\_Normal\_Plateau}

\defSub{a1}{The a1 ($a1$}
\defSub{a2}{The a2 ($a2$}
\defSub{sigma\_l}{The left-hand variance (sigma\_l) parameter}
\defSub{sigma\_r}{The right-hand variance (sigma\_r) parameter}
\defSub{alpha}{The maximum value of the selectivity}
\defSub{beta}{The minimum age/length for which the selectivity applies}

\commandlabsubarg{selectivity}{type}{Double\_Normal\_Stock\_Synthesis}

\defSub{peak}{Age or length of plateau (max selectivity)}
\defSub{y0}{Transformed selectivity for the first age or length bin}
\defSub{y1}{Transformed selectivity for the last age or length bins}
\defSub{descending}{The shape of descending limb in either ages or lengths}
\defSub{ascending}{The shape of ascending limb in either ages or lengths}
\defSub{width}{width of plateau how many ages or lengths are in the plateau}
\defSub{l}{min age or first length bin}
\defSub{l}{max age or last length bin}
\defSub{alpha}{The maximum value of the selectivity}

\commandlabsubarg{selectivity}{type}{Increasing}

\defSub{l}{The low value (L)}
\defSub{h}{The high value (H)}
\defSub{v}{The v parameter}
\defSub{alpha}{The maximum value of the selectivity }

\commandlabsubarg{selectivity}{type}{Inverse\_Logistic}

\defSub{a50}{The age or length where the selectivity is \(50\%\)}
\defSub{ato95}{The age or length between \(50\%\) and \(95\%\) selective}
\defSub{alpha}{The maximum value of the selectivity }
\defSub{beta}{The minimum age/length for which the selectivity applies}

\commandlabsubarg{selectivity}{type}{Knife\_Edge}

\defSub{e}{The edge value}
\defSub{alpha}{The maximum value of the selectivity }

\commandlabsubarg{selectivity}{type}{Logistic}

\defSub{a50}{The age or length where the selectivity is \(50\%\)}
\defSub{ato95}{The age or length between \(50\%\) and \(95\%\) selective}
\defSub{alpha}{The maximum value of the selectivity}
\defSub{beta}{The minimum age/length for which the selectivity applies}

\commandlabsubarg{selectivity}{type}{Logistic\_Producing}

\defSub{l}{The low value (L)}
\defSub{h}{The high value (H)}
\defSub{a50}{The a50 parameter}
\defSub{ato95}{the ato95 parameter}
\defSub{alpha}{The maximum value of the selectivity}

\commandlabsubarg{selectivity}{type}{compound\_left}

\defSub{a50}{The a50 ($a50$}
\defSub{ato95}{The age or length between \(50\%\) and \(95\%\) selective}
\defSub{a\_min}{The (a\_min) parameter}
\defSub{left\_mean}{The left\_mean parameter}
\defSub{sigma}{The sigma parameter}

\commandlabsubarg{selectivity}{type}{compound\_right}

\defSub{a50}{The a50 ($a50$}
\defSub{ato95}{The age or length between \(50\%\) and \(95\%\) selective}
\defSub{a\_min}{The (a\_min) parameter}
\defSub{left\_mean}{The left\_mean parameter}
\defSub{to\_right\_mean}{The to\_right\_mean parameter}
\defSub{sigma}{The sigma parameter}

\commandlabsubarg{selectivity}{type}{compound\_middle}

\defSub{a50}{The a50 ($a50$}
\defSub{ato95}{The age or length between \(50\%\) and \(95\%\) selective}
\defSub{a\_min}{The (a\_min) parameter}
\defSub{left\_mean}{The left\_mean parameter}
\defSub{to\_right\_mean}{The to\_right\_mean parameter}
\defSub{sigma}{The sigma parameter}

\commandlabsubarg{selectivity}{type}{compound\_all}

\defSub{a50}{The a50 ($a50$}
\defSub{ato95}{The age or length between \(50\%\) and \(95\%\) selective}
\defSub{a\_min}{The (a\_min) parameter}
\defSub{sigma}{The sigma parameter}

\commandlabsubarg{Selectivity}{type}{multi\_selectivity}

\defSub{years}{The years for which we want to apply the corresponding selectivity in}
\defSub{selectivity\_labels}{The labels of the selectivities, one for each year}
\defSub{default\_selectivity}{The selectivity used in missing years}
\defSub{projection\_selectivity}{The selectivity used in missing years in projections}

\defComLab{simulate}{Define an object of type \emph{Simulate}}

\defSub{label}{The label for the simulation command}
\defSub{type}{The type of simulation}
\defSub{years}{The years to simulate values for}
\defSub{parameter}{The parameter to generate simulated values for}

\commandlabsubarg{simulate}{type}{Constant}

\defSub{value}{The value to assign to the parameter in the simulations}

\defComLab{time\_step}{Define an object of type \emph{Time\_Step}}

\defSub{label}{The label of the time step}
\defSub{processes}{The labels of the processes that occur in this time step, in the order that they occur}

\defComLab{time\_varying}{Define an object of type \emph{Time\_Varying}}

\defSub{label}{The label of the time-varying object}
\defSub{type}{The type of the time-varying object}
\defSub{parameter}{The name of the parameter to vary in each year}
\defSub{years}{The years in which to vary the parameter}

\commandlabsubarg{time\_varying}{type}{Annual\_Shift}

\defSub{a}{Parameter A}
\defSub{b}{Parameter B}
\defSub{c}{Parameter C}
\defSub{scaling\_years}{The scaling years}
\defSub{values}{The values}

\commandlabsubarg{time\_varying}{type}{Constant}

\defSub{values}{The value to assign to addressable}

\commandlabsubarg{time\_varying}{type}{Exogenous}

\defSub{a}{The shift parameter}
\defSub{exogenous\_variable}{The values of exogenous variable for each year}

\commandlabsubarg{time\_varying}{type}{Linear}

\defSub{slope}{The slope of the linear trend (i.e., the additive amount per year)}
\defSub{intercept}{The intercept of the linear trend (, i.e. the value in the first year)}

\commandlabsubarg{time\_varying}{type}{Random\_Draw}

\defSub{mean}{The mean ($\mu$) of the random draw distribution}
\defSub{sigma}{The standard deviation ($\sigma$)  of the random draw distribution}
\defSub{lower\_bound}{The lower bound for the random draw}
\defSub{upper\_bound}{The upper bound for the random draw}
\defSub{distribution}{The distribution type}

\commandlabsubarg{time\_varying}{type}{Random\_Walk}

\defSub{mean}{The mean ($\mu$) of the random walk distribution}
\defSub{sigma}{The standard deviation ($\sigma$)  of the random walk distribution}
\defSub{lower\_bound}{The lower bound for the random walk}
\defSub{upper\_bound}{The upper bound for the random walk}
\defSub{rho}{The autocorrelation parameter ($\rho$)  of the random walk distribution}
\defSub{distribution}{The distribution type}

\normalsize


\let\stdsection\section % Force index to be a numbered section
\def\section*#1{\stdsection{#1}}
\printindex % Insert index
\let\section\stdsection

% Appendices
\clearemptydoublepage{}
\begin{appendices}
	\section{\I{Compiling \CNAME}}\label{sec:build_environment}

This section describes how to set up the environment on your local machine that will allow you to build and compile \CNAME. The build environment can be on either Microsoft Windows or Linux systems. At present the \CNAME\ build system supports Microsoft Windows 10+ and Linux (with GCC/G++ 4.9.0+). Apple OSX or other platforms are not currently supported.

\subsection{Overview}

The build system is made up of a collection of Python scripts that do the various tasks. These are located in \path{CASAL2/BuildSystem/buildtools/classes/}. Each Python script has its own set of functionality and undertakes one set of actions.

The top level of the build system can be found at \path{CASAL2/BuildSystem/}. In this directory you can run \texttt{doBuild.bat help} from a command terminal in Microsoft Windows systems or \texttt{./doBuild.sh help} from a terminal in Linux systems.

The script will take one or two parameters depending on what style of build you want to undertake. These commands allow the building of various stand-alone binaries, shared libraries, and the documentation. Note that you will need additional software installed on your system in order to build \CNAME.  The software requirements are described below.

A summary of all of the doBuild arguments can be found using the command \texttt{doBuild help} in the BuildSystem directory.

The arguments to doBuild are:

Usage: doBuild $<$build\_target$>$ $<$argument$>$
\begin{description}
  \item{\texttt{help}} Print out the doBuild help (this output)
  \item{\texttt{check}} Do a check of the build system
  \item{\texttt{clean}} Remove debug/release build files
  \item{\texttt{clean\_all}} Remove all build files and ALL prebuilt binaries
  \item{\texttt{version}} Build the current version files for C++, R, and LaTeX
\end{description}

Build required libraries (DLLs/shared objects for Casal2)
\begin{description}
  \item{\texttt{thirdparty}} Build the third party libraries
  \begin{description}
    \item{$<$option$>$} Optionally specify the target third party library to build, either adolc or betadiff (default is none)
  \end{description}
\end{description}

Build development and test versions (for development builds only)
\begin{description}
\item{\texttt{release}} Build stand-alone release executable
  \begin{description}
	\item{$<$option$>$} Optionally specify the target third party library to build, either adolc or betadiff (default is none)
  \end{description}
  \item{\texttt{debug}} Build stand-alone debug executable
  \begin{description}
	\item{$<$option$>$} Optionally specify the target third party library to build, either adolc or betadiff (default is none)
  \end{description}
  \item{\texttt{test}} Build stand-alone unit tests executable
  \item{\texttt{unittests}} Run the unit tests (requires that `test' has been built)
  \item{\texttt{modelrunner}} Run the test case models
\end{description}

Build the Casal2 end-user application
\begin{description}
  \item{\texttt{library}} Build shared library for use by front end application
  \begin{description}
    \item{$<$argument$>$} Required argument to specify the target library to build: release, adolc, betadiff, or test
  \end{description}
  \item{\texttt{frontend}} Build \CNAME\ front end application
\end{description}

Create the archive, R Library, documentation, and the installers
\begin{description}
  \item{\texttt{documentation}} Build the \CNAME\ user manuals
  \item{\texttt{rlibrary}} Create the R library
  \item{\texttt{archive}} Create a zipped archive of the \CNAME\ application.
  \begin{description}
     \item{$<$true$>$} if specified build skips everything but frontend
  \end{description}
  \item{\texttt{installer}} Create the Microsoft Windows installer package
  \item{\texttt{deb}} Create Linux Debian installer
\end{description}

The outputs from the build system commands will be placed in sub-folders of \path{CASAL2/BuildSystem/bin/<operating system>/<build_type>}

For example:

\path{CASAL2/BuildSystem/windows/debug}

\path{CASAL2/BuildSystem/windows/library_release}

\path{CASAL2/BuildSystem/windows/thirdparty/}

\path{CASAL2/BuildSystem/linux/library_release}

 The files \texttt{Casal2\_build.bat} for Windows and \texttt{Casal2\_build.sh} for Linux in the root folder contain all the calls in the correct order of \texttt{doBuild} required to successfully build \CNAME, the documentation, the Windows installer (Windows) or the Debian installer (Linux), the R-Libraries, and run all the test cases and unit tests.

\subsection{Building on Microsoft Windows}

\subsubsection{Prerequisite software}

The building of \CNAME\ requires additional build tools and software, including Python, git version control, GCC compiler, LaTeX compiler, and a Windows package builder. \CNAME\ can require specific implementations of these packages and versions in order to build without modifying the build scripts.

\paragraph*{C++ and Fortran compiler}

Source: tdm-gcc (MinGW-w64) from \url{https://jmeubank.github.io/tdm-gcc/}.

\CNAME\ is designed to compile under GCC on Microsoft Windows and Linux. While it may be possible to build the package using different compilers, the \CNAME\ Development Team does provide any assistance or recommendations. We recommend using 64-bit TDM-GCC with a version of at least 10.3.0. Ensure you have the "fortran" and "openmp" options installed as a part of the "gcc" install option drop-down tick boxes as these are required. For example, from  \url{https://jmeubank.github.io/tdm-gcc/articles/2021-05/10.3.0-release}, select the 64+32-bit MinGW-w64 edition, then select the Custom install and tick all boxes. 

Note that a common error that can be made is having a different GCC compiler in your path when attempting to compile. For example, \texttt{rtools} includes a version of the GCC compiler. We recommend removing these from your path prior to compiling.

\paragraph*{GIT version control}

Source: Command line GIT from \url{https://www.git-scm.com/downloads}.

\CNAME\ automatically adds a version details to its files and any output based on the GIT version of the latest commit to its repository. This includes the name of source repository that was used. The command line version of GIT is used  to generate the version details.

\paragraph*{MiKTeX LaTeX Processor}

Source: Portable version from \url{http://www.miktex.org/portable}.

The main user documentation for \CNAME\ is a PDF document generated from LaTeX. The LaTeX syntax sections of the documentation are generated, in part, directly from the code. In order to generate the user documentation, you will need the MiKTeX LaTeX compiler.

A number of additional LaTeX styles are used --- these will usually be identified doing the doBuild process and can be installed as required. 

\paragraph*{7-Zip}

Source: 7-Zip from \url{http://www.7-zip.org/download.html}.

The build system calls \texttt{7zip.exe} to unzip files in the build system; it is advised to have this in the path.

\paragraph*{Python}

Source: Python3 from \url{https://www.python.org/downloads/windows/}

Python is used to run the build scripts and set the required environment variables required to build \CNAME. 

\paragraph*{Python modules}

There are a number of Python3 modules that are required to build \CNAME. These can be installed with \texttt{python -m pip install \emph{module-name}}. For example, You may need to install \texttt{datetime}, \texttt{re}, and \texttt{distutils} Python modules. 

\paragraph*{Inno setup installer build (optional)}

Source: Inno Setup 5 from \url{http://www.jrsoftware.org/isdl.php}

If you wish to build a Microsoft Windows compatible Installer for \CNAME\ (recommended) then you will need the  `Inno Setup 5' application installed on the machine. The installation path must be \path{C:\Program Files (x86)\Inno Setup 5\} in order for the build scripts to find and use it.

\subsubsection{Pre-build requirements}

Prior to building \CNAME\ you will need to ensure you have both G++ and GIT in your path. You can check both of these by typing the following commands and checking that they return the correct version number:

\texttt{g++ -{}-version}

\texttt{git -{}-version}

This also allows you to check that there are no alternative versions of a GCC compiler that may confuse the \CNAME\ build. It’s also worth checking to ensure GFortran has been installed with the G++ compiler by typing:

\texttt{gfortran -{}-version}

If you wish to build the documentation, \texttt{bibtex} will also need to be in the path, e.g., to check, try:

\texttt{bibtex -{}-version}

\subsubsection{Building \CNAME}

The build process is relatively straightforward. Before you start the build process, you can run \texttt{doBuild check} from the command prompt to check if your build environment is complete. Make sure that you are within \path{CASAL2/BuildSystem/} to run \texttt{doBuild}. 

\texttt{doBuild check} will summarise Windows environment PATH as a part of its output, and this can be used to check that the paths for g++ and gfortran and the g++ point to where the correct version of GCC is installed. 

The build process is as follows: 
\begin{enumerate}
  \item Download a clone of the code on your local machine
  \item Navigate to the BuildSystem folder in \path{CASAL2/BuildSystem}
  \item You need to build the third party libraries with the following commands from the command prompt:
  \begin{itemize}
    \item \texttt{doBuild thirdparty}
  \end{itemize}
  \item You need to build the binary you want to use:
  \begin{itemize}
    \item \texttt{doBuild release}
  \end{itemize}
  \item You can build the documentation if you want:
  \begin{itemize}
    \item \texttt{doBuild documentation}
  \end{itemize}
\end{enumerate}

\subsection{Building on Linux}

This guide has been written against a fresh install of Ubuntu 20.04. With Ubuntu we use apt-get to install new packages. You’ll need to be familiar with the package manager for your distribution to correctly install the required prerequisite software. For this you will require administrator level access.

\subsubsection{Prerequisite software}

\paragraph*{G++ compiler}

If gfortran is not installed, install this with: \texttt{sudo apt-get install gfortran}.

\paragraph*{GIT version control}

Git may not be installed by default and it can be installed with \texttt{sudo apt-get install git}

\CNAME\ automatically adds a version details to its files and any output based on the GIT version of the latest commit to its repository. This includes the name of source repository that was used. The command line version of GIT is used  to generate the version details.

\paragraph*{CMake}

CMake is required to build multiple third-party libraries and the main code base. You can do this with \texttt{sudo apt-get install cmake}

\paragraph*{Python}

Python3 is used to run the build scripts and set the required environment variables required to build \CNAME. This is usually installed by default on Linux systems, but if not, it can be installed using: \texttt{sudo apt-get install python3}

\paragraph*{Python modules}

There are a number of Python3 modules that are required to build \CNAME. These can be installed with \texttt{sudo apt-get install \texttt{module-name}}. For example, You may need to install \texttt{datetime}, \texttt{re}, and \texttt{distutils} Python modules. 

\paragraph*{LaTeX}

LaTeX on Linux is required, and the Texlive LaTeX Processor is recommended. This can be installed with:

\texttt{sudo apt-get install texlive-binaries}
\texttt{sudo apt-get install texlive-latex-base}
\texttt{sudo apt-get install texlive-latex-recommended}
\texttt{sudo apt-get install texlive-latex-extra}

Alternatively you can install the complete package with 
\texttt{sudo apt-get install texlive-full}

A number of additional LaTeX styles are used --- these will usually be identified doing the doBuild process and can be installed as required. 

\subsubsection{Building \CNAME}

The build process is relatively straightforward. You can run \texttt{./doBuild.sh check} to see if your build environment is ready.

\begin{enumerate}
	\item Download a clone of the code on your local machine
	\item Navigate to the BuildSystem folder in \path{CASAL2/BuildSystem}
	\item You need to build the third party libraries with:
	\begin{itemize}
	    \item \texttt{./doBuild.sh thirdparty}
	\end{itemize}
	\item You need to build the binary you want to use:
	\begin{itemize}
		\item \texttt{./doBuild.sh release}
	\end{itemize}
	\item You can build the documentation:
	\begin{itemize}
		\item \texttt{./doBuild.sh documentation}
	\end{itemize}
\end{enumerate}

\subsection{Troubleshooting}

\subsubsection{Third-party C++ libraries}

It's possible that there will be build errors or issues building the C++ third-party libraries. If you encounter an error, then check the log files to locate the source of the problem. Each third-party build system stores a log of everything that was done. The files will be named

\begin{itemize}
	\item casal2\_unzip.log
	\item casal2\_configure.log
	\item casal2\_make.log
	\item casal2\_build.log
	\item \dots etc,.
\end{itemize}

Some of the third-party libraries require very specialised environments for compiling under GCC on Windows. These libraries are packaged with MSYS (MinGW Linux style shell system). The log files for these will be found in \path{ThirdParty/<library name>/msys/1.0/<library name>/}

e.g., \path{ThirdParty/adolc/msys/1.0/adolc/ADOL-C-2.5.2/casal2_make.log}\\
e.g., \path{ThirdParty/boost/boost_1_58_0/casal2_build.log}

A common issue when running doBuild thirdparty are Python error messages about missing modules, e.g., ModuleNotFoundError: No module named 'dateutil'. This type of error message indicates that a Python module (library) is missing and will need to be installed. For instance, to install the 'dateutil' module, type the following into a command prompt or terminal window: pip3 install python-dateutil.  

\subsubsection{Main code base}

If the unmodified code base does not compile, the most likely cause is an issue with the third-party libraries not being built correctly. As updates and revisions are outside the control of the Development Team, problems can arise that may require the developers of the third party libraries to resolve first. However, versions of these libraries are included in the \CNAME\ source code and these should work. For any specific issues contact a local expert with regard to your specific system environment, or else the \CNAME\ Development Team for help.

	%\clearemptydoublepage{}
	\section{\I{\CNAME\ build guidelines and validation}\label{sec:buildrules}}

\subsection{\CNAME\ coding practice and style}\label{subsec:codepractice}

\CNAME\ is written in C++ and uses the Google C++ style guide (see \url{https://google.github.io/styleguide/cppguide.html}). 

In general when editing or writing code for \CNAME\:

\begin{enumerate}
  \item Using consistent indentations inside functions and loops, and descriptive and human readable variable or function names.
  \item Use of the characters `\_' on the end of class variables defined in the .h files. 
  \item Annotate and comment the code, especially where it would help another contributor understand the program logic and rationale.
  \item Add descriptive log messages to aid in debugging and checking of the program logic flow.
  \item Implement unit tests, internal models, and external models to test and validate the new or changed functionality.
  \item Document the functionality in the \CNAME\ User Manual(s).
\end{enumerate}

\CNAME\ allows printing of logging messages art runtime using the --loglevel command line argument. The levels of logging in \CNAME\ are:

\begin{itemize}
	\item LOG\_MEDIUM()  usually reserved for iterative functionality (e.g. estimates during estimation phase)
	\item LOG\_FINE() the level of reporting between an actual report and a fine scale detail that end users are not interested in (Developers)
	\item LOG\_FINEST() Minor fine scale details within a function or routine.
	\item LOG\_TRACE() put at the beginning of every function
\end{itemize}

e.g., to run \CNAME\ with logging, use 

\texttt{casal2 -r --loglevel finest > my\_run.log 2> my\_run.err}

This will output all the logged information to \texttt{my\_run.err}.

\subsection{Units tests and model validation}

The \CNAME\ development places an emphasis on maintaining software integrity and reproducibility between revisions. \CNAME\ uses model validations and built in unit tests to validate and verify the code each time \CNAME\ is compiled and built. 

There are three different validation approaches in \CNAME. These are:

\begin{enumerate}
	\item Mocking specific classes.
	\item Implementing internal models (implemented in C++ source code with extension \texttt{.Test.cpp}) that have variable test cases for specific classes. 
	\item Implementing externally run models (found in the TestModel folder) that are validated to generate expected output.
\end{enumerate}

To implement mocking of classes and internal models, \CNAME\ uses the Google testing framework and the Google mocking framework.

To implement testing of full models, input configuration files are run using the compiled \CNAME\ binaries, and the output compared with expected output using @assert commands.

\subsubsection{Mocking specific classes}

Classes are unit tested using unit tests that are a part of the source code. These are designed to check the components of the code to validate that functions provide expected output. These unit tests are run each time \CNAME\ is compiled.

When adding unit tests, they should to be developed and tested outside of \CNAME\. This gives confidence that the test does not contain any calculation errors. 

Mocking specific classes is used to validate specific functionality and is encouraged because it is the easiest to isolate simple errors that may be introduced with code changes. 

As examples, see (i) the file \texttt{VonBertalanffy.Test.cpp} mocks the von Bertalanffy age-length class and tests the mean length calculation, and (ii) the \texttt{Partition} class has the file \texttt{Partition.Test.cpp} that validates user inputs and model expectations.

\subsubsection{Internal mocking of simple models}

Mocking of simple models is done using a number of internal models. Most of the functionality for implementing these are in the source folder \texttt{/CASAL2/source/TestResources}. 

These implements simple models and run test cases with differing class implementations by running an internal empty model and testing the output of classes from a model run. 

As examples, see (i) the \texttt{LogNormal.Test.cpp} in the \texttt{Projects} class that test the lognormal distribution when used for projections, and (ii) the \texttt{TagByLength} process in \texttt{TagByLength.Test.cpp} that tests functionality of the tagging process.

\subsubsection{External testing using test models}

External tests are run following compilation using the Python modelrunner.py scripts (i.e., using the \texttt{DoBuild modelrunner} script in the BuildSystem folder). These models are used to test model runs, minimisation routines, and MCMC output.

The test model input configuration files are located in the \texttt{TestModel} folder and the command calls to run these are in the modelrunner.py script.  

Contributors are encouraged to add additional models to the list of test models as these be used to validate the combined functionality of a range of interrelated commands and subcommands in \CNAME. 

\subsection{Verification}
After \CNAME\ executes Validate and Build it runs sanity checks in the verify state. These are business rules that can be checked across the entire system. This can be useful to suggest dependencies or configurations. For an example see the directory \subcommand{Processes\textbackslash Verification\textbackslash} in the source code. 

\subsection{Reporting (optional)}

Currently \CNAME\ has reports that are \R\ compatible, i.e., all output reports produced by \CNAME\ can be read into \R\ using the standard  \textbf{CASAL2} \R\ package. If you create a new report or modify an old one, you most follow the standard so that the report is \R\ compatible.

All reports must start with,
\texttt{*label (type)}
and end with,
\texttt{*end}

Depending on what type of information you wish to report, will depend on the syntax you need to use. For example

\paragraph*{$\{$d$\}$ (Dataframe)}
Report a dataframe
{\small{\begin{verbatim}
			*estimates (estimate_value)
			values {d}
			process[Recruitment_BOP].r0 process[Recruitment_ENLD].r0 
			2e+006 8e+006
			*end
\end{verbatim}}}

\paragraph*{$\{$m$\}$ (Matrix)}
Report a matrix
{\small{\begin{verbatim}
			*covar (covariance_matrix)
			Covariance_Matrix {m}
			2.29729e+010 -742.276 -70160.5
			-110126 -424507 -81300 
			-36283.4 955920 -52736.2 
			*end
\end{verbatim}}}

\paragraph*{$\{$L$\}$ (List)}
Report a List
{\small{\begin{verbatim}
			*weight_one (partition_mean_weight)
			year: 1900
			ENLD.EN.notag {L}
			mean_weights {L}
			0.0476604 0.111575 0.199705
			end {L}
			age_lengths {L}
			12.0314 16.2808 20.0135
			end {L}
			end {L}
			*end
\end{verbatim}}}

\subsection{Update the \CNAME\ User Manual(s)}

Contributors will need to add or modify sections of the user manual(s) to document their changes. This includes the section that describes the methods and the section where the specific syntax is defined. 

\subsection{Builds to pass before merging changes}

Once you have made changes, you must run the following before your changes can be included in the master code. 

\begin{itemize}
	\item Build the unittest version. See Section~\ref{sec:build_environment} for how to build unittest depending on your system.
	\item Run the standard and new unit tests to check that they all pass. To do this first compile the test executable using the script \texttt{DoBuild test}. Then move to the directory with the location of the executable (\texttt{BuildSystem/bin/OS/test}) and run it (open a command terminal and run \texttt{casal2}) to check all the unit-tests pass.
	\item Test that the debug and release of \CNAME\ compiles and runs with \texttt{DoBuild debug}
	\item Run the second phase of unit tests (requires that the debug version is built). This runs the tests that comprise of complete model runs using \texttt{DoBuild modelrunner}
	\item Build the archive using \texttt{DoBuild archive} which builds all required libraries. There are small nuances between Double classes, especially when reporting the class that mean seemingly simple changes can sometimes cause a break in the full build.
\end{itemize}



	%\include{Appendix2}
\end{appendices}

\end{document}
%End
