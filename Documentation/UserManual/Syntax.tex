\section{\I{Population command and subcommand syntax}\label{syntax:Population}}

The description of the methods for the population section is given in Section \ref{sec:Population}.

In the following section, the sub-section headers use a notation of the form "\textbf {@observation[label].type=abundance}" which, in this case, represents the input command fragment
{\small{\begin{verbatim}
@observation label # where label is a unique label for that observation
type=abundance
...
\end{verbatim}}}
The specific subcommands for a command are given within each command.

\subsection{\I{Model structure}}
\ifAgeBased
\defComLab{Model}{Define an object of type \emph{Model}}.
\defRef{sec:Model}
\label{syntax:Model}

\defSub{type}{Type of model (only type=age is currently implemented)}
\defType{String}
\defDefault{age}

\defSub{base\_weight\_units}{Define the units for the base weight measurement unit (grams, kilograms (kgs), or tonnes). This will be the default unit of any weight input values}
\defType{String}
\defDefault{tonnes}

\defSub{threads}{The number of threads to use for this model}
\defType{Non-negative integer}
\defDefault{1}
\defLowerBound{1 (inclusive)}

\subsubsection{Model of type Age}
\commandlabsubarg{Model}{type}{Age}.
\defRef{sec:Model-Age}
\label{syntax:Model-Age}

\defSub{final\_year}{Define the final year of the model, excluding years in the projection period}
\defType{Non-negative integer}
\defDefault{No default}
\defValue{Defines the last year of the model, i.e., the model is run from start\_year to final\_year}

\defSub{min\_age}{Minimum age of individuals in the population}
\defType{Non-negative integer}
\defDefault{0}
\defValue{$0 \le$ age\textlow{min} $\le$ age\textlow{max}}

\defSub{max\_age}{Maximum age of individuals in the population}
\defType{Non-negative integer}
\defDefault{0}
\defValue{$0 \le$ age\textlow{min} $\le$ age\textlow{max}}

\defSub{age\_plus}{Define the oldest age or extra length midpoint (plus group size) as a plus group}
\defType{Boolean}
\defDefault{true}
\defValue{true, false}

\defSub{initialisation\_phases}{Define the labels of the phases of the initialisation}
\defType{Vector of strings}
\defDefault{true}
\defValue{A list of valid labels defined by \texttt{@initialisation\_phase}}

\defSub{time\_steps}{Define the labels of the time steps, in the order that they are applied, to form the annual cycle}
\defType{Vector of strings}
\defDefault{No default}
\defValue{A list of valid labels defined by \texttt{@time\_step}}

\defSub{projection\_final\_year}{Define the final year of the model when running projections}
\defType{Non-negative integer}
\defDefault{0}
\defValue{A value greater than final\_year}

\defSub{length\_bins}{The minimum length in each length bin}
\defType{Vector of real numbers}
\defDefault{true}
\defValue{$0 \le$ length\textlow{min} $\le$ length\textlow{max}}

\defSub{length\_plus}{Specify whether there is a length plus group or not}
\defType{Boolean}
\defDefault{true}
\defValue{true, false}

\defSub{length\_plus\_group}{Mean length of length plus group}
\defType{Real number}
\defDefault{0}
\defValue{length\textlow{max} $<$ length\_plus\_group}


\else
\input{IncludedSyntax/ModelLength}
\fi

\subsection{\I{Initialisation}}
\input{IncludedSyntax/InitialisationPhase}

\subsection{\I{Categories}}
\ifAgeBased
\defComLab{Categories}{Define an object of type \emph{Categories}}.
\defRef{sec:Categories}
\label{syntax:Categories}

\defSub{format}{The format that the category names use}
\defType{String}
\defDefault{No default}

\defSub{names}{The names of the categories}
\defType{Vector of strings}
\defDefault{No default}

\defSub{age\_lengths}{The age-length relationship labels for each category}
\defType{Vector of strings}
\defDefault{true}

\defSub{growth\_increment}{The growth increment model label for each category}
\defType{Vector of strings}
\defDefault{No default}


\else
\input{IncludedSyntax/CategoriesLength}
\fi

\subsection{\I{Time-steps}}
\input{IncludedSyntax/TimeStep}

\subsection{\I{Processes}}
\ifAgeBased
\defComLab{process}{Define an object of type \emph{Process}}.
\defRef{sec:Process}
\label{syntax:Process}

\defSub{label}{The label of the process}
\defType{String}
\defDefault{No default}

\defSub{type}{The type of process}
\defType{String}
\defDefault{No default}

\subsubsection{Process of type Ageing}
\commandlabsubarg{process}{type}{Ageing}.
\defRef{sec:Process-Ageing}
\label{syntax:Process-Ageing}

\defSub{categories}{The labels of the categories to age}
\defType{Vector of strings}
\defDefault{No default}

%\subsubsection{Process of type Load Partition}
%\commandlabsubarg{process}{type}{Load\_Partition}.
%\defRef{sec:Process-LoadPartition}
%\label{syntax:Process-LoadPartition}
%
%\defSub{table}{The table of data specifying the n in each partition category and age}
%\defType{Data table with label = data}
%\defDefault{No default}
%\defValue{}
%\defNote{See \ref{sec:DataTable} for more details on specifying data tables}
%
\subsubsection{Process of type Maturation}
\commandlabsubarg{process}{type}{Maturation}.
\defRef{sec:Process-Maturation}
\label{syntax:Process-Maturation}

\defSub{from}{The list of categories to mature from}
\defType{Vector of strings}
\defDefault{No default}

\defSub{to}{The list of categories to mature to}
\defType{Vector of strings}
\defDefault{No default}

\defSub{selectivities}{The list of selectivities to use for maturation}
\defType{Vector of strings}
\defDefault{No default}

\defSub{years}{The years to be associated with the maturity rates}
\defType{Vector of non-negative integers}
\defDefault{No default}

\defSub{rates}{The rates to mature for each year}
\defType{Vector of real numbers (estimable)}
\defDefault{No default}

\subsubsection{Process of type Mortality Constant Rate}
\commandlabsubarg{process}{type}{Mortality\_Constant\_Rate}.
\defRef{sec:Process-MortalityConstantRate}
\label{syntax:Process-MortalityConstantRate}

\defSub{categories}{The list of category labels}
\defType{Vector of strings}
\defDefault{No default}

\defSub{m}{The mortality rates}
\defType{Real number (estimable)}
\defDefault{No default}
\defLowerBound{0.0 (inclusive)}

\defSub{time\_step\_proportions}{The time step proportions for the mortality rates}
\defType{Vector of real numbers}
\defDefault{false}
\defLowerBound{0.0 (inclusive)}
\defUpperBound{1.0 (inclusive)}
\defValue{The time step proportions must sum to one. If only one value is supplied, then the each time step is allocated an equal proportion. Otherwise the number of values must equal the number of time steps}

\defSub{relative\_m\_by\_age}{The list of mortality by age ogive labels for the categories}
\defType{Vector of strings}
\defDefault{No default}

\subsubsection{Process of type Mortality Constant Exploitation}
\commandlabsubarg{process}{type}{Mortality\_Constant\_Exploitation}.
\defRef{sec:Process-MortalityConstantExploitation}
\label{syntax:Process-MortalityConstantExploitation}

\defSub{categories}{The list of category labels}
\defType{Vector of strings}
\defDefault{No default}

\defSub{u}{The exploitation rates}
\defType{Real number (estimable)}
\defDefault{No default}
\defLowerBound{0.0 (inclusive)}
\defUpperBound{1.0 (inclusive)}

\defSub{time\_step\_proportions}{The time step proportions for the exploitation rates}
\defType{Vector of real numbers}
\defDefault{false}
\defLowerBound{0.0 (inclusive)}
\defUpperBound{1.0 (inclusive)}
\defValue{The time step proportions must sum to one. If only one value is supplied, then the each time step is allocated an equal proportion. Otherwise the number of values must equal the number of time steps}

\defSub{relative\_u\_by\_age}{The list of exploitation by age ogive labels for the categories}
\defType{Vector of strings}
\defDefault{No default}

\subsubsection{Process of type Mortality Disease Rate}
\commandlabsubarg{process}{type}{Mortality\_Disease\_Rate}.
\defRef{sec:Process-MortalityDiseaseRate}
\label{syntax:Process-MortalityDiseaseRate}

\defSub{categories}{The list of category labels}
\defType{Vector of strings}
\defDefault{No default}

\defSub{disease\_mortality\_rate}{The disease mortality rates}
\defType{Real number (estimable)}
\defDefault{No default}
\defLowerBound{0.0 (inclusive)}
\defUpperBound{10.0 (inclusive)}

\defSub{year\_effects}{Annual deviations around the disease mortality rate}
\defType{Vector of real numbers (estimable)}
\defDefault{No default}
\defLowerBound{0.0 (inclusive)}

\defSub{selectivities}{The list of selectivities}
\defType{Vector of strings}
\defDefault{No default}

\defSub{years}{Years in which to apply the disease mortality in}
\defType{Vector of non-negative integers}
\defDefault{No default}

\subsubsection{Process of type Mortality Event}
\commandlabsubarg{process}{type}{Mortality\_Event}.
\defRef{sec:Process-MortalityEvent}
\label{syntax:Process-MortalityEvent}

\defSub{categories}{The categories}
\defType{Vector of strings}
\defDefault{No default}

\defSub{years}{The years in which to apply the mortality process}
\defType{Vector of non-negative integers}
\defDefault{No default}

\defSub{catches}{The number of removals (catches) to apply for each year}
\defType{Vector of real numbers (estimable)}
\defDefault{No default}

\defSub{u\_max}{The maximum exploitation rate ($U\_{max}$)}
\defType{Real number}
\defDefault{0.99}
\defLowerBound{0.0 (inclusive)}
\defUpperBound{1.0 (inclusive)}

\defSub{selectivities}{The list of selectivities}
\defType{Vector of strings}
\defDefault{No default}

\defSub{penalty}{The label of the penalty to apply if the total number of removals cannot be taken}
\defType{String}
\defDefault{No default}

\subsubsection{Process of type Mortality Event Biomass}
\commandlabsubarg{process}{type}{Mortality\_Event\_Biomass}.
\defRef{sec:Process-MortalityEventBiomass}
\label{syntax:Process-MortalityEventBiomass}

\defSub{categories}{The category labels}
\defType{Vector of strings}
\defDefault{No default}

\defSub{selectivities}{The labels of the selectivities for each of the categories}
\defType{Vector of strings}
\defDefault{No default}

\defSub{years}{The years in which to apply the mortality process}
\defType{Vector of non-negative integers}
\defDefault{No default}

\defSub{catches}{The biomass of removals (catches) to apply for each year}
\defType{Vector of real numbers (estimable)}
\defDefault{No default}

\defSub{u\_max}{The maximum exploitation rate ($U\_{max}$)}
\defType{Real number (estimable)}
\defDefault{0.99}
\defLowerBound{0.0 (inclusive)}
\defUpperBound{1.0 (inclusive)}

\defSub{penalty}{The label of the penalty to apply if the total biomass of removals cannot be taken}
\defType{String}
\defDefault{No default}

\subsubsection{Process of type Mortality Holling Rate}
\commandlabsubarg{process}{type}{Mortality\_Holling\_Rate}.
\defRef{sec:Process-MortalityHollingRate}
\label{syntax:Process-MortalityHollingRate}

\defSub{prey\_categories}{The prey categories labels}
\defType{Vector of strings}
\defDefault{No default}

\defSub{predator\_categories}{The predator categories labels}
\defType{Vector of strings}
\defDefault{No default}

\defSub{is\_abundance}{Is vulnerable amount of prey and predator an abundance [true] or biomass [false]}
\defType{Boolean}
\defDefault{true}

\defSub{a}{Parameter a}
\defType{Real number (estimable)}
\defDefault{No default}
\defLowerBound{0.0 (inclusive)}

\defSub{b}{Parameter b}
\defType{Real number (estimable)}
\defDefault{No default}
\defLowerBound{0.0 (inclusive)}

\defSub{x}{This parameter controls the functional form: Holling function type 2 (x=2) or 3 (x=3), or generalised (Michaelis Menten, x>=1)}
\defType{Real number (estimable)}
\defDefault{No default}
\defLowerBound{1.0 (inclusive)}

\defSub{u\_max}{The maximum exploitation rate ($U\_{max}$)}
\defType{Real number}
\defDefault{0.99}
\defLowerBound{0.0 (inclusive)}
\defUpperBound{1.0 (exclusive)}

\defSub{prey\_selectivities}{The selectivities for prey categories}
\defType{Vector of strings}
\defDefault{true}

\defSub{predator\_selectivities}{The selectivities for predator categories}
\defType{Vector of strings}
\defDefault{true}

\defSub{penalty}{The label of penalty}
\defType{String}
\defDefault{No default}

\defSub{years}{The years in which to apply the mortality process}
\defType{Vector of non-negative integers}
\defDefault{No default}

\defSub{table}{The table of data specifying the predator selectivities}
\defType{Data table with label = \TODO{I don't know how this one works?}}
\defDefault{No default}
\defValue{}
\defNote{See \ref{sec:DataTable} for more details on specifying data tables}

\defSub{table}{The table of data specifying the prey selectivities}
\defType{Data table with label = \TODO{I don't know how this one works?}}
\defDefault{No default}
\defValue{}
\defNote{See \ref{sec:DataTable} for more details on specifying data tables}

\subsubsection{Process of type Mortality Initialisation Event}
\commandlabsubarg{process}{type}{mortality\_initialisation\_event}.
\defRef{sec:Process-MortalityInitialisationEvent}
\label{syntax:Process-MortalityInitialisationEvent}

\defSub{categories}{The categories}
\defType{Vector of strings}
\defDefault{No default}

\defSub{catch}{The number of removals (catches) to apply for each year}
\defType{Real number (estimable)}
\defDefault{No default}

\defSub{u\_max}{The maximum exploitation rate ($U\_{max}$)}
\defType{Real number}
\defDefault{0.99}
\defLowerBound{0.0 (inclusive)}
\defUpperBound{1.0 (exclusive)}

\defSub{selectivities}{The list of selectivities}
\defType{Vector of strings}
\defDefault{No default}

\defSub{penalty}{The label of the penalty to apply if the total number of removals cannot be taken}
\defType{String}
\defDefault{No default}

\defSub{table}{The table of data specifying the catches for each fishery, the categories, years, and the $U_max$}
\defType{Data table with label = \TODO{Craig? Can you specify this table?}}
\defDefault{No default}
\defNote{See \ref{sec:DataTable} for more details on specifying data tables}

\subsubsection{Process of type Mortality Initialisation Event Biomass}
\commandlabsubarg{process}{type}{Mortality\_Initialisation\_Event\_Biomass}.
\defRef{sec:Process-MortalityInitialisationEventBiomass}
\label{syntax:Process-MortalityInitialisationEventBiomass}

\defSub{categories}{The categories}
\defType{Vector of strings}
\defDefault{No default}

\defSub{catch}{The number of removals (catches) to apply for each year}
\defType{Real number (estimable)}
\defDefault{No default}

\defSub{u\_max}{The maximum exploitation rate ($U\_{max}$)}
\defType{Real number}
\defDefault{0.99}
\defLowerBound{0.0 (inclusive)}
\defUpperBound{1.0 (inclusive)}

\defSub{selectivities}{The list of selectivities}
\defType{Vector of strings}
\defDefault{No default}

\defSub{penalty}{The label of the penalty to apply if the total number of removals cannot be taken}
\defType{String}
\defDefault{No default}


\subsubsection{Process of type Mortality Initialisation Baranov}
\commandlabsubarg{process}{type}{mortality\_initialisation\_baranov}.
\defRef{sec:Process-MortalityInitialisationEventBiomass}
\label{syntax:Process-MortalityInitialisationBaranov}

\defSub{categories}{The categories}
\defType{Vector of strings}
\defDefault{No default}

\defSub{fishing\_mortality}{The fishing mortality to apply}
\defType{Real number (estimable)}
\defDefault{No default}

\defSub{selectivities}{The list of selectivities for each category}
\defType{Vector of strings}
\defDefault{No default}

\subsubsection{Process of type Mortality Hybrid}
\commandlabsubarg{process}{type}{mortality\_hybrid}.
\defRef{sec:Process-MortalityHybrid}
\label{syntax:Process-MortalityHybrid}

\defSub{categories}{The categories for to apply natural mortality to}
\defType{Vector of strings}
\defDefault{No default}

\defSub{m}{The natural mortality rates for each category}
\defType{Real number (estimable)}
\defDefault{No default}
\defLowerBound{0.0 (inclusive)}

\defSub{time\_step\_proportions}{The time step proportions for natural mortality}
\defType{Vector of real numbers}
\defDefault{true}
\defLowerBound{0.0 (inclusive)}
\defUpperBound{1.0 (inclusive)}
\defValue{Proportions must sum to one}

\defSub{biomass}{Switch to indicate if the catches are biomasses or abundances}
\defType{Boolean}
\defDefault{True}

\defSub{relative\_m\_by\_age}{The M-by-age selectivities to apply to each of the categories for natural mortality}
\defType{Vector of strings}
\defDefault{No default}

\defSub{max\_f}{Maximum \(F\) allowed}
\defType{Real number}
\defDefault{4.0}
\defLowerBound{0.0 (inclusive)}

\defSub{f\_iterations}{The number of tuning iterations to solve \(F\)}
\defType{integer}
\defDefault{4}
\defLowerBound{0 (inclusive)}

\defSub{table catches}{The table of data specifying the catches for each fishery and year}
\defDefault{No default}
\defNote{See below for example}
\begin{verbatim}
	table catches
	year Fishery1_label Fishery2_label
	1993 34 34 
	1994 23 34 
	end_table
\end{verbatim}

\defSub{table method}{The table of data specifying which fishery interacts with which category, selectivities, penalty, and time-step for each fishery and annual duration of the fishery.}
\defDefault{No default}
\defNote{See below for example}
\begin{verbatim}
	table method
	method         category selectivity annual_duration time_step penalty
	Fishery1_label male     fish_sel    1               step1     none
	Fishery2_label male     fish_sel    1               step1     none
	end_table
\end{verbatim}

\subsubsection{Process of type Mortality Instantaneous}
\commandlabsubarg{process}{type}{Mortality\_Instantaneous}.
\defRef{sec:Process-MortalityInstantaneous}
\label{syntax:Process-MortalityInstantaneous}

\defSub{categories}{The categories for instantaneous mortality}
\defType{Vector of strings}
\defDefault{No default}

\defSub{m}{The natural mortality rates for each category}
\defType{Real number (estimable)}
\defDefault{No default}
\defLowerBound{0.0 (inclusive)}

\defSub{time\_step\_proportions}{The time step proportions for natural mortality}
\defType{Vector of real numbers}
\defDefault{true}
\defLowerBound{0.0 (inclusive)}
\defUpperBound{1.0 (inclusive)}
\defValue{Proportions must sum to one}

\defSub{relative\_m\_by\_age}{The M-by-age selectivities to apply to each of the categories for natural mortality}
\defType{Vector of strings}
\defDefault{No default}

\defSub{table catches}{The table of data specifying the catches for each fishery and year}
\defDefault{No default}
\defValue{A column of catches as biomass or abundance; or a column of exploitation rates of biomass or abundance}
\defLowerBound{0 (inclusive)}
\defNote{See below for example}
\begin{verbatim}
	table catches
	year Fishery1_label Fishery2_label
	1993 34 34 
	1994 23 34 
	end_table
\end{verbatim}

\defSub{table method}{The table of data specifying which fishery interacts with which category, selectivities, u\_max, time-step, and penalty for each fishery and annual duration of the fishery.Optional columns can indicate the catches are in abundance and if the values are exploitation rates}
\defDefault{No default}
\defValue{Valid column names are: method, category, selectivity, u\_max, time\_step, and penalty; and optionally biomass, u, and age\_weight\_label}
\defNote{See below for example}
\begin{verbatim}
	table method
	method         category selectivity u_max time_step penalty
	Fishery1_label male     fish_sel    1     step1     CatchMustBeTaken
	Fishery2_label male     fish_sel    1     step1     CatchMustBeTaken
	end_table
\end{verbatim}

\subsubsection{Process of type Mortality Instantaneous Retained}
\commandlabsubarg{process}{type}{Mortality\_Instantaneous\_Retained}.
\defRef{sec:Process-MortalityInstantaneousRetained}
\label{syntax:Process-MortalityInstantaneousRetained}

\defSub{categories}{The categories for instantaneous mortality}
\defType{Vector of strings}
\defDefault{No default}

\defSub{m}{The natural mortality rates for each category}
\defType{Real number (estimable)}
\defDefault{No default}
\defLowerBound{0.0 (inclusive)}

\defSub{time\_step\_proportions}{The time step proportions for natural mortality}
\defType{Vector of real numbers}
\defDefault{No default}
\defLowerBound{0.0 (inclusive)}
\defUpperBound{1.0 (inclusive)}
\defValue{Proportions must sum to one}

\defSub{relative\_m\_by\_age}{The M-by-age selectivities to apply on the categories for natural mortality}
\defType{Vector of strings}
\defDefault{No default}

\defSub{table}{The table of data specifying the catches for each fishery, the categories, years, and the $U_max$}
\defType{Data table with label = \TODO{Craig? Can you specify this table?}}
\defDefault{No default}
\defValue{}
\defNote{See \ref{sec:DataTable} for more details on specifying data tables}

\subsubsection{Process of type Mortality Prey Suitability}
\commandlabsubarg{process}{type}{Mortality\_Prey\_Suitability}.
\defRef{sec:Process-MortalityPreySuitability}
\label{syntax:Process-MortalityPreySuitability}

\defSub{prey\_categories}{The prey categories labels}
\defType{Vector of strings}
\defDefault{No default}

\defSub{predator\_categories}{The predator categories labels}
\defType{Vector of strings}
\defDefault{No default}

\defSub{consumption\_rate}{The predator consumption rate}
\defType{Real number (estimable)}
\defDefault{No default}
\defLowerBound{0.0 (inclusive)}
\defUpperBound{1.0 (inclusive)}

\defSub{electivities}{The prey electivities}
\defType{Vector of real numbers (estimable)}
\defDefault{No default}
\defLowerBound{0.0 (inclusive)}
\defUpperBound{1.0 (inclusive)}

\defSub{u\_max}{The maximum exploitation rate ($U\_{max}$)}
\defType{Real number}
\defDefault{0.99}
\defLowerBound{0.0 (inclusive)}
\defUpperBound{1.0 (exclusive)}

\defSub{prey\_selectivities}{The selectivities for prey categories}
\defType{Vector of strings}
\defDefault{No default}

\defSub{predator\_selectivities}{The selectivities for predator categories}
\defType{Vector of strings}
\defDefault{No default}

\defSub{penalty}{The label of the penalty}
\defType{String}
\defDefault{No default}

\defSub{years}{The year that process occurs}
\defType{Vector of non-negative integers}
\defDefault{No default}
\subsubsection{Markovian Movement}
\commandlabsubarg{process}{type}{markovian\_movement}.
\defRef{sec:Process-MarkovianMovement}
\label{syntax:Process-MarkovianMovement}

\defSub{from}{The categories to transition from}
\defType{Vector of strings}
\defDefault{No default}
\defValue{Valid category labels}

\defSub{to}{The categories to transition to}
\defType{Vector of strings}
\defDefault{No default}
\defValue{Valid category labels}

\defSub{proportions}{The proportions to transition for each category}
\defType{Real number (estimable)}
\defDefault{No default}
\defLowerBound{0.0 (inclusive)}
\defUpperBound{1.0 (inclusive)}

\defSub{selectivities}{The selectivities to apply to each proportion}
\defType{Vector of strings}
\defDefault{No default}
\defValue{Valid selectivity labels}

\subsubsection{Process of type null process}
\commandlabsubarg{process}{type}{null\_process}.
\label{syntax:Process-Null}

The null\_process  type has no additional subcommands. Note that this process does nothing. It is included primarily as a means of replacing other processes with "no action" to allow for testing of alternative model structures.

\subsubsection{Process of type Recruitment Beverton Holt}
\commandlabsubarg{process}{type}{Recruitment\_Beverton\_Holt}.
\defRef{sec:Process-RecruitmentBevertonHolt}
\label{syntax:Process-RecruitmentBevertonHolt}

\defSub{categories}{The category labels}
\defType{Vector of strings}
\defDefault{No default}

\defSub{r0}{R0, the mean recruitment used to scale annual recruits or initialise the model}
\defType{Real number (estimable)}
\defDefault{No default}
\defLowerBound{0.0 (inclusive)}
\defValue{Use either R0 or B0, but not both}

\defSub{b0}{B0, the SSB corresponding to R0, and used to scale annual recruits or initialise the model}
\defType{Real number (estimable)}
\defDefault{No default}
\defLowerBound{0.0 (inclusive)}
\defValue{Use either R0 or B0, but not both}

\defSub{proportions}{The proportion for each category}
\defType{Real number (estimable)}
\defDefault{No default}

\defSub{age}{The age at recruitment}
\defType{Non-negative integer}
\defDefault{No default}

\defSub{ssb\_offset}{The spawning biomass year offset}
\defType{Non-negative integer}
\defDefault{No default}

\defSub{steepness}{Steepness (h)}
\defType{Real number (estimable)}
\defDefault{1.0}
\defLowerBound{0.2 (inclusive)}
\defUpperBound{1.0 (inclusive)}

\defSub{ssb}{The SSB label (i.e., the derived quantity label)}
\defType{String}
\defDefault{No default}

\defSub{b0\_initialisation\_phase}{The initialisation phase label that B0 is from}
\defType{String}
\defDefault{No default}

\defSub{ycs\_values}{Deprecated}
\defSub{ycs\_years}{Deprecated}
\defSub{standardise\_ycs\_years}{Deprecated}

\defSub{recruitment\_multipliers}{The recruitment values also termed year class strengths.}
\defType{Vector of real numbers (estimable)}
\defDefault{No default}

\defSub{standardise\_years}{The years that are included for year class standardisation, they refer to the recruited year not spawning or year class year.}
\defType{Vector of non-negative integers}
\defDefault{true}

\subsubsection{Process of type Recruitment Beverton Holt With Deviations}
\commandlabsubarg{process}{type}{Recruitment\_Beverton\_Holt\_With\_Deviations}.
\defRef{sec:Process-RecruitmentBevertonHoltWithDeviations}
\label{syntax:Process-RecruitmentBevertonHoltWithDeviations}

\defSub{categories}{The category labels}
\defType{Vector of strings}
\defDefault{No default}

\defSub{r0}{R0, the mean recruitment used to scale annual recruits or initialise the model}
\defType{Real number (estimable)}
\defDefault{No default}
\defLowerBound{0.0 (inclusive)}
\defValue{Use either R0 or B0, but not both}

\defSub{b0}{B0, the SSB corresponding to R0, and used to scale annual recruits or initialise the model}
\defType{Real number (estimable)}
\defDefault{No default}
\defLowerBound{0.0 (inclusive)}
\defValue{Use either R0 or B0, but not both}

\defSub{proportions}{The proportion for each category}
\defType{Real number (estimable)}
\defDefault{No default}

\defSub{age}{The age at recruitment}
\defType{Non-negative integer}
\defDefault{true}

\defSub{ssb\_offset}{The spawning biomass year offset}
\defType{Non-negative integer}
\defDefault{No default}

\defSub{steepness}{Steepness (h)}
\defType{Real number (estimable)}
\defDefault{1.0}
\defLowerBound{0.2 (inclusive)}
\defUpperBound{1.0 (inclusive)}

\defSub{ssb}{The SSB label (i.e., the derived quantity label)}
\defType{String}
\defDefault{No default}
\defValue(A valid derived quantity)

\defSub{sigma\_r}{The standard deviation of recruitment, $\sigma_R$}
\defType{Real number (estimable)}
\defDefault{No default}
\defLowerBound{0.0 (inclusive)}

\defSub{b\_max}{The maximum bias adjustment}
\defType{Real number (estimable)}
\defDefault{0.85}
\defLowerBound{0.0 (inclusive)}
\defUpperBound{1.0 (inclusive)}

\defSub{last\_year\_with\_no\_bias}{The last year with no bias adjustment}
\defType{Non-negative integer}
\defDefault{false}

\defSub{first\_year\_with\_bias}{The first year with full bias adjustment}
\defType{Non-negative integer}
\defDefault{false}

\defSub{last\_year\_with\_bias}{The last year with full bias adjustment}
\defType{Non-negative integer}
\defDefault{false}

\defSub{first\_recent\_year\_with\_no\_bias}{The first recent year with no bias adjustment}
\defType{Non-negative integer}
\defDefault{false}

\defSub{b0\_initialisation\_phase}{The initialisation phase label that B0 is from}
\defType{String}
\defDefault{No default}

\defSub{deviation\_values}{The recruitment deviation values}
\defType{Vector of real numbers (estimable)}
\defDefault{No default}

\defSub{deviation\_years}{Deprecated}

\subsubsection{Process of type Recruitment Ricker}
\commandlabsubarg{process}{type}{Recruitment\_Ricker}.
\defRef{sec:Process-RecruitmentRicker}
\label{syntax:Process-RecruitmentRicker}

\defSub{categories}{The category labels}
\defType{Vector of strings}
\defDefault{No default}

\defSub{r0}{R0, the mean recruitment used to scale annual recruits or initialise the model}
\defType{Real number (estimable)}
\defDefault{No default}
\defLowerBound{0.0 (inclusive)}
\defValue{Use either R0 or B0, but not both}

\defSub{b0}{B0, the SSB corresponding to R0, and used to scale annual recruits or initialise the model}
\defType{Real number (estimable)}
\defDefault{No default}
\defLowerBound{0.0 (inclusive)}
\defValue{Use either R0 or B0, but not both}

\defSub{proportions}{The proportion for each category}
\defType{Real number (estimable)}
\defDefault{No default}

\defSub{age}{The age at recruitment}
\defType{Non-negative integer}
\defDefault{No default}

\defSub{ssb\_offset}{The spawning biomass year offset}
\defType{Non-negative integer}
\defDefault{No default}

\defSub{steepness}{Steepness (h)}
\defType{Real number (estimable)}
\defDefault{1.0}
\defLowerBound{0.2 (inclusive)}
\defUpperBound{1.0 (inclusive)}

\defSub{ssb}{The SSB label (i.e., the derived quantity label)}
\defType{String}
\defDefault{No default}

\defSub{b0\_initialisation\_phase}{The initialisation phase label that B0 is from}
\defType{String}
\defDefault{No default}

\defSub{ycs\_values}{Deprecated}
\defSub{ycs\_years}{Deprecated}
\defSub{standardise\_ycs\_years}{Deprecated}

\defSub{recruitment\_multipliers}{The recruitment values also termed year class strengths.}
\defType{Vector of real numbers (estimable)}
\defDefault{No default}

\defSub{standardise\_years}{The years that are included for year class standardisation, they refer to the recruited year not spawning or year class year.}
\defType{Vector of non-negative integers}
\defDefault{true}

\subsubsection{Process of type Recruitment Constant}
\commandlabsubarg{process}{type}{Recruitment\_Constant}.
\defRef{sec:Process-RecruitmentConstant}
\label{syntax:Process-RecruitmentConstant}

\defSub{categories}{The categories}
\defType{Vector of strings}
\defDefault{No default}

\defSub{proportions}{The proportion for each category}
\defType{Real number (estimable)}
\defDefault{true}

\defSub{age}{The age at recruitment}
\defType{Non-negative integer}
\defDefault{No default}

\defSub{r0}{R0, the recruitment used for annual recruits and initialise the model}
\defType{Real number (estimable)}
\defDefault{No default}
\defLowerBound{0.0 (inclusive)}

\subsubsection{Process of type Survival Constant Rate}
\commandlabsubarg{process}{type}{Survival\_Constant\_Rate}.
\defRef{sec:Process-SurvivalConstantRate}
\label{syntax:Process-SurvivalConstantRate}

\defSub{categories}{The list of categories}
\defType{Vector of strings}
\defDefault{No default}

\defSub{s}{The survival rates}
\defType{Real number (estimable)}
\defDefault{No default}
\defLowerBound{0.0 (inclusive)}
\defUpperBound{1.0 (inclusive)}

\defSub{time\_step\_proportions}{The time step proportions for the survival rate $S$}
\defType{Vector of real numbers}
\defDefault{true}
\defLowerBound{0.0 (exclusive)}
\defUpperBound{1.0 (inclusive)}
\defValue{The proportions must sum to one}

\defSub{selectivities}{The selectivity labels for each category}
\defType{Vector of strings}
\defDefault{No default}

\subsubsection{Process of type Tag By Age}
\commandlabsubarg{process}{type}{Tag\_By\_Age}.
\defRef{sec:Process-TagByAge}
\label{syntax:Process-TagByAge}

\defSub{from}{The categories that are selected for tagging (i.e, transition from)}
\defType{Vector of strings}
\defDefault{No default}

\defSub{to}{The categories that have tags (i.e., transition to)}
\defType{Vector of strings}
\defDefault{No default}

\defSub{min\_age}{The minimum age tagged}
\defType{Non-negative integer}
\defDefault{No default}

\defSub{max\_age}{The maximum age tagged}
\defType{Non-negative integer}
\defDefault{No default}

\defSub{penalty}{The penalty label}
\defType{String}
\defDefault{No default}

\defSub{u\_max}{The maximum exploitation rate ($U\_{max}$)}
\defType{Real number}
\defDefault{0.99}
\defLowerBound{0.0 (inclusive)}
\defUpperBound{1.0 (exclusive)}

\defSub{years}{The years to execute the tagging in}
\defType{Vector of non-negative integers}
\defDefault{No default}
\defNote{This should list the years of data in the table numbers or table proportions described below}

\defSub{initial\_mortality}{The initial mortality value (as a instantaneous proportion)}
\defType{Real number (estimable)}
\defDefault{0.0}
\defLowerBound{0.0 (inclusive)}
\defUpperBound{1.0 (inclusive)}

\defSub{initial\_mortality\_selectivity}{The initial mortality selectivity label}
\defType{String}
\defDefault{No default}
\defValue{Valid selectivity labels}

\defSub{selectivities}{The selectivity labels}
\defType{Vector of strings}
\defDefault{No default}
\defValue{Valid selectivity labels}

\defSub{n}{the total number of tagged fish}
\defType{Vector of real numbers (estimable)}
\defDefault{No default}
\defNote{Only required if table proportions are also supplied}

\defSub{table}{The table of data specifying the either the numbers or proportions to tag from and to each category and year}
\defType{Data table with either label = numbers or label = proportions}
\defDefault{No default}
\defNote{If table proportions, then the total number (@process[label].n) should also be specified. See \ref{sec:DataTable} for more details on specifying data tables}

\defSub{table numbers}{The table of releases as numbers for the process}
\defType{Data table with label = numbers}
\defDefault{Can be replaced with 'table proportions' -- see below}
\defValue{A \(n_y \times (n_l\times n_c) + 1\) matrix, where $n_y=$ is the number of years, \(n_a\) are the ages and \(n_c\) are the number of categories. The first column is the year value for that row. See below for an example}
\defNote{example below}
\begin{verbatim}
	table numbers 
	1993 34 34 23 43
	1994 23 34 23 43
	end_table
\end{verbatim}

\defSub{table proportions}{The table of releases as numbers for the process}
\defType{Data table with label = proportions}
\defDefault{Can be replaced with table numbers -- see above}
\defValue{A \(n_y \times (n_l\times n_c) + 1\) matrix, where $n_y=$ is the number of years, \(n_a\) are the ages and \(n_c\) are the number of categories. The first column is the year value for that row. See below for an example}
\defNote{example below}
\begin{verbatim}
	n 200 300 ## need to specify n if you give proportions
	table proportions 
	1993 0.1 0.2 0.7
	1994 0.1 0.2 0.7
	end_table
\end{verbatim}

\defSub{tolerance}{Tolerance for checking the specified proportions sum to one}
\defType{Real number}
\defDefault{1e-5}
\defLowerBound{0 (inclusive)}
\defUpperBound{1.0 (inclusive)}

\subsubsection{Process of type Tag By Length}
\commandlabsubarg{process}{type}{Tag\_By\_Length}.
\defRef{sec:Process-TagByLength}
\label{syntax:Process-TagByLength}

\defSub{from}{The categories that are selected for tagging (i.e, transition from)}
\defType{Vector of strings}
\defDefault{No default}

\defSub{to}{The categories that have tags (i.e., transition to)}
\defType{Vector of strings}
\defDefault{No default}

\defSub{penalty}{The penalty label}
\defType{String}
\defDefault{No default}

\defSub{u\_max}{The maximum exploitation rate ($U\_{max}$)}
\defType{Real number}
\defDefault{0.99}
\defLowerBound{0.0 (inclusive)}
\defUpperBound{1.0 (exclusive)}

\defSub{compatibility\_option}{Backwards compatibility option: either casal2 (the default) or casal. This affects the penalty and age-length calculations}
\defType{string}
\defDefault{casal2}
\defValue{Valid options are \subcommand{casal2} \& \subcommand{casal}}

\defSub{years}{The years to execute the tagging events in}
\defType{Vector of non-negative integers}
\defDefault{No default}

\defSub{initial\_mortality}{The initial mortality value (as a proportion)}
\defType{Real number (estimable)}
\defDefault{0.0}
\defLowerBound{0.0 (inclusive)}
\defUpperBound{1.0 (inclusive)}

\defSub{initial\_mortality\_selectivity}{The initial mortality selectivity label}
\defType{String}
\defDefault{No default}
\defValue{A valid selectivity label}

\defSub{selectivities}{The selectivity labels}
\defType{Vector of strings}
\defDefault{No default}
\defValue{Valid selectivity labels}

\defSub{n}{the total number of tagged fish}
\defType{Vector of real numbers (estimable)}
\defDefault{No default}
\defNote{Only required if table proportions are also supplied}

\defSub{table}{The table of data specifying the either the numbers or proportions to tag from and to each category and year}
\defType{Data table with either label = numbers or label = proportions}
\defDefault{No default}
\defNote{If table proportions, then the total number (@process[label].n) should also be specified. See \ref{sec:DataTable} for more details on specifying data tables}

\defSub{table numbers}{The table of releases as numbers for the process}
\defType{Data table with label = numbers}
\defDefault{Can be replaced with 'table proportions' -- see below}
\defValue{A \(n_y \times (n_l\times n_c) + 1\) matrix, where $n_y=$ is the number of years, \(n_a\) are the number of ages ages and \(n_c\) are the number of categories. The first column is the year value for that row. See below for an example}
\defNote{example below}
\begin{verbatim}
	table numbers 
	1993 34 34 23 43
	1994 23 34 23 43
	end_table
\end{verbatim}

\defSub{table proportions}{The table of releases as numbers for the process}
\defType{Data table with label = proportions}
\defDefault{Can be replaced with table numbers -- see above}
\defValue{A \(n_y \times (n_l\times n_c) + 1\) matrix, where $n_y=$ is the number of years, \(n_a\) are the number of ages and \(n_c\) are the number of categories. The first column is the year value for that row. See below for an example}
\defNote{example below}
\begin{verbatim}
	n 200 300 ## need to specify n if you give proportions
	table proportions 
	1993 0.1 0.2 0.7
	1994 0.1 0.2 0.7
	end_table
\end{verbatim}

\defSub{tolerance}{Tolerance for checking the specified proportions sum to one}
\defType{Real number}
\defDefault{1e-5}
\defLowerBound{0 (inclusive)}
\defUpperBound{1.0 (inclusive)}

\subsubsection{Process of type Tag Loss}
\commandlabsubarg{process}{type}{Tag\_Loss}.
\defRef{sec:Process-TagLoss}
\label{syntax:Process-TagLoss}

\defSub{categories}{The list of categories}
\defType{Vector of strings}
\defDefault{No default}
\defValue{Valid category labels}

\defSub{tag\_loss\_rate}{The instantaneous tag loss rates}
\defType{Vector of real numbers (estimable)}
\defDefault{No default}
\defLowerBound{0.0 (inclusive)}
\defValue{The instantaneous rate of tag loss (supplied as either a single value that is applied to all categories, or a vector of length equal to the number of categories defined for this process)}

\defSub{time\_step\_proportions}{The time step proportions for tag loss}
\defType{Vector of real numbers (estimable)}
\defDefault{true}
\defLowerBound{0.0 (inclusive)}
\defUpperBound{1.0 (inclusive)}
\defNote{The sum of the values of time\_step\_proportions must equal 1.0}

\defSub{tag\_loss\_type}{The type of tag loss}
\defType{String}
\defDefault{single}
\defValue{Valid options are \subcommand{single} \& \subcommand{double}}

\defSub{selectivities}{The selectivities}
\defType{Vector of strings}
\defDefault{No default}

\defSub{year}{The year the first tagging release process was executed}
\defType{Non-negative integer}
\defDefault{No default}
\defNote{For the double tag loss rate, this is also assumed to be the first year in the calculation of the annual loss rates}

\subsubsection{Process of type Tag Loss Empirical}
\commandlabsubarg{process}{type}{Tag\_Loss\_Empirical}.
\defRef{sec:Process-TagLossEmpirical}
\label{syntax:Process-TagLossEmpirical}

\defSub{categories}{The list of categories}
\defType{Vector of strings}
\defDefault{No default}
\defValue{Valid category labels}

\defSub{tag\_loss\_rate}{The instantaneous tag loss rates}
\defType{Vector of real numbers (estimable)}
\defDefault{No default}
\defLowerBound{0.0 (inclusive)}
\defValue{The instantaneous rate of tag loss (supplied as either a single value that is applied to all years at liberty, or a vector of length equal to the number of years at liberty defined for this process)}

\defSub{time\_step\_proportions}{The time step proportions for tag loss}
\defType{Vector of real numbers (estimable)}
\defDefault{true}
\defLowerBound{0.0 (inclusive)}
\defUpperBound{1.0 (inclusive)}
\defNote{The sum of the values of time\_step\_proportions must equal 1.0}

\defSub{selectivities}{The selectivities}
\defType{Vector of strings}
\defDefault{No default}

\defSub{year}{The year the first tagging release process was executed}
\defType{Non-negative integer}
\defDefault{No default}
\defNote{For the tag loss rate, this is also assumed to be the first year at liberty in the calculation of the annual loss rates}

\defSub{years\_at\_liberty}{The years at liberty that the tag\_loss\_rate applies to}
\defType{Non-negative integer}
\defDefault{No default}
\defLowerBound{0 (inclusive)}
\defNote{Years at liberty must be a valid value between 0 and the maximum number of years in the model. The tag\_loss\_rate is not applied for years at liberty that are not specified}

\subsubsection{Process of type Transition Category}
\commandlabsubarg{process}{type}{Transition\_Category}.
\defRef{sec:Process-TransitionCategory}
\label{syntax:Process-TransitionCategory}

\defSub{from}{The categories to transition from}
\defType{Vector of strings}
\defDefault{No default}
\defValue{Valid category labels}

\defSub{to}{The categories to transition to}
\defType{Vector of strings}
\defDefault{No default}
\defValue{Valid category labels}

\defSub{proportions}{The proportions to transition for each category}
\defType{Real number (estimable)}
\defDefault{No default}
\defLowerBound{0.0 (inclusive)}
\defUpperBound{1.0 (inclusive)}

\defSub{selectivities}{The selectivities to apply to each proportion}
\defType{Vector of strings}
\defDefault{No default}
\defValue{Valid selectivity labels}

\defSub{include\_in\_mortality\_block}{Include this process within the mortality block}
\defType{Boolean}
\defDefault{false}
\defValue{Either true or false}
\defNote{Warning, if true, this may adversely effect derived quantities and observations that interpolate over the mortality block - see \defRef{sec:Process-TransitionCategory}}

\subsubsection{Process of type Transition Category By Age}
\commandlabsubarg{process}{type}{Transition\_Category\_By\_Age}.
\defRef{sec:Process-TransitionCategoryByAge}
\label{syntax:Process-TransitionCategoryByAge}

\defSub{from}{The categories to transition from}
\defType{Vector of strings}
\defDefault{No default}
\defValue{Valid category labels}

\defSub{to}{The categories to transition to}
\defType{Vector of strings}
\defDefault{No default}
\defValue{Valid category labels}

\defSub{min\_age}{The minimum age to transition}
\defType{Non-negative integer}
\defDefault{No default}
\defValue{Valid category labels}

\defSub{max\_age}{The maximum age to transition}
\defType{Non-negative integer}
\defDefault{No default}

\defSub{penalty}{The penalty label}
\defType{String}
\defDefault{No default}
\defValue{A valid penalty label}

\defSub{u\_max}{The maximum exploitation rate ($U\_{max}$)}
\defType{Real number (estimable)}
\defDefault{0.99}
\defLowerBound{0.0 (inclusive)}
\defUpperBound{1.0 (exclusive)}

\defSub{years}{The years to execute the transition in}
\defType{Vector of non-negative integers}
\defDefault{No default}

\defSub{table n}{The table of numbers at age to transition from and to each category}
\defType{See below for example}
\defDefault{No default}
\defValue{}
\defNote{See below for example}
\begin{verbatim}
table n
year 3 4 5 6
2008 1000 2000 3000 4000
end_table
\end{verbatim}

\else
\input{IncludedSyntax/ProcessLength}
\fi 

\subsection{\I{Time varying parameters}}
\input{IncludedSyntax/TimeVarying}

\subsection{\I{Derived quantities}}
\defComLab{Derived\_Quantity}{Define an object of type \emph{Derived\_Quantity}}.
\defRef{sec:DerivedQuantity}
\label{syntax:DerivedQuantity}

\defSub{type}{The type of derived quantity}
\defType{String}
\defDefault{No default}

\defSub{time\_step}{The time step in which to calculate the derived quantity}
\defType{String}
\defDefault{No default}

\defSub{categories}{The list of categories to use when calculating the derived quantity}
\defType{Vector of strings}
\defDefault{No default}

\defSub{selectivities}{The list of selectivities to use when calculating the derived quantity}
\defType{Vector of strings}
\defDefault{No default}

\defSub{time\_step\_proportion}{The proportion through the mortality block of the time step when the derived quantity is calculated}
\defType{Real number (estimable)}
\defDefault{0.5}
\defLowerBound{0.0 (inclusive)}
\defUpperBound{1.0 (inclusive)}

\defSub{time\_step\_proportion\_method}{The method for interpolating for the proportion through the mortality block}
\defType{String}
\defDefault{weighted\_sum}

\subsubsection{Derived\_Quantity of type Abundance}
\commandlabsubarg{Derived\_Quantity}{type}{Abundance}.
\defRef{sec:DerivedQuantity-Abundance}
\label{syntax:DerivedQuantity-Abundance}

The Abundance type has no additional subcommands.
\subsubsection{Derived\_Quantity of type Biomass}
\commandlabsubarg{Derived\_Quantity}{type}{Biomass}.
\defRef{sec:DerivedQuantity-Biomass}
\label{syntax:DerivedQuantity-Biomass}

\defSub{age\_weight\_labels}{The labels for the age-weights that correspond to each category for the biomass calculation}
\defType{Vector of strings}
\defDefault{No default}

\subsubsection{Derived\_Quantity of type Biomass\_Index}
\commandlabsubarg{Derived\_Quantity}{type}{Biomass\_Index}.
\defRef{sec:DerivedQuantity-BiomassIndex}
\label{syntax:DerivedQuantity-BiomassIndex}

\defSub{distribution}{The type of distribution for the biomass index}
\defType{String}
\defDefault{"lognormal"}

\defSub{cv}{The cv for the uncertainty for the distribution when generating the biomass index}
\defType{Real number}
\defDefault{0.2}
\defLowerBound{0.0 (inclusive)}

\defSub{bias}{The bias (a positive or negative proportion) when generating the biomass index}
\defType{Real number}
\defDefault{0.0}

\defSub{rho}{The autocorrelation in annual values when generating the biomass index}
\defType{Real number}
\defDefault{0.0}
\defLowerBound{0.0 (inclusive)}

\defSub{catchability}{The catchability to use when generating the biomass index}
\defType{String}
\defDefault{No default}



\ifAgeBased
\subsection{\I{Age-length relationship}}
\input{IncludedSyntax/AgeLength}
\subsection{\I{Age-weight}}
\input{IncludedSyntax/AgeWeight}
\else
\subsection{\I{Growth-Increment}}
\input{IncludedSyntax/GrowthIncrement}
\fi 

\subsection{\I{Length-weight}}
\input{IncludedSyntax/LengthWeight}

\subsection{\I{Selectivities}}
\defComLab{selectivity}{Define an object of type \emph{Selectivity}}.
\defRef{sec:Selectivity}
\label{syntax:Selectivity}

\defSub{label}{The label for the selectivity}
\defType{String}
\defDefault{No default}

\defSub{type}{The type of selectivity}
\defType{String}
\defDefault{No default}

\defSub{length\_based}{Is the selectivity length based?}
\defType{Boolean}
\defDefault{false}

\defSub{intervals}{The number of quantiles to evaluate a length-based selectivity over the age-length distribution}
\defType{Non-negative integer}
\defDefault{5}

\defSub{values}{}
\defType{Vector of addressables}
\defDefault{No default}

\defSub{length\_values}{}
\defType{Vector of addressables}
\defDefault{No default}

\subsubsection{Selectivity of type All Values}
\commandlabsubarg{selectivity}{type}{All\_Values}.
\defRef{sec:Selectivity-AllValues}
\label{syntax:Selectivity-AllValues}

\defSub{v}{The v parameter}
\defType{Vector of real numbers (estimable)}
\defDefault{No default}

\subsubsection{Selectivity of type All Values Bounded}
\commandlabsubarg{selectivity}{type}{All\_Values\_Bounded}.
\defRef{sec:Selectivity-AllValuesBounded}
\label{syntax:Selectivity-AllValuesBounded}

\defSub{l}{The low value (L)}
\defType{Non-negative integer}
\defDefault{No default}

\defSub{h}{The high value (H)}
\defType{Non-negative integer}
\defDefault{No default}

\defSub{v}{The v parameter}
\defType{Vector of real numbers (estimable)}
\defDefault{No default}

\subsubsection{Selectivity of type Constant}
\commandlabsubarg{selectivity}{type}{Constant}.
\defRef{sec:Selectivity-Constant}
\label{syntax:Selectivity-Constant}

\defSub{a}{The $a$ value in $ax^b + c$}
\defType{Real number (estimable)}
\defDefault{0.0}
\defNote{The defaults result in a simple linear constant where $x=1$ for all values of $x$}

\defSub{b}{The $b$ value in $ax^b + c$}
\defType{Real number (estimable)}
\defDefault{0.0}
\defLowerBound{0.0 (inclusive)}
\defNote{The defaults result in a simple linear constant where $x=1$ for all values of $x$}

\defSub{c}{The $c$ value in $ax^b + c$}
\defType{Real number (estimable)}
\defDefault{1.0}
\defLowerBound{0.0 (inclusive)}
\defNote{The defaults result in a simple linear constant where $x=1$ for all values of $x$}

\defSub{beta}{The minimum age/length for which the selectivity applies}
\defType{Real number (constant)}
\defDefault{0.0}
\defLowerBound{0.0 (inclusive)}

\subsubsection{Selectivity of type Double Exponential}
\commandlabsubarg{selectivity}{type}{Double\_Exponential}.
\defRef{sec:Selectivity-DoubleExponential}
\label{syntax:Selectivity-DoubleExponential}

\defSub{x0}{The $x0$ parameter}
\defType{Real number (estimable)}
\defDefault{No default}

\defSub{x1}{The $x1$ parameter}
\defType{Real number (estimable)}
\defDefault{No default}

\defSub{x2}{The $x2$ parameter}
\defType{Real number (estimable)}
\defDefault{No default}

\defSub{y0}{The $y0$ parameter}
\defType{Real number (estimable)}
\defDefault{No default}
\defLowerBound{0.0 (inclusive)}

\defSub{y1}{The $y1$ parameter}
\defType{Real number (estimable)}
\defDefault{No default}
\defLowerBound{0.0 (inclusive)}

\defSub{y2}{The $y2$ parameter}
\defType{Real number (estimable)}
\defDefault{No default}
\defLowerBound{0.0 (inclusive)}

\defSub{alpha}{The maximum value of the selectivity}
\defType{Real number (estimable)}
\defDefault{1.0}
\defLowerBound{0.0 (exclusive)}

\defSub{beta}{The minimum age/length for which the selectivity applies}
\defType{Real number (constant)}
\defDefault{0}
\defLowerBound{0.0 (inclusive)}

\subsubsection{Selectivity of type Double Normal}
\commandlabsubarg{selectivity}{type}{Double\_Normal}.
\defRef{sec:Selectivity-DoubleNormal}
\label{syntax:Selectivity-DoubleNormal}

\defSub{mu}{The mean ($\mu$}
\defType{Real number (estimable)}
\defDefault{No default}

\defSub{sigma\_l}{The left-hand variance (sigma\_l) parameter}
\defType{Real number (estimable)}
\defDefault{No default}
\defLowerBound{0.0 (exclusive)}

\defSub{sigma\_r}{The right-hand variance (sigma\_r) parameter}
\defType{Real number (estimable)}
\defDefault{No default}
\defLowerBound{0.0 (exclusive)}

\defSub{alpha}{The maximum value of the selectivity}
\defType{Real number (estimable)}
\defDefault{1.0}
\defLowerBound{0.0 (exclusive)}

\defSub{beta}{The minimum age/length for which the selectivity applies}
\defType{Real number (constant)}
\defDefault{0}
\defLowerBound{0.0 (inclusive)}

\subsubsection{Selectivity of type Double Normal Plateau}
\commandlabsubarg{selectivity}{type}{Double\_Normal\_Plateau}.
\defRef{sec:Selectivity-DoubleNormalPlateau}
\label{syntax:Selectivity-DoubleNormalPlateau}

\defSub{a1}{The a1 ($a1$}
\defType{Real number (estimable)}
\defDefault{No default}

\defSub{a2}{The a2 ($a2$}
\defType{Real number (estimable)}
\defDefault{No default}

\defSub{sigma\_l}{The left-hand variance (sigma\_l) parameter}
\defType{Real number (estimable)}
\defDefault{No default}
\defLowerBound{0.0 (exclusive)}

\defSub{sigma\_r}{The right-hand variance (sigma\_r) parameter}
\defType{Real number (estimable)}
\defDefault{No default}
\defLowerBound{0.0 (exclusive)}

\defSub{alpha}{The maximum value of the selectivity}
\defType{Real number (estimable)}
\defDefault{1.0}
\defLowerBound{0.0 (exclusive)}

\defSub{beta}{The minimum age/length for which the selectivity applies}
\defType{Real number (constant)}
\defDefault{0}
\defLowerBound{0.0 (inclusive)}

\subsubsection{Selectivity of type Double Normal Stock Synthesis}
\commandlabsubarg{selectivity}{type}{Double\_Normal\_Stock\_Synthesis}.
\defRef{sec:Selectivity-DoubleNormalStockSynthesis}
\label{syntax:Selectivity-DoubleNormalStockSynthesis}

\defSub{peak}{Age or length of plateau (max selectivity)}
\defType{Real number (estimable)}
\defLowerBound{0.0 (exclusive)}

\defSub{y0}{Transformed selectivity for the first age or length bin}
\defType{Real number (estimable)}
\defLowerBound{-20}
\defUpperBound{0}

\defSub{y1}{Transformed selectivity for the last age or length bins}
\defType{Real number (estimable)}
\defLowerBound{-20}
\defUpperBound{10}

\defSub{descending}{The shape of descending limb in either ages or lengths}
\defType{Real number (estimable)}
\defDefault{No default}

\defSub{ascending}{The shape of ascending limb in either ages or lengths}
\defType{Real number (estimable)}
\defDefault{No default}

\defSub{width}{width of plateau how many ages or lengths are in the plateau}
\defType{Real number (estimable)}
\defDefault{No default}
\defLowerBound{0.0 (exclusive)}

\defSub{l}{min age or first length bin}
\defType{Real number}
\defDefault{No default}
\defLowerBound{0.0 (exclusive)}

\defSub{l}{max age or last length bin}
\defType{Real number}
\defDefault{No default}
\defLowerBound{0.0 (exclusive)}

\defSub{alpha}{The maximum value of the selectivity}
\defType{Real number (estimable)}
\defDefault{1.0}
\defLowerBound{0.0 (exclusive)}

\subsubsection{Selectivity of type Increasing}
\commandlabsubarg{selectivity}{type}{Increasing}.
\defRef{sec:Selectivity-Increasing}
\label{syntax:Selectivity-Increasing}

\defSub{l}{The low value (L)}
\defType{Non-negative integer}
\defDefault{No default}

\defSub{h}{The high value (H)}
\defType{Non-negative integer}
\defDefault{No default}

\defSub{v}{The v parameter}
\defType{Vector of real numbers (estimable)}
\defDefault{No default}

\defSub{alpha}{The maximum value of the selectivity }
\defType{Real number (estimable)}
\defDefault{1.0}
\defLowerBound{0.0 (exclusive)}

\subsubsection{Selectivity of type Inverse Logistic}
\commandlabsubarg{selectivity}{type}{Inverse\_Logistic}.
\defRef{sec:Selectivity-InverseLogistic}
\label{syntax:Selectivity-InverseLogistic}

\defSub{a50}{The age or length where the selectivity is \(50\%\)}
\defType{Real number (estimable)}
\defDefault{No default}

\defSub{ato95}{The age or length between \(50\%\) and \(95\%\) selective}
\defType{Real number (estimable)}
\defDefault{No default}
\defLowerBound{0.0 (exclusive)}

\defSub{alpha}{The maximum value of the selectivity }
\defType{Real number (estimable)}
\defDefault{1.0}
\defLowerBound{0.0 (exclusive)}

\defSub{beta}{The minimum age/length for which the selectivity applies}
\defType{Real number (constant)}
\defDefault{0}
\defLowerBound{0.0 (inclusive)}

\subsubsection{Selectivity of type Knife Edge}
\commandlabsubarg{selectivity}{type}{Knife\_Edge}.
\defRef{sec:Selectivity-KnifeEdge}
\label{syntax:Selectivity-KnifeEdge}

\defSub{e}{The edge value}
\defType{Real number (estimable)}
\defDefault{No default}

\defSub{alpha}{The maximum value of the selectivity }
\defType{Real number (estimable)}
\defDefault{1.0}

\subsubsection{Selectivity of type Logistic}
\commandlabsubarg{selectivity}{type}{Logistic}.
\defRef{sec:Selectivity-Logistic}
\label{syntax:Selectivity-Logistic}

\defSub{a50}{The age or length where the selectivity is \(50\%\)}
\defType{Real number (estimable)}
\defDefault{No default}

\defSub{ato95}{The age or length between \(50\%\) and \(95\%\) selective}
\defType{Real number (estimable)}
\defDefault{No default}
\defLowerBound{0.0 (exclusive)}

\defSub{alpha}{The maximum value of the selectivity}
\defType{Real number (estimable)}
\defDefault{1.0}
\defLowerBound{0.0 (exclusive)}

\defSub{beta}{The minimum age/length for which the selectivity applies}
\defType{Real number (constant)}
\defDefault{0}
\defLowerBound{0.0 (inclusive)}

\subsubsection{Selectivity of type Logistic Producing}
\commandlabsubarg{selectivity}{type}{Logistic\_Producing}.
\defRef{sec:Selectivity-LogisticProducing}
\label{syntax:Selectivity-LogisticProducing}

\defSub{l}{The low value (L)}
\defType{Non-negative integer}
\defDefault{No default}

\defSub{h}{The high value (H)}
\defType{Non-negative integer}
\defDefault{No default}

\defSub{a50}{The a50 parameter}
\defType{Real number (estimable)}
\defDefault{No default}

\defSub{ato95}{the ato95 parameter}
\defType{Real number (estimable)}
\defDefault{No default}
\defLowerBound{0.0 (exclusive)}

\defSub{alpha}{The maximum value of the selectivity}
\defType{Real number (estimable)}
\defDefault{1.0}
\defLowerBound{0.0 (exclusive)}

\subsubsection{Selectivity of type Compound Left }
\commandlabsubarg{selectivity}{type}{compound\_left}.
\defRef{sec:Selectivity-CompoundLeft}
\label{syntax:Selectivity-CompoundLeft}

\defSub{a50}{The a50 ($a50$}
\defType{Real number (estimable)}
\defDefault{No default}

\defSub{ato95}{The age or length between \(50\%\) and \(95\%\) selective}
\defType{Real number (estimable)}
\defDefault{No default}
\defLowerBound{0.0 (exclusive)}

\defSub{a\_min}{The (a\_min) parameter}
\defType{Real number (estimable)}
\defDefault{No default}
\defLowerBound{0.0 (exclusive)}

\defSub{left\_mean}{The left\_mean parameter}
\defType{Real number (estimable)}
\defDefault{1.0}
\defLowerBound{0.0 (exclusive)}

\defSub{sigma}{The sigma parameter}
\defType{Real number (estimable)}
\defDefault{1.0}
\defLowerBound{0.0 (exclusive)}


\subsubsection{Selectivity of type Compound Right }
\commandlabsubarg{selectivity}{type}{compound\_right}.
\defRef{sec:Selectivity-CompoundRight}
\label{syntax:Selectivity-CompoundRight}

\defSub{a50}{The a50 ($a50$}
\defType{Real number (estimable)}
\defDefault{No default}

\defSub{ato95}{The age or length between \(50\%\) and \(95\%\) selective}
\defType{Real number (estimable)}
\defDefault{No default}
\defLowerBound{0.0 (exclusive)}

\defSub{a\_min}{The (a\_min) parameter}
\defType{Real number (estimable)}
\defDefault{No default}
\defLowerBound{0.0 (exclusive)}

\defSub{left\_mean}{The left\_mean parameter}
\defType{Real number (estimable)}
\defDefault{1.0}
\defLowerBound{0.0 (exclusive)}

\defSub{to\_right\_mean}{The to\_right\_mean parameter}
\defType{Real number (estimable)}
\defDefault{1.0}
\defLowerBound{0.0 (exclusive)}

\defSub{sigma}{The sigma parameter}
\defType{Real number (estimable)}
\defDefault{1.0}
\defLowerBound{0.0 (exclusive)}

\subsubsection{Selectivity of type Compound Middle }
\commandlabsubarg{selectivity}{type}{compound\_middle}.
%\defRef{sec:Selectivity-CompoundMiddle}
\label{syntax:Selectivity-CompoundMiddle}

\defSub{a50}{The a50 ($a50$}
\defType{Real number (estimable)}
\defDefault{No default}

\defSub{ato95}{The age or length between \(50\%\) and \(95\%\) selective}
\defType{Real number (estimable)}
\defDefault{No default}
\defLowerBound{0.0 (exclusive)}

\defSub{a\_min}{The (a\_min) parameter}
\defType{Real number (estimable)}
\defDefault{No default}
\defLowerBound{0.0 (exclusive)}

\defSub{left\_mean}{The left\_mean parameter}
\defType{Real number (estimable)}
\defDefault{1.0}
\defLowerBound{0.0 (exclusive)}

\defSub{to\_right\_mean}{The to\_right\_mean parameter}
\defType{Real number (estimable)}
\defDefault{1.0}
\defLowerBound{0.0 (exclusive)}

\defSub{sigma}{The sigma parameter}
\defType{Real number (estimable)}
\defDefault{1.0}
\defLowerBound{0.0 (exclusive)}

\subsubsection{Selectivity of type Compound All }
\commandlabsubarg{selectivity}{type}{compound\_all}.
\defRef{sec:Selectivity-CompoundAll}
\label{syntax:Selectivity-CompoundAll}

\defSub{a50}{The a50 ($a50$}
\defType{Real number (estimable)}
\defDefault{No default}

\defSub{ato95}{The age or length between \(50\%\) and \(95\%\) selective}
\defType{Real number (estimable)}
\defDefault{No default}
\defLowerBound{0.0 (exclusive)}

\defSub{a\_min}{The (a\_min) parameter}
\defType{Real number (estimable)}
\defDefault{No default}
\defLowerBound{0.0 (exclusive)}

\defSub{sigma}{The sigma parameter}
\defType{Real number (estimable)}
\defDefault{1.0}
\defLowerBound{0.0 (exclusive)}

\subsubsection{Selectivity of type Multi-Selectivity}
\commandlabsubarg{Selectivity}{type}{multi\_selectivity}.
\defRef{sec:Selectivity-MultiSelectivities}
\label{syntax:Selectivity-MultiSelectivities}

\defSub{years}{The years for which we want to apply the corresponding selectivity in}
\defType{Vector of integer for all model years to apply corresponding selectivity.}
\defDefault{No default}

\defSub{selectivity\_labels}{The labels of the selectivities, one for each year}
\defType{Vector of strings defining the labels of the selectivities to be used for each year}
\defDefault{No default}

\defSub{default\_selectivity}{The selectivity used in missing years}
\defType{string}
\defDefault{No default}

\defSub{projection\_selectivity}{The selectivity used in missing years in projections}
\defType{string}
\defDefault{Defaults to \argument{default\_selectivity} if not supplied}


\subsection{\I{Projections}}
\defComLab{project}{Define an object of type \emph{Project}}.
\defRef{sec:Project}
\label{syntax:Project}

\defSub{label}{Label}
\defType{String}
\defDefault{No default}

\defSub{type}{Type}
\defType{String}
\defDefault{No default}

\defSub{years}{Years to recalculate the values}
\defType{Vector of non-negative integers}
\defDefault{No default}

\defSub{parameter}{Parameter to project}
\defType{String}
\defDefault{No default}

\subsubsection{Project of type Constant}
\commandlabsubarg{project}{type}{Constant}.
\defRef{sec:Project-Constant}
\label{syntax:Project-Constant}

\defSub{values}{The values to assign to the addressable}
\defType{Vector of real numbers (estimable)}
\defDefault{No default}

\defSub{multiplier}{Multiplier that is applied to the projected value}
\defType{Real number (estimable)}
\defDefault{1.0}
\defLowerBound{0.0 (exclusive)}
\defValue{A vector of length 1 (for a constant value for all years), or a vector of length years (for a specific value for each year)}

\subsubsection{Project of type Multiple Values}
\commandlabsubarg{project}{type}{multiple\_values}.
\defRef{sec:Project-MultipleValues}
\label{syntax:Multiple-Values}

\defSub{multiplier}{Multiplier that is applied to the projected value}
\defType{Real number (estimable)}
\defDefault{1.0}
\defLowerBound{0.0 (exclusive)}
\defValue{A vector of length 1 (for a constant value for all years), or a vector of length years (for a specific value for each year)}

\defSub{table values}{The table of data specifying the projected values to use for each row of the supplied free parameter file (\texttt{-i} or \texttt{-I}) with one column of data for each projected year}
\defType{table with label = values}
\defDefault{No default}
\defValue{A $n_i \times n_y$ matrix. Where \(n_i\) is the number of rows (parameter sets) in the free parameter file. And \(n_y\) is the number of projection years defined by the input \texttt{years}}
\defNote{example below}
\begin{verbatim}
	@project future_disease_rates
	type multiple_values
	parameter process[dtransition].proportions{disease}  	
	years 2024:2026
	table values
	# 2024 2025 2026
	   0.1  0.2  0.3
	   0.3  0.4  0.5
	end_table
\end{verbatim}

\subsubsection{Project of type Empirical Sampling}
\commandlabsubarg{project}{type}{Empirical\_Sampling}.
\defRef{sec:Project-EmpiricalSampling}
\label{syntax:Project-EmpiricalSampling}

\defSub{start\_year}{The start year of sampling}
\defType{Non-negative integer}
\defDefault{No default}

\defSub{final\_year}{The final year of sampling}
\defType{Non-negative integer}
\defDefault{No default}

\defSub{multiplier}{Multiplier that is applied to the projected value}
\defType{Real number (estimable)}
\defDefault{1.0}
\defLowerBound{0.0 (exclusive)}
\defValue{A vector of length 1 (for a constant value for all years), or a vector of length years (for a specific value for each year)}

\subsubsection{Project of type Lognormal}
\commandlabsubarg{project}{type}{Lognormal}.
\defRef{sec:Project-LogNormal}
\label{syntax:Project-LogNormal}

\defSub{mean}{The mean of the lognormal process}
\defType{Real number (estimable)}
\defDefault{0.0}

\defSub{sigma}{The standard deviation (sigma) of the lognormal sampling}
\defType{Real number (estimable)}
\defDefault{No default}
\defLowerBound{0.0 (inclusive)}

\defSub{multiplier}{Multiplier that is applied to the projected value}
\defType{Real number (estimable)}
\defDefault{1.0}
\defLowerBound{0.0 (exclusive)}
\defValue{A vector of length 1 (for a constant value for all years), or a vector of length years (for a specific value for each year)}

\subsubsection{Project of type Lognormal Empirical}
\commandlabsubarg{project}{type}{Lognormal\_Empirical}.
\defRef{sec:Project-LogNormalEmpirical}
\label{syntax:Project-LogNormalEmpirical}

\defSub{mean}{The mean of the Gaussian process}
\defType{Real number (estimable)}
\defDefault{0.0}

\defSub{start\_year}{The start year of sampling}
\defType{Non-negative integer}
\defDefault{No default}

\defSub{final\_year}{The final year of sampling}
\defType{Non-negative integer}
\defDefault{No default}

\defSub{multiplier}{Multiplier that is applied to the projected value}
\defType{Real number (estimable)}
\defDefault{1.0}
\defLowerBound{0.0 (exclusive)}
\defValue{A vector of length 1 (for a constant value for all years), or a vector of length years (for a specific value for each year)}

\subsubsection{Project of type Harvest Strategy Constant Catch}
\commandlabsubarg{Project}{type}{harvest\_strategy\_constant\_catch}.
\defRef{sec:Project-HarvestStrategyConstantCatch}
\label{syntax:Project-HarvestStrategyConstantCatch}

\defSub{catch}{The catch to apply}
\defType{Real number (estimable)}
\defDefault{0.0}
\defLowerBound{0 (inclusive)}

\defSub{alpha}{The multiplier on the proportional change in biomass applied to the catch}
\defType{Real number (estimable)}
\defDefault{1.0}
\defLowerBound{0 (exclusive)}

\defSub{min\_delta}{The minimum difference (proportion) in catch required before it is updated}
\defType{Real number (estimable)}
\defDefault{0.0}
\defLowerBound{0 (inclusive)}

\defSub{max\_delta}{The maximum difference (proportion) in catch that can be applied}
\defType{Real number (estimable)}
\defDefault{0.0}
\defLowerBound{0 (inclusive)}
\defNote(Use max\_delta = 0 to have no maximum)

\defSub{update\_frequency\_years}{The number of years between updates}
\defType{Non-negative integer}
\defDefault{1}
\defLowerBound{1 (inclusive)}

\defSub{biomass\_index\_lag\_years}{The lag (years) of the derived\_quantity that is used for the calculation of the catch}
\defType{Non-negative integer}
\defDefault{1}
\defLowerBound{1 (inclusive)}

\defSub{current\_catch}{The current catch to apply at the start of the projections (applied until first\_year)}
\defType{Real number (estimable)}
\defDefault{0}
\defLowerBound{0 (inclusive)}

\defSub{multiplier}{Multiplier that is applied to the projected value}
\defType{Vector of real numbers (estimable)}
\defDefault{1.0}
\defLowerBound{0 (exclusive)}

\defSub{first\_year}{The first year in which to consider an update using the harvest strategy rule}
\defType{Non-negative integer}
\defDefault{0}

\subsubsection{Project of type Harvest Strategy Constant U}
\commandlabsubarg{Project}{type}{Harvest\_Strategy\_Constant\_U}.
\defRef{sec:Project-HarvestStrategyConstantU}
\label{syntax:Project-HarvestStrategyConstantU}

\defSub{u}{The exploitation rate to apply}
\defType{Real number (estimable)}
\defDefault{0.0}
\defLowerBound{0 (inclusive)}

\defSub{min\_delta}{The minimum difference (proportion) in catch required before it is updated}
\defType{Real number (estimable)}
\defDefault{0.0}
\defLowerBound{0 (inclusive)}

\defSub{max\_delta}{The maximum difference (proportion) in catch that can be applied}
\defType{Real number (estimable)}
\defDefault{0.0}
\defLowerBound{0 (inclusive)}
\defNote(Use max\_delta = 0 to have no maximum)

\defSub{update\_frequency\_years}{The number of years between updates}
\defType{Non-negative integer}
\defDefault{1}
\defLowerBound{1 (inclusive)}

\defSub{biomass\_index\_lag\_years}{The lag (years) of the derived\_quantity that is used for the calculation of the catch}
\defType{Non-negative integer}
\defDefault{1}
\defLowerBound{1 (inclusive)}

\defSub{current\_catch}{The current catch to apply at the start of the projections (applied until first\_year)}
\defType{Real number (estimable)}
\defDefault{0}
\defLowerBound{0 (inclusive)}

\defSub{multiplier}{Multiplier that is applied to the projected value}
\defType{Vector of real numbers (estimable)}
\defDefault{1.0}
\defLowerBound{0 (exclusive)}

\defSub{first\_year}{The first year in which to consider an update using the harvest strategy rule}
\defType{Non-negative integer}
\defDefault{0}

\subsubsection{Project of type Harvest Strategy Ramp U}
\commandlabsubarg{Project}{type}{harvest\_strategy\_ramp\_u}.
\defRef{sec:Project-HarvestStrategyRampU}
\label{syntax:Project-HarvestStrategyRampU}

\defSub{u}{The exploitation rates to apply}
\defType{Vector of real numbers (estimable)}
\defDefault{0.0}
\defLowerBound{0 (inclusive)}

\defSub{reference\_points}{The reference points for each exploitation rate}
\defType{Vector of real numbers (estimable)}
\defDefault{0.0}
\defLowerBound{0 (inclusive)}

\defSub{reference\_index}{The biomass index for reference points (i.e., the derived quantity label for the calculation of reference points)}
\defType{String}
\defDefault{No default}

\defSub{min\_delta}{The minimum difference (proportion) in catch required before it is updated}
\defType{Real number (estimable)}
\defDefault{0.0}
\defLowerBound{0 (inclusive)}

\defSub{max\_delta}{The maximum difference (proportion) in catch that can be applied}
\defType{Real number (estimable)}
\defDefault{0.0}
\defLowerBound{0 (inclusive)}
\defNote(Use max\_delta = 0 to have no maximum)

\defSub{update\_frequency\_years}{The number of years between updates}
\defType{Non-negative integer}
\defDefault{1}
\defLowerBound{1 (inclusive)}

\defSub{biomass\_index\_lag\_years}{The lag (years) of the derived\_quantity that is used for the calculation of the catch}
\defType{Non-negative integer}
\defDefault{1}
\defLowerBound{1 (inclusive)}

\defSub{current\_catch}{The current catch to apply at the start of the projections (applied until first\_year)}
\defType{Real number (estimable)}
\defDefault{0}
\defLowerBound{0 (inclusive)}

\defSub{multiplier}{Multiplier that is applied to the projected value}
\defType{Vector of real numbers (estimable)}
\defDefault{1.0}
\defLowerBound{0 (exclusive)}

\defSub{first\_year}{The first year in which to consider an update using the harvest strategy rule}
\defType{Non-negative integer}
\defDefault{0}

%\subsubsection{Project of type User Defined}
%\commandlabsubarg{project}{type}{User\_Defined}.
%\defRef{sec:Project-UserDefined}
%\label{syntax:Project-UserDefined}
%
%\defSub{equation}{The equation to do a test run of}
%\defType{Vector of strings}
%\defDefault{No default}
%


 \pagebreak 
 \section{\I{Estimation command and subcommand syntax}\label{syntax:Estimation}}

The description of methods for the estimation section is given in Section \ref{sec:Estimation}.

In the following section, the sub-section headers use a notation of the form "\textbf {@observation[label].type=abundance}" which, in this case, represents the input command fragment
{\small{\begin{verbatim}
@observation label # where label is a unique label for that observation
type=abundance
...
\end{verbatim}}}
The specific subcommands for a command are given within each command.

\subsection{\I{Estimation methods}}
\defComLab{estimate}{Define an object of type \emph{Estimate}}.
\defRef{sec:Estimation}
\label{syntax:Estimate}

\defSub{label}{The label of the estimate}
\defType{String}
\defDefault{No default}

\defSub{type}{The type of prior for the estimate}
\defType{String}
\defDefault{No default}

\defSub{parameter}{The name of the parameter to estimate}
\defType{String}
\defDefault{No default}

\defSub{lower\_bound}{The lower bound for the parameter}
\defType{Real number (estimable)}
\defDefault{No default}

\defSub{upper\_bound}{The upper bound for the parameter}
\defType{Real number (estimable)}
\defDefault{No default}

\defSub{same}{List of other parameters that are constrained to have the same value as this parameter}
\defType{Vector of strings}
\defDefault{No default}

\defSub{estimation\_phase}{The first estimation phase to allow this to be estimated}
\defType{Non-negative integer}
\defDefault{1}
\defValue{Phases must be number sequentially and start at one}

\defSub{mcmc\_fixed}{Indicates if this parameter is estimated at the point estimate but fixed during MCMC estimation run}
\defType{Boolean}
\defDefault{false}

\subsubsection{Estimate of type Uniform}
\commandlabsubarg{estimate}{type}{Uniform}.
\defRef{sec:Prior-Uniform}
\label{syntax:Estimate-Uniform}

The Uniform type has no additional subcommands.
\subsubsection{Estimate of type Uniform-Log}
\commandlabsubarg{estimate}{type}{Uniform\_Log}.
\defRef{sec:Prior-UniformLog}
\label{syntax:Estimate-UniformLog}

The Uniform\_Log type has no additional subcommands.

\subsubsection{Estimate of prior type Normal}
\commandlabsubarg{estimate}{type}{Normal}.
\defRef{sec:Prior-Normal}
\label{syntax:Estimate-Normal}

\defSub{mu}{The normal prior mean (mu) parameter}
\defType{Real number (estimable)}
\defDefault{No default}

\defSub{cv}{The normal standard deviation (sigma) parameter}
\defType{Real number (estimable)}
\defDefault{No default}
\defLowerBound{0.0 (exclusive)}

\subsubsection{Estimate of prior type Normal By Stdev}
\commandlabsubarg{estimate}{type}{Normal\_By\_Stdev}.
\defRef{sec:Prior-NormalByStdev}
\label{syntax:Estimate-NormalByStdev}

\defSub{mu}{The normal prior mean (mu) parameter}
\defType{Real number (estimable)}
\defDefault{No default}

\defSub{sigma}{The normal standard deviation (sigma) parameter}
\defType{Real number (estimable)}
\defDefault{No default}
\defLowerBound{0.0 (exclusive)}

\defSub{lognormal\_transformation}{Add a Jacobian if the derived outcome of the estimate is assumed to be lognormal, e.g., used for recruitment deviations in the recruitment process. See the User Manual for more information}
\defType{Boolean}
\defDefault{false}

\subsubsection{Estimate of prior type Lognormal}
\commandlabsubarg{estimate}{type}{Lognormal}.
\defRef{sec:Prior-Lognormal}
\label{syntax:Estimate-Lognormal}

\defSub{mu}{The lognormal prior mean (mu) parameter}
\defType{Real number (estimable)}
\defDefault{No default}
\defLowerBound{0.0 (exclusive)}

\defSub{cv}{The lognormal variance (cv) parameter}
\defType{Real number (estimable)}
\defDefault{No default}
\defLowerBound{0.0 (exclusive)}

\subsubsection{Estimate of prior type Normal-Log}
\commandlabsubarg{estimate}{type}{Normal\_Log}.
\defRef{sec:Prior-NormalLog}
\label{syntax:Estimate-NormalLog}

\defSub{mu}{The normal-log prior mean (mu) parameter}
\defType{Real number (estimable)}
\defDefault{No default}

\defSub{sigma}{The normal-log prior standard deviation (sigma) parameter}
\defType{Real number (estimable)}
\defDefault{No default}
\defLowerBound{0.0 (exclusive)}

\subsubsection{Estimate of prior type Beta}
\commandlabsubarg{estimate}{type}{Beta}.
\defRef{sec:Prior-Beta}
\label{syntax:Estimate-Beta}

\defSub{mu}{Beta prior  mean (mu) parameter}
\defType{Real number (estimable)}
\defDefault{No default}

\defSub{sigma}{Beta prior standard deviation (sigma) parameter}
\defType{Real number (estimable)}
\defDefault{No default}
\defLowerBound{0.0 (exclusive)}

\defSub{a}{Beta prior lower bound of the range (A) parameter}
\defType{Real number (estimable)}
\defDefault{No default}

\defSub{b}{Beta prior upper bound of the range (B) parameter}
\defType{Real number (estimable)}
\defDefault{No default}

\subsubsection{Estimate of prior type Student's t}
\commandlabsubarg{estimate}{type}{students\_t}.
\defRef{sec:Prior-Studentst}
\label{syntax:Estimate-Studentst}

\defSub{mu}{The Student's t prior location (mu) parameter}
\defType{Real number (estimable)}
\defDefault{No default}

\defSub{sigma}{The Student's t scale (sigma) parameter}
\defType{Real number (estimable)}
\defDefault{No default}
\defLowerBound{0.0 (exclusive)}

\defSub{df}{The Student's t degrees of freedom (df) parameter}
\defType{Real number (constant)}
\defDefault{No default}
\defLowerBound{0.0 (exclusive)}


\subsection{\I{Point estimation}}
\defComLab{minimiser}{Define an object of type \emph{Minimiser}}.
\defRef{sec:Minimiser}
\label{syntax:Minimiser}

\defSub{label}{The minimiser label}
\defType{String}
\defDefault{No default}

\defSub{type}{The type of minimiser to use}
\defType{String}
\defDefault{No default}

\defSub{active}{Indicates if this minimiser is active}
\defType{Boolean}
\defDefault{false}

\defSub{covariance}{Indicates if a covariance matrix should be generated}
\defType{Boolean}
\defDefault{true}

\subsubsection{Minimiser of type ADOLC}
\commandlabsubarg{minimiser}{type}{ADOLC}.
\defRef{sec:Minimiser-ADOLC}
\label{syntax:Minimiser-ADOLC}

\defSub{iterations}{The maximum number of iterations}
\defType{Integer}
\defDefault{1000}
\defLowerBound{1 (inclusive)}

\defSub{evaluations}{The maximum number of evaluations}
\defType{Integer}
\defDefault{4000}
\defLowerBound{1 (inclusive)}

\defSub{tolerance}{The tolerance of the gradient for convergence}
\defType{Real number}
\defDefault{1e-5}
\defLowerBound{0.0 (exclusive)}

\defSub{step\_size}{The minimum step-size before minimisation fails}
\defType{Real number}
\defDefault{1e-7}
\defLowerBound{0.0 (exclusive)}

\defSub{parameter\_transformation}{The choice of parametrisation used to scale the parameters for ADOLC}
\defType{string}
\defDefault{sin\_transformation}
\defValue{Either \subcommand{sin\_transform} or \subcommand{tan\_transform}. See \ref{sec:Minimiser-ADOLC} for more information}

\subsubsection{Minimiser of type Betadiff}
\commandlabsubarg{minimiser}{type}{Betadiff}.
\defRef{sec:Minimiser-BetaDiff}
\label{syntax:Minimiser-BetaDiff}

\defSub{iterations}{The maximum number of iterations}
\defType{Integer}
\defDefault{1000}
\defLowerBound{1 (inclusive)}

\defSub{evaluations}{The maximum number of evaluations}
\defType{Integer}
\defDefault{4000}
\defLowerBound{1 (inclusive)}

\defSub{tolerance}{The tolerance of the gradient for convergence}
\defType{Real number}
\defDefault{1e-5}
\defLowerBound{0.0 (exclusive)}

\subsubsection{Minimiser of type DESolver}
\commandlabsubarg{minimiser}{type}{de\_solver}.
\defRef{sec:Minimiser-DESolver}
\label{syntax:Minimiser-DESolver}

\defSub{population\_size}{The number of candidate solutions to have in the population}
\defType{Non-negative integer}
\defDefault{25}
\defLowerBound{1 (inclusive)}

\defSub{crossover\_probability}{The minimiser's crossover probability}
\defType{Real number}
\defDefault{0.9}
\defLowerBound{0.0 (inclusive)}
\defUpperBound{1.0 (inclusive)}

\defSub{difference\_scale}{The scale to apply to new solutions when comparing candidates}
\defType{Real number}
\defDefault{0.02}

\defSub{max\_generations}{The maximum number of iterations to run}
\defType{1000}
\defDefault{No default}

\defSub{tolerance}{The total variance between the population and best candidate before acceptance}
\defType{Real number}
\defDefault{1e-5}
\defLowerBound{0.0 (exclusive)}

\subsubsection{Minimiser of type Deltadiff}
\commandlabsubarg{minimiser}{type}{Deltadiff}.
\defRef{sec:Minimiser-DeltaDiff}
\label{syntax:Minimiser-DeltaDiff}

\defSub{iterations}{Maximum number of iterations}
\defType{Integer}
\defDefault{1000}
\defLowerBound{1 (inclusive)}

\defSub{evaluations}{Maximum number of evaluations}
\defType{Integer}
\defDefault{4000}
\defLowerBound{1 (inclusive)}

\defSub{tolerance}{Tolerance of the gradient for convergence}
\defType{Real number}
\defDefault{1e-5}
\defLowerBound{0 (exclusive)}

\defSub{step\_size}{Minimum Step-size before minimisation fails}
\defType{Real number}
\defDefault{1e-7}
\defLowerBound{0 (exclusive)}

\subsubsection{Minimiser of type Numerical Differences}
\commandlabsubarg{minimiser}{type}{Numerical\_Differences}.
\defRef{sec:Minimiser-GammaDiff}
\label{syntax:Minimiser-GammaDiff}

\defSub{iterations}{The maximum number of iterations}
\defType{Integer}
\defDefault{1000}
\defLowerBound{1 (inclusive)}

\defSub{evaluations}{The maximum number of evaluations}
\defType{Integer}
\defDefault{4000}
\defLowerBound{1 (inclusive)}

\defSub{tolerance}{The tolerance of the gradient for convergence}
\defType{Real number}
\defDefault{1e-5}
\defLowerBound{0.0 (exclusive)}

\defSub{step\_size}{The minimum step-size before minimisation fails}
\defType{Real number}
\defDefault{1e-7}
\defLowerBound{0.0 (exclusive)}



\subsection{\I{Markov chain Monte Carlo (MCMC)}}
\defComLab{mcmc}{Define an object of type \emph{MCMC}}.
\defRef{sec:MCMC}
\label{syntax:MCMC}

\defSub{label}{The label of the MCMC}
\defType{String}
\defDefault{No default}

\defSub{type}{The MCMC method}
\defType{String}
\defDefault{No default}

\defSub{length}{The number of iterations for the MCMC (including the burn in period)}
\defType{Non-negative integer}
\defDefault{No default}
\defLowerBound{1 (inclusive)}

\defSub{burn\_in}{The number of iterations for the burn\_in period of the MCMC}
\defType{Non-negative integer}
\defDefault{0}
\defLowerBound{0 (inclusive)}

\defSub{active}{Indicates if this is the active MCMC algorithm}
\defType{Boolean}
\defDefault{true}

\defSub{step\_size}{Initial step-size (as a multiplier of the approximate covariance matrix)}
\defType{Real number}
\defDefault{The default is $2.4d^{-0.5}$}
\defLowerBound{0 (inclusive)}
\defNote{If the value is set to zero or the subcommand is omitted, the default value is used instead}

\defSub{start}{The covariance multiplier for the starting point of the MCMC}
\defType{Real number}
\defDefault{0.0}
\defLowerBound{0.0 (inclusive)}
\defValue{If zero, then the MCMC starts at the point estimate (i.e., the MPD). Otherwise a random (multivariate normal) jump from the point estimate with \subcommand{start} used as the standard deviation multiplier}

\defSub{adjust\_parameters\_at\_bounds}{Adjust the start point for parameters at bounds}
\defType{Boolean}
\defDefault{false}
\defValue{If true, then the MCMC will adjust the start point of any parameters at a bound to a random uniform location between the lower and upper bound}

\defSub{keep}{The spacing between recorded values in the MCMC}
\defType{Non-negative integer}
\defDefault{1}
\defLowerBound{1 (inclusive)}

\defSub{max\_correlation}{The maximum absolute correlation in the covariance matrix of the proposal distribution}
\defType{Real number}
\defDefault{0.8}
\defLowerBound{0.0 (exclusive)}
\defUpperBound{1.0 (inclusive)}

\defSub{covariance\_adjustment\_method}{The method for adjusting small variances in the covariance proposal matrix}
\defType{String}
\defDefault{correlation}
\defValue{Either covariance, correlation, or none}

\defSub{correlation\_adjustment\_diff}{The minimum non-zero variance times the range of the bounds in the covariance matrix of the proposal distribution}
\defType{Real number}
\defDefault{0.0001}
\defLowerBound{0.0 (exclusive)}

\defSub{proposal\_distribution}{The shape of the proposal distribution (either the t or the normal distribution)}
\defType{String}
\defDefault{t}
\defValue{Either t or normal}

\defSub{df}{The degrees of freedom of the multivariate t proposal distribution}
\defType{Non-negative integer}
\defDefault{4}
\defLowerBound{1}

\defSub{adapt\_stepsize\_at}{The iteration numbers in which to check and resize the MCMC step-size}
\defType{Vector of non-negative integers}
\defDefault{true}
\defLowerBound{0 (inclusive)}
\defValue{If zero, then no step-size adaption is applied. Otherwise must be a positive integer less than less than the burn\_in}

\defSub{adapt\_stepsize\_method}{The method to use to adapt the step-size}
\defType{String}
\defDefault{ratio}
\defValue{Either \subcommand{double\_half} or \subcommand{ratio}}

\defSub{adapt\_covariance\_matrix\_at}{The iteration number in which to adapt the covariance matrix}
\defType{Non-negative integer}
\defDefault{0}
\defLowerBound{0 (inclusive)}
\defValue{If zero, then no covariance matrix adaption is applied. Otherwise must be a positive integer that is less than the burn\_in}

\subsubsection{MCMC of type Hamiltonian Monte Carlo}
\commandlabsubarg{mcmc}{type}{Hamiltonian}.
\label{syntax:MCMC-HamiltonianMonteCarlo}

\defSub{leapfrog\_steps}{Number of leapfrog steps}
\defType{Non-negative integer}
\defDefault{1}
\defLowerBound{0 (exclusive)}

\defSub{leapfrog\_delta}{Amount to leapfrog per step}
\defType{Real number}
\defDefault{1e-7}
\defLowerBound{0 (exclusive)}

\defSub{gradient\_step\_size}{Step-size to use when calculating gradient}
\defType{Real number}
\defDefault{1e-7}
\defLowerBound{1e-13 (inclusive)}

\subsubsection{MCMC of type Random Walk Metropolis Hastings}
\commandlabsubarg{mcmc}{type}{Random\_Walk}.
\defRef{sec:MCMC-RandomWalkMetropolisHastings}
\label{syntax:MCMC-RandomWalkMetropolisHastings}

The Random\_Walk type has no additional subcommands.


\subsection{\I{Posterior profiles}}
\defComLab{profile}{Define an object of type \emph{Profile}}.
\defRef{sec:Profile}
\label{syntax:Profile}

\defSub{label}{The label of the profile}
\defType{String}
\defDefault{No default}

\defSub{steps}{The number of steps between the lower and upper bound}
\defType{Non-negative integer}
\defDefault{No default}
\defValue{A positive integer $\ge 2$}

\defSub{lower\_bound}{The lower value of the range}
\defType{Real number (estimable)}
\defDefault{No default}

\defSub{upper\_bound}{The upper value of the range}
\defType{Real number (estimable)}
\defDefault{No default}

\defSub{parameter}{The free parameter to profile}
\defType{String}
\defDefault{No default}

\defSub{same}{A free parameter that is constrained to have the same value as the parameter being profiled}
\defType{String}
\defDefault{No default}

\defSub{transformation}{The transformation to apply to the upper and lower bounds}
\defType{String}
\defDefault{'none'}
\defValue{'none', 'log', 'square\_root', and 'inverse'}
\defNote{This specifies that the upper and lower bounds should be transformed from natural space into the transformed space before being evaluated and used for the profile}


\subsection{\I{Catchability constants}}
\input{IncludedSyntax/Catchability}

\subsection{\I{Penalties}}
\input{IncludedSyntax/Penalty}

\subsection{\I{Additional priors}}
\defComLab{Additional\_Prior}{Define an object of type \emph{Additional\_Prior}}.
\defRef{sec:AdditionalPrior}
\label{syntax:AdditionalPrior}

\defSub{label}{The label for the additional prior}
\defType{String}
\defDefault{No default}

\defSub{type}{The additional prior type}
\defType{String}
\defDefault{No default}

\subsubsection{Additional\_Prior of type Beta}
\commandlabsubarg{Additional\_Prior}{type}{Beta}.
\defRef{sec:AdditionalPrior-Beta}
\label{syntax:AdditionalPrior-Beta}

\defSub{parameter}{The name of the parameter for the additional prior}
\defType{String}
\defDefault{No default}

\defSub{mu}{Beta distribution mean (mu) parameter}
\defType{Real number (estimable)}
\defDefault{No default}

\defSub{sigma}{Beta distribution variance (sigma) parameter}
\defType{Real number (estimable)}
\defDefault{No default}
\defLowerBound{0.0 (exclusive)}

\defSub{a}{Beta distribution lower bound, of the range (A) parameter}
\defType{Real number (estimable)}
\defDefault{No default}

\defSub{b}{Beta distribution upper bound of the range (B) parameter}
\defType{Real number (estimable)}
\defDefault{No default}

\subsubsection{Additional\_Prior of type Element\_Difference}
\commandlabsubarg{Additional\_Prior}{type}{Element\_Difference}.
\defRef{sec:AdditionalPrior-ElementDifference}
\label{syntax:AdditionalPrior-ElementDifference}

\defSub{parameter}{The name of the parameter for the additional prior}
\defType{String}
\defDefault{No default}

\defSub{second\_parameter}{The name of the second parameter for comparing}
\defType{String}
\defDefault{No default}

\defSub{multiplier}{Multiply the penalty by this factor}
\defType{Real number (estimable)}
\defDefault{1}

\subsubsection{Additional\_Prior of type Log\_Normal}
\commandlabsubarg{Additional\_Prior}{type}{Log\_Normal}.
\defRef{sec:AdditionalPrior-LogNormal}
\label{syntax:AdditionalPrior-LogNormal}

\defSub{parameter}{The name of the parameter for the additional prior}
\defType{String}
\defDefault{No default}

\defSub{mu}{The lognormal prior mean (mu) parameter}
\defType{Real number (estimable)}
\defDefault{No default}
\defLowerBound{0.0 (exclusive)}

\defSub{cv}{The lognormal CV parameter}
\defType{Real number (estimable)}
\defDefault{No default}
\defLowerBound{0.0 (exclusive)}

\subsubsection{Additional\_Prior of type Ratio}
\commandlabsubarg{Additional\_Prior}{type}{Ratio}.
\defRef{sec:AdditionalPrior-Ratio}
\label{syntax:AdditionalPrior-Ratio}

\defSub{parameter}{The name of the parameter for the additional prior}
\defType{String}
\defDefault{No default}

\defSub{second\_parameter}{The name of the parameter on the denominator}
\defType{String}
\defDefault{No default}

\defSub{mu}{The lognormal prior mean (mu) of the ratio}
\defType{Real number (estimable)}
\defDefault{No default}
\defLowerBound{0.0 (exclusive)}

\defSub{cv}{The lognormal CV parameter for the ratio}
\defType{Real number (estimable)}
\defDefault{No default}
\defLowerBound{0.0 (exclusive)}

\subsubsection{Additional\_Prior of type Sum}
\commandlabsubarg{Additional\_Prior}{type}{Sum}.
\defRef{sec:AdditionalPrior-Sum}
\label{syntax:AdditionalPrior-Sum}

\defSub{parameters}{The names of the parameters for summing}
\defType{Vector of strings}
\defDefault{No default}

\defSub{distribution}{The additional prior distribution to apply}
\defType{String}
\defDefault{lognormal}

\defSub{mu}{Mean of the distribution}
\defType{Real number (estimable)}
\defDefault{1.0}
\defLowerBound{0.0 (exclusive)}

\defSub{cv}{cv of the distribution}
\defType{Real number (estimable)}
\defDefault{No default}
\defLowerBound{0.0 (exclusive)}

\subsubsection{Additional\_Prior of type Uniform}
\commandlabsubarg{Additional\_Prior}{type}{Uniform}.
\defRef{sec:AdditionalPrior-Uniform}
\label{syntax:AdditionalPrior-Uniform}

\defSub{parameter}{The name of the parameter for the additional prior}
\defType{String}
\defDefault{No default}

\subsubsection{Additional\_Prior of type Uniform\_Log}
\commandlabsubarg{Additional\_Prior}{type}{Uniform\_Log}.
\defRef{sec:AdditionalPrior-UniformLog}
\label{syntax:AdditionalPrior-UniformLog}

\defSub{parameter}{The name of the parameter for the additional prior}
\defType{String}
\defDefault{No default}

\subsubsection{Additional\_Prior of type Vector\_Average}
\commandlabsubarg{Additional\_Prior}{type}{Vector\_Average}.
\defRef{sec:AdditionalPrior-VectorAverage}
\label{syntax:AdditionalPrior-VectorAverage}

\defSub{parameter}{The name of the parameter for the additional prior}
\defType{String}
\defDefault{No default}

\defSub{method}{Which calculation method to use: k, l, or m}
\defType{String}
\defDefault{k}

\defSub{k}{The k value to use in the calculation}
\defType{Real number (estimable)}
\defDefault{No default}

\defSub{multiplier}{Multiplier for the penalty amount}
\defType{Real number (estimable)}
\defDefault{1}

\subsubsection{Additional\_Prior of type Vector\_Smoothing}
\commandlabsubarg{Additional\_Prior}{type}{Vector\_Smoothing}.
\defRef{sec:AdditionalPrior-VectorSmoothing}
\label{syntax:AdditionalPrior-VectorSmoothing}

\defSub{parameter}{The name of the parameter for the additional prior}
\defType{String}
\defDefault{No default}

\defSub{log\_scale}{Should the sums of squares be calculated on the log scale?}
\defType{Boolean}
\defDefault{false}

\defSub{multiplier}{Multiply the penalty by this factor}
\defType{Real number (estimable)}
\defDefault{1}

\defSub{lower\_bound}{The first element to apply the penalty to in the vector}
\defType{Non-negative integer}
\defDefault{0}

\defSub{upper\_bound}{The last element to apply the penalty to in the vector}
\defType{Non-negative integer}
\defDefault{0}

\defSub{r}{The rth difference that the penalty is applied to}
\defType{Non-negative integer}
\defDefault{2}



\subsection{\I{Parameter transformations}}
\input{IncludedSyntax/ParameterTransformation.tex}

 \pagebreak 
 \section{\I{Observation command and subcommand syntax}\label{syntax:Observations}}

The description of methods for the observation section is given in Section \ref{sec:Observation}.

In the following section, the sub-section headers use a notation of the form "\textbf {@observation[label].type=abundance}" which, in this case, represents the input command fragment
{\small{\begin{verbatim}
@observation label # where label is a unique label for that observation
type=abundance
...
\end{verbatim}}}
The specific subcommands for a command are given within each command.

\subsection{\I{Observation types}}\label{syntax:ObservationTypes}

The description of the observations is given in Section \ref{sec:Observation}. The observation types available are:

\begin{description}
  \item Observations of proportions of individuals by age class
  \item Observations of proportions of individuals by category and age class
  \item Relative and absolute abundance observations
  \item Relative and absolute biomass observations
\end{description}

Each type of observation requires a set of subcommands and arguments specific to that process.

\ifAgeBased
\defComLab{observation}{Define an object of type \emph{Observation}}.
\defRef{sec:Observation}
\label{syntax:Observation}

\defSub{label}{The label of the observation}
\defType{String}
\defDefault{No default}

\defSub{type}{The type of observation}
\defType{String}
\defDefault{No default}

\defSub{likelihood}{The type of likelihood to use}
\defType{String}
\defDefault{No default}

\defSub{categories}{The category labels to use}
\defType{Vector of strings}
\defDefault{true}

\defSub{delta}{The robustification value (delta) for the likelihood}
\defType{Real number (estimable)}
\defDefault{1e-11}
\defLowerBound{0.0 (inclusive)}

\defSub{simulation\_likelihood}{The simulation likelihood to use}
\defType{String}
\defDefault{No default}

\defSub{likelihood\_multiplier}{The likelihood multiplier}
\defType{Real number (estimable)}
\defDefault{1.0}
\defLowerBound{0.0 (inclusive)}

\defSub{error\_value\_multiplier}{The error value multiplier for likelihood}
\defType{Real number (estimable)}
\defDefault{1.0}
\defLowerBound{0.0 (inclusive)}

\defSub{table}{The table of data specifying the observed values}
\defType{Data table with label = obs}
\defDefault{No default}
\defValue{A $n*m$ matrix, where $n=$ the years and $m=$ either the number of ages, lengths, or abundance/biomass observation for each year defined in the model. Each row starts with the year. The table ends with `end\_table'}
\defNote{See \ref{sec:DataTable} for more details each observation may have custom table labels.}

\defSub{table}{The table of data specifying the observed error values}
\defType{Data table with label = error\_values}
\defDefault{No default}
\defValue{A $n*m$ matrix, where $n=$ the years and $m=$ either the number of ages, lengths, or abundance/biomass observation for each year defined in the model. Each row starts with the year. The table ends with `end\_table'}
\defNote{See \ref{sec:DataTable} for more details on specifying data tables.  each observation may have custom table labels.}

\subsubsection{Observation of type Abundance}
\commandlabsubarg{observation}{type}{Abundance}.
\defRef{sec:Observation-Abundance}
\label{syntax:Observation-Abundance}

\defSub{time\_step}{The label of the time step that the observation occurs in}
\defType{String}
\defDefault{No default}

\defSub{catchability}{The label of the catchability coefficient (q)}
\defType{String}
\defDefault{No default}

\defSub{selectivities}{The labels of the selectivities}
\defType{Vector of strings}
\defDefault{true}

\defSub{process\_error}{The process error}
\defType{Real number (estimable)}
\defDefault{0.0}
\defLowerBound{0.0 (inclusive)}

\defSub{years}{The years for which there are observations}
\defType{Vector of non-negative integers}
\defDefault{No default}

\defSub{table obs}{The table of data specifying the observed and error values}
\defType{Data table with label = obs}
\defDefault{No default}
\defValue{A $n*3$ matrix, where $n=$ the years and a column for year, observation and error value. See below for example.}
\defNote{example below}
\begin{verbatim}
table obs 
# year observation error_value
1993 238.2 0.12
1994 170 0.16
1995 216.2 0.18
2004 46.9 0.20
end_table
\end{verbatim}

\subsubsection{Observation of type Biomass}
\commandlabsubarg{observation}{type}{Biomass}.
\defRef{sec:Observation-Biomass}
\label{syntax:Observation-Biomass}

\defSub{time\_step}{The label of the time step that the observation occurs in}
\defType{String}
\defDefault{No default}

\defSub{catchability}{The label of the catchability coefficient (q)}
\defType{String}
\defDefault{No default}

\defSub{selectivities}{The labels of the selectivities}
\defType{Vector of strings}
\defDefault{true}

\defSub{process\_error}{The process error}
\defType{Real number (estimable)}
\defDefault{0.0}
\defLowerBound{0.0 (inclusive)}

\defSub{age\_weight\_labels}{The labels for the \command{$age\_weight$} block which corresponds to each category, to use the weight calculation method for biomass calculations)}
\defType{Vector of strings}
\defDefault{No default}

\defSub{years}{The years of the observed values}
\defType{Vector of non-negative integers}
\defDefault{No default}

\defSub{table obs}{The table of data specifying the observed and error values}
\defType{Data table with label = obs}
\defDefault{No default}
\defValue{A $n*3$ matrix, where $n=$ the years and a column for year, observation and error value. See below for example.}
\defNote{example below}
\begin{verbatim}
table obs 
# year observation error_value
1993 238.2 0.12
1994 170 0.16
1995 216.2 0.18
2004 46.9 0.20
end_table
\end{verbatim}

\subsubsection{Observation of type Process Removals By Age}
\commandlabsubarg{observation}{type}{Process\_Removals\_By\_Age}.
\defRef{sec:Observation-ProcessRemovalsByAge}
\label{syntax:Observation-ProcessRemovalsByAge}

\defSub{min\_age}{The minimum age}
\defType{Non-negative integer}
\defDefault{No default}

\defSub{max\_age}{The maximum age}
\defType{Non-negative integer}
\defDefault{No default}

\defSub{sum\_to\_one}{Scale year (row) observed values by the total so they sum to equal 1}
\defType{Boolean}
\defDefault{false}

\defSub{simulated\_data\_sum\_to\_one}{Whether simulated data is discrete or scaled by totals to be proportions for each year}
\defType{Boolean}
\defDefault{true}


\defSub{plus\_group}{Is the maximum age the age plus group}
\defType{Boolean}
\defDefault{true}

\defSub{time\_step}{The label of time-step that the observation occurs in}
\defType{Vector of strings}
\defDefault{No default}

\defSub{years}{The years for which there are observations}
\defType{Vector of non-negative integers}
\defDefault{No default}

\defSub{process\_errors}{The process errors to use}
\defType{Vector of real numbers (estimable) of length equal to the number of years}
\defDefault{0.0}
\defNote{If only one value is supplied, it will be repeated for all years in the observation}

\defSub{ageing\_error}{The label of the ageing error to use}
\defType{String}
\defDefault{No default}

\defSub{method\_of\_removal}{The label of the observed method of removals}
\defType{Vector of strings}
\defDefault{No default}


\defSub{mortality\_process}{The label of the mortality instantaneous process for the observation}
\defType{String}
\defDefault{No default}
\defNote{Allowed mortality process types are \subcommand{mortality\_instantaneous} and \subcommand{mortality\_hybrid}}

\defSub{table obs}{The table of data specifying the observed values}
\defType{Data table with label = obs}
\defDefault{No default}
\defValue{A $n\times m$ matrix, where $n=$ the years and $m$ is categories \(\times\) length bins. See below for example.}
\defNote{example below}
\begin{verbatim}
table obs 
1993 0.1 0.2 0.3
1994 0.1 0.2 0.3
end_table
\end{verbatim}
\defSub{table error\_values}{The table of data specifying the error values}
\defType{Data table with label = error\_values}
\defDefault{No default}
\defValue{Can be specified two ways either as a $n\times 1$ matrix with an error value for each year. Or a $n\times m$ matrix, where $n=$ the years and $m$ is categories \(\times\) length bins. See below for example.}
\defNote{example below}
\begin{verbatim}
table error_values 
1993 234
1994 343
end_table
\end{verbatim}
\subsubsection{Observation of type Process Removals By Age Retained}
\commandlabsubarg{observation}{type}{Process\_Removals\_By\_Age\_Retained}.
\defRef{sec:Observation-ProcessRemovalsByAgeRetained}
\label{syntax:Observation-ProcessRemovalsByAgeRetained}

\defSub{min\_age}{The minimum age}
\defType{Non-negative integer}
\defDefault{No default}

\defSub{max\_age}{The maximum age}
\defType{Non-negative integer}
\defDefault{No default}

\defSub{plus\_group}{Is the maximum age the age plus group?}
\defType{Boolean}
\defDefault{true}

\defSub{time\_step}{The label of the time step that the observation occurs in}
\defType{Vector of strings}
\defDefault{No default}


\defSub{sum\_to\_one}{Scale the year (row) observed values by the total, so they sum to 1}
\defType{Boolean}
\defDefault{false}

\defSub{simulated\_data\_sum\_to\_one}{Whether simulated data is discrete or scaled by totals to be proportions for each year}
\defType{Boolean}
\defDefault{true}

\defSub{years}{The years for which there are observations}
\defType{Vector of non-negative integers}
\defDefault{No default}

\defSub{process\_errors}{The process errors to use}
\defType{Vector of real numbers (estimable) of length equal to the number of years}
\defDefault{0.0}
\defNote{If only one value is supplied, it will be repeated for all years in the observation}

\defSub{ageing\_error}{The label of the ageing error to use}
\defType{String}
\defDefault{No default}

\defSub{method\_of\_removal}{The label of observed method of removals}
\defType{Vector of strings}
\defDefault{No default}


\defSub{mortality\_process}{The label of the mortality instantaneous process for the observation}
\defType{String}
\defDefault{No default}
\defNote{Allowed mortality process types are \subcommand{mortality\_instantaneous\_retained}}

\defSub{table obs}{The table of data specifying the observed values}
\defType{Data table with label = obs}
\defDefault{No default}
\defValue{A $n\times m$ matrix, where $n=$ the years and $m$ is categories \(\times\) length bins. See below for example.}
\defNote{example below}
\begin{verbatim}
table obs 
1993 0.1 0.2 0.3
1994 0.1 0.2 0.3
end_table
\end{verbatim}
\defSub{table error\_values}{The table of data specifying the error values}
\defType{Data table with label = error\_values}
\defDefault{No default}
\defValue{Can be specified two ways either as a $n\times 1$ matrix with an error value for each year. Or a $n\times m$ matrix, where $n=$ the years and $m$ is categories \(\times\) length bins. See below for example.}
\defNote{example below}
\begin{verbatim}
table error_values 
1993 234
1994 343
end_table
\end{verbatim}
\subsubsection{Observation of type Process Removals By Age Retained Total}
\commandlabsubarg{observation}{type}{Process\_Removals\_By\_Age\_Retained\_Total}.
\defRef{sec:Observation-ProcessRemovalsByAgeRetainedTotal}
\label{syntax:Observation-ProcessRemovalsByAgeRetainedTotal}

\defSub{min\_age}{The minimum age}
\defType{Non-negative integer}
\defDefault{No default}

\defSub{max\_age}{The maximum age}
\defType{Non-negative integer}
\defDefault{No default}

\defSub{plus\_group}{Is the maximum age the age plus group?}
\defType{Boolean}
\defDefault{true}

\defSub{time\_step}{The label of the time step that the observation occurs in}
\defType{Vector of strings}
\defDefault{No default}

\defSub{sum\_to\_one}{Scale the year (row) observed values by the total, so they sum to 1}
\defType{Boolean}
\defDefault{false}

\defSub{simulated\_data\_sum\_to\_one}{Whether simulated data is discrete or scaled by totals to be proportions for each year}
\defType{Boolean}
\defDefault{true}

\defSub{years}{The years for which there are observations}
\defType{Vector of non-negative integers}
\defDefault{No default}

\defSub{process\_errors}{The process errors to use}
\defType{Vector of real numbers (estimable) of length equal to the number of years}
\defDefault{0.0}
\defNote{If only one value is supplied, it will be repeated for all years in the observation}

\defSub{ageing\_error}{The label of the ageing error to use}
\defType{String}
\defDefault{No default}

\defSub{method\_of\_removal}{The label of observed method of removals}
\defType{Vector of strings}
\defDefault{No default}

\defSub{mortality\_process}{The label of the mortality process for this observation}
\defType{String}
\defDefault{No default}
\defNote{Allowed mortality process types are \subcommand{mortality\_instantaneous\_retained}}

\defSub{table obs}{The table of data specifying the observed values}
\defType{Data table with label = obs}
\defDefault{No default}
\defValue{A $n\times m$ matrix, where $n=$ the years and $m$ is categories \(\times\) length bins. See below for example.}
\defNote{example below}
\begin{verbatim}
table obs 
1993 0.1 0.2 0.3
1994 0.1 0.2 0.3
end_table
\end{verbatim}
\defSub{table error\_values}{The table of data specifying the error values}
\defType{Data table with label = error\_values}
\defDefault{No default}
\defValue{Can be specified two ways either as a $n\times 1$ matrix with an error value for each year. Or a $n\times m$ matrix, where $n=$ the years and $m$ is categories \(\times\) length bins. See below for example.}
\defNote{example below}
\begin{verbatim}
table error_values 
1993 234
1994 343
end_table
\end{verbatim}
\subsubsection{Observation of type Process Removals By Length}
\commandlabsubarg{observation}{type}{Process\_Removals\_By\_Length}.
\defRef{sec:Observation-ProcessRemovalsByLength}
\label{syntax:Observation-ProcessRemovalsByLength}

\defSub{time\_step}{The time step to execute in}
\defType{String}
\defDefault{No default}

\defSub{years}{The years for which there are observations}
\defType{Vector of non-negative integers}
\defDefault{No default}

\defSub{process\_errors}{The process errors to use}
\defType{Vector of real numbers (estimable) of length equal to the number of years}
\defDefault{0.0}
\defNote{If only one value is supplied, it will be repeated for all years in the observation}

\defSub{method\_of\_removal}{The label of observed method of removals}
\defType{String}
\defDefault{No default}

\defSub{length\_bins}{The length bins}
\defType{Vector of real numbers (estimable)}
\defDefault{No default}

\defSub{sum\_to\_one}{Scale the year (row) observed values by the total, so they sum to 1}
\defType{Boolean}
\defDefault{false}

\defSub{simulated\_data\_sum\_to\_one}{Whether simulated data is discrete or scaled by totals to be proportions for each year}
\defType{Boolean}
\defDefault{true}

\defSub{plus\_group}{Is the last length bin a plus group? (defaults to @model value)}
\defType{Boolean}
\defDefault{model}

\defSub{mortality\_process}{The label of the mortality instantaneous process for the observation}
\defType{String}
\defDefault{No default}
\defNote{Allowed mortality process types are \subcommand{mortality\_instantaneous} and \subcommand{mortality\_hybrid}}

\defSub{table obs}{The table of data specifying the observed values}
\defType{Data table with label = obs}
\defDefault{No default}
\defValue{A $n\times m$ matrix, where $n=$ the years and $m$ is categories \(\times\) length bins. See below for example.}
\defNote{example below}
\begin{verbatim}
table obs 
1993 0.1 0.2 0.3
1994 0.1 0.2 0.3
end_table
\end{verbatim}
\defSub{table error\_values}{The table of data specifying the error values}
\defType{Data table with label = error\_values}
\defDefault{No default}
\defValue{Can be specified two ways either as a $n\times 1$ matrix with an error value for each year. Or a $n\times m$ matrix, where $n=$ the years and $m$ is categories \(\times\) length bins. See below for example.}
\defNote{example below}
\begin{verbatim}
table error_values 
1993 234
1994 343
end_table
\end{verbatim}
\subsubsection{Observation of type Process Removals By Length Retained}
\commandlabsubarg{observation}{type}{Process\_Removals\_By\_Length\_Retained}.
\defRef{sec:Observation-ProcessRemovalsByLengthRetained}
\label{syntax:Observation-ProcessRemovalsByLengthRetained}

\defSub{time\_step}{The time step to execute in}
\defType{String}
\defDefault{No default}

\defSub{years}{The years for which there are observations}
\defType{Vector of non-negative integers}
\defDefault{No default}

\defSub{process\_errors}{The process errors to use}
\defType{Vector of real numbers (estimable) of length equal to the number of years}
\defDefault{0.0}
\defNote{If only one value is supplied, it will be repeated for all years in the observation}

\defSub{method\_of\_removal}{The label of observed method of removals}
\defType{String}
\defDefault{No default}

\defSub{length\_bins}{The length bins}
\defType{Vector of real numbers (estimable)}
\defDefault{No default}

\defSub{sum\_to\_one}{Scale the year (row) observed values by the total, so they sum to 1}
\defType{Boolean}
\defDefault{false}

\defSub{simulated\_data\_sum\_to\_one}{Whether simulated data is discrete or scaled by totals to be proportions for each year}
\defType{Boolean}
\defDefault{true}

\defSub{plus\_group}{Is the last length bin a plus group? (defaults to @model value)}
\defType{Boolean}
\defDefault{model}

\defSub{mortality\_process}{The label of the mortality instantaneous process for the observation}
\defType{String}
\defDefault{No default}
\defNote{Allowed mortality process types are \subcommand{mortality\_instantaneous\_retained}}

\defSub{table obs}{The table of data specifying the observed values}
\defType{Data table with label = obs}
\defDefault{No default}
\defValue{A $n\times m$ matrix, where $n=$ the years and $m$ is categories \(\times\) length bins. See below for example.}
\defNote{example below}
\begin{verbatim}
table obs 
1993 0.1 0.2 0.3
1994 0.1 0.2 0.3
end_table
\end{verbatim}
\defSub{table error\_values}{The table of data specifying the error values}
\defType{Data table with label = error\_values}
\defDefault{No default}
\defValue{Can be specified two ways either as a $n\times 1$ matrix with an error value for each year. Or a $n\times m$ matrix, where $n=$ the years and $m$ is categories \(\times\) length bins. See below for example.}
\defNote{example below}
\begin{verbatim}
table error_values 
1993 234
1994 343
end_table
\end{verbatim}

\subsubsection{Observation of type Process Removals By Length Retained Total}
\commandlabsubarg{observation}{type}{Process\_Removals\_By\_Length\_Retained\_Total}.
\defRef{sec:Observation-ProcessRemovalsByLengthRetainedTotal}
\label{syntax:Observation-ProcessRemovalsByLengthRetainedTotal}

\defSub{time\_step}{The time step to execute in}
\defType{String}
\defDefault{No default}

\defSub{years}{The years for which there are observations}
\defType{Vector of non-negative integers}
\defDefault{No default}

\defSub{process\_errors}{The process errors to use}
\defType{Vector of real numbers (estimable) of length equal to the number of years}
\defDefault{0.0}
\defNote{If only one value is supplied, it will be repeated for all years in the observation}

\defSub{method\_of\_removal}{The label of observed method of removals}
\defType{String}
\defDefault{No default}

\defSub{length\_bins}{The length bins}
\defType{Vector of real numbers (estimable)}
\defDefault{No default}

\defSub{plus\_group}{Is the last length bin a plus group? (defaults to @model value)}
\defType{Boolean}
\defDefault{model}

\defSub{sum\_to\_one}{Scale the year (row) observed values by the total, so they sum to 1}
\defType{Boolean}
\defDefault{false}

\defSub{simulated\_data\_sum\_to\_one}{Whether simulated data is discrete or scaled by totals to be proportions for each year}
\defType{Boolean}
\defDefault{true}

\defSub{mortality\_process}{The label of the mortality instantaneous process for the observation}
\defType{String}
\defDefault{No default}
\defNote{Allowed mortality process types are \subcommand{mortality\_instantaneous\_retained}}

\defSub{table obs}{The table of data specifying the observed values}
\defType{Data table with label = obs}
\defDefault{No default}
\defValue{A $n\times m$ matrix, where $n=$ the years and $m$ is categories \(\times\) length bins. See below for example.}
\defNote{example below}
\begin{verbatim}
table obs 
1993 0.1 0.2 0.3
1994 0.1 0.2 0.3
end_table
\end{verbatim}
\defSub{table error\_values}{The table of data specifying the error values}
\defType{Data table with label = error\_values}
\defDefault{No default}
\defValue{Can be specified two ways either as a $n\times 1$ matrix with an error value for each year. Or a $n\times m$ matrix, where $n=$ the years and $m$ is categories \(\times\) length bins. See below for example.}
\defNote{example below}
\begin{verbatim}
table error_values 
1993 234
1994 343
end_table
\end{verbatim}

\subsubsection{Observation of type Process Removals By Weight}
\commandlabsubarg{observation}{type}{Process\_Removals\_By\_Weight}.
%\defRef{sec:Observation-ProcessRemovalsByWeight}
\label{syntax:Observation-ProcessRemovalsByWeight}

\defSub{mortality\_process}{The label of the mortality instantaneous process for the observation}
\defType{String}
\defDefault{No default}
\defNote{Allowed mortality process types are \subcommand{mortality\_instantaneous}}

\defSub{method\_of\_removal}{The label of observed method of removals}
\defType{String}
\defDefault{No default}

\defSub{time\_step}{The time step to execute in}
\defType{String}
\defDefault{No default}

\defSub{years}{The years for which there are observations}
\defType{Vector of non-negative integers}
\defDefault{No default}

\defSub{process\_errors}{The process errors to use}
\defType{Vector of real numbers (estimable) of length equal to the number of years}
\defDefault{0.0}
\defNote{If only one value is supplied, it will be repeated for all years in the observation}

\defSub{length\_weight\_cv}{The CV for the length-weight relationship}
\defType{Real number (estimable)}
\defDefault{0.10}
\defLowerBound{0.0 (exclusive)}

\defSub{length\_weight\_distribution}{The distribution of the length-weight relationship}
\defType{String}
\defDefault{normal}

\defSub{length\_bins}{The length bins}
\defType{Vector of real numbers (estimable)}
\defDefault{No default}

\defSub{length\_bins\_n}{The average number in each length bin}
\defType{Vector of real numbers (estimable)}
\defDefault{No default}

\defSub{units}{The units for the weight bins (grams, kilograms (kgs), or tonnes)}
\defType{String}
\defDefault{kgs}

\defSub{fishbox\_weight}{The target weight of each box}
\defType{Real number (estimable)}
\defDefault{20.0}
\defLowerBound{0.0 (exclusive)}

\defSub{weight\_bins}{The weight bins}
\defType{Vector of real numbers (estimable)}
\defDefault{No default}

\subsubsection{Observation of type Proportions At Age}
\commandlabsubarg{observation}{type}{Proportions\_At\_Age}.
\defRef{sec:Observation-ProportionsAtAge}
\label{syntax:Observation-ProportionsAtAge}

\defSub{min\_age}{The minimum age}
\defType{Non-negative integer}
\defDefault{No default}

\defSub{max\_age}{The maximum age}
\defType{Non-negative integer}
\defDefault{No default}

\defSub{plus\_group}{Is the maximum age the age plus group?}
\defType{Boolean}
\defDefault{true}

\defSub{time\_step}{The label of the time step that the observation occurs in}
\defType{String}
\defDefault{No default}

\defSub{years}{The years of the observed values}
\defType{Vector of non-negative integers}
\defDefault{No default}

\defSub{selectivities}{The labels of the selectivities}
\defType{Vector of strings}
\defDefault{true}

\defSub{process\_errors}{The process errors to use}
\defType{Vector of real numbers (estimable) of length equal to the number of years}
\defDefault{0.0}
\defNote{If only one value is supplied, it will be repeated for all years in the observation}

\defSub{ageing\_error}{The label of ageing error to use}
\defType{String}
\defDefault{No default}

\defSub{sum\_to\_one}{Scale the year (row) observed values by the total, so they sum to 1}
\defType{Boolean}
\defDefault{false}

\defSub{simulated\_data\_sum\_to\_one}{Whether simulated data is discrete or scaled by totals to be proportions for each year}
\defType{Boolean}
\defDefault{true}

\defSub{table obs}{The table of data specifying the observed values}
\defType{Data table with label = obs}
\defDefault{No default}
\defValue{A $n\times m$ matrix, where $n=$ the years and $m$ is categories \(\times\) length bins. See below for example.}
\defNote{example below}
\begin{verbatim}
table obs 
1993 0.1 0.2 0.3
1994 0.1 0.2 0.3
end_table
\end{verbatim}
\defSub{table error\_values}{The table of data specifying the error values}
\defType{Data table with label = error\_values}
\defDefault{No default}
\defValue{Can be specified two ways either as a $n\times 1$ matrix with an error value for each year. Or a $n\times m$ matrix, where $n=$ the years and $m$ is categories \(\times\) length bins. See below for example.}
\defNote{example below}
\begin{verbatim}
table error_values 
1993 234
1994 343
end_table
\end{verbatim}

\subsubsection{Observation of type Proportions At Length}
\commandlabsubarg{observation}{type}{Proportions\_At\_Length}.
\defRef{sec:Observation-ProportionsAtLength}
\label{syntax:Observation-ProportionsAtLength}

\defSub{time\_step}{The label of the time step that the observation occurs in}
\defType{String}
\defDefault{No default}

\defSub{years}{The years for which there are observations}
\defType{Vector of non-negative integers}
\defDefault{No default}

\defSub{selectivities}{The labels of the selectivities}
\defType{Vector of strings}
\defDefault{true}

\defSub{process\_errors}{The process errors to use}
\defType{Vector of real numbers (estimable) of length equal to the number of years}
\defDefault{0.0}
\defNote{If only one value is supplied, it will be repeated for all years in the observation}

\defSub{length\_bins}{The length bins}
\defType{Vector of real numbers (estimable)}
\defDefault{true}

\defSub{plus\_group}{Is the last length bin a plus group?}
\defType{Boolean}
\defDefault{true if the value of \commandsub{model}{length\_plus} is true, otherwise false}

\defSub{sum\_to\_one}{Scale the year (row) observed values by the total, so they sum to 1}
\defType{Boolean}
\defDefault{false}

\defSub{simulated\_data\_sum\_to\_one}{Whether simulated data is discrete or scaled by totals to be proportions for each year}
\defType{Boolean}
\defDefault{true}

\defSub{table obs}{The table of data specifying the observed values}
\defType{Data table with label = obs}
\defDefault{No default}
\defValue{A $n\times m$ matrix, where $n=$ the years and $m$ is categories \(\times\) length bins. See below for example.}
\defNote{example below}
\begin{verbatim}
table obs 
1993 0.1 0.2 0.3
1994 0.1 0.2 0.3
end_table
\end{verbatim}
\defSub{table error\_values}{The table of data specifying the error values}
\defType{Data table with label = error\_values}
\defDefault{No default}
\defValue{Can be specified two ways either as a $n\times 1$ matrix with an error value for each year. Or a $n\times m$ matrix, where $n=$ the years and $m$ is categories \(\times\) length bins. See below for example.}
\defNote{example below}
\begin{verbatim}
table error_values 
1993 234
1994 343
end_table
\end{verbatim}

\subsubsection{Observation of type Proportions By Category}
\commandlabsubarg{observation}{type}{Proportions\_By\_Category}.
\defRef{sec:Observation-ProportionsByCategory}
\label{syntax:Observation-ProportionsByCategory}

\defSub{min\_age}{The minimum age}
\defType{Non-negative integer}
\defDefault{No default}

\defSub{max\_age}{The maximum age}
\defType{Non-negative integer}
\defDefault{No default}

\defSub{time\_step}{The label of the time step that the observation occurs in}
\defType{String}
\defDefault{No default}

\defSub{plus\_group}{Use the age plus group?}
\defType{Boolean}
\defDefault{true}

\defSub{years}{The years for which there are observations}
\defType{Vector of non-negative integers}
\defDefault{No default}

\defSub{selectivities}{The labels of the selectivities}
\defType{Vector of strings}
\defDefault{true}

\defSub{categories2}{The target categories}
\defType{Vector of strings}
\defDefault{No default}

\defSub{selectivities2}{The target selectivities}
\defType{Vector of strings}
\defDefault{No default}

\subsubsection{Observation of type Proportions Mature By Age}
\commandlabsubarg{observation}{type}{Proportions\_Mature\_By\_Age}.
%\defRef{sec:Observation-ProportionsMatureByAge}
\label{syntax:Observation-ProportionsMatureByAge}

\defSub{min\_age}{The minimum age}
\defType{Non-negative integer}
\defDefault{No default}

\defSub{max\_age}{The maximum age}
\defType{Non-negative integer}
\defDefault{No default}

\defSub{time\_step}{The label of time-step that the observation occurs in}
\defType{String}
\defDefault{No default}

\defSub{plus\_group}{Use the age plus group?}
\defType{Boolean}
\defDefault{true}

\defSub{years}{The years for which there are observations}
\defType{Vector of non-negative integers}
\defDefault{No default}

\defSub{ageing\_error}{The label of ageing error to use}
\defType{String}
\defDefault{No default}

\defSub{total\_categories}{All category labels that were vulnerable to sampling at the time of this observation (not including the categories already given)}
\defType{Vector of strings}
\defDefault{true}

\defSub{time\_step\_proportion}{The proportion through the mortality block of the time step when the observation is evaluated}
\defType{Real number (estimable)}
\defDefault{0.5}
\defLowerBound{0.0 (inclusive)}
\defUpperBound{1.0 (inclusive)}

\subsubsection{Observation of type Proportions Migrating}
\commandlabsubarg{observation}{type}{Proportions\_Migrating}.
\defRef{sec:Observation-ProportionsMigrating}
\label{syntax:Observation-ProportionsMigrating}

\defSub{min\_age}{The minimum age}
\defType{Non-negative integer}
\defDefault{No default}

\defSub{max\_age}{The maximum age}
\defType{Non-negative integer}
\defDefault{No default}

\defSub{time\_step}{The label of the time step that the observation occurs in}
\defType{String}
\defDefault{No default}

\defSub{plus\_group}{Is the maximum age the age plus group?}
\defType{Boolean}
\defDefault{true}

\defSub{years}{The years for which there are observations}
\defType{Vector of non-negative integers}
\defDefault{No default}

\defSub{process\_errors}{The process errors to use}
\defType{Vector of real numbers (estimable) of length equal to the number of years}
\defDefault{0.0}
\defNote{If only one value is supplied, it will be repeated for all years in the observation}

\defSub{ageing\_error}{The label of the ageing error to use}
\defType{String}
\defDefault{No default}

\defSub{process}{The process label}
\defType{String}
\defDefault{No default}

\subsubsection{Observation of type Tag Recapture by Fishery}
\commandlabsubarg{observation}{type}{tag\_recapture\_by\_fishery}.
\defRef{sec:Observation-TagRecaptureByFishery}
\label{syntax:Observation-TagRecaptureByFishery}

\defSub{tagged\_categories}{The tagged categories that we want to generate recaptures for. Categories need to be space separated no use of the '+' category syntax.}
\defType{Vector of strings}
\defDefault{No default}

\defSub{time\_step}{The label of time-step that the observation occurs in}
\defType{Vector of strings}
\defDefault{No default}

\defSub{reporting\_rate}{The reporting rate for this observation}
\defType{Real number (estimable)}
\defDefault{No default}
\defLowerBound{0.0 (inclusive)}
\defUpperBound{1.0 (inclusive)}


\defSub{years}{The years for which there are observations}
\defType{Vector of non-negative integers}
\defDefault{No default}

\defSub{method\_of\_removal}{The label of the observed method of removals}
\defType{Vector of strings}
\defDefault{No default}


\defSub{mortality\_process}{The label of the mortality instantaneous process for the observation}
\defType{String}
\defDefault{No default}
\defNote{Allowed mortality process types are \subcommand{mortality\_instantaneous} and \subcommand{mortality\_hybrid}}

\defSub{table recaptured}{The table of recaptures in each year}
\defType{Data table with label = \subcommand{recaptured}}
\defDefault{No default}
\defValue{A $n\times 2$ matrix, where $n=$ the years and the first column specifies the year and the second column specifies the observed tag recaptures.}
\defNote{example below}
\begin{verbatim}
table recaptured
2000 10120
2001 90123
end_table
\end{verbatim}

\subsubsection{Observation of type Tag Recapture By Age}
\commandlabsubarg{observation}{type}{Tag\_Recapture\_By\_Age}.
\defRef{sec:Observation-TagRecaptureByAge}
\label{syntax:Observation-TagRecaptureByAge}

\defSub{min\_age}{The minimum age}
\defType{Non-negative integer}
\defDefault{No default}

\defSub{max\_age}{The maximum age}
\defType{Non-negative integer}
\defDefault{No default}

\defSub{plus\_group}{Is the maximum age the age plus group?}
\defType{Boolean}
\defDefault{true}

\defSub{years}{The years for which there are observations}
\defType{Vector of non-negative integers}
\defDefault{No default}

\defSub{time\_step}{The label of the time step that the observation occurs in}
\defType{String}
\defDefault{No default}

\defSub{selectivities}{The labels of the selectivities used for untagged categories}
\defType{Vector of strings}
\defDefault{true}

\defSub{tagged\_selectivities}{The labels of the tag category selectivities}
\defType{Vector of strings}
\defDefault{true}

\defSub{tagged\_categories}{The  categories of tagged individuals}
\defType{Vector of strings}
\defDefault{No default}

\defSub{detection}{The probability of detecting a recaptured individual}
\defType{Real number (estimable)}
\defDefault{No default}
\defLowerBound{0.0 (inclusive)}
\defUpperBound{1.0 (inclusive)}

\defSub{dispersion}{The over-dispersion parameter, $\phi$}
\defType{Vector of real numbers, one for each year of recaptures}
\defDefault{No default}
\defLowerBound{0.0}

\defSub{overlap\_scalar}{The overlap\_scalar parameter, $k$}
\defType{Vector of real numbers, one for each year of recaptures (if only one value is supplied, it is repeated for each year of recaptures)}
\defDefault{1.0}
\defLowerBound{0.0 (inclusive)}
\defNote{See Section \ref{sec:Observation-TagRecaptures} for more information}

\defSub{time\_step\_proportion}{The proportion through the mortality block of the time step when the observation is evaluated}
\defType{Real number (estimable)}
\defDefault{0.5}
\defLowerBound{0.0 (inclusive)}
\defUpperBound{1.0 (inclusive)}

\defSub{table recaptured}{The table of data specifying the recaptures}
\defType{Data table with label = recaptured}
\defDefault{No default}
\defValue{A $n\times m$ matrix, where $n=$ the years and $m$ is categories \(\times\) length bins. See below for example.}
\defNote{example below}
\begin{verbatim}
table recaptured 
1993 1 32 25
1994 3 4 43
end_table
\end{verbatim}

\defSub{table scanned}{The table of data specifying the scanned fish}
\defType{Data table with label = scanned}
\defDefault{No default}
\defValue{A $n\times m$ matrix, where $n=$ the years and $m$ is categories \(\times\) length bins. See below for example.}
\defNote{example below}
\begin{verbatim}
table scanned 
1993 1 32 25
1994 3 4 43
end_table
\end{verbatim}

\subsubsection{Observation of type Tag Recapture By Length}
\commandlabsubarg{observation}{type}{Tag\_Recapture\_By\_Length}.
\defRef{sec:Observation-TagRecaptureByLength}
\label{syntax:Observation-TagRecaptureByLength}

\defSub{years}{The years for which there are observations}
\defType{Vector of non-negative integers}
\defDefault{No default}

\defSub{time\_step}{The time step to execute in}
\defType{String}
\defDefault{No default}

\defSub{length\_bins}{The length bins}
\defType{Vector of real numbers (estimable)}
\defDefault{true}

\defSub{selectivities}{The labels of the selectivities used for untagged categories}
\defType{Vector of strings}
\defDefault{true}

\defSub{tagged\_selectivities}{The labels of the tag category selectivities}
\defType{Vector of strings}
\defDefault{No default}

\defSub{tagged\_categories}{The  categories of tagged individuals}
\defType{Vector of strings}
\defDefault{No default}

\defSub{detection}{The probability of detecting a recaptured individual}
\defType{Real number (estimable)}
\defDefault{No default}
\defLowerBound{0.0 (inclusive)}
\defUpperBound{1.0 (inclusive)}

\defSub{dispersion}{The over-dispersion parameter, $\phi$}
\defType{Vector of real numbers, one for each year of recaptures}
\defDefault{No default}
\defLowerBound{0.0}

\defSub{overlap\_scalar}{The overlap\_scalar parameter, $k$}
\defType{Vector of real numbers, one for each year of recaptures (if only one value is supplied, it is repeated for each year of recaptures)}
\defDefault{1.0}
\defLowerBound{0.0 (inclusive)}
\defNote{See Section \ref{sec:Observation-TagRecaptures} for more information}

\defSub{time\_step\_proportion}{The proportion through the mortality block of the time step when the observation is evaluated}
\defType{Real number (estimable)}
\defDefault{0.5}
\defLowerBound{0.0 (inclusive)}
\defUpperBound{1.0 (inclusive)}

\defSub{table recaptured}{The table of data specifying the recaptures}
\defType{Data table with label = recaptured}
\defDefault{No default}
\defValue{A $n\times m$ matrix, where $n=$ the years and $m$ is categories \(\times\) length bins. See below for example.}
\defNote{example below}
\begin{verbatim}
table recaptured 
1993 1 32 25
1994 3 4 43
end_table
\end{verbatim}

\defSub{table scanned}{The table of data specifying the scanned fish}
\defType{Data table with label = scanned}
\defDefault{No default}
\defValue{A $n\times m$ matrix, where $n=$ the years and $m$ is categories \(\times\) length bins. See below for example.}
\defNote{example below}
\begin{verbatim}
table scanned 
1993 1 32 25
1994 3 4 43
end_table
\end{verbatim}

\subsubsection{Observation of type Age Length}
\commandlabsubarg{observation}{type}{age\_length}.
\defRef{sec:Observation-AgeSize}
\label{syntax:Observation-AgeLength}

\defSub{time\_step}{The label of the time step that the observation occurs in}
\defType{String}
\defDefault{No default}

\defSub{selectivities}{The labels of the selectivities, one for each combined category}
\defType{Vector of strings}
\defDefault{true}

\defSub{numerator\_categories}{A combined category label that defines categories that make up the numerator}
\defType{Vector of strings}
\defNote{These categories are required to have the same age-length definition and have the same selectivity.}
\defDefault{the values defined in categories}

\defSub{year}{The year this observation occurred in}
\defType{Vector of non-negative integers}
\defDefault{No default}

\defSub{sample\_type}{The sample type}
\defType{string}
\defDefault{length}
\defallowed{age, length, random}

\defSub{ages}{vector of observed ages}
\defType{Vector of positive integers}
\defNote{Needs to be integers, with model age definition, and same number of elements as lengths}
\defDefault{No default}

\defSub{lengths}{vector of observed lengths}
\defType{Vector of real numbers}
\defNote{same number of elements as ages}
\defDefault{No default}

\defSub{ageing\_error}{The label of ageing error to use}
\defType{String}
\defDefault{No ageing error}


\else
\input{IncludedSyntax/ObservationLength}
\fi 

\subsection{\I{Likelihoods}}
\defComLab{Likelihood}{Define an object of type \emph{Likelihood}}.
\defRef{sec:Likelihood}
\label{syntax:Likelihood}

\subsubsection{Likelihood of type Bernoulli}
\commandlabsubarg{Likelihood}{type}{Bernoulli}.
\defRef{sec:Likelihood-Bernoulli}
\label{syntax:Likelihood-Bernoulli}

The Bernoulli type has no additional subcommands.
\subsubsection{Likelihood of type Binomial}
\commandlabsubarg{Likelihood}{type}{Binomial}.
\defRef{sec:Likelihood-Binomial}
\label{syntax:Likelihood-Binomial}

The Binomial type has no additional subcommands.
\subsubsection{Likelihood of type Binomial\_Approx}
\commandlabsubarg{Likelihood}{type}{Binomial\_Approx}.
\defRef{sec:Likelihood-BinomialApprox}
\label{syntax:Likelihood-BinomialApprox}

The Binomial\_Approx type has no additional subcommands.
\subsubsection{Likelihood of type Dirichlet}
\commandlabsubarg{Likelihood}{type}{Dirichlet}.
\defRef{sec:Likelihood-Dirichlet}
\label{syntax:Likelihood-Dirichlet}

The Dirichlet type has no additional subcommands.
\subsubsection{Likelihood of type Dirichlet\_Multinomial}
\commandlabsubarg{Likelihood}{type}{Dirichlet\_Multinomial}.
\defRef{sec:Likelihood-DirichletMultinomial}
\label{syntax:Likelihood-DirichletMultinomial}

\defSub{label}{Label of the Dirichlet-multinomial distribution}
\defType{String}
\defDefault{No default}

\defSub{type}{Type of likelihood}
\defType{String}
\defDefault{No default}

\defSub{theta}{Theta parameter (account for overdispersion)}
\defType{Real number (estimable)}
\defDefault{false}
\defLowerBound{0.0 (inclusive)}

\subsubsection{Likelihood of type Log\_Normal}
\commandlabsubarg{Likelihood}{type}{Log\_Normal}.
\defRef{sec:Likelihood-LogNormal}
\label{syntax:Likelihood-LogNormal}

The Log\_Normal type has no additional subcommands.
\subsubsection{Likelihood of type Log\_Normal\_With\_Q}
\commandlabsubarg{Likelihood}{type}{Log\_Normal\_With\_Q}.
\defRef{sec:Likelihood-LogNormalWithQ}
\label{syntax:Likelihood-LogNormalWithQ}

The Log\_Normal\_With\_Q type has no additional subcommands.
\subsubsection{Likelihood of type Multinomial}
\commandlabsubarg{Likelihood}{type}{Multinomial}.
\defRef{sec:Likelihood-Multinomial}
\label{syntax:Likelihood-Multinomial}

The Multinomial type has no additional subcommands.
\subsubsection{Likelihood of type Normal}
\commandlabsubarg{Likelihood}{type}{Normal}.
\defRef{sec:Likelihood-Normal}
\label{syntax:Likelihood-Normal}

The Normal type has no additional subcommands.
\subsubsection{Likelihood of type Poisson}
\commandlabsubarg{Likelihood}{type}{Poisson}.
\defRef{sec:Likelihood-Poisson}
\label{syntax:Likelihood-Poisson}

The Poisson type has no additional subcommands.
\subsubsection{Likelihood of type Pseudo}
\commandlabsubarg{Likelihood}{type}{Pseudo}.
\defRef{sec:Likelihood-Pseudo}
\label{syntax:Likelihood-Pseudo}

The Pseudo type has no additional subcommands.


\ifAgeBased
\subsection{\I{Defining ageing error}}\label{syntax:AgeingError}

The methods for including ageing error into estimation for observations are:

\begin{itemize}
	\item None
	\item Data
	\item Normal
	\item Off-by-one
\end{itemize}

Each type of ageing error has a set of subcommands and arguments specific to its type.

\input{IncludedSyntax/AgeingError}
\fi % end if
%\subsection{\I{Simulating observations}}
%\input{IncludedSyntax/Simulate}

 \pagebreak 
 \section{\I{Report command and subcommand syntax}\label{syntax:Reports}}

The description of each report is given in Section \ref{sec:Report}.

\subsection{\I{Report commands and subcommands}}

\defComLab{report}{Define an object of type \emph{Report}}.
\defRef{sec:Report}
\label{syntax:Report}

\defSub{label}{The report label}
\defType{String}
\defDefault{No default}

\defSub{type}{The report type}
\defType{String}
\defDefault{No default}

\defSub{file\_name}{The file name. If not supplied, then output is directed to standard out}
\defType{String}
\defDefault{No default}

\defSub{write\_mode}{Specify if any previous file with the same name should be overwritten, appended to, or is generated using a sequential suffix}
\defType{String}
\defDefault{overwrite}
\defValue{valid options are \subcommand{append}, \subcommand{overwrite}, \subcommand{incremental\_suffix}}

\defSub{format}{Report output format}
\defType{String}
\defDefault{r}
\defValue{Either \R\ for formatting for reading into \R\, or \texttt{none} for no formatting}


\subsubsection{Report of type Default}
\commandlabsubarg{report}{type}{Default}.
\defRef{sec:Report-Default}
\label{syntax:Report-Default}

\defSub{catchabilities}{Report catchabilities}
\defType{Boolean}
\defDefault{false}
\defNote{Reports all valid catchabilities}

\defSub{derived\_quantities}{Report derived quantities}
\defType{Boolean}
\defDefault{false}
\defNote{Reports all valid derived quantities}

\defSub{observations}{Report observations}
\defType{Boolean}
\defDefault{false}
\defNote{Reports all valid observations}

\defSub{processes}{Report processes}
\defType{Boolean}
\defDefault{false}
\defNote{Reports all valid processes}

\defSub{projects}{Report projects}
\defType{Boolean}
\defDefault{false}
\defNote{Reports all valid projections}

\defSub{selectivities}{Report selectivities}
\defType{Boolean}
\defDefault{false}
\defNote{Reports all valid selectivities}

\defSub{time\_varying}{Report time-varying parameters}
\defType{Boolean}
\defDefault{false}
\defNote{Reports all valid time-varying parameters}

\defSub{parameter\_transformations}{Report all parameter transformations}
\defType{Boolean}
\defDefault{false}
\defNote{Reports all valid parameter transformations}

\subsubsection{Report of type Addressable}
\commandlabsubarg{report}{type}{Addressable}.
\defRef{sec:Report-Addressable}
\label{syntax:Report-Addressable}

\defSub{parameter}{The addressable parameter name}
\defType{String}
\defDefault{No default}

\defSub{years}{Define the years that the report is generated for}
\defType{Vector of non-negative integers}
\defDefault{No default}

\defSub{time\_step}{Defines the time-step that the report applies to}
\defType{String}
\defDefault{No default}
\defValue{A valid time step label}

\ifAgeBased
\subsubsection{Report of type Age Length}
\commandlabsubarg{report}{type}{Age\_Length}.
\defRef{sec:Report-AgeLength}
\label{syntax:Report-AgeLength}

\defSub{time\_step}{The time step label}
\defType{String}
\defDefault{No default}
\defValue{A valid time step label}

\defSub{years}{The years for the report}
\defType{Vector of non-negative integers}
\defDefault{All years}

\defSub{age\_length}{The age-length label}
\defType{String}
\defDefault{No default}
\else
\subsubsection{Report of type Growth Increment model}
\commandlabsubarg{report}{type}{growth\_increment}.
\defRef{sec:Report-GrowthIncrement}
\label{syntax:Report-GrowthIncrement}

\defSub{time\_step}{The time step label}
\defType{String}
\defDefault{No default}
\defValue{A valid time step label}

\defSub{years}{The years for the report}
\defType{Vector of non-negative integers}
\defDefault{All years}

\defSub{growth\_increment}{The growth-increment label}
\defType{String}
\defDefault{No default}
\fi

\ifAgeBased
\subsubsection{Report of type Ageing Error Matrix}
\commandlabsubarg{report}{type}{Ageing\_Error\_Matrix}.
\defRef{sec:Report-AgeingErrorMatrix}
\label{syntax:Report-AgeingErrorMatrix}

\defSub{ageing\_error}{The ageing error label}
\defType{String}
\defDefault{No default}
\fi

\subsubsection{Report of type Catchability}
\commandlabsubarg{report}{type}{Catchability}.
\defRef{sec:Report-Catchability}
\label{syntax:Report-Catchability}

\defSub{catchability}{The catchability label}
\defType{String}
\defDefault{No default}
\defValue{If not specified, then the label of the report is assumed to be the category label}

%\subsubsection{Report of type Category List}
%\commandlabsubarg{report}{type}{Category\_List}.
%\defRef{sec:Report-CategoryList}
%\label{syntax:Report-CategoryList}

%The Category\_List report has no additional subcommands.

\subsubsection{Report of type Correlation Matrix}
\commandlabsubarg{report}{type}{Correlation\_Matrix}.
\defRef{sec:Report-CorrelationMatrix}
\label{syntax:Report-CorrelationMatrix}

The Correlation\_Matrix report has no additional subcommands.

\subsubsection{Report of type Covariance Matrix}
\commandlabsubarg{report}{type}{Covariance\_Matrix}.
\defRef{sec:Report-CovarianceMatrix}
\label{syntax:Report-CovarianceMatrix}

The Covariance\_Matrix type has no additional subcommands.

\subsubsection{Report of type Derived Quantity}
\commandlabsubarg{report}{type}{Derived\_Quantity}.
\defRef{sec:Report-DerivedQuantity}
\label{syntax:Report-DerivedQuantity}

\defSub{derived\_quantity}{The derived quantity label}
\defType{String}
\defDefault{No default}
\defValue{If not specified, then the label of the report is assumed to be the derived quantity label}
 
\subsubsection{Report of type Equation Test}
\commandlabsubarg{report}{type}{Equation\_Test}.
\defRef{sec:eq_parser}
\label{syntax:Report-EquationTest}

\defSub{equation}{The equation to do a test run of}
\defType{Vector of strings}
\defDefault{No default}

\subsubsection{Report of type Estimate Summary}
\commandlabsubarg{report}{type}{Estimate\_Summary}.
\defRef{sec:Report-EstimateSummary}
\label{syntax:Report-EstimateSummary}

\defValue{A summary of the estimated (free parameters)}

The Estimate\_Summary type has no additional subcommands.

\subsubsection{Report of type Estimate Value}
\commandlabsubarg{report}{type}{Estimate\_Value}.
\defRef{sec:Report-EstimateValue}
\label{syntax:Report-EstimateValue}

\defValue{The free parameters and their values, in a format suitable for use with \texttt{-i}}

The Estimate\_Value report has no additional subcommands.

\subsubsection{Report of type Estimation Result}
\commandlabsubarg{report}{type}{Estimation\_Result}.
\defRef{sec:Report-EstimationResult}
\label{syntax:Report-EstimationResult}

\defValue{A summary of the results of the minimisation}

The Estimation\_Result report has no additional subcommands.

\subsubsection{Report of type Hessian Matrix}
\commandlabsubarg{report}{type}{Hessian\_Matrix}.
\defRef{sec:Report-HessianMatrix}
\label{syntax:Report-HessianMatrix}

The Hessian\_Matrix report has no additional subcommands.

\subsubsection{Report of type Initialisation}
\commandlabsubarg{report}{type}{Initialisation}.
\defRef{sec:Report-Initialisation}
\label{syntax:Report-Initialisation}

The Initialisation report has no additional subcommands.

\subsubsection{Report of type Initialisation\_Partition}
\commandlabsubarg{report}{type}{Initialisation\_Partition}.
\defRef{sec:Report-InitialisationPartition}
\label{syntax:Report-InitialisationPartition}

\subsubsection{Report of type MCMC Covariance}
\commandlabsubarg{report}{type}{MCMC\_Covariance}.
\defRef{sec:Report-MCMCCovariance}
\label{syntax:Report-MCMCCovariance}\\
\defValue{This will output the covariance matrices (the initial covariance matrix and the covariance matrix if adapted ) used for the MCMC chain.}

The MCMC\_Covariance report has no additional subcommands.  

\subsubsection{Report of type MCMC Objective}
\commandlabsubarg{report}{type}{MCMC\_Objective}.
\defRef{sec:Report-MCMCObjective}
\label{syntax:Report-MCMCObjective}

The MCMC\_Objective report has no additional subcommands.

\defSub{file\_name}{The file name. If not supplied the default filename is used}
\defType{string}
\defDefault{objectives}

\defSub{write\_mode}{Has a different default to the rest of the reports.}
\defType{String}
\defDefault{\subcommand{incremental\_suffix}}
\defValue{valid options are \subcommand{append}, \subcommand{overwrite}, \subcommand{incremental\_suffix}}


\subsubsection{Report of type MCMC Sample}
\commandlabsubarg{report}{type}{MCMC\_Sample}.
\defRef{sec:Report-MCMCSample}
\label{syntax:Report-MCMCSample}


\defSub{file\_name}{The file name. If not supplied the default filename is used}
\defType{string}
\defDefault{samples}

\defSub{write\_mode}{Has a different default to the rest of the reports.}
\defType{String}
\defDefault{\subcommand{incremental\_suffix}}
\defValue{valid options are \subcommand{append}, \subcommand{overwrite}, \subcommand{incremental\_suffix}}

The MCMC\_Sample report has no additional subcommands.

%\subsubsection{Report of type MPD}
%\commandlabsubarg{report}{type}{MPD}.
%\defRef{sec:Report-MPD}
%\label{syntax:Report-MPD}\\
%\defValue{An MPD report, consisting of the free parameters and the covariance matrix}

%The MPD report has no additional subcommands.

\subsubsection{Report of type Objective Function}
\commandlabsubarg{report}{type}{Objective\_Function}.
\defRef{sec:Report-ObjectiveFunction}
\label{syntax:Report-ObjectiveFunction}

The Objective\_Function type has no additional subcommands.

\subsubsection{Report of type Observation}
\commandlabsubarg{report}{type}{Observation}.
\defRef{sec:Report-Observation}
\label{syntax:Report-Observation}

\defSub{observation}{The observation label}
\defType{String}
\defDefault{No default}

\defSub{normalised\_residuals}{Print Normalised Residuals?}
\defType{Boolean}
\defDefault{true}
\defNote{Only generated if valid for associated likelihood}

\defSub{pearsons\_residuals}{Print Pearsons Residuals?}
\defType{Boolean}
\defDefault{true}
\defNote{Only generated if valid for associated likelihood}

\subsubsection{Report of type Output Parameters}
\commandlabsubarg{report}{type}{Output\_Parameters}.
\defRef{sec:Report-OutputParameters}
\label{syntax:Report-OutputParameters}

The Output\_Parameters report has no additional subcommands.

\subsubsection{Report of Parameter transformations}
\commandlabsubarg{report}{type}{parameter\_transformation}.
\defRef{sec:Report-ParameterTransformations}
\label{syntax:Report-ParameterTransformation}

\defSub{parameter\_transformation}{label of parameter transformation block}
\defType{String}
\defDefault{No default}


\subsubsection{Report of type Partition}
\commandlabsubarg{report}{type}{Partition}.
\defRef{sec:Report-Partition}
\label{syntax:Report-Partition}

\defSub{time\_step}{Time Step label}
\defType{String}
\defDefault{No default}

\defSub{years}{Years}
\defType{Vector of non-negative integers}
\defDefault{All years}

\subsubsection{Report of type Partition Biomass}
\commandlabsubarg{report}{type}{Partition\_Biomass}.
\defRef{sec:Report-PartitionBiomass}
\label{syntax:Report-PartitionBiomass}

\defSub{time\_step}{The time step label}
\defType{String}
\defDefault{No default}

\defSub{years}{The years for the report}
\defType{Vector of non-negative integers}
\defDefault{All years}

\subsubsection{Report of type Process}
\commandlabsubarg{report}{type}{Process}.
\defRef{sec:Report-Process}
\label{syntax:Report-Process}

\defSub{process}{The process label that is reported}
\defType{String}
\defDefault{No default}
\defValue{A valid process label}
\defValue{If not specified, then the label of the report is assumed to be the process label}


\subsubsection{Report of type Profile}
\commandlabsubarg{report}{type}{Profile}.
\defRef{sec:Profile}
\label{syntax:Report-Profile}

\subsubsection{Report of type Project}
\commandlabsubarg{report}{type}{Project}.
\defRef{sec:Project}
\label{syntax:Report-Project}

\defSub{project}{The project label that is reported}
\defType{String}
\defDefault{No default}
\defValue{If not specified, then the label of the report is assumed to be the projection label}

\subsubsection{Report of type Random Number Seed}
\commandlabsubarg{report}{type}{Random\_Number\_Seed}.
\defRef{sec:Report-RandomNumberSeed}
\label{syntax:Report-RandomNumberSeed}

The Random\_Number\_Seed type has no additional subcommands.

\subsubsection{Report of type Selectivity}
\commandlabsubarg{report}{type}{Selectivity}.
\defRef{sec:Report-Selectivity}
\label{syntax:Report-Selectivity}

\defSub{selectivity}{Selectivity name}
\defType{String}
\defDefault{No default}
\defValue{If not specified, then the label of the report is assumed to be the selectivity label}

\defSub{length\_values}{Length bins for reporting if a length-based selectivity in an age-based model}
\defType{Vector of real numbers}
\defDefault{If not specified and this is a length-based selectivity in an age-based model, then length bins specified for the model will be used}
\defNote{It is a fatal error if this is a report for a length-based selectivity in an age-based model, but neither the length values or \command{model.length\_bins} were supplied}

\subsubsection{Report of type Selectivity By Year}
\commandlabsubarg{report}{type}{selectivity\_by\_year}.
\defRef{sec:Report-SelectivityByYear}
\label{syntax:Report-SelectivityByYear}

\defSub{selectivity}{Selectivity name}
\defType{String}
\defDefault{No default}
\defValue{If not specified, then the label of the report is assumed to be the selectivity label}

\defSub{years}{years to report the selectivity in}
\defType{String}
\defDefault{true}
\defValue{If not specified will print for all years in of the model}

\defSub{time\_step}{Time step label}
\defType{String}
\defDefault{No default}
\defNote{This should not matter, but is required in order to identify the time step for each year when values are printed.}

\subsubsection{Report of type Simulated Observation}
\commandlabsubarg{report}{type}{Simulated\_Observation}.
\defRef{sec:Report-SimulatedObservation}
\label{syntax:Report-SimulatedObservation}

\defSub{observation}{The observation label}
\defType{String}
\defDefault{No default}
\defValue{If not specified, then the label of the report is assumed to be the observation label}

\subsubsection{Report of type Time Varying}
\commandlabsubarg{report}{type}{Time\_Varying}.
\defRef{sec:Report-TimeVarying}
\label{syntax:Report-TimeVarying}

\defSub{time\_varying}{The time varying label that is reported}
\defType{String}
\defDefault{No default}
\defValue{If not specified, then the label of the report is assumed to be the time varying label}



 \pagebreak 
 \section{\I{Including commands from other files}\label{syntax:General}}

\defComArg{include}{file}{\I{Include an external file}}

\defArg{file}{The name of the external \config\ to include}
\defType{string}
\defDefault{No default}
\defValue{A valid \config}
\defNote{If \texttt{file\_name} includes a space character, then it must be enclosed in quotes, for example \command{include} \argument{\ "my file.csl2"}. Also note that the \command{include} does not denote the end of the previous command block as is the case for all other commands}

 \pagebreak 
 \section{\I{Validating model values using asserts}}\label{syntax:Assert}

\CNAME\ can validate or check certain addressables parameters as a part of testing and validation with the assert command. Asserts check the value of a specific addressables (for example, and observations, parameters, or the objective function). Asserts are one aspect of the internal tests \CNAME\ uses to ensure accuracy across versions and revisions (see Section \ref{sec:Assert})

\subsection{Assert syntax}

\input{IncludedSyntax/Assert}

